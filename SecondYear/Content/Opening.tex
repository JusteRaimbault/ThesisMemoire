


% Chapter 

\chapter{Thematic and General Perspectives} % Chapter title

\label{ch:opening} % For referencing the chapter elsewhere, use \autoref{ch:name} 

%----------------------------------------------------------------------------------------



\headercit{}{}{}

\bigskip






%----------------------------------------------------------------------------------------

\newpage


\section{Développement Spécifiques}

% list here ideas that were not evoked in particular sections, but worth mentioning as each could easily bring significant knowledge.


Le mode de communication scientifique actuel est loin d'être optimal % TODO cit peer-review broken ; paper format ?
et les initiatives se multiplient pour proposer des modèles alternatifs : la revue post-publication en est une, l'utilisation de systèmes de contrôle de version et de dépôts publics une autre, ou la publication éclair de pistes de recherche (Journal of Brief Ideas). % TODO : Journal of Design and Science ; Journal mec Paris Sud (CEA ?)
Les descriptions courtes de pistes de recherche sont souvent reléguées à la discussion ou la conclusion des articles, qui s'écrivent de manière conventionnelle, souvent avec un biais pour justifier a posteriori l'intérêt de \emph{sa nouvelle méthode} qu'il faut malheureusement vendre. On fait alors des plans sur la comète, propose des développements ayant peu de rapport, ou des domaines d'application \emph{qui auront un impact} (lire qui sont à la mode ou qui reçoivent le plus de financements à la période de l'écriture). % note : science anarchiste ? lire Feyerabend
Ce manuscrit tombe bien évidemment partiellement sous ces critiques, et encore plus les articles qui lui sont associés.


Nous proposons dans cette section un exercice pas forcément conventionnel : proposer des idées et développements possibles, en s'efforçant de concrétiser les questions de recherche et/ou points techniques autant que possible, afin que ceux-ci ne s'apparentent pas à une bouteille à la mer.



\subsection{Epistémologie Quantitative}

% - full-text mining ?
% - integrated platform -> mention here CybergeoNetworks ?
%



\subsection{Modèles Multi-scalaires}






\subsection{Vers des Modèles Opérationnels}






%----------------------------------------------------------------------------------------

\newpage


\section{Vers un Programme de Recherche}


\subsection{Pour une Géographie Intégrée Alternative}

% develop here position for a renewal of TQG : additional three dimensions in the knowledge framework ; position at the core of fundamental CS - cannot ignore fundamental questions.




\subsection{Axes de Recherche}






