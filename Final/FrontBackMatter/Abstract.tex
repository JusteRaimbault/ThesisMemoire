% Abstract

%\renewcommand{\abstractname}{Abstract} % Uncomment to change the name of the abstract

\pdfbookmark[1]{Abstract}{Abstract} % Bookmark name visible in a PDF viewer

\begingroup
\let\clearpage\relax
\let\cleardoublepage\relax
\let\cleardoublepage\relax

%\chapter*{Abstract}{Résumé}
\chapter*{Résumé}
%Short summary of the contents\dots a great guide by 
%Kent Beck how to write good abstracts can be found here:  
%\begin{center}
%\url{https://plg.uwaterloo.ca/~migod/research/beckOOPSLA.html}
%\end{center}


La question ouverte de l'existence des effets structurants des infrastructures de transports sur les territoires, est l'une des facettes de dynamiques territoriales complexes, au sein desquelles territoires et réseaux de transport seraient en co-évolution. L'objectif de cette thèse est de mettre à l'épreuve cette vision des interactions entre réseaux et territoires, autant sur le plan conceptuel qu'empirique, dans le but de l'intégrer au sein de modèles de simulation des systèmes territoriaux. La nature intrinsèquement multi-disciplinaire de la question nous conduit dans un premier temps à une analyse d'épistémologie quantitative, qui permet de dresser une carte du paysage scientifique et une description précise de la structure des différents modèles dans chaque discipline. La définition de la co-évolution et une méthode de caractérisation empirique, basée sur une analyse de correlations spatio-temporelles, est élaborée. Deux pistes complémentaires de modélisation, correspondant à des ontologies et des échelles différentes sont alors explorées. A l'échelle macroscopique, nous développons une famille de modèles dans la lignée des modèles d'interaction au sein des systèmes de villes développés par la Théorie Evolutive des Villes. Leur exploration montre qu'ils capturent effectivement des dynamiques de co-évolution, et leur calibration sur données démographiques pour le système de villes français (1830-1999) quantifie l'évolution des processus d'interaction comme l'effet tunnel ou le rôle de la centralité. A l'échelle mesoscopique, un modèle de morphogenèse capture la co-évolution de la forme urbaine et de la topologie du réseau. Il est calibré sur les indicateurs correspondants pour la forme et la topologie locales calculés pour l'ensemble de l'Europe. De multiples processus d'évolution du réseau (planification coût-bénéfices, rupture de potentiel, auto-organisation) sont montrés complémentaires pour produire l'ensemble des configurations réelles. La calibration est également faite au second ordre, c'est à dire sur les interactions entre indicateurs, pour lequel le modèle reproduit une grande diversité de situations existantes. Ces résultats suggèrent d'une part une construction théorique intégrant Théorie Evolutive Urbaine et Morphogenèse, et ouvrent d'autre part l'exploration d'une nouvelle génération de modèles urbains intégrant les dynamiques co-évolutives dans une perspective multi-échelles.




%-----------------------------------------------

\newpage

\cn{
\chapter*{建模交通网络和地域的共同演变 : 摘要}
}

\cn{运输基础设施对领土体系结构效应存在的问题远未得到解决。% La question de l'existence d'effets structurants des infrastructures de transports sur les systèmes territoriaux est loin d'être résolue. 
这是复杂的地域动态的一个方面,其中领土和交通网络正在共同演变。% C'est l'une des facettes de dynamiques territoriales complexes, au sein desquelles territoires et réseaux de transport sont en co-évolution.
这篇论文的目的是测试网络和地域之间的相互作用。%L'objectif de cette thèse est de mettre à l'épreuve cette vision des interactions entre réseaux et territoires.
它将在概念和经验上做到这一点,目的是将其整合到地域系统的模拟模型中。% Elle le fera autant sur le plan conceptuel que le plan empirique, dans le but de l'intégrer au sein de modèles de simulation des systèmes territoriaux.
我们正在处理的问题本质上是多学科的。% La problématique que nous traitons est intrinsèquement multi-disciplinaire.
出于这个原因,我们首先进行量化的认识论分析。% Pour cette raison, nous procédons dans un premier temps à une analyse d'épistémologie quantitative. 
它可以绘制科学的景观图,并精确地描述每个学科不同模型的结构。% Elle permet de dresser une carte du paysage scientifique et une description précise de la structure des différents modèles dans chaque discipline.
我们制定了一个共同进化的定义,并开发了一个基于时空相关分析的经验表征方法。% Nous élaborons une définition de la co-évolution et élaborons une méthode de caractérisation empirique basée sur une analyse de correlations spatio-temporelles.
探索两个互补的建模轨道。 它们对应于不同的本体和尺度。% Deux pistes complémentaires de modélisation sont explorées. Elles correspondent à des ontologies et des échelles différentes.
在宏观层面上,我们根据城市演变理论发展起来的城市体系内的相互作用模型发展了一个模型家族。% A l'échelle macroscopique, nous développons une famille de modèles dans la lignée des modèles d'interaction au sein des systèmes de villes développés par la Théorie Evolutive des Villes.
他们的探索表明,他们实际上捕捉到共同演化的动力。 他们对法国城市系统(1830-1999)的人口统计数据的校准量化了互动过程的演变。 这些例如是隧道效应或网络中心性的影响。% Leur exploration montre qu'ils capturent effectivement des dynamiques de co-évolution. Leur calibration sur données démographiques pour le système de villes français (1830-1999) quantifie l'évolution de processus d'interaction. Ceux-ci sont par exemple l'effet tunnel ou l'impact de la centralité dans le réseau.
在介观尺度上,形态演化模型捕捉城市形态和网络拓扑的共同演化。% A l'échelle mesoscopique, un modèle d'évolution morphologique capture la co-évolution de la forme urbaine et de la topologie du réseau.
根据整个欧洲计算的局部形态和拓扑结构的相应指标进行校准。% Il est calibré sur les indicateurs correspondants pour la forme et la topologie locales calculés pour l'ensemble de l'Europe.
网络演进的多个过程被考虑到:成本效益计划,潜在的突破,自组织。 它们似乎是互补的,可以产生所有的真实配置。% De multiples processus d'évolution du réseau sont pris en compte : planification coût-bénéfices, rupture de potentiel, auto-organisation. Ils apparaissent complémentaires pour produire l'ensemble des configurations réelles. 
校准也是按照第二顺序进行的,也就是指标之间的相互作用,模型重现了现有情况的多样性。% La calibration est également faite au second ordre, c'est à dire sur les interactions entre indicateurs, pour lequel le modèle reproduit une grande diversité de situations existantes.
这些结果一方面表明了把城市演变理论与形式演变相结合的理论建构。 另一方面,他们开辟了新一代城市模式的探索,这些模型将不得不整合多尺度协同进化动力学。% Ces résultats suggèrent d'une part une construction théorique intégrant Théorie Evolutive Urbaine et évolution de la forme. Ils ouvrent d'autre part l'exploration d'une nouvelle génération de modèles urbains qui devra intègre les dynamiques co-évolutives multi-échelles.
}





%-----------------------------------------------

\newpage

\chapter*{Abstract}











% old abstract

%Territorial systems exhibit complexity at any levels and for most of their aspects. Related disciplines \comment{(Florent) geography, planning, socio, economy}
%generally embrace complex systems science approaches to tackle their understanding and the associated dramatic \comment{(Florent)un peu fort ?}
%social and environmental issues. Choosing a specific angle of lecture of territories, \comment{(Florent)trop vite ds problématique, développer d'abord aspect dynamique}, idem
%it appears, following territorial theories of networks, that real networks play a crucial role in system dynamics, and in particular transportation networks. Taking furthermore a modeling paradigm, we ask to what extent a modeling approach to territorial systems as networked human territories can help disentangling complexly involved processes. We propose to build an associated theory, relying on a vision of human territories as networked, combined with the evolutive urban theory and insights from morphogenesis and co-evolution, that we call a \emph{theory of co-evolutive networked territorial systems}. \comment{(Florent) pas clair, pas forcément nommer}
% It is then embedded into a more general epistemological framework insisting on the notions of emergence and modularity. Quantitative epistemological analysis \comment{(Florent)trop précis}
% confirm the manual literature review and guide research towards co-evolutive models of networks and territories. We search for stylized facts in empirical datasets to also guide model construction. Methodological developments allow to expect information on dynamical processes from static correlations between urban morphology and network shape. The first modeling experiments include a calibrated spatial model of urban growth, giving an insight into theoretical assumption of network necessity. This model is then weakly coupled with a network generation heuristic to explore the space of feasible correlations. It paves the road for both comparison with real correlations and a strongly coupled calibrated model. We also explore novel paradigms such as the role of governance processes in network growth, through a game-theoretic agent-based model. These preliminary results provide the roadmap towards a family of comprehensive operational models of co-evolution between networks and territories that aim to disentangle their circular causalities.
%\comment{(Florent) trop de concepts dans l'abstract, peut pas apporter qqchse à tous}
%\comment{(Florent) commencer par expliquer ce que sont causalités circulaires et pourquoi difficiles à modéliser}
%\comment{(Arnaud) complexly : ?}
%\comment{(Arnaud) théorie des systèmes territoriaux en réseau co-évolutifs ?}





\endgroup			

\vfill