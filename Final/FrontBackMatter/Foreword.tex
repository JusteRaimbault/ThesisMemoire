
% Abstract

%\renewcommand{\abstractname}{Abstract} % Uncomment to change the name of the abstract

%\pdfbookmark[1]{Reading Notes}{Reading Notes} % Bookmark name visible in a PDF viewer
\pdfbookmark[1]{Notes de Lecture}{Notes de Lecture}

\begingroup
\let\clearpage\relax
\let\cleardoublepage\relax
\let\cleardoublepage\relax

%\chapter*{Reading Notes}{Notes de Lecture}

\chapter*{Notes de Lecture}



Cette thèse devait initialement être rédigée en anglais pour sa version originale. Un premier tiers et la majorité des articles l'ont été, pour être repris et traduits par la suite, afin de répondre à une contrainte administrative d'un autre âge. Elle avait également été conçue comme une ``thèse à articles'', mais les fortes recommandations du CNU ont vite eu vent de cette ambition. Ainsi, la version courante est passée par maintes transformations et ``lissages'', afin de lui donner une forme, un fond et une identité ``classiques''. Nous nous excusons préalablement auprès du lecteur si des écueils de traduction ou d'articulation subsistent et perturbent la fluidité de la lecture.

L'ensemble des figures est produit par l'auteur, sauf la figure~\ref{fig:computation:xkcd} (source xkcd). La grande majorité des figures est \emph{directement} reproductible, c'est à dire pouvant être obtenue par execution des scripts. L'ensemble des codes sources, des modèles à l'interprétation des résultats et à cette propre rédaction, est disponible de manière ouverte avec l'ensemble de son historique atomique (\emph{commits}) sur le dépôt du projet\footnote{à \url{https://github.com/JusteRaimbault/CityNetwork}}. L'ensemble des jeux de données produits dans ce cadre est ouvert, et l'ensemble des données utilisées sont ouvertes ou rendues ouvertes (de manière agrégée correspondant au niveau d'utilisation par les modèles dans le cas d'une base tierce fermée).

Ce mémoire en lui-même a été relu par les lecteurs suivants (ordre alphabétique) : Arnaud Banos (AB), Florent Le Néchet (FL), Clémentine Cottineau (CC), dans l'esprit d'une revue ouverte : en suivant les commits successifs à \url{https://github.com/JusteRaimbault/ThesisMemoire}, l'utilisation de commandes spécifiques permet de retracer l'ensemble du processus de revue.

Les noms en Mandarin (villes, lieux, personnes, etc.) sont transcrits en système \emph{pinyin}.




\endgroup			

\vfill







%\textit{Cette thèse est un voyage, tout d'abord géographique au travers des territoires très divers que nous visiterons tout autour du monde et dans des mondes qui n'existent pas. Un voyage entre des disciplines qui n'ont pas forcément l'habitude de se parler. Un voyage au delà des illusions et des idéaux naïfs sur une recherche qui serait surhumaine, un voyage initiatique dans la médiocrité quotidienne et l'étroitesse d'esprit, particulièrement dans ses relations à l'enseignement. Un trip sous drogues diverses qui aura cherché un sens jusqu'au bout pour comprendre que le sens du sens lui-même n'en avait pas. Une exploration préliminaire effleurant l'immensité des voyages qui nous attendent plus tard.}


