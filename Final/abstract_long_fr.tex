\documentclass[9pt]{article}

\usepackage[utf8]{inputenc}
\usepackage[T1]{fontenc}

\usepackage[margin=1.5cm]{geometry}

\newcommand{\noun}[1]{\textsc{#1}}

\usepackage[strict]{changepage}



\begin{document}




\title{Caractérisation et modélisation de la co-évolution des réseaux de transport et des territoires
}
\author{}

\date{}

\maketitle

%\justify

\pagenumbering{gobble}

\medskip
%\vspace{1cm}

\textit{Juste Raimbault, Thèse de Doctorat en Géographie de l'Université Paris VII - Denis Diderot, soutenue le 10 juin 2018.}

\medskip

%\vspace{1cm}

%Soutenue publiquement le 11 juin 2018 devant le jury composé de :
%\bigskip
%\begin{adjustwidth*}{-0.5cm}{-2cm}
%\begin{minipage}{0.28\linewidth}
%\raggedright
%\textbf{\noun{Denise Pumain}}\\
%\textbf{\noun{Didier Josselin}}\\
%\textbf{\noun{Catherine Morency}}\\
%\textbf{\noun{Olivier Bonin}}\\
%\textbf{\noun{Anne Ruas}}\\
%\textbf{\noun{Arnaud Banos}}\\
%\textbf{\noun{Florent Le Néchet}}
%\end{minipage}
%\begin{minipage}{0.7\linewidth}
%\raggedright
%Professeure, Université Paris 1 (Présidente du Jury)\\
%Directeur de Recherche, CNRS (Rapporteur)\\
%Professeure, Ecole Polytechnique de Montréal (Rapporteuse)\\
%Chargé de Recherche, IFSTTAR (Examinateur)\\
%Directrice de Recherche, IFSTTAR (Examinatrice)\\
%Directeur de Recherche, CNRS (Directeur)\\
%Maître de Conférence, Université Paris-Est (Directeur)\\
%\end{minipage}
%\end{adjustwidth*}


%\vspace{1.5cm}



%\textbf{Caractérisation et modélisation de la co-évolution des réseaux de transport et des territoires}
%\bigskip

\noindent\textbf{Mots-clés : } Territoires ; Réseaux de Transport ; Co-évolution ; Morphogenèse ; Théorie Évolutive des Villes ; Épistémologie Quantitative ; Systèmes de Villes ; Morphologie Urbaine ; Grand Paris ; Delta de la Rivière des Perles

\bigskip


\section*{Résumé}



L'identification d'effets structurants des infrastructures de transports sur la dynamique des territoires reste un défi scientifique ouvert. Cette question est une des facettes de recherches sur la complexité des dynamiques territoriales, au sein desquelles territoires et réseaux de transport seraient en co-évolution. L'objectif de cette thèse est de mettre à l'épreuve cette vision des interactions entre réseaux et territoires, autant sur le plan conceptuel que sur le plan empirique, en les intégrant au sein de modèles de simulation des systèmes territoriaux. La problématique est ainsi double: d'une part comment caractériser les interactions entre réseaux de transport et territoires et leur co-évolution le cas échéant, et d'autre part comment modéliser celle-ci.


La nature intrinsèquement pluri-disciplinaire de la question nous conduit à mener un travail d'épistémologie quantitative, qui permet de dresser une carte du paysage scientifique et une description des éléments communs et des spécificités des modèles traitant la co-évolution entre réseaux et territoires dans chaque discipline. Les cartes scientifique par réseaux de citation et réseaux sémantiques révèlent des approches complémentaires allant de l'économie à la planification et la géographie, ainsi que la caractérisation de chaque domaine en termes d'interdisciplinarité. Nous menons également une revue systématique et une modélographie liant caractéristiques des modèles et positionnement scientifique. Ce travail suggère la nécessité d'aborder plusieurs échelles et permet de mieux situer les choix ontologiques faits pour la modélisation.


Nous proposons ensuite une définition de la co-évolution, ainsi qu'une méthode de caractérisation empirique qui lui est associée, basée sur une analyse de corrélations spatio-temporelles retardées similaire à une causalité de Granger. Cette méthode qui se veut très générique est testée sur données synthétiques montrant l'existence de régimes de co-évolution dans un modèle simple de morphogenèse urbaine. Elle est également appliquée aux dynamiques territoriales du Grand Paris, montrant des dynamiques de type effet structurant pour des variables foncières, ainsi qu'au système de ville sud-africain sur le temps long au sein duquel est mise en évidence une inversion du sens de causalité entre accessibilité ferroviaire et population due aux évènements historiques.


Deux pistes complémentaires de modélisation, correspondant à des ontologies et des échelles différentes sont alors explorées. À l'échelle macroscopique, nous construisons une famille de modèles dans la lignée des modèles d'interaction au sein des systèmes de villes développés par la Théorie Evolutive des Villes. Une première exploration d'un modèle avec réseau statique permet l'identification d'effets de réseau dans les dynamiques des systèmes de villes. L'extension en modèle co-évolutif montre que ce type de modèle capture effectivement des dynamiques de co-évolution en termes de régimes co-évolutif identifié par la méthode de caractérisation. La calibration sur des données démographiques pour le système de villes français (1830-1999), couplées aux données du réseau ferroviaire, permet de quantifier l'évolution des processus d'interaction entre réseaux et territoires comme l'effet tunnel ou le rôle de la centralité.


A l'échelle mésoscopique, des modèles de morphogenèse sont introduits. Un modèle de type reaction-diffusion avec densité de population seule permet une bonne reproduction de formes urbaines existantes. Son extension capture la co-évolution de la forme urbaine et de la topologie du réseau, permettant d'inclure à la fois forme et fonction urbaines. Il est calibré sur les indicateurs correspondants pour la forme et la topologie locales calculés pour l'ensemble de l'Europe. De multiples processus d'évolution du réseau s'avèrent être complémentaires pour reproduire la grande variété des configurations observées, au niveau des indicateurs ainsi que des interactions entre indicateurs. Par ailleurs, un modèle à l'échelle métropolitaine, simulant l'extension du réseau d'infrastructures de transport en incluant des processus de gouvernance, en couplage à un modèle d'évolution d'usage du sol, est étudié pour des régions urbaines synthétiques et appliqué au Delta de la Rivière des Perles.


Les résultats de cette thèse suggèrent ainsi de nouvelles pistes d'exploration des modèles urbains intégrant les dynamiques co-évolutives dans une perspective multi-échelles.









%%%%%%%%%%%%%%%%%%%%
%% Biblio
%%%%%%%%%%%%%%%%%%%%


%\begin{multicols}{2}

%\setstretch{0.3}
%\setlength{\parskip}{-0.4em}


%\bibliographystyle{apalike}
%\bibliography{/Users/Juste/Documents/ComplexSystems/CityNetwork/Biblio/Bibtex/CityNetwork}

%\end{multicols}

\end{document}
