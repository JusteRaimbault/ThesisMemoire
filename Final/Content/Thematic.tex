

% Chapter 




%\chapter{Interactions between Networks and Territories}{Interactions entre Réseaux et Territoires} % Chapter title
\chapter{Interactions entre Réseaux et Territoires}


\label{ch:thematic} % For referencing the chapter elsewhere, use \autoref{ch:name} 




%----------------------------------------------------------------------------------------

%\headercit{If you are embarrassed by the precedence of the chicken by the egg or of the egg by the chicken, it is because you are assuming that animals have always be the way they are}{Denis Diderot}{\cite{diderot1965entretien}}

%\headercit{Si la question de la priorit{\'e} de l'\oe{}uf sur la poule ou de la poule sur l'\oe{}uf vous embarrasse, c'est que vous supposez que les animaux ont {\'e}t{\'e} originairement ce qu'ils sont {\`a} pr{\'e}sent.
%}{Denis Diderot}{\cite{diderot1965entretien}}


\bigskip

%Ce chapitre introductif est destiné à poser le cadre thématique, les contextes géographiques sur lesquels les développements suivants se baseront.\comment[AB]{ya til besoin de le dire ?} 

\bpar{
This analogy is ideal to evoke the questions of causality and processes in territorial systems. When trying to tackle naively our preliminary question, some observers have qualified the identification of causalities in complex systems as ``chicken and egg'' problems : if one effect appears to cause another and reciprocally, how can one disentangle effective processes ? This vision is often present in reductionist approaches that do not postulate an intrinsic complexity in studied systems. The idea that Diderot suggests is the notion of \emph{co-evolution} that is a central phenomenon in evolutive dynamics of Complex Adaptive Systems as \noun{Holland} develops in~\cite{holland2012signals}. He links the notion of emergence (that is ignored in a reductionist vision), in particular the emergence of structures at an upper scales from the interactions between agents at a given scale, materialized generally by boundaries, that become crucial in the coevolution of agents at any scales : the emergence of one structure will be simultaneous with one other, each exploiting their interrelations and generated environments conditioned by their boundaries. We shall explore these ideas in the case of territorial systems in the following.
}{
Pour mieux visualiser les notions de causalités circulaires dans les systèmes complexes, et pourquoi celles-ci peuvent conduire à des paradoxes en apparence, l'image fournie par \noun{Diderot} dans~\cite{diderot1965entretien} est éclairante : ``\textit{Si la question de la priorit{\'e} de l'\oe{}uf sur la poule ou de la poule sur l'\oe{}uf vous embarrasse, c'est que vous supposez que les animaux ont {\'e}t{\'e} originairement ce qu'ils sont {\`a} pr{\'e}sent}''. En voulant traiter naïvement des questions similaires induites par notre problématique introduite précédemment, les causalités au sein de systèmes complexes géographiques peuvent être présentées comme un problème ``de poule et {\oe}uf'' : si un effet semble causer l'autre et réciproquement, est-il possible et même pertinent de vouloir isoler les processus correspondants, s'ils font en fait partie d'un système plus large qui évolue à d'autres échelles ? Une vision réductrice, qui consisterait à attribuer des rôles systématiques à l'une composante ou l'autre, s'oppose à l'idée suggérée par \noun{Diderot} qui rejoint celle de \emph{co-évolution}. L'un des enjeux est donc de dresser un aperçu des processus d'interactions entre réseaux et territoires, afin de préciser la définition de la co-évolution, ce qui sera fait à l'issue d'un travail similaire pour les approches par la modélisation, à la fin de la première partie.
}

%Nous allons revenir sur le cas des réseaux de transport et des territoires, introduit précédemment de manière préliminaire, pour voir dans quelle mesure ceux-ci mobilisent intrinsèquement ces concepts, à la fois dans leur construction théorique, mais aussi dans leur diverses manifestations empiriques.\comment[AB]{phrase a supprimer ?}

\bpar{
This introductive chapter aims to set up the thematic scene, the geographical context in which further developments will root. It is not supposed to be understood as an exhaustive literature review nor the fundamental theoretical basement of our work (the first will be an object of chapter~\ref{ch:quantepistemo} whereas the second will be earlier tackled in chapter~\ref{ch:theory}), but more as narration aimed to introduce typical objects and views and construct naturally research questions.
}{
Ce chapitre doit être lu comme la construction introduisant nos objets et positions d'étude, et sera complété par une revue de littérature exhaustive sur le sujet précis de la modélisation des interactions, qui fera l'objet du Chapitre~\ref{ch:modelinginteractions}. Dans une première section~\ref{sec:networkterritories}, nous préciserons l'approche prise de l'objet territoire, et dans quelle mesure celui-ci naturellement implique la considération des réseaux de transport pour la compréhension des dynamiques couplées. Cela permet de construire un cadre de lecture définissant les systèmes territoriaux, particulièrement adapté à notre approche par la co-évolution. Ces considérations abstraites seront illustrées par des cas d'étude empiriques dans la deuxième section~\ref{sec:casestudies}, choisis très différents pour comprendre les enjeux d'universalité sous-jacents : la métropole du Grand Paris et le Delta de la rivière des Perles en Chine. Enfin, dans la troisième section~\ref{sec:qualitative}, des éléments d'observation de terrain effectués en Chine préciseront et complexifient la construction de ce cadre théorique et empirique.
}

%Nous ne proposons pas non plus tandis que le second sera traité systématiquement dans le chapitre~\ref{ch:theory} lorsque le recul nécessaire aura été progressivement construit.
%ni comme les fondations théoriques fondamentales de notre travail


\stars


\textit{Ce chapitre est entièrement inédit.}







%-------------------------------



























