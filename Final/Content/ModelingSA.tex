



%----------------------------------------------------------------------------------------

\newpage

\section{Modeling Interactions}{Modéliser les interactions}
\label{sec:modelingsa}


%----------------------------------------------------------------------------------------




\subsection{Modeling in Quantitative Geography}{Modélisation en Géographie Quantitative}


\subsubsection{History}{Histoire}

\bpar{
Modeling has in Theoretical and Quantitative Geography (TQG) a privileged role. \cite{cuyala2014analyse} proposes an analysis of the spatio-temporal development of French speaking TQG scientific movement and underlines the emergence of the discipline as the combination between quantitative analysis (e.g. spatial analysis or modeling and simulation practices) and theoretical constructions. This dynamic can be tracked back to the end of the seventies, and is closely linked to the growing use and appropriation of mathematical tools~\cite{pumain2002role}. The integration of these two components allows to construct theories from empirical stylized facts, which then produce theoretical hypothesis that can be tested on empirical data. This approach is born under the influence of the \emph{New Geography} in Anglo-Saxon countries and in Sweden.
}{
La modélisation joue en géographie théorique et quantitative un rôle fondamental. \cite{cuyala2014analyse} procède à une analyse spatio-temporelle du mouvement de la géographie théorique et quantitative en langue française et souligne l'émergence de la discipline comme une combinaison d'analyses quantitatives (e.g. analyse spatiale et pratiques de modélisation et de simulation) et de construction théoriques. Cette dynamique est datée à la fin des années 1970, et est intimement liée à l'utilisation et l'appropriation des outils mathématiques~\cite{pumain2002role}. L'intégration de ces deux composantes permet la construction de théories à partir de faits stylisés empiriques, qui produisent à leur tour des hypothèses théoriques pouvant être testées sur les données empiriques. Cette approche est née sous l'influence de la \emph{New Geography} dans les pays Anglo-saxons et en Suède.
}


\bpar{
Concerning urban modeling n itself, other fields than geography have proposed simulation models approximatively at the same period. For example, the \noun{Lowry} model, developed by~\cite{lowry1964model} with the objective to be applied directly to the Pittsburg metropolitan region, assumes a system of equations for the localization of actives and employments in different areas. This model has been a cornerstone of urban modeling, since as shows \cite{goldner1971lowry} it already had less than ten years after a broad heritage of conceptual and operational developments\footnote{\cite{goldner1971lowry} makes the hypothesis that this success is due to the combination of three factors: a possibility of an immediate operational application, a causal structure of the model easy to grasp (actives relocate depending on employments), and a flexible frame that can be extended or adapted.}. Relatively similar models are still largely used nowadays.
}{
Concernant la modélisation urbaine en elle-même, d'autre champs que la géographie ont proposé des modèles de simulation à peu près à la même période. Par exemple, le modèle de \noun{Lowry}, développé par~\cite{lowry1964model} dans un but appliqué immédiat à la région métropolitaine de Pittsburg, suppose un système d'équations pour la localisation des actifs et des emplois dans différentes zones. Ce modèle a été une pierre angulaire de la modélisation urbaine, puisque comme le montre~\cite{goldner1971lowry} il avait déjà moins d'une dizaine d'années plus tard un conséquent héritage de développements conceptuels et opérationnels\footnote{\cite{goldner1971lowry} fait l'hypothèse que ce succès est du à la combinaison de trois facteurs : une possibilité d'application opérationnelle directe, une structure causale du modèle simple à appréhender (les actifs se localisent en fonction des emplois), et un cadre flexible pouvant être étendu ou adapté.}. Des modèles relativement similaires sont toujours largement utilisés aujourd'hui.
}




% Lowry : https://www.rand.org/content/dam/rand/pubs/research_memoranda/2006/RM4035.pdf
% Alonso : http://onlinelibrary.wiley.com/doi/10.1111/j.1435-5597.1967.tb01370.x/epdf
% http://www.sciencedirect.com/science/article/pii/S009411909792074X prefatt-diff ?


\subsubsection{Simulation of models and intensive computation}{Simulation de modèle et calcul intensif}


\bpar{
A broad history of the genesis of models of simulation in geography is done by~\cite{rey2015plateforme} with a particular emphasis on the notion of validation of models (we will come back on the role of these aspects in our work in~\ref{ch:positioning}). The use of computation ressources for the simulation of models is anterior to the introduction of current paradigms of complexity, coming back for exemple to \noun{Forrester}, a computer scientist pioneer in spatial economics models inspired by cybernetics\footnote{Which was, together with the systemic trend, precursors of current paradigms of complexity as we already developed.}. With the increase of computational capabilities, epistemological transformations have also occurred, with the emergence of explicative models as experimental tools. \noun{Rey} compares the dynamism of seventies when computation centers were opened to geographers to the current democratization of High Performance Computing\footnote{The development of the first urban simulation models coincides with the opening of the first computation centers to social sciences and humanities, as recalls also \noun{Pumain} (interview on 31/03/2017, see Appendix~\ref{app:sec:interviews}) for example for the implementation of the \noun{Allen} entropy model.}. Today, this ease of use is in particular exemplified by grid computing with a transparent use, i.e. without the need for advanced technical skills related to mechanisms of computation distribution. This way, \cite{schmitt2014half} givef an exemple of the possibilities offered in terms of model validation and calibration, reducing the computational time from 30 years to one week - these techniques will play a crucial role in the results we will obtain in the following. This evolution is also accompanied by an evolution of modeling practices~\cite{banos2013pour} and techniques~\cite{10.1371/journal.pone.0138212}.
}{
Une histoire étendue de la genèse des modèles de simulation en géographie est faite par \cite{rey2015plateforme} avec une attention particulière pour la notion de validation de modèles (nous reviendrons sur la place de ces aspects dans notre travail en~\ref{ch:positioning}). L'utilisation de ressources de calcul pour la simulation de modèles est antérieure à l'introduction des paradigmes de la complexité actuels, remontant par exemple à \noun{Forrester}, informaticien qui a été pionnier des modèles d'économie spatiale inspirés par la cybernétique\footnote{Celle-ci, ainsi que le courant systémique, sont comme nous l'avons déjà développé précurseurs des paradigmes actuels de la complexité.}. Avec l'augmentation des potentialités de calcul, des transformations épistémologiques ont également suivi, avec l'apparition de models explicatifs comme outils expérimentaux. \noun{Rey} compare le dynamisme des années soixante-dix quand les centres de calcul furent ouverts aux géographes à la démocratisation actuelle du Calcul Haute Performance\footnote{Le développement des premiers modèles de simulation urbaine coincide avec l'ouverture des premiers centres de calcul aux sciences humaines, comme le rappelle par ailleurs \noun{Pumain} (entretien du 31/03/2017, voir Annexe~\ref{app:sec:interviews}) par exemple pour l'implémentation du modèle d'entropie d'\noun{Allen}.}. Aujourd'hui, cette facilité d'accès consiste entre autres à du calcul sur grille dont l'utilisation est rendue transparente, c'est-à-dire sans besoin de compétences techniques pointues liées au mécanismes de la distribution des calculs. Ainsi~\cite{schmitt2014half} donnent un exemple des possibilités offertes en termes de calibration et de validation de modèle, réduisant le temps de calcul nécessaire de 30 ans à une semaine - ces techniques jouent un rôle clé pour les résultats que nous obtiendrons par la suite. Cette évolution est également accompagnée par une évolution des pratiques~\cite{banos2013pour} et techniques~\cite{10.1371/journal.pone.0138212} de modélisation.
}


\bpar{
Modeling, and in particular computational models of simulation, is seen by many as a fundamental building brick of knowledge: \cite{livet2010} recalls the combination of empirical, conceptual (theoretical) and modeling domains, with constructive feedbacks between each domain. A model can be an exploration tool to test assumptions, an empirical tool to validate a theory against datasets, an explicative tool to reveal causalities and internal processes of a system, a constructive tool to iteratively build a theory jointly with associated models. These are examples among others: \cite{varenne2010simulations} proposes a classification of diverse functions of a model. We will consider modeling as a fundamental instrument of knowledge on processes within systems, and more particularly in our case within complex adaptive systems. We recall thus that our research question will focus on \emph{models which ontology is mainly composed by interactions between transportation networks and territories}.
}{
La modélisation, et en particulier les modèles de simulation, est vue par beaucoup comme une brique fondamentale de la connaissance : \cite{livet2010} rappelle la combinaison des domaines empirique, conceptuel (théorique) et de la modélisation, avec des retroactions constructives entre chaque. Un modèle peut être un outil d'exploration pour tester des hypothèses, un outil empirique pour valider une théorie sur des jeux de données, un outil explicatif pour révéler des causalités et ainsi des processus internes au système, un outil constructif pour construire itérativement une théorie conjointement avec celle des modèles associés. Ce sont des exemples de fonctions parmi d'autres : \cite{varenne2010simulations} propose une classification des diverses fonctions d'un modèle. Nous considérons la modélisation comme un instrument fondamental de connaissance des processus au sein d'un système, plus particulièrement dans notre cas au sein d'un système complexe adaptatif. Nous rappelons ainsi que notre question de recherche s'intéressera aux \emph{modèles dont l'ontologie contient une part non négligeable d'interactions entre réseaux et territoires}.
}




%%%%%%%%%%%%%%%%%%%%%%%%%%%
\subsection{Modeling networks and territories}{Modéliser les territoires et réseaux}


\bpar{
We develop now an overview of different approaches modeling interactions between networks and territories. First of all, we need to notice a high contingency of scientific constructions underlying these. Indeed, according to~\cite{bretagnolle2002time}, the ``\textit{ideas of specialists in planning aimed to give definitions of city systems, since 1830, are closely linked to the historical transformations of communication networks}''. The historical context (and consequently the socio-economical and technological contexts) conditions strongly the formulated theories. It implies that ontologies and corresponding models addressed by geographers and planners are closely linked to their current historical preoccupations, thus necessarily limited in scope and/or operationnal purpose. In a perspectivist vision of science~\cite{giere2010scientific}, such boundaries are the essence of the scientific entreprise, and as we will argue in chapter~\ref{ch:theory} their combination and coupling in the case of models is generally a source of knowledge.
}{
Développons à présent un aperçu des différentes approches modélisant des interactions entre réseaux de transport et territoires. Remarquons de manière préliminaire une forte contingence des constructions scientifiques sous-jacentes à celles-ci. En effet, selon~\cite{bretagnolle2002time}, ``\textit{les idées des spécialistes de la planification cherchant à donner des définitions des systèmes de ville, depuis 1830, sont étroitement liées aux transformations des réseaux de communication}''. Le contexte historique (et donc socio-économique et technologique) conditionne fortement les théories formulées. Cela implique que les ontologies et les modèles correspondants proposés par les géographes et les planificateurs sont fortement liés aux préoccupations historiques courantes, ce qui limite nécessairement leur portée théorique et/ou opérationnelle. Au delà de la question de la définition du système qui joue également un rôle central, on comprend bien l'impact que peut avoir cette influence sur la portée des modèles développés. Dans une vision perspectiviste de la science~\cite{giere2010scientific} de telles limites sont l'essence de l'entreprise scientifique, et comme nous suggèrerons dans le chapitre~\ref{ch:theory} leur combinaison et couplage dans le cas de modèles est généralement une source de connaissance.
}


%C'est en quelque sorte la prophétie auto-réalisatrice inversée, au sens où elle est déjà réalisée avant d'être formulée\comment[FL]{sens ?}. % -> notion de coevol des connaissance avec les objets - on revient sur la reflexivité - trop glissant pour développer.

% \comment[FL]{pas par essence mais cela peut l'etre effectivement}[(JR) pas d'accord : opérer un couplage de modele implique un couplage des ontologies et necessairement un accroissement des connaissances (si celui-ci est utile c'est un autre problème)]


\bpar{
The entry we take here to sketch an overview of models is complementary to the one taken in chapter~\ref{ch:thematic}, by declining them through their main object (i.e. the relations Network $\rightarrow$ Territory, Territory $\rightarrow$ Network and Territory $\leftrightarrow$ Network)\footnote{We recall the meaning of this notation introduced in chapter~\ref{ch:thematic}: a cirect arrow correspond to processes that we can relatively univocally attribute to the origin, whereas a reciprocal arrow assumes the intrinsic existence of reciprocal interaction, generally in coincidence with the emergence of entities playing a role in these.}.
}{
L'entrée que nous proposons ici pour dresser un aperçu des modèles est complémentaire à celle prise au chapitre~\ref{ch:thematic}, en regardant par objet principal (c'est-à-dire les relations Réseau $\rightarrow$ Territoire, Territoire $\rightarrow$ Réseau et Territoire $\leftrightarrow$ Réseau)\footnote{Nous rappelons la signification de cette notation introduite au chapitre~\ref{ch:thematic} : une flèche directe signifie des processus qu'on peut attribuer relativement de manière univoque à l'origine, tandis qu'une flèche réciproque suppose intrinsèquement l'existence d'interactions réciproques, généralement en coincidence avec l'émergence d'entités jouant un rôle dans celles-ci.}.
}


\bpar{
The reference frame for scales is also the one introduced in chapter~\ref{ch:thematic}, knowling that we do not consider the microscopic scales by choice to discard daily mobility. We have therefore roughly mesoscopic and macroscopic temporal and spatial scales.
}{
Le cadre de lecture des échelles est également celui proposé au chapitre~\ref{ch:thematic}, sachant que nous ne nous intéressons pas aux échelles microscopiques par choix de ne pas considérer la mobilité quotidienne. On a ainsi schématiquement des échelles temporelles et spatiales mesoscopiques et des échelles macroscopiques.
}


\bpar{
We have seen that the correspondence to temporal and spatial scales is not systematic (see the provisional double entry typology for processes). On the contrary, the correspondence to fields of study and types of stakeholders is more systematic. This literature review is thus done following the latest logic.
}{
Nous avons vu que la correspondance à des échelles temporelles et spatiales n'est pas systématique (voir la typologie provisoire à double entrée des processus). Par contre, celle à des domaines particuliers et à des acteurs l'est plus. Cette revue de littérature est donc orientée dans cette seconde direction.
}


%\subsubsection{Land-Use Transportation Interaction Models}{Modèles LUTI}
\subsubsection{Territories}{Territoires}


\bpar{
The main current dealing with the modeling of the influence of transportation networks on territories lies in the field of planning, at medium temporal and spatial scales (the scales of metropolitan accessibility we developed before). Models in geography at other scales, such as the Simpop models already described~\cite{pumain2012multi}, do not include a particular ontology for transportation networks, and even if they include networks between cities as carriers of exchanges, they do not allow to study in particular the relations between networks and territories. We will come back later on extensions that are relevant for our question. First, let recall the context of models closer to planning studies.
}{
Le courant principal s'intéressant à la modélisation de l'influence du réseau de transport sur les territoires se trouve dans le champ de la planification, à des échelles spatiales et temporelles moyennes (les échelles de l'accessibilité métropolitaine que nous avons développées ci-dessus). Des modèles en géographie à d'autres échelles, comme les modèles Simpop déjà évoqués~\cite{pumain2012multi}, ne supposent pas une ontologie particulière pour le réseau de transport, et s'ils incluent des réseaux entre les villes comme porteur des échanges, ne permettent cependant pas d'étudier en particulier les relations entre réseaux et territoires. Nous reviendrons plus loin sur des extensions pertinentes pour notre question. Revoyons pour commencer un contexte de modèles plus proches des études de planification.
}





\paragraph{LUTI models}{Modèles LUTI}

\bpar{
These approaches are generally named as \emph{models of the interaction between land-use and transportation} (\emph{LUTI}, for \textit{Land-Use Transport Interaction}). Land-use generally means the spatial distribution of territorial activities, generally classified into more or less precise typologies (for example housing, industry, tertiary, natural space). These works can be difficult to apprehend as they relate to different scientific disciplines\footnote{We make here the choice to gather numerous approaches having the common characteristic to principally model the evolution of land-use, on medium temporal and spatial scales. The unity and the relative positioning of these approaches covering from economics to planning, remain an open question that to the best of our knowledge has never been frontally tackled. The work done in~\ref{sec:quantepistemo} introduces elements of answer through an approach in quantitative epistemology.}. Their general principle is to model and simulate the evolution of the spatial distribution of activities, taking transportation networks as a context and significant drivers of localizations. To understand the underlying conceptual frame to most approaches, the Frame~\ref{frame:modelingsa:wegener} sums up the one given by \cite{wegener2004land}\footnote{A more general frame that we already developed, that allows to bridge it with our frame, is the one given by~\cite{le2010approche}, which situates the triad Transportation system/Localization system/Activities system within the relation with agents: agents creating demand, agents building the city, external factors.}.
}{
Ces approches sont désignées de manière générale comme \emph{modèles d'interaction entre usage du sol et transport} (\emph{LUTI}, pour \textit{Land-Use Transport Interaction}). Il est entendu par usage du sol la répartition des activités territoriales, généralement réparties en typologies plus ou moins précises (par exemple logements, industrie, tertiaire, espace naturel). Ces travaux peuvent être difficiles à cerner car liés à différentes disciplines scientifiques\footnote{Nous prenons le parti ici de rassembler de nombreuses approches ayant la caractéristique commune de modéliser principalement l'évolution de l'usage du sol, sur des échelles temporelles et spatiales moyennes. L'unité et le positionnement relatif de ces approches couvrant de l'économie à la planification, sont une question ouverte qui n'a à notre connaissance jamais été traitée de front. Le travail mené en~\ref{sec:quantepistemo} donne des pistes de réponse par une approche d'épistémologie quantitative.}. Leur principe général est de modéliser et simuler l'évolution de la distribution spatiale des activités, en prenant le réseau de transport comme contexte et déterminant significatif des localisations. Pour comprendre le cadre conceptuel sous-jacent à la majorité des travaux, l'Encadré~\ref{frame:modelingsa:wegener} résume celui issu de \cite{wegener2004land}\footnote{Un cadre plus général que nous avons déjà développé, qui permet de faire le pont avec notre cadre, est celui de~\cite{le2010approche}, qui replace le triptyque Système de transport/Système de localisation/Système d'activités en relation avec les agents : agents demandeurs, agents aménageurs, facteurs externes.}.
}


\bpar{
For example, from the point of view of urban economics, propositions for such models have existed for a relatively long time: \cite{putman1975urban} recalls the frame of urban economics in which main components are employments, demography and transportation, and reviews economic models of localization that relate to the \noun{Lowry} model already mentioned.
}{
Par exemple, du point de vue de l'économie urbaine, les propositions de tels modèles existent depuis un certain temps : \cite{putman1975urban} rappelle le cadre d'économie urbaine où les principales composantes sont les emplois, la démographie et le transport, et passe en revue des modèles économiques de localisation qui s'apparentent au modèle de \noun{Lowry} déjà mentionné.
}





%%%%%%%%%%%%%%
\begin{figure}[h!]
	\begin{mdframed}
		
		\bpar{
		\cite{wegener2004land} introduces a general theoretical and empirical frame for land-use transport interaction models. The four concepts included are land-use, localization of activities, the transportation system and the distribution of accessibility. A cycle of circular effects are summed up in the following loop:
		}{
		\cite{wegener2004land} pose un cadre général théorique et empirique pour les modèles d'interaction entre transport et usage du sol. Les quatre concepts mobilisés sont l'usage du sol, la localisation des activités, le système de transport et la distribution de l'accessibilité. Un cycle d'effets circulaires sont résumés dans la boucle suivante :
		}		
		
		\begin{center}
		\bigskip
		\bpar{
		\tikzmark{Activities} $\longrightarrow$ Transportation system $\longrightarrow$ Accessibility $\longrightarrow$ Land-\tikzmark{use}\arrow{use}{Activities}
		}{
		 \tikzmark{Activités} $\longrightarrow$ Système de Transport $\longrightarrow$ Accessibilité $\longrightarrow$ Usage \tikzmark{du} sol\arrow{du}{Activités}
		}		
		\bigskip
		\end{center}
		
		\bpar{
		The transportation system is assumed with a \emph{fixed infrastructure}, i.e. effects of the distribution of activities are effects on the \emph{use} of the transportation system (and thus link to \emph{mobility} in our more general frame): modal choice, frequency of trips, length of travels.
		}{
		Le système de transport est supposé \emph{à infrastructure fixe}, c'est-à-dire que les effets de la distribution des activités sont ceux sur \emph{l'utilisation} du système de transport (donc liés à la \emph{mobilité} dans notre cadre plus général) : choix modal, fréquence des voyages, longueur des voyages.
		}
		
		\bpar{
		The theoretically expected effects are classified according to the direction of the relation (\textit{Land-use}$\rightarrow$\textit{Transport} or \textit{Transport}$\rightarrow$\textit{Land-use}, and a loop \textit{Transport}$\rightarrow$\textit{Transport} that is not taken into account in our case), and according to the acting factor (residential density, of employments, localization, accessibility, transportation costs) and also by the aspect that is modified (length and frequency of trips, modal choice, densities, localizations). We can for example take:
		\begin{itemize}
		\item \textit{Land-use}$\rightarrow$\textit{Transport}: a minimal residential density is necessary for the efficiency of public transportation, a concentration of employments implies longer trips, larger cities have a greater proportion of the modal part of public transportation.
		\item \textit{Transport}$\rightarrow$\textit{Land-use}: a high accessibility implies higher prices and an increased development of residential housing, companies locate for a better accessibility to transportation at a larger scale.
		\item \textit{Transport}$\rightarrow$ \textit{Transport}: places with a good accessibility will produce more and longer trips, modal choice and transportation cost are highly correlated. 
		\end{itemize}
		}{
		Les effets théoriquement attendus sont classés selon les directions de la relation (\textit{Usage du sol}$\rightarrow$\textit{Transport} ou \textit{Transport}$\rightarrow$\textit{Usage du sol}, ainsi qu'une boucle \textit{Transport}$\rightarrow$\textit{Transport}, qui fait partie des éléments ignorés dans notre cas), et par ailleurs par facteur agissant (densité résidentielle, d'emplois, localisation, accessibilité, coûts de transport) ainsi que par aspect affecté (longueur et fréquence des voyages, choix de mode, densités, localisations).  On peut par exemple citer :
		\begin{itemize}
			\item \textit{Usage du sol}$\rightarrow$\textit{Transport} : une densité résidentielle minimale est nécessaire pour l'efficience du transport public, une concentration des emplois implique des voyages plus long, les villes plus grandes ont une part modale plus importante pour les transports en commun.
			\item \textit{Transport}$\rightarrow$\textit{Usage du sol} : une forte accessibilité implique des prix plus élevés et un développement accru pour le résidentiel, les entreprises se localisent pour une meilleure accessibilité aux moyens de transport à grande échelle. 
			\item \textit{Transport}$\rightarrow$\textit{Transport} : les lieux avec une bonne accessibilité produiront plus et de plus longs voyages, le choix modal et le coût de transport sont fortement corrélés.
		\end{itemize}
		}
		
		\bpar{
		These theoretical effects are then compared to empirical observations, which for most of them give the way processes are implemented. Some are not observed in practice, whereas most converge with theoretical expectations.		
		}{
		Ces effets théoriques sont par ailleurs comparés aux observations empiriques, qui pour la plupart donnent la manière d'implémentation du processus. Certains ne sont pas observés en pratique, tandis que la plupart sont en accord avec les attentes théoriques.
		}		
		
		\bigskip
		
		\bpar{
		\textit{Comment 1: An uniscalar framework ?} This framework takes schematically into account two main scales, the scale of daily mobility and the scale of the localization of activities. Knowing that in practice mobility behaviors are generally taken into account as average flows, it often reduces to a unique mesoscopic scale. All in all, it does not allow to take into account dynamics on longer time scales, that would include the evolution of the transportation network infrastructure or structural dynamics of systems of cities on long time periods.
		}{
		\textit{Commentaire 1 : Un cadre uniscalaire ?} Ce cadre prend en compte schématiquement deux échelles principales, celle de la mobilité quotidienne et celle de la localisation des activités. Sachant qu'en pratique les comportement de mobilité sont généralement pris en compte sous forme de flux moyens, il se réduit souvent à une unique échelle mesoscopique. Dans tous les cas, il ne permet pas de tenir compte de dynamiques sur le temps plus long comprenant l'évolution de l'infrastructure du système de transport ou des dynamiques structurelles des systèmes de villes sur le temps long.
		}		
		
		\bigskip
		
		\bpar{
		\textit{Comment 2: A systematic view of structuring effects ?} Furthermore, critics of the rhetoric of structuring effects may find in this framework its strong presence, since direct effects of accessibility on land-use and then the localization of activities are assumed here. These critics can be undermined by observing that these are theoretical expected effects, and that the framework is put into perspective of empirical effects indeed observed. We will however always take it with caution, by situating it in terms of context and scales.
		}{
		\textit{Commentaire 2 : Une vision systématique des effets structurants ?} Par ailleurs, les pourfendeurs de la rhétorique des effets structurants trouveront en ce cadre leur bête noire, puisque les effets directs de l'accessibilité sur l'usage du sol puis sur la localisation des activités sont postulés ici. Ces critiques pourront être refoulées par l'observation qu'il s'agit des effets attendus théoriques, et que le cadre est mis en perspective des effets empiriques effectivement observés. Il sera à cependant à prendre avec précaution, en le situant toujours en terme de contexte et d'échelles.
		}

		
		\bigskip
		
		\framecaption{\textbf{Conceptual framework of land-use transport interactions according to \cite{wegener2004land}.}\label{frame:modelingsa:wegener}}{\textbf{Cadre conceptuel des interactions entre transport et usage du sol selon \cite{wegener2004land}.}\label{frame:modelingsa:wegener}}		
	\end{mdframed}
\end{figure}
%%%%%%%%%%%%%%



\bpar{
\cite{wegener2004land} give more recently a state of the art of empirical studies and in modeling on this type of approach of interactions between land-use and transport. The theoretical positioning is closer of disciplines such as transportation socio-economics and planning (see the disciplinary landscapes described in~\ref{sec:quantepistemo}). \cite{wegener2004land} compare and classify seventeen models, among which no one includes an endogenous evolution of the transportation network on relatively short time scales for simulations (of the order of the decade). We find again indeed the correspondance with typically mesoscopic scales previously established. A complementary review is done by~\cite{chang2006models}, broadening the context with the inclusion of more general classes of models, such as spatial interactions models (which contain trafic assignment and four steps models), planing models based on operational research (optimization of locations of different activities, generally homes and employments), the microscopic models of random utility, and models of the real estate market.
}{
\cite{wegener2004land} donnent plus récemment un état de l'art des études empiriques et de modélisation sur ce type d'approche des interactions entre usage du sol et transport. Le positionnement théorique est plutôt proche des disciplines de la socio-économie des transports et de la planification (voir les paysages disciplinaires dressés en~\ref{sec:quantepistemo}). \cite{wegener2004land} comparent et classifient dix-sept modèles, parmi lesquels aucun n'inclut une évolution endogène du réseau de transport sur les échelles de temps relativement courtes (de l'ordre de la décade) des simulations. On retrouve bien la correspondance avec les échelles typiquement mesoscopiques établies précédemment. Une revue complémentaire est faite par~\cite{chang2006models}, élargissant le contexte avec l'inclusion de classes plus générales de modèles, comme des modèles d'interactions spatiales (parmi lesquels l'attribution du traffic et les modèles à quatre temps), les modèles de planification basés sur la recherche opérationnelle (optimisation des localisations des différentes activités, généralement résidences et emplois), les modèles microscopiques d'utilité aléatoire, et les modèles de marché foncier.
}



%%%%%%%%%%%%%%
\begin{figure}
	\begin{mdframed}
	
	\bpar{
	The Pirandello\textregistered model\footnote{The origin of the name is not given, but strongly suggests the influence of its original creators \noun{V. Piron} and \noun{J. Delons}.} is presented in \cite{delons:hal-00319087} as one of the first attempts to develop an operational Luti model in France. The model is based on four fundamental economic processes: the real estate market and the dwellings offer, the residential mobility of households, the attribution of travel destinations, the model choice. The model is static, i.e. computes n equilibrium for spatial distributions of actives and employments, and also for transportation flows.
	}{
	Le modèle Pirandello\textregistered\footnote{L'origine du nom n'est pas donnée, mais suggère fortement l'influence de ses créateurs originaux \noun{V. Piron} et \noun{J. Delons}.} est présenté dans \cite{delons:hal-00319087} comme l'une des premières tentatives de développement de modèle Luti opérationnel en France. Le modèle se base sur quatre processus économiques fondamentaux : le marché du foncier et l'offre de logement, la mobilité résidentielle des ménages, l'attribution des destinations de déplacement, le choix modal. Le modèle est statique, c'est-à-dire calcule un équilibre pour les distributions spatiales des actifs et des emplois, ainsi que pour les flux de transport.
	}
	
	
	\bpar{
	The fundamental processes taken into account and their implementation are the following:
	\begin{itemize}
		\item Residential choices of households are based on a utility function taking into account (i) a confort term as a Cobb-Douglas of housing surface and income, corrected by a linear preference for individual dwellings; (ii) an accessibility term based on generalized cost (aggregation of transportation cost and time, with a value of time); (iii) the dwelling price and the local tax as a function of the housing surface; (iv) a fixed effect by income and by area; and (v) a random term assumed to follow a Gumbel law. Location probabilities for an income group are then given by a discrete choice model given this utility.
		\item The housing prices are formed following a scaling law of population.
		\item A local bidding mechanism answers to the demand previously obtained, as a function of an exogenous dwelling offer.
		\item Companies locate by maximizing their profit, function of the productivity (Cobb-Douglas in the salary and the accessibility) and the real estate price, under the constraint of a fixed spatial distribution of the number of employments, of the office surface, and of the total production of the region.
		\item Transportation is taken into account through a four steps model, which distributes model choices and destination choices with a discrete choice model, and flows are assigned according to a Wardrop equilibrium (see~\ref{sec:reproducibility}), what allows to adjust the values of accessibility given a spatial distribution of activities.
	\end{itemize}
	}{
	Les processus fondamentaux pris en compte et leur implémentation sont les suivants :
	\begin{itemize}
		\item Les choix résidentiels des ménages se basent sur une fonction d'utilité prenant en compte (i) un terme de confort en Cobb-Douglas de la surface et du revenu, corrigée par une préférence linéaire pour les logements individuels ; (ii) un terme d'accessibilité basé sur le coût généralisé (agrégation du coût de transport et du temps, avec un prix du temps) ; (iii) le prix du logement et de la taxe locale en fonction de la surface de logement ; (iv) un effet fixe par revenu et par zone ; et (v) un terme aléatoire supposé suivre une loi de Gumbel. Les probabilités de localisation pour une tranche de revenus sont alors données par un modèle de choix discret étant donné cette utilité.
		\item Les prix du logement sont formés selon une loi d'échelle de la population.
		\item Un système d'enchère local répond à la demande obtenue précédemment, en fonction d'une offre de logement exogène.
		\item Les entreprises se localisent en maximisant leur profit, fonction de la productivité (Cobb-Douglas du salaire et de l'accessibilité) et du prix du foncier, sous contrainte d'une distribution spatiale fixée du nombre d'emplois, de la surface de bureaux, et de la production totale de la région.
		\item Le transport est pris en compte par un modèle à quatre étapes, qui distribue les choix modaux et les choix de destination par un modèle de choix discrets, et les flux assignés selon un équilibre de Wardrop (voir~\ref{sec:reproducibility}), ce qui permet d'ajuster les valeurs de l'accessibilité étant donné une distribution spatiale des activités.
	\end{itemize}
	}
	
	
	\bpar{
	The mechanism to combine these different processes to obtain a global equilibrium is detailed by~\cite{kryvobokov2013comparison}, and consists in the establishment of three sub-equilibriums at different scales: transportation flows (giving costs) on a short term, location and real estate prices on the middle term, land prices and available terrains (fixed in an exogenous way for all the modeled period).
	}{
	Le mécanisme de combinaison de ces différents processus pour obtenir un équilibre global est détaillé par~\cite{kryvobokov2013comparison}, et consiste à l'établissement de trois sous-équilibres à différentes échelles : flux de transport (donnant les coûts) sur le court terme, localisation et prix de l'immobilier sur le moyen terme, prix du foncier et terrain disponibles (fixés de manière exogène pour l'ensemble de la période modélisée).
	}
	
	\bigskip
	
	
	\bpar{
	\textit{Commentary: Equilibrium, operational model and calibration.} A certain number of remarks can be done concerning this model, the most important for our approach are: (i) the equilibrium assumption can be a powerful tool to understand the structure of the attractors of the system, but has no empirical foundation, and even less for the coupling of equilibriums at different scales; (ii) thus, the operational nature of the model can be discussed, since the study of the impact of scenarios on the movements of attractors can difficultly allow to infer on local dynamics of the system; and (iii) sub-models are calibrated more or less rigorously and relatively separately, but the conditions of a calibration by decomposition are an open question still not well explored and linked to the nature of model coupling. In our sense, such a micro-based model would in any case be in better consistence with a philosophy of dynamical generative modeling and parsimony (see\ref{sec:computation}).
	}{
	\textit{Commentaire : Équilibre, modèle opérationnel et calibration.} Un certain nombre de remarques peuvent être faites à ce modèle, les plus importantes pour notre approche étant : (i) l'hypothèse d'équilibre peut être un outil puissant pour comprendre la structure des attracteurs du système, mais n'a pas de fondement empirique, et encore moins pour le couplage d'équilibres à différentes échelles ; (ii) ainsi, la nature opérationnelle du modèle est discutable, puisque l'étude de l'impact de scenarios sur les déplacements des attracteurs permet difficilement d'inférer sur les dynamiques locales du système ; et (iii) les sous-modèles sont calibrés plus ou moins rigoureusement et relativement séparément, or les conditions d'un calibrage par décomposition sont une question ouverte encore peu explorée et liée à la nature du couplage de modèles. En notre sens, un tel modèle micro-fondé serait dans tous les cas en meilleure cohérence avec une philosophie de modélisation générative dynamique et de parcimonie (voir~\ref{sec:computation}).
	}
	
	\medskip
	
	\framecaption{\textbf{The Pirandello model.}\label{frame:modelingsa:pirandello}}{\textbf{Le modèle Pirandello.}\label{frame:modelingsa:pirandello}}
	
	\end{mdframed}
\end{figure}
%%%%%%%%%%%%%%	

	
	
	
	
%%%%%%%%%%%%%%
\begin{figure}
	\begin{mdframed}
	
	\bpar{
	The Nedum2D model, described in details by~\cite{viguie2014downscaling}, is focused on the localization of actives and their interaction with land rent and real estate promoters: it is a model inspired by the Fujita-Ogawa model~\cite{fujita1982multiple}, inheriting from the literature in Urban Economics.
	}{
	Le modèle Nedum2D, décrit en détail dans~\cite{viguie2014downscaling}, se concentre sur la localisation des actifs et leur interaction avec la rente foncière et les promoteurs immobiliers : il s'agit d'un modèle inspiré du modèle de Fujita-Ogawa~\cite{fujita1982multiple}, héritant de la littérature en Economie Urbaine.
	}
	
	\bpar{
	The processes included in the model are, with each its own time scale fixed by a parameter:
	\begin{itemize}
		\item Households make a compromise between housing surface and available budget without transportation costs and rent, following a Cobb-Douglas function for the corresponding utility. This process induces a dynamic for housing surface as a function of the distance to the center.
		\item They relocate in order to have an expected utility larger than the average.
		\item Rents evolve to maximize the occupation or in response to an external demand.
		\item New buildings are built by promoters that aim at maximizing their profits.
	\end{itemize}
	}{
	Les processus inclus dans le modèle sont, chacun ayant une échelle de temps particulière fixée par un paramètre :
	\begin{itemize}
		\item Les ménages font un compromis entre surface de logement et budget disponible hors coûts de transports et loyer, suivant une fonction de Cobb-Douglas pour l'utilité correspondante. Ce processus induit une dynamique pour la surface des logements en fonction de la distance au centre.
		\item Ils se relocalisent pour avoir une utilité moyenne plus grande que la moyenne.
		\item Les loyers évoluent pour maximiser l'occupation ou en réponse à une demande extérieure.
		\item De nouveaux bâtiments sont construits par des promoteurs cherchant à maximiser leur profits.
	\end{itemize}
	}
	
	\bpar{
	This model is dynamical and simulates the evolution of these different variables in space (the formulation above is monocentric, a polycentric extension and one taking into account an exogenous distribution of employments exist) and time. Its spatial scale is metropolitan, and the time scale can range from a medium scale (decade) to longer time-periods (century), knowing that the latest has a low credibility since it keeps static numerous other components of the urban system.
	}{
	Ce modèle est dynamique et simule l'évolution de ces différentes variables dans l'espace (la formulation ci-dessus est monocentrique, une variante polycentrique et prenant en compte une distribution exogène d'emplois existent) et dans le temps. Son échelle spatiale est métropolitaine, et l'échelle de temps peut s'étendre d'une échelle moyenne (décade) à du temps plus long (siècle), sachant que cette dernière est peu crédible puisque qu'elle garde statique de nombreuses autres composantes du système urbain.
	}
	
	\medskip
	
	\bpar{
	\textit{Comment: extension of ontologies.} The coupling of Nedum with a model for traffic assignment, the Modus model\footnote{In the frame of the current research project ANR VITE! (see \url{http://www.agence-nationale-recherche.fr/Projet-ANR-14-CE22-0013}).}, aims at including the feedback of congestion in the transportation system on costs, and thus on the localization and on the urban structure. Fundamental questions arise from the first coupling experiments:
	\begin{itemize}
		\item Is the masterplan \emph{Schéma Directeur} really useful, since is seems to only accompany already existing dynamics ? In other words, \textit{is the governance process endogenous} ? Does the Sdrif in fact capture an intrinsic dynamic on a longer time ?
		\item The coupling of models raises in itself technical difficulties, for communication between modules already implemented in different languages and for convergence of the coupled model in a reasonable number of iterations.
		\item It furthermore raises ontological difficulties: each model includes opposite mechanisms for the same ontology (aggregation effect against congestion effect for the distribution of population). The question is then if a specific coupling ontology is necessary (for example with specific equations integrating these contradictory effects), to allow on the one hand a better convergence, on the other hand a better ontological consistency.
	\end{itemize}
	}{
	\textit{Commentaire : extension des ontologies.} Le couplage de Nedum avec un modèle d'attribution de transport, le modèle Modus\footnote{Dans le cadre du projet en cours de réalisation ANR VITE! (voir \url{http://www.agence-nationale-recherche.fr/Projet-ANR-14-CE22-0013}).}, vise à inclure la retroaction de la congestion dans le système de transport sur les coûts, et donc sur la localisation et la structure urbaine. Des questions fondamentales se dégagent des premières expériences de couplage :
	\begin{itemize}
		\item Le schéma directeur est-il vraiment utile, puisqu'il ne semble qu'accompagner des dynamiques déjà présentes ? En d'autres termes, \textit{le processus de gouvernance est-il endogène} ? Le Sdrif capture-t-il en fait une dynamique intrinsèque sur le temps plus long ?
		\item Le couplage des modèles pose en lui-même des difficultés techniques, de communication entre des modules déjà implémentés dans différents langages et de convergence du modèle couplé en un nombre raisonnable d'itérations.
		\item Il pose d'autre part des difficultés ontologiques : chaque modèle inclut des mécanismes opposés pour la même ontologie (effet d'agrégation contre congestion pour la distribution de la population). La question se pose s'il faut rajouter spécifiquement une ontologie de couplage (par exemple des equations spécifiques intégrant ces effets contradictoires), pour permettre d'une part une meilleure convergence, d'autre part une meilleure cohérence ontologique.
	\end{itemize}
	}	
	
	
	% couplage avec modus http://www.agence-nationale-recherche.fr/Projet-ANR-14-CE22-0013
	% seminaire lvmt (27/11) : points cruciaux soulevés
	%  - endogeneite de la gouvernance, ou le sdrif sert-il a rien ? accompagne dynamique locale. negociations locales. modele calibre sur amenagements passes : le sdrif n'est pas une rupture : dynamique sur le temps long intrinseque, capturee par le modele.
	%  - difficulte technique du couplage : ouverture etc
	%  - question endogeneite : White, exploration modeles, mutlimodeling, parcimonie.
	%  - difficulte ontologique du couplage : modele ont processus oppose, instable numeriquement : domaines de validités locaux ? (au sens des hypotheses) - en ouverture : enrome a travail a faire sur le couplage, du point de vue des domaines de connaissance (differentes dimensions du couplage)
	
	\medskip
	
	\framecaption{\textbf{The Nedum model.}\label{frame:modelingsa:nedum}}{\textbf{Le modèle Nedum.}\label{frame:modelingsa:nedum}}
	
	\end{mdframed}
\end{figure}
%%%%%%%%%%%%%%




\bpar{
In order to give a better intuition of the logic underlying some Luti models, we detail in Frame~\ref{frame:modelingsa:pirandello} and in Frame~\ref{frame:modelingsa:nedum} the structures, the ontologies, and assumptions of two models developed in the specific case of Ile-de-France (allowing on the one hand a comparison between both and on the other hand echoing the thematic development of~\ref{sec:casestudies}). Even for very close ontologies (real estate prices, households localizations), we see the variety of possible assumptions and of issues raised by the models.
}{
Afin de donner une meilleure intuition de la logique sous-jacente à certains modèles Luti, nous détaillons en Encadré~\ref{frame:modelingsa:pirandello} et en Encadré~\ref{frame:modelingsa:nedum} les structures, les ontologies et les hypothèses de deux modèles développés dans le cas spécifique de l'Ile-de-France (permettant d'une part la comparaison entre les deux et d'autre part donnant un écho au développement thématique de~\ref{sec:casestudies}). Même pour des ontologies très proches (prix immobiliers, localisation des ménages), on voit la variété d'hypothèses possibles et de problématiques soulevées par les modèles.
}



\paragraph{Very different operational models}{Des modèles opérationnels très variés}


\bpar{
The variety of existing models lead to operational comparisons: \cite{paulley1991overview} synthesize a project comparing different model applied to different cities. Their result allow on the one hand to classify interventions depending on their impact on the level of interaction between transportation and land-use, and on the other hand to show that the effects of interventions strongly depend on the size of the city and on its socio-economic characteristics.
}{
La variété des modèles existants a conduit à des comparaisons opérationnelles : \cite{paulley1991overview} rendent compte d'un projet comparant différents modèles appliqués à différentes villes. Leurs résultats permettent d'une part de classifier des interventions en fonction de leur impact sur le niveau d'interaction entre transport et usage du sol, et d'autre part de montrer que l'effet des interventions dépend fortement de la taille de la ville et de ses caractéristiques socio-économiques.
}


\bpar{
Ontologies of processes, and more particularly on the question of equilibrium, are also varied. The respective advantages of a static approach (computation of a static equilibrium of households localisation for a given specification of their utility functions) and of a dynamical approach (out-of-equilibrium simulation of residential dynamics) has been studied by~\cite{kryvobokov2013comparison}, within a metropolitan frame on time scales of the order of the decade. The authors show that results are roughly comparable and that each model has its utility depending on the question asked.
}{
Les ontologies des processus, et notamment sur la question de l'équilibre, sont aussi variées. Les avantages respectifs d'une approche statique (calcul d'un équilibre statique de la localisation des ménages pour une certaine spécification de leur fonctions d'utilité) et d'une approche dynamique (simulation hors équilibre des dynamiques résidentielles) a été étudié par~\cite{kryvobokov2013comparison}, dans un cadre métropolitain sur des échelles de temps de l'ordre de la décennie. Les auteurs montrent que les résultats sont globalement comparables et que chaque modèle a son utilité selon la question posée.
}



\bpar{
Different aspects of the same system can be included within diverse models, as show for example~\cite{wegener1991one}, and traffic, residential and employments dynamics, the evolution of land-use as a consequence, also influenced by a static transportation network, are generally taken into account. \cite{iacono2008models} covers a similar horizon with an additional development on cellular automata models for the evolution of land-use and agent-based models. The temporal range of application of these models, around the decade, and their operational nature, make them useful for planning, what is rather far of our focus to obtain explicative models of geographical processes. Indeed, it is often more relevant for a model used in planning to be understandable as an anticipation tool, or even a communication tool, than to be faithful to territorial processes, at the cost of an abstraction.
}{
Différents aspects du même système peuvent être traduits par divers modèles, comme le montre par exemple~\cite{wegener1991one}, et le trafic, les dynamiques résidentielles et d'emploi, l'évolution de l'usage du sol en découlant, influencée aussi par un réseau de transport statique, sont généralement pris en compte. \cite{iacono2008models} couvre un horizon similaire avec un développement supplémentaire sur les modèles à automates cellulaires d'évolution d'usage du sol et les modèles à base d'agents. La portée temporelle d'application de ces modèles, de l'ordre de la décennie, et leur nature opérationnelle les rend utiles pour la planification, ce qui est assez loin de notre souci d'obtenir des modèles explicatifs de processus géographiques. En effet, il est souvent plus pertinent pour un modèle utilisé en planification d'être lisible comme outil d'anticipation, voire de communication, que d'être fidèle aux processus territoriaux au prix d'une abstraction.
}



\paragraph{Perspectives for LUTI models}{Perspectives pour les modèles LUTI}

\bpar{
\cite{timmermans2003saga} formulates doubts regarding the possibility of interaction models that would be really integrated, i.e. producing endogenous transportation patterns and being detached from artefacts such as accessibility for which the influence of its artificial nature remains to be established, in particular because of the lack of data and a difficulty to model governance and planning processes. It is interesting to note that current priorities for the development of LUTI models seem to be centered on a better integration of new technologies and a better integration with planning and decision-making processes, for example through visualization interfaces as proposed by~\cite{JTLU611}. They do not aim at being extended on problematics of territorial dynamics including the network on longer time scales for example, what confirms the range and the logic of use and development of this type of models.
}{
\cite{timmermans2003saga} émet des doutes quant à la possibilité de modèles d'interaction réellement intégrés, c'est-à-dire produisant des motifs de transports endogènes et se détachant d'artefacts comme l'accessibilité dont l'influence du caractère artificiel reste à établir, notamment à cause du manque de données et une difficulté à modéliser les processus de gouvernance et de planification. Il est intéressant de noter que les priorités actuelles de développement des modèles LUTI semblent centrées sur une meilleure intégration des nouvelles technologies et une meilleure intégration avec la planification et les processus de prise de décision, par exemple via des interfaces de visualisation comme le propose~\cite{JTLU611}. Ils ne cherchent pas à s'étendre à des problématiques de dynamiques territoriales incluant le réseau sur de plus longues échelles par exemple, ce qui confirme la portée et la logique d'utilisation et de développement de ce type de modèles.
}

\bpar{
A generalization of this type of approach at a smaller scale, such as the one proposed by \cite{russo2012unifying}, consists in the coupling between a LUTI at the mesoscopic scale to macroeconomic models at the macroscopic scale\footnote{\cite{russo2012unifying} indeed generalizes the framework of LUTI models to propose a framework of interaction between spatial economy and transportation (\emph{Spatial Economics and Transport Interactions}). This framework includes LUTI models at the urban scale, and at the national level macroeconomic models simulating production and consumption, competition between activities, production of the stock of the offer of transportation. Transportation models still assume a fixed network and establish equilibria within it, what implies a small spatial scale and a short time scale.}. These do not consider the evolution of the transportation network in an explicit manner but are interested only to abstract patterns of demand and offer. Urban economics have developed specific approaches that are similar in their context: \cite{masso2000} for example describes an integrated model coupling urban development, relocations and equilibrium of transportation flows.
}{
Une généralisation de ce type d'approche à une plus petite échelle, comme celle proposée par \cite{russo2012unifying}, consiste au couplage du LUTI à l'échelle mesoscopique à des modèles macroéconomiques à l'échelle macroscopique\footnote{\cite{russo2012unifying} généralise en fait le cadre des LUTI pour proposer un cadre d'interaction entre Economie Spatiale et Transports (\emph{Spatial Economics and Transport Interactions}). Celui-ci inclut les LUTI à l'échelle urbaine, et au niveau national les modèles macroéconomiques simulant production et consommation, compétition des activités, production du stock d'offre de transport. Les modèles de transport supposent toujours réseau fixe et établissent des équilibres au sein de celui-ci, ce qui implique une petite échelle spatiale et une courte échelle temporelle.}. Ceux-ci ne considèrent pas l'évolution du réseau de transport de manière explicite mais s'intéressent seulement aux motifs abstraits d'offre et demande. L'économie urbaine a développé des approches spécifiques similaires dans leur démarche : \cite{masso2000} décrit par exemple un modèle intégré couplant développement urbain, relocalisation et équilibre des flux de transports.
}


\bpar{
Thus, we can synthesize this type of approach, that we can designate through a semantic shortcut as \emph{LUTI approaches}, by the fundamental following characteristics: (i) models aiming at understanding an evolution of the territory, within the context of a given transportation network; (ii) models in a logic of planning and applicability, being themselves often implied in decision-making; and (iii) models at medium scales, in space (metropolitan scale) and in time (decade).
}{
Ainsi, nous pouvons synthétiser ce type d'approche, qu'on pourra désigner par abus de langage \emph{approche LUTI}, par les caractéristiques fondamentales suivantes : (i) modèles visant à comprendre une évolution du territoire, dans le contexte d'un réseau de transport donné ; (ii) modèles dans une logique de planification et d'applicabilité, étant souvent impliqués eux-même dans les prises de décision ; et (iii) modèles à des échelles moyennes, dans l'espace (métropole) et dans le temps (décennie).
}


\subsubsection{Network Growth}{Croissance du Réseau}


\bpar{
We can now switch to the ``opposite'' paradigm, focused on the evolution of the network. It may seem strange to consider a variable network while neglecting the evolution of the territory, when considering the overview of some potential evolution mechanisms we previously reviewed (potential breakdown, self-reinforcements, network planning) which occur at mainly longer time scales than territorial evolutions. We will see here that there is no paradox, since (i) either the modeling focuses on the evolution of \emph{network properties}, at a short scale (micro) for congestion, capacity, tarification processes, mainly from an economic point of view; (ii) or territorial components playing indeed a role on the network are stable on the long scales considered.
}{
Passons à présent au paradigme ``opposé'', centré sur l'évolution du réseau. Il peut sembler incongru de considérer un réseau variable en négligeant les variations du territoire, au regard de l'aperçu de certains des mécanismes potentiels d'évolution revus précédemment (rupture de potentiel, auto-renforcements, planification du réseau) qui se produisent à des échelles de temps majoritairement plus longues que les évolutions territoriales. On verra ici qu'il n'y a pas de paradoxe, vu que (i) soit la modélisation s'intéresse à l'évolution des \emph{propriétés du réseau}, à une courte échelle (micro) pour des processus de congestion, de capacité, de tarification, principalement d'un point de vue économique ; (ii) soit les composantes territoriales jouant en effet sur le réseau sont stables au échelles longues considérés.
}


\bpar{
Network growth is the subject of modeling approaches which aim at explaining the growth of transportation networks. They generally take a \emph{bottom-up} and endogenous point of view, i.e. aiming at unveiling local rules that would allow to reproduce the growth of the network on long time scales (often the road network). As we will see, it can be a topological growth (creation of new links) or the growth of link capacities in relation with their use, depending on scales and ontologies considered. To simplify, we distinguish broad disciplinary streams having studied the modeling of the growth of transportation networks: these are respectively linked to transportation economics, physics, transportation geography, and biology.
}{
La croissance de réseaux est l'objet de démarches de modélisation qui cherchent à expliquer la croissance des réseaux de transport. Ils prennent généralement un point de vue \emph{bottom-up} et endogène, c'est-à-dire cherchant à mettre en évidence des règles locales qui permettraient de reproduire la croissance du réseau sur de longues échelles de temps (souvent le réseau routier). Comme nous allons le voir, il peut s'agir de la croissance topologique (création de nouveaux liens) ou la croissance des capacités des liens en relation avec leur utilisation, selon les échelles et les ontologies considérées. Nous distinguons pour simplifier des grands courants disciplinaires s'étant intéressé à la modélisation de la croissance des réseaux de transport : ceux-ci sont liés respectivement à l'économie des transports, la physique, la géographie des transport et la biologie.
}


\bpar{
We thus partly rejoin the classification by~\cite{xie2009modeling}, which proposes an extended review of modeling the growth of transportation networks, in a perspective of transportation economics but broadened to other fields. \cite{xie2009modeling} distinguishes broad disciplinary streams having studied the growth of transportation networks: transportation geography has developed very early models based on empirical facts but which have focused on reproducing topology rather than mechanisms\footnote{According to \cite{xie2009modeling}, the contribution of geography consists in limited efforts at the period of \cite{haggett1970network}, we will therefore build on this review and not give a more thorough development.}; statistical models on case studies produce very limited conclusions on causal relations between network growth and demand (growth being in that case conditioned to demand data); economists have studied the production of infrastructure both from a microscopic and macroscopic point of view, generally not spatialized; network science has produced stylized models of network growth which are based on topological and structural rules rather than rules built on processes corresponding to empirical facts.
}{
On rejoint ainsi partiellement la classification de~\cite{xie2009modeling}, qui propose une revue étendue de la modélisation de croissance des réseaux, dans une perspective d'économie des transports mais en élargissant à d'autres champs. \cite{xie2009modeling} distingue des grands courants disciplinaires ayant étudié la croissance des réseaux de transport : la géographie des transports a développé très tôt des modèles basés sur des faits empiriques mais qui ont visé à reproduire la topologie plutôt que sur les mécanismes\footnote{Selon \cite{xie2009modeling}, la contribution de la géographie consiste en des efforts limités à l'époque de \cite{haggett1970network}, nous nous baserons donc sur cette revue et ne donnerons pas de développement approfondi.} ; les modèles statistiques sur des cas d'étude fournissent des conclusions très mitigées sur les relations causales entre croissance du réseau et demande (la croissance étant dans ce cas conditionnée aux données de demande) ; les économistes ont étudié la production d'infrastructure à la fois d'un point de vue microscopique et macroscopique, généralement non spatialisés ; la science des réseaux a produit des modèles stylisés de croissance de réseau qui se basent sur des règles topologiques et structurelles plutôt que des règles se reposant sur des processus correspondant à des réalités empiriques.
}



\paragraph{Economics}{Economie}

\bpar{
Economists have proposed models of this type: \cite{zhang2007economics} reviews transportation economics literature on network growth, recalling the three main features studied by economists on that subject, that are road pricing, infrastructure investment and ownership regime, and finally describes an analytical model combining the three. These three classes of processes are related to an interaction between microscopic economic agents (users of the network) and governance agents. Models can include a detailed description of planning processes, such as~\cite{levinson2012forecasting} which combines qualitative surveys with statistics to parametrize a network growth model.  \cite{xie2009jurisdictional} compares the relative influence of centralized (planning by a governance structure) and decentralized growth processes (local growth which does not enters the frame of a global planning). 
 \cite{yerra2005emergence} shows through a reinforcement economic model including investment rule based on traffic assignment that local rules are enough to make hierarchy of roads emerge for a fixed land-use. \cite{levinson2003induced} proceed to an empirical study of drivers of road network growth for \emph{Twin Cities} in the United States (Minneapolis-Saint-Paul), establishing that basic variables (length, accessibility change) have the expected behavior, and that there exists a difference between the levels of investment, implying that local growth is not affected by costs, what could correspond to an equity of territories in terms of accessibility. The same data are used by~\cite{zhang2016model} to calibrate a network growth model which superimposes investment decisions with network use patterns. \cite{yerra2005emergence} shows with an economic model based on self-reinforcement processes (i.e. that include a positive feedback of flows on capacity) and which includes an investment rule based on traffic assignment, that local rules are sufficient to make a hierarchy of the road network emerge with a fixed land-use. A synthesis of these works gravitating around \noun{Levinson} is done in~\cite{xie2011evolving}.
}{
Les économistes ont proposé des modèles de ce type : \cite{zhang2007economics} passe en revue la littérature en économie des transports sur la croissance des réseaux, rappelant les trois aspects principalement traités par les économistes sur le sujet, qui sont la tarification routière, l'investissement en infrastructures et le régime de propriété, et propose finalement un modèle analytique combinant les trois. Ces trois classes de processus relèvent d'une interaction entre les agents économiques microscopiques (utilisateurs du réseau) et les agents de gouvernance. Les modèles peuvent inclure une description détaillée des processus de planification, comme~\cite{levinson2012forecasting} qui combinent des enquêtes qualitatives et des statistiques pour paramétrer un modèle de croissance de réseau. \cite{xie2009jurisdictional} comparent l'influence relative des processus de croissance centralisés (planification par une structure de gouvernance) et décentralisés (croissance locale ne rentrant pas dans le cadre d'une planification globale). \cite{levinson2003induced} procèdent à une étude empirique des déterminants de la croissance du réseau routier pour les \emph{Twin Cities} aux Etats-Unis (Minneapolis-Saint-Paul), établissant que les variables basiques (longueur, changement dans l'accessibilité) ont le comportement attendu, et qu'il existe une différence entre les niveaux d'investissement, impliquant que la croissance locale n'est pas affectée par les coûts, ce qui peut correspondre à une équité des territoires en termes d'accessibilité. Ces données sont utilisées par~\cite{zhang2016model} pour calibrer un modèle de croissance de réseau qui superpose les décisions d'investissement aux motifs d'utilisation du réseau. \cite{yerra2005emergence} montre avec un modèle économique basé sur des processus d'auto-renforcement (c'est-à-dire incluant une rétroaction positive des flux sur la capacité) et incluant une règle d'investissement basée sur l'attribution du trafic, que des règles locales sont suffisantes pour faire émerger une hiérarchie du réseau routier à usage du sol fixé. Une synthèse de ces travaux gravitant autour de \noun{Levinson} est faite dans~\cite{xie2011evolving}.
}

% sur ownership structure et pricing :
% https://sci-hub.cc/https://link.springer.com/article/10.1007/s11067-015-9309-3
% relié à gouvernance, mais trop loin du sujet ici.


\paragraph{Physics}{Physique}

\bpar{
Physics has recently introduced infrastructure network growth models, largely inspired by this economic literature: a model which is very similar to the last we described is given by~\cite{louf2013emergence} with simpler cost-benefit functions by obtaining a similar conclusion. Given a distribution of nodes (cities)\footnote{We are here in a case in which the assumption of non-evolving city populations whereas the networks is iteratively established finds little empirical or thematic support, since we showed that network and cities had comparable evolution time scales. This models is thus closer to produce in the proper sense a \emph{potential network} given a distribution of cities, and must be interpreted with caution.} which population follows a power law, two cities will be connected by a road link if a cost-benefit utility function, which linearly combines potential gravity flow and construction cost\footnote{What gives a cost function of the form $C = \beta / d_{ij}^{\alpha} - d_{ij}$, where $\alpha$ and $\beta$ are parameters.}, has a positive value. These simple local assumptions are sufficient to make a complex network emerge with phase transitions as a function of the relative weight parameter in the cost function, leading to the emergence of hierarchy. \cite{zhao2016population} apply this model in an iterative way to connect intra-urban areas, and shows that taking into account populations in the cost function significantly changes the topologies obtained.
}{
La physique a introduit récemment des modèles de croissance des réseaux d'infrastructure, en s'inspirant largement de cette littérature économique : un modèle très similaire au dernier cité est donné par~\cite{louf2013emergence} avec des fonctions coûts-bénéfices plus simples mais obtenant une conclusion similaire. Étant donné une distribution de noeuds (villes)\footnote{Nous nous trouvons ici dans un cas où l'hypothèse de non-évolution des population des villes tandis que le réseau s'établit itérativement trouve peu de support empirique ou thématique, puisqu'on a montré que réseau et villes avaient des échelles de temps d'évolution comparables. Ce modèle produit donc plus à proprement parler un \emph{réseau potentiel} étant donné une distribution de villes, et il est à interpréter avec précaution.} dont la population suit une loi puissance, deux villes seront connectées par un lien routier si une fonction d'utilité coût-bénéfice, combinant linéairement flux gravitaire potentiel et coût de construction\footnote{Ce qui donne une fonction de coût de la forme $C = \beta / d_{ij}^{\alpha} - d_{ij}$, où $\alpha$ et $\beta$ sont des paramètres.}, a une valeur positive. Ces hypothèses locales simples suffisent à faire émerger un réseau complexe et des transitions de phase en fonction du paramètre de poids relatif dans le coût, conduisant à l'apparition de la hiérarchie. \cite{zhao2016population} applique ce modèle de manière itérative pour connecter des zones intra-urbaines, et montre que la prise en compte des populations dans la fonction de coût change significativement les topologies obtenues.
}



\bpar{
An other class of models, close to procedural models in their ideas, are based on local geometric optimization processes, and aim at resembling real networks in their topology. \cite{bottinelli2017balancing} thus study a tree growth model applied to ant tracks, in which maintenance cost and construction cost both influence the choice of new links. The morphogenesis model by~\cite{courtat2011mathematics} which uses a compromise between realization of interaction potentials and construction cost, and also connectivity rules, reproduces in a stylized way real patterns of street networks. A very close model is described in~\cite{rui2013exploring}, but including supplementary rules for local optimization (taking into account degree for the connection of new links). Optimal network design, belonging more to the field of engineering, uses similar paradigms: \cite{vitins2010patterns} explore the influence of different rules of a shape grammar (in particular connection patterns between links of different hierarchical levels) on performances of networks generated by a genetic algorithm.
}{
Une autre classe de modèles, proche dans leur idée des modèles procéduraux, se basent sur des processus d'optimisation géométrique locale, et visent à ressembler à des réseaux réels dans leur topologie. \cite{bottinelli2017balancing} étudient ainsi un modèle de croissance d'arbre appliqué aux pistes de fourmis, dans lequel coût de maintenance et coût de construction influencent tous les deux les choix de nouveaux liens. Le modèle de morphogenèse de~\cite{courtat2011mathematics} qui utilise un compromis entre réalisation des potentiels d'interaction et coût de construction, ainsi que des règles de connectivité, reproduit de manière stylisée des motifs réels des réseaux de rues. Un modèle très proche est décrit dans~\cite{rui2013exploring}, tout en incluant des règles supplémentaires pour l'optimisation locale (prise en compte du degré pour la connection de nouveaux liens). La conception optimale de réseau, plutôt pratiquée par l'ingénierie, utilise des paradigmes similaires : \cite{vitins2010patterns} explorent l'influence de différentes règles d'une grammaire de formes (notamment les motifs de connection entre les liens de différents niveaux hiérarchiques) sur les performances de réseaux générés par algorithme génétique.
}


\bpar{
We can detail the mechanisms of one of these geometrical growth models. \cite{barthelemy2008modeling} describe a model based on a local optimization of energy which generates road networks with a globally reasonable shape. The model assumes ``centers'', which correspond to nodes of a road network, and road segments in space linking these centers. The model starts with initial connected centers, and proceeds by iterations to simulate network growth the following way:
\begin{enumerate}
	\item New centers are randomly added following an exogenous probability distribution, at fixed duration time steps.
	\item The network grows following a cost minimization rule: centers are grouped by projection on the network; each group makes a fixed length segment grow in the average direction towards the group starting from the projection (except if it vanishes in length, a segment then grows in the direction of each point).
\end{enumerate}
This model is adjusted in order that areas of parcels delimited by the network follow a power law with an exponent similar to the one observed for the city of Dresde. It has the advantage to be simple, to have few parameters (probability distribution for centers, length of segments built), to rely on reasonable local rules. This last point has also a dark side, since we can then expect the model to only capture few complexity, by neglecting numerous processes unveiled in chapter~\ref{ch:thematic} such as governance.
}{
Détaillons les mécanismes de l'un de ces modèles de croissance géométrique. \cite{barthelemy2008modeling} décrivent un modèle basé sur une optimisation locale de l'énergie qui génère des réseaux routiers à l'aspect globalement crédible. Le modèle suppose des ``centres'', qui correspondent à des noeuds d'un réseau routier, et des segments de route dans l'espace reliant ces centres. Le modèle part de centres initiaux connectés, et procède par itérations pour simuler la croissance du réseau de la façon suivante :
\begin{enumerate}
	\item De nouveaux centres sont ajoutés aléatoirement suivant une distribution de probabilité exogène, aux pas de temps multiple d'une durée fixée.
	\item Le réseau croit suivant une règle de minimisation de coût : les centres sont groupés par projection sur le réseau ; chaque groupe fait croître un segment de longueur fixée dans la direction moyenne vers l'ensemble du groupe à partir de la projection (sauf si celle-ci est nulle, un segment croit alors en direction de chaque point).
\end{enumerate}
Ce modèle est ajusté pour que les aires des parcelles délimitées par le réseau suivent une loi d'échelle avec un exposant similaire à celui observé pour la ville de Dresde. Il a l'avantage d'être simple, d'avoir peu de paramètres (distribution de probabilité pour les centres, taille des tronçons construits), de reposer sur des règles locales crédibles. Cette dernière propriété est à double tranchant, puisqu'on peut alors s'attendre à ce que le modèle ne puisse capturer que peu de complexité, en négligeant de nombreux processus mis en valeur au chapitre~\ref{ch:thematic} comme la gouvernance.
}


% engineering : congestion / capacity
% http://www.sciencedirect.com/science/article/pii/S1877705816003131

%  La simplicité des hypothèses dans ce genre de modèle permet dans certains cas d'inclure des processus qui serait par ailleurs difficile à intégrer : 






%\paragraph{Transport geography}{Géographie des transports}

% la géographie des transports a développé très tôt des modèles basés sur des faits empiriques mais qui ont visé à reproduire la topologie plutôt que sur les mécanismes
% \comment[FL]{point tres important faire une section a part}

% Un peu de geo dans Ducruet :
% https://halshs.archives-ouvertes.fr/file/index/docid/605653/filename/Ducruet_Lugo_SAGE_Handbook_of_Transport_Studies_draft.pdf
% mais sinon remonte majoritairement a Hagget et Chorley, et des travaux contemporains lies a modelisation procedurale






\paragraph{Biological networks}{Réseaux biologiques}


\bpar{
Finally, an interesting and original approach to network growth are biological networks. This approach belongs to the field of morphogenetic engineering, which aims at conceiving artificial complex systems inspired from natural complex systems and on which a control of emerging properties is possible~\cite{doursat2012morphogenetic}. \emph{Physarum machines}, which are models of a self-organized mould (\emph{slime mould}) have been proved to solve in an efficient way difficult problems (in the sense of their computational complexity, see~\ref{sec:epistemology}) such as routing problems~\cite{tero2006physarum} or NP-complete navigation problems such as the Traveling Salesman Problem~\cite{zhu2013amoeba}. These properties allow these systems to produce networks with Pareto-efficient properties for cost and robustness~\cite{tero2010rules} which are typical of empirical properties of real networks, and furthermore relatively close to these in terms of shape (under certain conditions, see~\cite{adamatzky2010road}).
}{
Enfin, une approche originale et intéressante pour la croissance des réseaux est le réseau biologique. Cette approche appartient au champ de l'ingénierie morphogénétique, qui vise à concevoir des systèmes complexes artificiels inspirés de systèmes complexes naturels et sur lesquels un contrôle des propriétés émergentes est possible~\cite{doursat2012morphogenetic}. Les \emph{machines Physarum}, qui sont des modèles d'une moisissure auto-organisée (\emph{slime mould}) ont été prouvés comme résolvant de manière efficiente des problèmes difficiles (au sens de leur complexité computationnelle, voir~\ref{sec:epistemology}) comme des problèmes de routage~\cite{tero2006physarum} ou des problèmes de navigation NP-complets comme le Problème du Voyageur de Commerce~\cite{zhu2013amoeba}. Ces propriétés permettent à ces systèmes de produire des réseaux ayant des propriétés de coût-robustesse Pareto-efficientes~\cite{tero2010rules} qui sont typiques des propriétés empiriques des réseaux réels, et de plus relativement proches en forme de ceux-ci (sous certaines conditions, voir~\cite{adamatzky2010road}).
}

\bpar{
This type of models can have an interest in our case since self-reinforcement processes based on flows are analogous to link reinforcement mechanisms in transportation economics. This type of heuristic has been tested to generate the French railway network by~\cite{mimeur:tel-01451164}, making an interesting bridge with investment models by \noun{Levinson} we previously described\footnote{Knowing that for this study, validation criteria that were applied remain however limited, either at a level inappropriate to the stylized facts studied (number of intersection or of branches) or too general and that can be reproduced by any model (total length and percentage of population deserved), and belong to criteria of form that are typical to procedural modeling which can only difficultly account of internal dynamics of a system as previously developed. Furthermore, taking for an external validation the production of a hierarchical network reveals an incomplete exploration of the structure and the behavior of the model, since through its preferential attachment mechanisms it must mechanically produce a hierarchy. Thus, a particular caution will have to be given to the choice of validation criteria.}.
}{
Ce type de modèles peut être d'intérêt dans notre cas puisque les processus d'auto-renforcement basés sur les flots sont analogues aux mécanismes de renforcement de lien en économie des transports. Ce type d'heuristique a été testé pour générer le réseau ferré Français par~\cite{mimeur:tel-01451164}, faisant un pont intéressant avec les modèles d'investissement de \noun{Levinson} décrits précédemment\footnote{Sachant que pour cette étude, les critères de validation appliqués restent toutefois limités, soit à un niveau inadapté aux faits stylisés étudiés (nombre d'intersection ou de branches) soit trop généraux et pouvant être produit par n'importe quel modèle (longueur totale et pourcentage de population desservie), et relèvent de critères de forme typique de la modélisation procédurale qui ne peuvent que difficilement rendre compte des dynamiques internes d'un système comme développé précédemment. De plus, prendre pour validation externe la production d'un réseau hiérarchique découle d'une exploration incomplète de la structure et du comportement du modèle, puisque celui-ci par ses mécanismes d'attachement préférentiel doit mécaniquement produire une hiérarchie. Ainsi, une attention particulière devra être donnée au choix des critères de validation.}.
}

% ce qui est porteur de sens au regard des liens entre différents types de complexité développés en~\ref{sec:epistemology}

% bio-inspired design :
% http://journals.sagepub.com/doi/abs/10.1177/2399808317690156
%  cool mais loin de la pb



\paragraph{Procedural modeling}{Modélisation procédurale}



\bpar{
Finally, we can mention other tentatives such as~\cite{de2007netlogo,yamins2003growing}, which are closer to procedural modeling~\cite{lechner2004procedural,watson2008procedural} and therefore have only little interest in our case since they can difficultly be used as explicative models\footnote{Following~\cite{varenne2017theories}, an explicative model allows to produce an explanation to observed regularities or laws, for example by suggesting processes which can be at their origin. If model processes are explicitly detached from a reasonable ontology, they can not be potential explanations. We will give in~\ref{sec:computation} a development of this notion in the frame of a more global reflexion on the epistemology of modeling.}. Procedural modeling consists in generating structures in a way similar to shape grammars\footnote{A shape grammar is a formal system (i.e a set of initial symbos, axioms, and a set of transformation rules) which acts on geometrical objects. Starting from initial patterns, they allow to generate classes of objects.}, but it also concentrates generally on the faithful reproduction of local form, without considering macroscopic emerging properties. Classifying them as morphogenesis models is incorrects and corresponds to a misunderstanding of mechanisms of \emph{Pattern Oriented Modeling}~\cite{grimm2005pattern}\footnote{\emph{Pattern Oriented Modeling} consists in seeking to explain observed patterns, generally at multiple scales, in a \emph{bottom-up} way. Procedural modeling does not correspond to that, since it aims at reproducing and not at explaining.} on the one hand and of the epistemology of morphogenesis on the other hand (see~\ref{sec:interdiscmorphogenesis}). We will use this type of models (exponential mixture to produce a population density for example) to generate initial synthetic data uniquely to parametrize other complex models (see~\ref{sec:computation} and \ref{sec:correlatedsyntheticdata}).
}{
Finalement, nous pouvons mentionner d'autres tentatives comme~\cite{de2007netlogo,yamins2003growing}, qui sont plus proches de la modélisation procédurale~\cite{lechner2004procedural,watson2008procedural} et pour cette raison n'ont que peu d'intérêt pour notre cas puisqu'ils peuvent difficilement être utilisés comme modèles explicatifs\footnote{Suivant~\cite{varenne2017theories}, un modèle explicatif permet de produire une explication à des régularités ou des lois observées, par exemple en suggérant des processus pouvant en être à l'origine. Si les processus du modèle sont explicitement dissociés d'une ontologie raisonnable, ceux-ci ne peuvent être explications potentielles. Nous donnerons en~\ref{sec:computation} un développement de cette notion dans le cadre d'une réflexion plus générale sur l'épistémologie de la modélisation.}. La modélisation procédurale consiste à générer des structures à la manière des grammaires de forme\footnote{Une grammaire de forme est un système formel (c'est-à-dire un ensemble de symboles initiaux, les axiomes, et un ensemble de règles de transformation) qui agit sur des objets géométriques. Partant de motifs initiaux, elles permettent de générer des classes d'objets.}, mais celle-ci se concentre généralement sur la reproduction fidèle de forme locale, sans tenir compte des propriétés macroscopiques émergentes. Les classifier comme modèles de morphogenèse n'est pas correct et correspond à une incompréhension des mécanismes du \emph{Pattern Oriented Modeling}~\cite{grimm2005pattern}\footnote{Le \emph{Pattern Oriented Modeling} consiste à chercher à expliquer des motifs observés, généralement à plusieurs échelles, dans une démarche \emph{bottom-up}. La modélisation procédurale n'en relève pas, puisqu'elle vise à reproduire et non à expliquer.} d'une part et de l'épistémologie de la morphogenèse d'autre part (voir~\ref{sec:interdiscmorphogenesis}). Nous utiliserons ce type de modèle (mélange d'exponentielles pour produire une densité de population par exemple) pour générer des données synthétiques initiales uniquement pour paramétrer d'autres modèles complexes (voir~\ref{sec:computation} et \ref{sec:correlatedsyntheticdata}).
}






%%%%%%%%%%%%%%%%%%
\subsection{Modeling co-evolution}{Modéliser la co-évolution}



\bpar{
We can now switch to models that integrate dynamically the paradigm Territory $\leftrightarrow$ Network, which as we recall assumes that the conditioning of one by the other can not be identified. The ontologies used, as we will see, often couple\footnote{We recall the definition of model coupling, which corresponds to the one of system or process coupling given in introduction: it is the construction of a model that is simultaneously the extension of each initial model.} network elements with territorial components, but this positioning is not necessary and some elements may be hybrid (for example a governance structure for the transportation network may simultaneously belong to both aspects). In our reading of models, these different specifications will naturally arise.
}{
Nous pouvons à présent nous intéresser aux modèles intégrant dynamiquement le paradigme Territoire $\leftrightarrow$ Réseau, qui on le rappelle suppose qu'un conditionnement de l'un par l'autre n'est pas identifiable. Les ontologies utilisées, comme nous le verrons, couplent\footnote{Nous rappelons la définition du couplage de modèle, qui correspond à celle de couplage de système ou de processus donnée en introduction : il s'agit de la constitution d'un modèle qui est simultanément une extension de chacun des modèles initiaux.} souvent des éléments de réseau avec des composantes territoriales, mais cette position n'est pas une nécessité et certains éléments peuvent être hybrides (par exemple une structure de gouvernance du système de transport peut relever simultanément des deux aspects). Dans notre lecture des modèles, ces différentes spécifications se dégageront naturellement.
}

% sur couplage dans rui2011 : \comment[FL]{en parlant comme cela, tu considere comme aquis que dans ce type de modele, on travaille par sous-blocs ayant des liens avec les autres : cest discutable donc a discuter}[(JR) ontologies separees (cf chap 1), donc necessairement decomposition modulaire avec ontologie de couplage (cf chap 9) $\rightarrow$ a positionner en intro de la sous-partie]



\bpar{
We will broadly designate by model of co-evolution simulation models that include a coupling of urban growth dynamics and transportation network growth dynamics. These are relatively rare, and for most of them still at the stage of stylized models. The efforts being relatively sparse and in very different domains, there is not much unity in these approaches, beside the abstraction of the assumption of an interdependency between networks and territorial characteristics in time. We propose to review them still through the prism of scales.
}{
Nous désignerons largement par modèle de co-évolution les modèles de simulation qui incluent un couplage des dynamiques de la croissance urbaine et du réseau de transport. Ceux-ci sont relativement rares, et pour la plupart au stade de modèles stylisés. Les efforts étant assez disparates et dans des domaines très variés, il y a peu d'unité dans ces approches, si ce n'est l'abstraction de l'hypothèse d'interdépendance entre réseaux et caractéristiques du territoire dans le temps. Nous proposons de les passer en revue toujours avec la grille de lecture des échelles.
}



\subsubsection{Microscopic and mesoscopic scales}{Echelle microscopique et mesoscopique}


\paragraph{Geometrical Models}{Modèles géométriques}

\bpar{
\cite{achibet2014model} describes a co-evolution model at a very large scale (scale of the building), in which evolution of both network and buildings are ruled by a same agent, influenced differently by network topology and population density, and that can be understood as an agent of urban development. The model allows to simulate an auto-organized urban extension and to produce district configurations. Even if it strongly couples territorial components (buildings) and the road network, described results do not imply any conclusion on the processes of co-evolution themselves.
 }{
\cite{achibet2014model} décrit un modèle de co-évolution à une très grande échelle (échelle du bâtiment), dans lequel l'évolution du réseau et des bâtiments sont tous les deux régis par un agent commun, influencé différemment par la topologie du réseau et la densité de population, qui peut être compris comme un agent développeur. Le modèle permet de simuler une extension urbaine auto-organisée et de produire des configurations de quartier. Bien qu'il couple fortement composantes territoriales (bâtiments) et réseau routier, les résultats présentés ne permettent pas de tirer de conclusion sur les processus de co-évolution en eux-mêmes.
}



%\cite{ruas2011conception} regles procedurale, echelle micro.
% http://florence.curie.free.fr/pdf/jfsma2010.pdf
% http://florence.curie.free.fr/pdf/agile2010.pdf
% http://geopensim.ign.fr/publication.html
% https://scholar.google.fr/scholar?cites=18028265655543708128&as_sdt=2005&sciodt=0,5&hl=fr
%  -> too few information


\bpar{
A generalization of the geometrical local optimization model described before is developed in~\cite{barthelemy2009co}. It aims at capturing the co-evolution of network topology with the density of its nodes. The localization of new nodes is simultaneously influenced by density and centrality, yielding the looping of the strong coupling. More precisely, the global behavior of the model is the same, as the network extension behavior. Centers then localize following a utility function that is a linear combination of average betweenness centrality in a neighborhood and of the opposite of density (dispersion due to higher price as a function of density). This utility is used to compute the probability of localization of new centers following a discrete choices model. The model allows to show that the influence of centrality reinforces aggregation phenomena (in particular through an analytical resolution on a one-dimensional version of the model), and furthermore reproduces exponentially decreasing density profiles (Clarcke's law) which are observed empirically.  
}{
Une généralisation du modèle d'optimisation locale géométrique décrit précédemment est développée dans~\cite{barthelemy2009co}, et cherche à capturer la co-évolution entre topologie du réseau et densité de ses noeuds. La localisation de nouveaux noeuds est influencée à la fois par la densité et la centralité, permettant de boucler le couplage fort. Plus précisément, le fonctionnement global du modèle est le même, ainsi que la règle de croissance du réseau. Les centres se localisent quant à eux selon une fonction d'utilité qui est une combinaison linéaire de la centralité de chemin moyenne dans un voisinage et de l'opposée de la densité (dispersion due aux prix plus élevés en fonction de la densité). Cette utilité permet de définir la probabilité de localisation des nouveaux centres suivant un modèle de choix discrets. Le modèle permet de montrer que l'influence de la centralité accentue les phénomènes d'agrégation (notamment par une résolution analytique sur une version en une dimension du modèle), et reproduit par ailleurs des profils exponentiels décroissants pour la densité (loi de Clarke), observés empiriquement.
}
% note : pas d'effet contradictoires dans ce modèle ? : que du renforcement : doit capturer moins de formes meme que le morphogenese simple ? en fait si, densité répulse.


\bpar{
\cite{ding2017heuristic} introduce a model of co-evolution between different layers of the transportation network, and show the existence of an optimal coupling parameter in terms of inequalities for the centrality in network conception: if the road network is assimilated at a fine granularity to a population distribution, this model can be compared with the precedent model of co-evolution between the transportation network and the territory.
}{
\cite{ding2017heuristic} introduisent un modèle de co-évolution entre différentes couches du réseau de transport, et montrent l'existence d'un paramètre de couplage optimal en terme d'inégalités de centralité pour la conception d'un réseau : si on assimile le réseau routier à granularité très fine à une distribution de population, ce modèle se rapproche du précédent modèle de co-évolution entre réseau de transport et territoire.
}


%\comment{\cite{stanley2017simple} check if coevolution : repartition of activities depends on roads networks, not clear how it grows.}
% -- on hold for now --

\paragraph{Economic models}{Modèles économiques}


\bpar{
\cite{levinson2007co} take an economic approach, which is richer from the point of view of network development processes implied, similar to a four step model (i.e. including the generation of origin-destination flows and the assignment of traffic in the network) including travel cost and congestion, coupled with a road investment module simulating toll revenues for constructing agents, and a land-use evolution module updating actives and employments through discrete choice modeling. The exploration experiments show that co-evolving network and land uses lead to positive feedbacks reinforcing hierarchies. These are however far from satisfying, since network topology does not evolve as only capacities and flows change within the network, what implies that more complex mechanisms (such as the planning of new infrastructures) on longer time scales are not taken into account. \cite{li2016integrated} have recently extended this model by adding endogenous real estate prices and an optimization heuristic with a genetic algorithm for deciding agents.
}{
\cite{levinson2007co} prennent une approche économique plus riche du point de vue des processus de développement de réseau impliqués, similaire à un modèle à quatre étapes (c'est-à-dire incluant une génération de flux origine-destination et une attribution du trafic dans le réseau) qui inclut coût de transport et congestion, couplé avec un module d'investissement routier qui simule les revenus des péages pour les agents qui construisent, et un module d'évolution d'usage du sol qui simule les relocalisations des actifs et des emplois. Les expériences d'exploration de ce modèle montrent que l'usage du sol et le réseau en co-évolution conduisent à des retroactions positives renforçant les hiérarchies. Elles sont cependant loin d'être satisfaisantes puisque la topologie du réseau n'évolue pas à proprement parler puisque seules les capacités et les flux changent dans le réseau, ce qui signifie que des mécanismes plus complexes (comme la planification de nouvelles infrastructures) sur de plus longues échelles de temps ne sont pas pris en compte. \cite{li2016integrated} ont récemment étendu ce modèle par l'ajout de prix immobiliers endogènes et d'une heuristique d'optimisation par algorithme génétique pour les agents décideurs.
}

\bpar{
From an other point of view, \cite{levinson2005paving} is also presented as a model of co-evolution, but corresponds more to a predictive model based on Markov chains, and thus closer to a statistical analysis than a simulation model based on these processes. \cite{rui2011urban} describe a model in which the coupling between land-use and network topology is done with a weak paradigm, land-use and accessibility having no feedback on network topology, the land-use model being conditioned to the growth of the autonomous network.
}{
D'un autre point de vue, \cite{levinson2005paving} est aussi présenté comme un modèle de co-évolution mais repose sur un modèle prédictif à chaîne de Markov, et donc plus proche d'une analyse statistique que d'un modèle de simulation basé sur des processus. \cite{rui2011urban} décrivent un modèle dans lequel le couplage entre usage du sol et la topologie du réseau est fait par un paradigme faible, l'usage du sol et l'accessibilité n'ayant pas de retroaction sur la topologie du réseau, le modèle d'usage du sol étant conditionné à la croissance du réseau autonome.
}

%Ce modèle est mis en perspective avec d'autres modèles d'usage du sol et de croissance de réseau dans~\cite{rui2013urban}\comment[FL]{et alors ?}.



\paragraph{Cellular automatons}{Automates cellulaires}

\bpar{
A simple hybrid model explored and applied to a stylized planning example of the functionnal distribution of a new district in~\cite{raimbault2014hybrid}, relies on mechanisms of accessibility to urban activities for the growth of settlements with a network adapting to the urban shape. The rules for network growth are too simple to capture more elaborated processes than just a simple systematic connection (such as potential breakdown for example), but the model produces at a large scale a broad range of urban shapes reproducing typical patterns of human settlements. This model is inspired by~\cite{moreno2012automate} for its core mechanisms but yield a much broader generation of forms by taking into account urban functions.
}{
Un modèle hybride simple exploré et appliqué à un exemple stylisé de planification de la répartition fonctionnelle d'un nouveau quartier dans~\cite{raimbault2014hybrid}, repose sur les mécanismes d'accès aux activités urbaines pour la croissance des établissements avec un réseau s'adaptant à la forme urbaine. Les règles pour la croissance du réseau sont trop simples pour capturer des processus plus élaborés qu'une simple connection systématique (comme une rupture de potentiel par exemple), mais le modèle produit à une grande échelle une large gamme de formes urbaines qui reproduisent les motifs typiques des établissements humains. Ce modèle s'inspire de~\cite{moreno2012automate} pour ses mécanismes de base mais permet une génération de formes bien plus larges par la prise en compte des fonctions urbaines.
}


\bpar{
At these relatively large scales, spanning from the urban to the metropolitan scale, mechanisms of population localization influenced by accessibility coupled to mechanisms of network growth optimizing some particular functions seem to be the rule for this kind of models: in the same way, \cite{wu2017city} couple a cellular automaton for population diffusion to a network optimizing local cost that depends on the geometry and on population distribution.
}{
A ces échelles relativement grandes, s'étendant de l'échelle urbaine à celle métropolitaine, les mécanismes de localisation de population influencée par l'accessibilité couplés à des mécanismes de croissance de réseau optimisant certaines fonctions semblent être la règle pour ces modèles : de la même façon, \cite{wu2017city} couplent un automate cellulaire de diffusion de population à un réseau optimisant un coût local dépendant de la géométrie et de la distribution de population.
}


\bpar{
Models answering to more remote questions can furthermore be linked to our problem: for example, in a conceptual way, a certain form of strong coupling is also used in~\cite{bigotte2010integrated} which by an approach of operational research propose a network design algorithm to optimize the accessibility to amenities, taking into account both network hierarchy and the hierarchy of connected centers.
}{
Des modèles répondant à des problématiques assez lointaines peuvent par ailleurs être reliés à notre question : par exemple, de manière conceptuelle, une certaine forme de couplage fort est également utilisé dans~\cite{bigotte2010integrated} qui par une approche de recherche opérationnelle proposent un algorithme de conception de réseau pour optimiser l'accessibilité aux services, prenant en compte à la fois la hiérarchie du réseau et celle des centres connectés.
}


\bpar{
This way, co-evolution models at the microscopic and mesoscopic scales globally have the following structure: (i) processes of localization or relocalization of activities (actives, buildings) influenced by their own distribution and network characteristics; (ii) network evolution, that can be topological or not, answering to very diverse rules: local optimization, fixed rules, planning by deciding agents. This diversity suggests the necessity to take into account the superposition of multiple processes ruling network evolution.
}{
Ainsi, les modèles de co-évolution aux échelles microscopique et mesoscopiques suivent globalement la structure suivante : (i) processus de localisation ou relocalisation des activités (actifs, bâtiments) influencés par leur propre distribution et les caractéristique du réseau ; (ii) evolution du réseau, topologique ou non, répondant à des règles très diverses : optimisation locale, règles fixes, planification par des agents décideurs. Cette diversité suggère la nécessité de prendre en compte la superposition de multiples processus régissant l'évolution du réseau.
}




\subsubsection{Urban systems modeling}{Modélisation de Systèmes Urbains}


\bpar{
At a macroscopic scale, co-evolution can be taken into account in models of urban systems. \cite{baptiste1999interactions} propose to couple an urban growth model based on migrations (introduced by the application of synergetics to systems of cities by~\cite{sanders1992systeme}) with a mechanism of self-reinforcement of capacities for the road network without topological modification. More precisely, the general principles of the model are the following.
\begin{itemize}
\item Attractivity and repulsion indicators allow for each city to determine emigration and immigration rates and to make populations evolve.
\item Network topology is fixed in time, but capacities of links evolve. The rule is an increase in capacity when the flow becomes greater given a fixed parameter threshold during a given number of iterations. Flows are affected with a gravity model of interaction between cities.
\end{itemize}
}{
A une échelle macroscopique, la co-évolution est parfois prise en compte dans des modèles de systèmes urbains. \cite{baptiste1999interactions} propose de coupler un modèle de croissance urbaine basé sur les migrations (introduit par l'application de la synergétique aux systèmes de villes par~\cite{sanders1992systeme}) avec un mécanisme d'auto-renforcement des capacités pour le réseau routier sans modification topologique. Plus précisément, les principes généraux du modèle sont les suivants.
\begin{itemize}
	\item Des indicateurs d'attractivité et de répulsion permettent pour chaque ville de déterminer des taux d'émigration et d'immigration et de faire évoluer les population.
	\item La topologie du réseau est fixée dans le temps, mais les capacités des liens évoluent. La règle est une augmentation de la capacité lorsque le flux dépasse celle-ci par un seuil donné comme paramètre pendant un certain nombre d'itérations. Les flux sont affectés par modèle gravitaire d'interaction entre les villes.
\end{itemize}
}

% en fait combinaison d'un gravitaire et d'un radiation ; plus riche que le notre ! - serait intéressant a explorer - plus generalement besoin d'un benchmark des modeles de systemes urbains, en termes synthetiques et d'application a divers systemes de villes. --> absolument developper ca

\bpar{
The last version of this model is presented by~\cite{baptistemodeling}. General conclusions that can be obtained from this work are that this coupling yield a hierarchical configuration\footnote{But we also know that simpler models, only a preferential for example, allow to reproduce this stylized fact. The model must have as an objective to answer to broader questions, such as the fine understanding of co-evolution processes, what is not done here. However, one of its operational objectives is otherwise fulfilled, through the application to France and the study of the impact of a high speed line project, recalling the multiple possible functions of a model (see~\ref{sec:computation}).} and that the addition of the network produces a less hierarchical space, allowing medium-sized cities to benefit from the feedback of the transportation network.
}{
Sa dernière version est présentée par~\cite{baptistemodeling}. Les conclusions générales qui peuvent être tirées de ce travail sont que ce couplage permet de faire émerger une configuration hiérarchique\footnote{Mais on sait par ailleurs que des modèles plus simples, un attachement préférentiel uniquement par exemple, permettent de reproduire ce fait stylisé. Le modèle doit avoir pour objectif de répondre à des problématiques plus larges, comme la compréhension fine des processus de co-évolution, ce qui n'est pas fait ici. Cependant, l'un de ses objectifs opérationnels est par ailleurs rempli, par l'application à la France et l'étude de l'impact d'un projet de Ligne à Grande Vitesse, rappelant les multiples fonctions possibles d'un modèle (voir~\ref{sec:computation}).} et que l'ajout du réseau produit un espace moins hiérarchique, permettant à des villes moyennes de bénéficier de la rétroaction du réseau de transport.
}




\bpar{
The model proposed by~\cite{blumenfeld2010network} can be seen as a bridge between the mesoscopic scale and the approaches of urban systems, since it simulates migrations between cities and network growth induced by potential breakdown when detours are too large. In the continuity of Simpop models for systems of cities, \cite{schmitt2014modelisation} describes the SimpopNet model which aims at precisely integrating co-evolution processes in systems of cities on long time scales, typically via rules for hierarchical network development as a function of the dynamics of cities, coupled with these that depends on network topology. Unfortunately the model was not explored nor further studied, and furthermore stayed at a toy-level. \cite{cottineau2014evolution} proposes an endogenous transportation network growth as the last building brick of the Marius modeling framework, but it stays at a conceptual level since this brick has not been specified nor implemented yet. To the best of our knowledge, there exists no model which is empirical or applied to a concrete case based on an approach of co-evolution by urban systems from the point of view of the evolutive urban theory.
}{
Le modèle proposé par~\cite{blumenfeld2010network} peut être vu comme un pont entre l'échelle mesoscopique et les approches de type système urbain, puisqu'il simule les migrations entre villes et la croissance du réseau induite par une rupture de potentiel lorsque les détours sont trop grands. Dans la continuité des modèles Simpop pour modéliser les systèmes de villes, \cite{schmitt2014modelisation} décrit le modèle SimpopNet qui vise à précisément intégrer les processus de co-évolution dans les systèmes de villes à longue échelle temporelle, typiquement par des règles pour un développement hiérarchique du réseau comme fonction des dynamiques des villes, couplées à celles-ci qui dépendent de la topologie du réseau. Malheureusement le modèle n'a pas été exploré ni étudié de manière plus approfondie, et de plus est resté au niveau de modèle jouet. \cite{cottineau2014evolution} propose une croissance endogène des réseaux de transport comme la dernière brique de construction du cadre de modélisation Marius, mais cela reste à un niveau conceptuel puisque cette brique n'a pas encore été spécifiée ni implémentée. Il n'existe à notre connaissance pas de modèle empirique ou appliqué à un cas concret se basant sur une approche de la co-évolution par les systèmes urbains vus par la théorie évolutive des villes.
}

\bpar{
We can see well the opposition to epistemological principles of economic geography: \cite{fujita1999evolution} introduce for example an evolutionary model able to reproduce and urban hierarchy and an organization typical of central place theory~\cite{banos2011christaller}, but that still relies on the notion of successive equilibriums, and moreover considers a ``Krugman-like'' model, i.e. a one dimensional and isotropic space, in which agents are homogeneously distributed\footnote{The absence of a real space is not an issue in this economic approach that aims at understanding processes out of their context. In our case, the structure of the geographical space is not separable, and indeed at the core of the issues we are interested in.}. This approach can be instructive on economic processes in themselves but more difficultly on geographical processes, since these imply the embedding of economic processes in the geographical space which spatial particularities not taken into account in this approach are crucial. Our work will focus on demonstrating to what extent this structure of space can be important and also explicative, since networks, and even more physical networks induce spatio-temporal processes that are path-dependent and thus sensitive to local singularities and prone to bifurcations induced by the combination of these with processes at other scales (for example the centrality inducing a flow).
}{
On voit bien l'opposition aux principes épistémologiques de l'économie géographique : \cite{fujita1999evolution} introduisent par exemple un modèle évolutionnaire capable de reproduire une hiérarchie urbaine et une organisation typique de la théorie des places centrales~\cite{banos2011christaller}, mais qui repose toujours sur la notion d'équilibres successifs, et surtout considère un modèle ``à-la-Krugman'' c'est-à-dire un espace à une dimension, isotrope, et dans lequel les agents sont répartis de manière homogène\footnote{L'absence d'espace réel n'est pas un problème dans cette approche économique qui vise à comprendre des processus hors-sol. Dans notre cas, la structure de l'espace géographique n'est pas séparable, et même au coeur des problématiques qui nous intéressent.}. Cette approche peut être instructive sur les processus économiques en eux-mêmes mais plus difficilement sur les processus géographiques, puisque ceux-ci impliquent un déroulement des processus économiques dans l'espace géographique dont les particularités spatiales qui ne sont pas prises en compte dans cette approche sont essentielles. Notre travail s'attachera à montrer dans quelle mesure cette structure de l'espace peut être importante et également explicative, puisque les réseaux, et encore plus les réseaux physiques induisent des processus spatio-temporels dépendants au chemin et donc sensibles aux singularités locales et propices aux bifurcations induites par la combinaison de celles-ci et de processus à d'autres échelles (par exemple la centralité induisant un flux).
}


%\comment[AB]{attention, ça n’est pas forcément une limite fondamentale, tout dépend des objectifs. Peut être suffisant en particulier pour étudier l’apparition de ces « particularités » dont tu parles ensuite}[(JR) oui en effet, on rejoint la question de representation des territoires, qu'est ce qui est necessaire a mettre ou non, selon l'objectif poursuivi (// la meme discussion plus loin) // idee de com pour le cist]



%\bpar{
%We shall position more in that stream of research in this thesis.
%}{
%Nous nous positionnerons particulièrement dans cette lignée de recherche dans cette thèse, vu l'importance que prendra la Théorie Evolutive dans notre démarche théorique et de modélisation comme nous le détaillerons par la suite. Typiquement, les hypothèses ontologiques fondamentales telles le rôle prépondérant des interactions et de la configuration géographique sont représentatives de cette approche si on les considère conjointement.
%}


\bpar{
At the macroscopic scale, existing models are based on the evolution of agents (generally cities) as a consequence of their interactions, carried by the network, whereas the evolution of the network can follow different rules: self-reinforcement, potential breakdown. The general structure is globally the same than at larger scales, but ontologies stay fundamentally different.
}{
A l'échelle macroscopique, les modèles existants se basent sur des évolutions des agents (souvent les villes)  en conséquence de leurs interactions, portées par le réseau, tandis que l'évolution du réseau peut répondre à différentes règles : auto-renforcement, rupture de potentiel. La structure générale est globalement la même qu'à des échelles plus grandes, mais les ontologies restent fondamentalement différentes.
}




\subsubsection*{Synthesis}{Synthèse}



\bpar{
It is crucial at this stage to risk a synthesis and put into perspective all models that we reviewed, since even if it will necessarily be reducing and simplifying, it gives the foundations for the analyses that will follow.
}{
Il est essentiel à ce stade de s'oser à une synthèse et une mise en perspective de l'ensemble des modèles que nous avons passé en revue, puisque même si celle-ci sera nécessairement réductrice et simplificatrice, elle donne les fondations pour les analyses qui suivront.
}


% Type & Classe & Echelles de Temps & Echelles d'Espace & Avantages & Objectif & Application & Paradigme


\bpar{
We will synthesize the broad types of models that we reviewed in the following table, by classing them by type (relation between networks and territories), by class (broad classes corresponding to the stratification of the review), and by giving the temporal and spatial scales concerned, the functions, the type of result obtained, the paradigms used. It is given in Table~\ref{tab:modelingsa:synthesis}.
}{
Nous synthétisons les grands types de modèles que nous avons passé en revue dans le tableau suivant, en les classant par type (relation entre réseaux et territoires), par classe (grandes classes correspondant à la stratification de la revue), et en précisant les échelles temporelles et spatiales concernées, les fonctions, le type de résultats obtenus, les paradigmes utilisés. Celle-ci est donnée en Table~\ref{tab:modelingsa:synthesis}.
}

%%%%%%%%%%%%%
\begin{table}%[h!]
\caption[Synthesis of modeling approaches][Synthèse des approches de modélisation]{\textbf{Synthesis of modeling approaches.} The type gives the sense of the relation; the class is the scientific field in which the model is inserted; scales correspond to our simplified scales; functions are given in the sense of~\ref{sec:computation}; we finally give the type of results they provide and the paradigms used.\label{tab:modelingsa:synthesis}}{\textbf{Synthèse des approches de modélisation.} Le type donne le sens de la relation ; la classe est le champ scientifique dans lequel le modèle se place ; les échelles correspondent à nos échelles simplifiées ; les fonctions sont données au sens de~\ref{sec:computation} ; nous donnons enfin le type de résultat qu'ils fournissent et les paradigmes utilisés.\label{tab:modelingsa:synthesis}}
\bpar{
\begin{tabular}{|p{2.5cm}|p{2cm}|p{2.5cm}|p{2.5cm}|p{2.1cm}|p{2.2cm}|p{2cm}|}
\hline
Type & Class & Temporal Scale & Spatial scale & Function & Results & Paradigms\\ \hline
Networks $\rightarrow$ Territories & LUTI & Medium & Mesoscopic & Planning, Prediction & Land-use simulation & Urban economics \\ \hline
\multirow{3}{*}{Territories $\rightarrow$}& Networks Economics & Medium & Mesoscopic & Explanation & Role of economic processes & Economics, Governance\\\cline{2-7}
Networks& Geometrical growth & Long & Meso or Macro & Explanation & Reproduction of stylized shapes & Simulation models, Local optimization \\\cline{2-7}
& Biological networks & Long & Mesoscopic & Optimization & Production of optimal networks & Self-organized network \\ \hline
\multirow{2}{*}{Territories $\leftrightarrow$}& Networks Economics & Medium & Mesoscopic & Explanation & Reinforcement effects & Economics\\\cline{2-7}
Networks & Geometrical growth & Long or NA & Micro, Meso or Macro & Explanation & Reproduction of stylized shapes & Simulation models, Local optimization \\\cline{2-7}
& Urban Systems & Medium, Long & Macroscopic & Explanation, prospection & Stylized facts & Complex geography\\\hline
\end{tabular}
}{
\begin{tabular}{|p{2.5cm}|p{2cm}|p{2.5cm}|p{2.5cm}|p{2.1cm}|p{2.2cm}|p{2cm}|}
\hline
Type & Classe & Echelle Temporelle & Echelle Spatiale & Fonction & Résultats & Paradigmes\\ \hline
Réseaux $\rightarrow$ Territoires & LUTI & Moyenne & Mesoscopique & Planification, Prédiction & Simulation de l'usage du sol & Économie urbaine \\ \hline
\multirow{3}{*}{Territoires $\rightarrow$}& Économie des Réseaux & Moyenne & Mesoscopique & Explication & Rôle de processus économiques & Économie, Gouvernance\\\cline{2-7}
Réseaux& Croissance géométrique & Longue & Meso ou Macro & Explication & Reproduction de formes stylisées & Modèles de Simulation, Optimisation locale \\\cline{2-7}
& Réseaux biologiques & Longue & Mesoscopique & Optimisation & Production de réseaux optimaux & Réseau auto-organisé \\ \hline
\multirow{2}{*}{Territoires $\leftrightarrow$}& Économie des Réseaux & Moyenne & Mesoscopique & Explication & Effets de renforcement & Économie\\\cline{2-7}
Réseaux & Croissance géométrique & Longue ou NA & Micro, Meso ou Macro & Explication & Reproduction de formes stylisées & Modèles de Simulation, Optimisation locale \\\cline{2-7}
& Systèmes Urbains & Moyenne, Longue & Macroscopique & Explication, prospection & Faits stylisés & Géographie complexe\\\hline
\end{tabular}
}
\end{table}
%%%%%%%%%%%%%

% Rappel : processus :
% Micro & Motifs de mobilité & Congestion du réseau ; Externalités négatives & Mobilité et structure sociale \\ \hline
% Meso & Relocalisations ; Effets locaux des infrastructures & Rupture de potentiel & Planification métropolitaine ; TOD \\ \hline
% Macro & Interactions entre villes ; Effet tunnel & Différenciation hiérarchique de l'accessibilité & Planification à grande échelle ; Dynamique structurelle ; Bifurcations\\ 





\subsubsection*{An neglected coevolution ?}{Une co-évolution négligée ?}

\bpar{
The unbalance between the last section accounting for models integrating effectively a strongly coupled dynamic (and possibly a co-evolution) and the preceding sections leads to an interrogation: are models integrating co-evolution marginal? Is it possible then to explain this marginality?
}{
Le déséquilibre entre la dernière section rendant compte des modèles intégrant effectivement une dynamique fortement couplée (et possiblement une co-évolution) et les précédentes interroge : les modèles intégrant la co-évolution sont-ils si marginaux ? Est-il alors possible d'expliquer cette marginalité ?
}

% Hadri : suppr boulot planificateurs ?

\bpar{
The aim of the two following sections will be to propose elements of answer to these questions through epistemological analyses by increasing the knowledge on concerned fields and of the corresponding models.
}{
L'objet des deux sections qui suivent sera de proposer des éléments de réponse à ces questions par des analyses épistémologiques en accroissant la connaissance des champs concernés et des modèles correspondants.
}




\stars


\bpar{
We have thus given in this section a broad overview of models focusing on interactions between transportation networks and territories, including co-evolution models. We begin thus to foresee a refinement of the definition of the concept of co-evolution in that frame.
}{
Nous avons ainsi donné dans cette section un aperçu large des modèles s'intéressant aux interactions entre réseaux de transport et territoires, incluant les modèles de co-évolution. Nous commençons donc à entrevoir une précision de la définition du concept de co-évolution dans ce cadre.
}


\bpar{
We propose in the next section to proceed to a more systematic mapping of this scientific landscape, in order to reinforce the epistemological viewpoint and better situate the positioning we will take and the models we will introduce in the following.
}{
Nous proposons dans la section suivante de dresser une cartographie plus systématique de ce paysage scientifique, afin de renforcer le point de vue épistémologique et mieux situer la position que nous prendrons et les modèles que nous introduirons par la suite.
}


\stars







