




\newpage

%----------------------------------------------------------------------------------------

\section{An Interdisciplinary Approach to Morphogenesis}{Une Approche Interdisciplinaire de la Morphogenèse}

\label{sec:interdiscmorphogenesis}


%----------------------------------------------------------------------------------------




\bpar{
A first crucial step is a clarification of what is meant by morphogenesis. Initially introduced in biology, its transfert to other field was accompanied by a deformation of associated concepts. We adapt here the text of~\cite{antelope2016interdisciplinary} which proposes an interdisciplinary entry on morphogenesis. As an essential building block of our constructions, it is indeed necessary to give it a rigorous and clear skeleton. We make the choice of a crossed approach, in the spirit of an applied perspectivism as introduced in section~\ref{sec:epistemo-position}, to obtain concepts as generic and broad as possible.
}{
Une première étape essentielle est la clarification de ce qui est entendu par le terme de morphogenèse. Initialement introduit en biologie, son transfert à d'autres champs s'est accompagné d'une déformation des concepts associés. Nous adaptons et traduisons ici le texte de~\cite{antelope2016interdisciplinary} qui propose une entrée interdisciplinaire sur la morphogenèse. Brique essentielle de nos constructions, il est en effet crucial de lui donner une armature rigoureuse et claire. Nous prenons le parti d'une vision croisée, dans l'idée d'un perspectivisme appliqué comme introduit en section~\ref{sec:epistemology}, pour obtenir des concepts aussi génériques et larges que possible.\comment[FL]{TB}
}


\bpar{
The notion of morphogenesis seems to play an important role in the study of a broad range of complex systems. If the concept was introduced in embryology to design growth of organisms, it was rapidly used in various fields, e.g. urbanism, geomorphology and even psychology. However, the use of the concept seems generally fuzzy and to have a field-specific definition for each use. We propose in this section an epistemological study, starting with a broad interdisciplinary review and extracting essential notions linked to morphogenesis across fields. It allows to build a consistent general meta-framework for morphogenesis. Further work may include concrete application of the framework on particular cases to operate interdisciplinary transfers of concepts, and quantitative text analysis to strengthen qualitative results.
}{
La notion de morphogenèse semble jouer un rôle important dans l'étude d'une large gamme de systèmes complexes. Si le concept a été introduit initialement en embryologie pour désigner la croissance des organismes, il a été rapidement utilisé dans différentes disciplines, e.g. l'urbanisme, la géomorphologie, et même la psychologie. Toutefois, l'utilisation du concept semble généralement floue et avoir une définition spécifique à chaque champ pour chacune de ses utilisations. Nous menons dans cette section une étude épistémologique, commençant par une revue interdisciplinaire large puis en extrayant les notions essentielles liées à la morphogenèse dans chaque champ. Cela permet de construire un meta-cadre général consistent pour la morphogenèse. Des applications peuvent inclure une application concrète du cadre sur des cas particuliers pour opérer un transfert interdisciplinaire de concepts, et des analyses quantitatives de texte pour renforcer ces résultats qualitatifs.
}



\paragraph{Context}{Contexte}



\bpar{
During every historical period, people use the main technological advance as a metaphor to explain other phenomena in nature. First, nature was mechanical, then electrical, and now computational. Here, we suggest that taking an alternative metaphor might allow us to better study some properties of a system, and study how the concept of morphogenesis that originated in the study of developmental biology, can be used across systems. Morphogenesis is a very powerful metaphor that is distinct from the previous three that have been very popular in history. Unlike the mechanical, electrical or computational explanations of nature, morphogenesis is not a human designed process. Morphogenesis emphasizes the role of change and growth, rather than a static state. As \cite{thompson1942growth} already pointed out, ``natural history deals with ephemeral and accidental, not eternal nor universal things''. The goal of this paper is to study three questions: 
\begin{enumerate}
\item How is morphogenesis defined in different fields? 
\item Are there fields that use approaches and concepts that embody the notion of morphogenesis but do not use the word?
\item Can approaches to study morphogenesis be applied across different fields?
\end{enumerate}
A similar effort is described in~\cite{bourgine2010morphogenesis}, but it consists more of a collection of viewpoints from subjects that can be related to morphogenesis rather than an epistemological reconstruction of the notion as we propose to do. Furthermore, examples are far from exhausted and our review is thus complementary.
}{
Durant chaque période historique, l'avancée technologique principale a été utilisée comme une métaphore pour expliquer d'autres phénomènes de la nature. D'abord, la nature a été mécanique, puis électrique, et à présent computationnelle. Ici, nous suggérons qu'une métaphore alternative peut permettre de mieux étudier les propriétés d'un système, et ainsi comprendre comment le concept de morphogenèse qui a trouvé son origine en biologie du développement, peut être utilisé pour d'autres types de systèmes. La morphogenèse est une métaphore très puissante qui est bien distincte des trois précédentes qui ont été très populaires dans l'histoire. Contrairement aux explications mécaniques, électriques ou computationnelles de la nature, la morphogenèse n'est pas un processus conçu par l'homme. La morphogenèse met l'accent sur le rôle du changement et de la croissance, plutôt qu'un état statique. Comme \cite{thompson1942growth} mentionnait déjà, ``l'histoire naturelle traite de l'éphémère et les accidents, pas par des choses éternelles ou universelles''. Le but de notre exercice est de répondre à trois questions : (i) comment la morphogenèse est définie dans différents champs; (ii) existe-t-il des champs qui utilisent des approches et concepts incluant la notion de morphogenèse mais sans utiliser le terme; (iii) dans quelle mesure les approches étudiant la morphogenèse peuvent-elles être transférées entre les champs ? Un effort similaire a été mené par~\cite{bourgine2010morphogenesis} mais consiste plus en une collection de points de vue de sujets liés à la morphogenèse plutôt qu'une reconstruction épistémologique de la notion comme nous proposons de faire. De plus, les exemples sur ce sujet sont loin d'être épuisés et notre revue est pour cela complémentaire.
}

\comment[CC]{du coup tu traduis l'ensemble de l'article plutot que de reprendre et adapter des morceaux? ca peut faire un peu patchwork, notamment dans les objectifs comme ici ou dans les references disciplinaires... comme plus bas. On risque de te dire que tu es loin des reseaux et des territoires!}


\bpar{
The rest of this section is organized as follows : we provide first a compartmentalized review of the notion of morphogenesis across various fields, ranging from biology to social sciences, psychology and territorial sciences. A synthesis is then made and a framework as general as possible proposed. We finally discuss further developments and potential application of this epistemological analysis.
}{
La suite de cette section est organisée de la façon suivante : nous produisons d'abord une revue autonome de la notion de morphogenèse pour différents champs, s'étendant de la biologie aux sciences sociales, la psychologie et les sciences territoriales. Une synthèse est ensuite faite et un cadre aussi général que possible proposé. Nous discutons finalement des développements futurs et des applications potentielles de cette analyse épistémologique.
}




\subsection{Reviews}{Revues}

Nous proposons un aperçu large de la manière dont est utilisée la notion de morphogenèse dans des domaines a priori très éloignés. Notre revue ne se prétend pas exhaustive et nous n'utilisons pas de méthode systématique, l'idée étant de mobiliser et de croiser différentes conceptions pertinentes de la notion.

\subsubsection{Developmental Biology}{Biologie du Développement}


\bpar{
In developmental biology, morphogenesis refers to the mechanisms of how an organism acquires its shape and different functional units, starting from only one cell. Generally, these mechanisms need to work reliably in order to guarantee similar outcomes for every individual. This often requires cells to know their position relative to some reference frame, in order to differentiate (a term used to describe how cells acquire a specific fate, becoming, say, skin cells, as opposed to neurons) or to decide whether or not to divide (which is often necessary for growth). The following section describes models that have been applied in developmental biology.
%Afterwards, we briefly describe how positional information and physical forces (created by biochemical reactions), can play together to determine the shape of tissues, organs and whole organisms. Finally, we will use the development of the nematode worm \textit{C. elegans} as an example for a comparatively well-studied system with many open questions to be asked.
}{
En biologie du développement, la morphogenèse fait référence aux mécanismes conduisant un organisme à acquérir sa forme et différentes unités fonctionnelles, en partant d'une unique cellule. De manière générale, ces mécanismes doivent être fiables pour garantir une issue similaire pour chaque individu. Cela suppose que les cellules connaissent leur position par rapport à un cadre de référence afin de se différencier, c'est à dire prendre une fonction particulière, ou pour décider si elles doivent se diviser ou non, ce qui est une étape cruciale lors de la croissance. Nous décrivons par la suite les modèles qui ont été appliqués en biologie du développement.
}


\paragraph{Reaction-Diffusion mechanisms}{Mécanismes de réaction-diffusion}

\bpar{
Alan Turing used the term reaction-diffusion system in his seminal 1952 paper 'The Chemical Basis of Morphogenesis' to describe simple patterning in a (theoretical) ring of cells~\cite{turing1952chemical}. Even though this work is now considered one of the most fundamental contributions to the field of pattern formation, it took many years until his work started getting recognition as an actual model for biological systems. Gierer{\&}Meinhardt~\cite{gierer1972theory} then suggested using similar models also for intracellular polarity - a ubiquitous phenomenon in biology in which a cell establishes and maintains two different regions within itself - an important capability of most cell types. 
These reaction diffusion networks are one example of the emergence of patterns from a homogeneous state. Using this framework we can recapitulate many pattern formation mechanisms in development, such as coloration, segmentation as well as establishment and maintenance of cell polarity. These larger scale patterns are generated by the interaction of a few species of chemicals. Every chemical species also undergoes diffusion, production and degradation. Thus it is possible to represent this model using a system of partial differential equations, and certain parameters will generate stable patterns from homogeneous initial condition, where random perturbations are amplified by the system. With only a few molecular species, very complex patterns can be formed~\cite{kondo2010reaction}. One of the most studied reaction-diffusion model capable of producing stable patterns comprises of two types of molecules, one activator and one repressor. The difference in diffusion rate between the two molecules is what amplifies random noise in the system~\cite{Turing1952,gierer1972theory}. The most well-studied reaction-diffusion system explaining coloration is in zebrafish. Cells called melanophores and xanthophores produce black and yellow pigments respectively~\cite{nakamasu2009interactions}. The interaction of melanophores and xanthophores produces the striped pattern on zebrafish. Melanophore growth is promoted by long-distance interaction with xanthophores. Short distance interactions between the two cell types inhibit each other. Polarity formation in yeast division can also be explained by a reaction-diffusion mechanism involving the small Rho-GTPase Cdc42. Cdc42 has two forms, one active membrane bound and one inactive cytoplasmic~\cite{goryachev2008dynamics} Phenomena like body segmentation in \textit{Drosophila melanogaster} usually involves a more complex system than the two previously discussed examples, because the pattern it generates needs to be robust to ensure the correct function, and thus cannot be sensitive to variation in initial conditions. 
}{
Le terme de réaction-diffusion avait été utilisé par \noun{Alan Turing} dans son article séminal de 1952~\cite{turing1952chemical}, pour décrire l'émergence de motifs dans un anneau théorique de cellules. Bien que ce travail soit aujourd'hui reconnu comme l'une des contributions les plus fondamentales dans le champ de la formation de motifs, il a fallu des années pour qu'il trouve une reconnaissance comme modèle effectif pour les systèmes biologiques. \cite{gierer1972theory} a plus tard suggéré d'utiliser des modèles similaires pour expliquer la polarité intracellulaire, qui correspond à la capacité d'une cellule à différencier des zones dans son intérieur. Ces réseaux de réaction-diffusion sont un exemple de l'émergence de motifs à partir d'un état homogène, parmi d'autres comme la coloration ou la segmentation. Ces motifs à grande échelle sont générés par l'interaction entre un petit nombre d'espèces chimiques, chacune suivant une diffusion, une production et une dégradation. Il est ainsi possible d'utiliser des systèmes d'équations aux dérivées partielles, pour lesquelles certains paramètres généreront des motifs stables à partir de conditions initiales homogènes, où les perturbations aléatoires sont amplifiées par le système. Des motifs complexes peuvent être produits à partir d'un nombre très restreint d'espèces~\cite{kondo2010reaction}. L'une des réactions capables de produire des motifs stables les plus étudiées comporte deux types de molécules, un activateur et un répresseur. La différence dans le taux de diffusion entre les deux molécules est responsable de l'amplification du bruit dans le système~\cite{gierer1972theory}. Le système à l'origine d'une coloration le plus étudié sont les réactions responsables des rayures jaunes et noires du poisson zèbre~\cite{nakamasu2009interactions}. L'émergence de la polarité cellulaire est expliquée chez certaines levures par un mécanisme similaire~\cite{goryachev2008dynamics}. Des exemples impliquant des fonctions comme la segmentation du corps de \textit{Drosophila melanogaster} impliquent des réseaux d'espèces chimiques bien plus complexes pour assurer la robustesse de l'émergence de ces fonctions.
}


\paragraph{The French Flag Model}{Le modèle French Flag}

\bpar{
Similar to the reaction-diffusion framework the French Flag Model was initially conceived to explain differentiation of cells in a regular fashion~\cite{Wolpert1969}. For example, stripes of cells within a tissue might need to assume different fates. Similar to a French flag, which has three differently colored stripes, cells within a tissue were thought to assume different fates if they are exposed to different concentrations of a certain protein - generically called a morphogen. This requires a graded concentration of the morphogen, which can be achieved if the morphogen is only produced locally and then diffuses across the tissue, away from the source. In order to achieve a stable gradient, on top of local protein production the system also needs either long range inhibition, a sink opposite to the source, or a degradation mechanism within the tissue (reviewed in~\cite{Rogers2011,Wolpert2011}). Once the gradient is set up, it can be interpreted linearly (for example by increasing the expression of a gene linearly with morphogen concentration) or switch-like by feedback mechanisms which are thought to further amplify the morphogen signal and is then translated to a specific cell fate at each position. This can be achieved via a variety of genetic circuits depending on the tissue. To the author's knowledge there is no single well understood system, but there is evidence for such mechanisms at work at least on a coarse grain level~\cite{Wolpert2011}. For rigorous verification of these models, precise measurements of molecular mobilities as well as production and decay rates are necessary, alas very hard to obtain (fluorescent tags that are often used might change the molecules behavior, \textit{in vivo} measurements are hard to perform, etc.). Also, experimenters are often confronted with biological redundancy, which can obscure effects of individual proteins. 
}{
De façon similaire, le modèle French Flag a été conçu initialement pour expliquer la différentiation des cellules de manière régulière~\cite{Wolpert1969}. Le modèle suppose un gradient de concentration d'une protéine, généralement appelée le morphogen, auquel les cellules d'un tissu réagiront différemment selon leur niveau (d'ou les rayures du drapeau). Un tel gradient doit être produit par une diffusion, à partir d'une source, complété par un mécanisme de stabilisation impliquant un puits ou une dégradation locale dans le tissu (mécanismes qui sont passés en revue par \cite{Rogers2011}). Le gradient peut ensuite être utilisé localement de manière linéaire (l'expression d'un gène variant de manière linéaire par exemple) ou par seuils grâce à des boucles de retroaction locales. D'après \cite{Wolpert2011}, aucun de ces systèmes n'est parfaitement bien compris, mais les preuves empiriques de leur existence sont claires à une granularité assez grande. Les expériences nécessaires pour leur vérification exacte sont en effet très difficiles et encore hors de portée pour la plupart.
}

\paragraph{Forces as drivers of cell and tissue shape}{Forces inter-cellulaires}


\bpar{
Epithelial rearrangements are often driven by intracellular forces that are generated by motor proteins acting on the cytoskeleton (e.g. kinesins walking along microtubules, actomyosin-mediated cortical tension)~\cite{Lecuit2007,Heisenberg2013}. These forces are then mediated between cells via cell-cell junctions forming an adhesive ring around a cell. These junctions are dynamic and can be remodelled, which can lead to seemingly fluid behavior when external stress is applied for a prolonged time. On short timescales, however, cells exhibit an elastic response, assuming their previous shape if an intermittent external force is no longer present. Tissues need to grow and often change shape during development. This can driven by divisions, cell death, cell extrusion, or intercalation~\cite{Guillot2013}.
An example of a well-studied tissue shape change can be found during mesoderm invagination in \textit{Drosophila melanogaster}. In this case cells that initially form a flat layer become a long furrow by constricting their cell membrane area on one side~\cite{Lecuit2007}. 
}{
Les réagencements cellulaires sont souvent conduits par des forces physiques intracellulaires~\cite{Heisenberg2013}, qui sont ensuite transmises entre cellules, par des jonctions intercellulaires modulables. Ce phénomène peut conduire à un comportement quasi-fluidique lorsqu'un stress extérieur est appliqué pour une certaine durée. A de plus petites échelles temporelles, les cellules gardent cependant un comportement élastique et gardent leur forme lorsque aucune force extérieure n'est appliquée. Pour que le tissu change de forme, ont lieu des divisions, morts, extrusions ou intercalages de cellules~\cite{Guillot2013}. Un exemple de dynamique de tissu bien étudiée est présent chez \textit{Drosophila melanogaster} également. Dans ce cas, des cellules formant initialement une couche plate deviennent un long sillon en contractant la membrane cellulaire d'un coté~\cite{Lecuit2007}.
}

\comment[CC]{c'est la ou je me dis que n'importe quel geographe te dira: "hors sujet". Soit tu limites le nombre d'exemples dans les autres disciplines, soit il faut donner un exemple, un parallele, une application de ces mecanismes dans le champs du transport/territoire pour montrer l'interet de passer tous les mecanismes en revue.}

% supplementary example of C. Elegans not necessary

%\paragraph{Embryonic development of \textit{C. elegans} as an example of morphogenesis}{}

%The emergence of the \textit{C. elegans} body shape is a remarkably well-regulated process, giving rise to a completely determined cell lineage, reproducing animals with a layout that is identical down to the single cell level~\cite{Sulston1983}. This fully determined development (which results in great experimental reproducibility) makes it an excellent model to investigate how molecular characteristics can give rise to complex animal development. Prior to fertilization, any part of the oocyte could later become head or tail. After sperm entry however, the future anterior-posterior (front-rear) axis will be determined by the (somewhat random) localization of sperm-donated centrosomes. However, this polarity cue needs robust amplification to allow reproducible development. Thus this region serves as a seed to establish cell polarity via mechano-chemical coupling~\cite{Goehring2013} (also see previous section on reaction-diffusion mechanisms). Membrane proteins (chemical) are pulled to one side of the cell by cortical flows (mechanical). These cortical flows emanate at the membrane locus closest to the centrosome - which serves thus as a trigger for more global symmetry breaking~\cite{Rose2014}. The difference in membrane proteins between the front and rear side of the cell is then translated into a cytoplasmic difference in protein content between the two sides of the cell. Consequently, the subsequent division is asymmetric, producing two daughter cells with different protein content. This achieves full specification of the future head-tail axis by the time the embryo reaches the two cell stage. At this point it is still rotationally symmetric around the anterior-posterior axis. Then, between the two and four cell stage, symmetry is broken again: the dorsal/ventral axis is defined by another asymmetric cell division. Similarly, left-right symmetry is broken between the four and six cell stage. At this point in development, the future body axes have been set up, it is now clear which sides will become head and tail, left and right as well as the dorsal and ventral part of the adult worm. The known mechanisms for this setting up of the body axes have been reviewed in~\cite{Rose2014}. The timing of these symmetry breaking events are in contrast to many other animals including vertebrates, who break symmetry much later (during gastrulation). 
%



\subsubsection*{Artificial Life}{Artificial Life}


\bpar{
As reviewed in~\cite{crosato2014self}, the notion of \emph{Programmable Self-Assembly} seems for students of Artificial Life to be very close to the biological concept of morphogenesis : ``The greater example of Programmable Self-Assembly in nature is probably the cell organisation in multicellular organisms, which is encoded by the DNA.'' An important approach is Doursat's concept of Morphogenetic Engineering, that focuses on designing complex systems from the bottom-up. A review of the field is done in~\cite{doursat2013review} : an essential distinction between self-organization and morphogenesis that it introduced is the presence of an architecture. An example of a heterogeneous swarm of particles, yielding complex architectures is described in~\cite{doursat2008programmable}. The processes of local interactions (corresponding in biology to local physical forces) and positional information through gradient propagation are both integrated in the swarm model and allow  bottom-up assembly of complex patterns. The combination of a chemical reaction layer with a hydrodynamic layer also provides an interesting model of morphogenesis in~\cite{cussat2012synthesis}.
}{
La notion de \emph{Programmable Self-Assembly} semble être en \emph{Artificial Life}\footnote{au sens du domaine scientifique propre, animé par exemple par la \emph{International Society for Artificial Life} (voir \texttt{http://alife.org/}).} très proche du concept biologique de morphogenèse : \cite{crosato2014self} note dans une large revue que ``le meilleur exemple de \emph{Programmable Self-Assembly} dans la nature est probablement l'organisation des cellules en organismes multi-cellulaires, qui est encodée par l'ADN''. Une approche importante dans ce champ est le concept de \emph{Morphogenetic Engineering} introduit par \noun{Doursat}, qui se concentre sur la conception de systèmes complexes par le bas. Une revue du champ est faite dans~\cite{doursat2013review}. Une distinction essentielle entre auto-organisation et morphogenèse qui y est introduite est la présence d'une \emph{architecture}, au sens d'une structure macroscopique bien discernable ayant des propriétés fonctionnelles (mais que nous ne considérerons pas nécessairement téléonomique~\cite{monod1970hasard} pour garder un certain niveau de généralité). Un exemple d'une nuée hétérogène de particules, produisant des architectures complexes, est décrit dans~\cite{doursat2008programmable}. Les processus d'interactions locales (correspondant en biologie aux forces physiques locales) et l'information de position par la propagation du gradient sont tous deux intégrés dans le modèle et permettent l'émergence par le bas de motifs complexes. \cite{sayama2009swarm} développe des modèles similaires en y ajoutant la possibilité d'évolution des espèces de particules, dirigée de manière interactive par le modélisateur, ce qui permet de effectivement orienter l'architecture émergente. Dans quelle mesure ces systèmes artificiels sont proches de systèmes vivants est une question ouverte : \cite{Schmickl_2016} exhibent des règles de mouvement similaires qui conduisent à l'émergence de structures aux propriétés de reproduction, avec différentes fonctions dans un écosystème propre, qu'ils qualifient de ``imitant la vie''\footnote{Nous développerons plus en détails plus loin les concepts nécessaires pour creuser cette affirmation.}. L'ajout d'un environnement avec ses propriétés propres influence fortement les dynamiques morphogénétiques, comme montré par \cite{cussat2012synthesis} qui combine une couche de réaction chimique avec une couche hydrodynamique dans laquelle cette première prend place. L'application de ces méthodes à des questions concrètes d'ingénierie commence à être développée avec des résultats prometteurs : \cite{Aage:2017aa} utilise un modèle de morphogenèse pour la conception de la structure interne d'une aile d'avion et obtient des gains de masse allant jusqu'à 5\%, et une structure finale très proche de formes aux fonctions similaires dans la nature comme une aile d'oiseau. Dans ce dernier cas, le bio-mimétisme est émergent, les processus de morphogenèse faisant le lien.
}



\subsubsection*{Territorial Sciences}{Sciences Territoriales}

\bpar{
The concept is used in various disciplines dealing with territories and the built environment: geography, urban planning and design, urbanism, architecture. There does not seem to be a unified view nor theory within these fields, not even within each field itself.
}{
Le concept est utilisé dans de nombreuses disciplines s'intéressant aux territoires et à l'environnement bâti: géographie, planification et design urbains, urbanisme, architecture. Il ne semble pas exister de vue unifiée ni de théorie entre les champs ni dans chaque champ lui-même.
}

%%%%%%%%%%%%%%%%%%%%%%%
\paragraph{Built Environment}{Environnement bâti}

\bpar{
Architecture and Urbanism are disciplines studying human settlements and the built environment at relatively small scales. \noun{Olsen}'s theory of Urban Metabolism \cite{olsen1982urban} links city morphogenesis with urban metabolism and urban ecology. The city is seen as a living organism with different time scales of evolution (the life cycles). The study of Urban Morphology~\cite{moudon1997urban}, which focuses on morphogenetic processes, is presented as an emerging field in itself, across geography, architecture and urban planning: this view emphasizes the crucial role of the form in these kind of processes. \cite{burke1972dublin} studies the growth of a particular city during a given period of time, and attributes the evolution of urban morphology to \emph{morphogenetic agents}, i.e. people and developers. At another scale, in architecture, a building can be seen as the results of micro-processes making sense and a particular architectural style interpreted through the use of generative shape grammars~\cite{ceccarini2001essai}. This methodology is not far from the work of \noun{C. Alexander}, an architect who produced a theory of design process \cite{mehaffy2007notes}, inspired from computer science and biology and linked in some aspects to complexity. The notion of morphogenesis is in that case however quite loose, as referring to the process of form generation in general, such as~\cite{whitehand1999urban} that studies concrete changes in house forms as witnesses of urban morphogenesis. \noun{Dollens} refers to autopoiesis~\cite{dollens2014alan}, implying a particular case of morphogenesis, to advocate Turing's influence on current design thinking, and to propose a more organic approach to architecture.
}{
L'architecture et l'urbanisme sont des disciplines étudiant les établissements humains et l'environnement bâti à des échelles généralement grandes\footnote{nous n'incluons pas l'aménagement du territoire, mais considérons bien le contexte de projets urbains qui ne dépassent jamais l'échelle métropolitaine}. La théorie du Métabolisme Urbain de \noun{Olsen}~\cite{olsen1982urban} relie la morphogenèse de la ville à son métabolisme et à l'écologie urbaine. La ville est vue comme une organisme vivant avec différentes échelles de temps d'évolution (les cycles de vie). L'étude de la Morphologie Urbaine~\cite{moudon1997urban}, qui s'intéresse aux processus morphogénétiques, est présenté comme un champ émergent en lui-même, à l'interface de la géographie, l'architecture et la planification urbaine : cette vision appuie sur le rôle crucial de la forme dans ce genre de processus. \cite{burke1972dublin} étudie la croissance d'une ville particulière (Dublin) durant une période temporelle donnée, et attribue l'évolution de la morphologie urbaine aux \emph{agents morphogénétiques}, i.e. les habitants et les promoteurs. A une autre échelle, en architecture, un bâtiment peut être vu comme le résultat de processus microscopiques ayant un sens propre, et un style architectural particulier peut être interprété par l'utilisation d'une grammaire générative de formes~\cite{ceccarini2001essai}. Cette méthodologie se rapproche du travail de \noun{C. Alexander}, un architecte ayant produit une théorie des processus de design~\cite{mehaffy2007notes}, inspirée de l'informatique et de la biologie et liée par certains aspects à la complexité. La notion de morphogenèse est dans ce cas cependant assez floue, puisqu'elle se rapporte au processus de la génération de forme en général, de la même façon que~\cite{whitehand1999urban} étudie les changements concrets dans la forme des maisons comme un témoin de la morphogenèse urbaine, montrant par exemple que les quartiers de plus forte densité étaient plus susceptibles à la contagion des adaptations mineures par les habitants. \noun{Dollens} fait référence à l'autopoièse~\cite{dollens2014alan}\comment[CC]{dire ce que c'est en 2 mots avant de le relier a d'autres concepts}, impliquant un cas particulier de morphogenèse, pour défendre l'influence de \noun{Turing} sur la pensée contemporaine en design, et pour proposer une approche plus organique de l'architecture. \cite{desmarais1992premisses} soutient que les structures humaines sont porteuses d'une morphologie abstraite, et que celle-ci est générée par des processus porteurs de sens, rejoignant la conception de~\cite{ceccarini2001essai}. Cela fait echo aux usages de la morphogenèse en psychologie comme nous verrons plus loin : l'élaboration de la forme concrète va alors de pair avec le processus cognitif qui est lui-même une morphogenèse. \cite{levy2005formes} soulève la difficulté d'une définition propre du terme de forme urbaine, et propose de le revisiter en liant la production de la forme à celle du sens dans l'ensemble de la dynamique du système. Ce positionnement rejoint partiellement celui que nous prendrons plus loin pour définir la morphogenèse.
}





%%%%%%%%%%%%%%%%%%%%%%%
\paragraph{Modeling}{Modélisation}


\bpar{
The Urban growth modeling literature often refers to the growth process as morphogenesis when the scale implied allows to exhibit shape patterns. An example of the emergence of qualitatively different urban functions, based on the Alonso-Muth model is proposed in~\cite{bonin2012modele}. \cite{makse1998modeling} studies a model of urban growth involving the local urban form. In this case the local spatial correlations induce urban structure when the cities gain new inhabitants. More heterogeneous models imply a coupling between city components and transportation networks. \cite{achibet2014model} describe a model of co-evolution between road network and urban blocks structure. At a larger scale and involving more abstract functions, \cite{raimbault2014hybrid} couples city growth with network growth, including local feedback of the form through a density constraint and global feedback of position through network centrality and accessibility to amenities. These two mechanisms are analogous to the local interaction and global information diffusion flow in biology.
}{
La littérature de modélisation de la croissance urbaine se réfère souvent au processus de croissance comme morphogenèse quand l'échelle impliquée permet de révéler des motifs de forme. Un exemple de l'émergence de fonctions urbaines qualitativement différenciées, basé sur le modèle d'Alonso-Muth, est proposé dans~\cite{bonin2012modele}. \cite{makse1998modeling} étudie un modèle de croissance urbaine impliquant la forme urbaine locale. Dans ce cas les corrélations spatiales locales induisent la structure urbaine quand les villes gagnent de nouveaux habitants. Des modèles plus hétérogènes impliquent un couplage entre les composantes urbaines et les réseaux de transport. \cite{achibet2014model} décrit un modèle de co-évolution entre réseau de rues et la structure des blocs urbains. A une plus petite échelle et impliquant des fonctions plus abstraites, \cite{raimbault2014hybrid} couple croissance urbaine et croissance de réseau, incluant une rétroaction locale de la forme par une contrainte de densité et une rétroaction globale de la position par la centralité de réseau et l'accessibilité aux aménités. Ces deux mécanismes sont analogues aux interactions locales et à la diffusion du flux d'information global en biologie. 
}

\comment{\cite{bonin2014modelisation} precedent modele revisité : parallele precis avec les substances morphogenes : reaction diffusion.}

\comment{le simpopnano comme un modele de morphogenese \cite{louail2009geometrie}}



%%%%%%%%%%%%%%%%%%%%%%%
\paragraph{Archeology}{Archéologie}


\bpar{
The morphogenesis of past human settlements viewed from Thom's Catastrophe Theory point of view, is introduced by~\cite{renfrew1978trajectory}. Sudden changes (qualitative changes, or regime shifts) have occurred at any time and can be viewed as bifurcations during the morphogenesis process. Another simplified way to see this is to interpret the transition as a change of meta-parameters of a stationary dynamic.
}{
La morphogenèse des établissements humains du passé, vue du point de vue de la Théorie des Catastrophes de \noun{Thom}, est introduite dans~\cite{renfrew1978trajectory}. Des changement soudains (changement qualitatifs, ou changements de régime) se sont produits à toute époque et peuvent être vus comme des bifurcations durant le processus de morphogenèse. Une autre manière simplifiée de le comprendre est d'interpréter la transition comme un changement des meta-paramètres d'une dynamique stationnaire.
}





\subsubsection*{Social Science and Psychology}{Sciences Sociales et Psychologie}

\bpar{
Morphogenesis has been occasionally used as a suitable metaphor to understand different processes in social science and various psychological fields. For example, in developmental psychology one can think of the relation to evolution of human cultural behavior and learning, epigenetic neural systems, and their influence on neural development and behavior throughout life~\cite{hart_held_2013}.
In Clinical Psychology and Psychopathology, analogies are used for the emergence of psychical structures and the self-organization of forms of relation with the self and the Other. Additionally, “psychological morphogenesis” is akin to the outcome of the complexity of psychological dynamics undergoing creative emergence. Therefore, in “successful” psychotherapy this generation of novelty would be fostered~\cite{piers_self-organizing_2007}.
Moreover, in the field of neuroscience there are a plethora of morphogenetic phenomena related to the structure of the brain, dendritic morphogenesis and neural nets being some remarkable examples~\cite{_issues_2013}. In social psychology we have noteworthy illustrations like the morphogenetic approach proposed by Margaret Archer as applied to the problem of structure and agency, that is, how we both shape society and are shaped by it in a dynamic interplay~\cite{archer_margaret_1999}. 
%Thus the morphogenetic approach offers a new understanding of social change and of the subject within it. Furthermore, its application to the field of psychoanalysis has been evoked as early as 1918 to understand the formation of psychical structures and their dynamics, the pervasive repetition of early development, and the symptom’s self-organization, or the relation to the morphogenetic qualities of drive theory \cite{benedek_instinct_1973}. After an extensive review of the available bibliography contained in the database Psychoanalytic Electronic Publishing, it emerges that morphogenesis has come to be mainly used after the sixties, and moreover thanks to the spreading of the ideas of René Thom on Structural Stability and Morphogenesis \cite{de_luca_picione_processes_2016}, undoubtedly thanks to Lacanian discourse and its movement towards topology \cite{nasio_five_1998}.  
Nonetheless, more than a systematic and widespread unity throughout these different fields, we encounter multiple uses that are sometimes discontinuous, and one could argue that the utility of morphogenesis could be more tangible on an epistemological level. This would consist of a shared perception of morphogenesis’s descriptive power to further understand the emergence and structure of various phenomena.
}{
La morphogenèse a été occasionnellement utilisée comme une métaphore efficace pour comprendre différents processus en sciences sociales et dans divers champs de la psychologie. En psychologie du développement par exemple, l'influence des processus d'apprentissage culturel sur le comportement sont une bonne illustration~\cite{hart_held_2013}. Pour la psychologie clinique, des analogies sont utilisées pour l'auto-organisation des relations avec le Moi et l'Autre, ainsi que pour les dynamiques impliquant l'émergence créative, qui doit être encouragée pour une psychothérapie ``aboutie''~\cite{piers_self-organizing_2007}. D'autres part, en neurosciences, la structure du cerveau en elle-même et la mise en place des réseau de neurones est typiquement l'issue de processus morphogénétiques~\cite{_issues_2013}. En psychologie sociale, la co-évolution de l'individu et de la société peut également être vu par ce prisme~\cite{archer_margaret_1999}. La théorie de \noun{René Thom} que nous détaillerons plus loin a certainement joué un rôle dans l'utilisation de ce concept en psychologie~\cite{de_luca_picione_processes_2016}. Toutefois, au delà d'une unité systématique au travers de ces différents champs, les usages sont plutôt discontinus, et on pourrait supposer que l'utilité du concept de morphogenèse réside plutôt dans sa portée épistémologique. Celle-ci consisterait dans une perception partagée du pouvoir descriptif de la morphogenèse pour mieux comprendre l'émergence de la structure des divers phénomènes.
}

\paragraph{A Sidenote on Autopoiesis}{Autopoièse}

\bpar{
It is interesting to note that Varela and Maturana’s theory of autopoiesis in biology, from which they develop an observer-dependent interpretation of cognition, language, and consciousness, had a constructive epistemological impact on social science, philosophy and psychology, even if sometimes latent. For example, an application in sociology can be found in Niklas Luhmann's Systems Theory. His generalized view of autopoiesis conceptualizes systems as self-producing, not in terms of their physical components, but in terms of their organization, which can be measured in terms of information and complexity\cite{gershenson_requisite_2014}. 
These views provide insight on the interpenetration between social and psychical systems. In Luhmann's theory, the 'human being' is not conceptualized as forming a systemic unity, but instead is understood as a conglomerate of organic and psychical systems, with language being the most important evolutionary achievement for the coupling of social and psychical systems. Language is thus a social phenomenon, yet thought processes are structured in a complementary way to language, as thoughts are broken down into chunks of sentences and words. \cite{seidl_luhmanns_2004} We could further assess the epistemological significance of this if we consider the conception of the subject as dynamic and recursive, thus in a movement that can interact with its environment. This stance stems away from classically static conceptions of the human psyche, and echoes some contemporary clinical approaches in psychology and psychoanalysis. One concept that clearly illustrates this is Pichon Riviere’s  notion of ECRO (Schema Conceptual Referential and Operative), as the working processes which constitute the tools from which the subjects mental operations flow\cite{pichon_riviere_processus_2004}.
% Thus, autopoiesis could be viewed as a necessary but insufficient condition for cognition\cite{bitbol2004autopoiesis}. % explained further
  Moreover, the interpenetration of the psychological and the social and the importance of language points us in the direction of psychoanalytical theory and clinical practice, with Jacques Lacan’s views on linguistics and the big Other as well as Sigmund Freud’s psychoanalytic anthropology that emphasizes the links between the neurotic patient’s symptom and sociocultural phenomena \cite{freud_totem_1989}.
%Finally, the philosopher and lacanian psychoanalyst Slavoj Žižek, in his discussion of Hegel argues that: "Hegel is – to use today's terms – the ultimate thinker of autopoiesis, of the process of the emergence of necessary features out of chaotic contingency, the thinker of contingency's gradual self-organization, of the gradual rise of order out of chaos."\cite{zizek_less_2013} % do not understand position - skip
}{
La notion d'autopoièse provenant de la biologie, que nous détaillerons plus loin, fournit une interprétation dépendante de l'observateur de la cognition et de la conscience. Celle-ci a eu des impacts en psychologie et sociologie, comme certaines théories des systèmes~\cite{gershenson_requisite_2014}. Les systèmes sociaux et psychiques sont alors compris comme des systèmes fortement couplés, comme le témoigne le language qui est un phénomène social profondément ancré dans les manifestations cognitives~\cite{seidl_luhmanns_2004}. Ces approches rejoignent également les visions du sujet comme dynamique et récursif~\cite{pichon_riviere_processus_2004}. L'interpénétration du social et du psychologique trouvent echo chez l'anthropologie psychanalytique de \noun{Freud} qui appuie les relations entre les symptômes neurotiques et les phénomènes socio-culturels~\cite{freud_totem_1989}. \comment[FL]{cf abercrombie : ?}
}

\subsubsection{History of the notion}{Histoire de la notion}

\bpar{
The study of morphogenesis started with embryology between just before 1930's. This is about the same time as Hodgkin and Lister, reported seeing red blood cells under a microscope, and less than 10 years before Dujardin's discovery of cellular movement in Amoeba. \cite{abercrombie1977concepts} Using google book, the first use of the word morphogenesis in a book is in 1871, saw a large peak in usage between 1907-1909, and continued to increase in usage until the 1990's before slowing decreasing in usage.
}{
L'étude de la morphogenèse a démarré avec l'embryologie juste avant les années 30. Il s'agit environ de la même période à laquelle les mouvement cellulaires de bactéries ont été découverts~\cite{abercrombie1977concepts}. Les statistiques issues de Google Books donne le premier usage du mot dans un livre en 1871. L'usage montre ensuite un pic d'utilisation entre 1907 et 1909, pour continuer d'augmenter jusqu'en 1990 avant de décroître progressivement.
}


\subsubsection{Others}{Autres}
\comment[CC]{c'est moche comme nom de section. Et on perd la logique de la structure du chapitre.}



\paragraph{Epistemology}{Epistémologie}

\bpar{
Morphogenesis is also used to study science itself: for example~\cite{gilbert2003morphogenesis} studies the evolution of evolutionary developmental biology through the metaphor of morphogenesis. He sees scientific ideas as interacting agents from which emerge new phenotype through differentiation processes, what is designed as the morphogenesis of the field.
}{
La morphogenèse peut aussi être utilisée pour étudier la science elle-même : par exemple~\cite{gilbert2003morphogenesis} étudie l'évolution de la biologie évolutionnaire du développement par la métaphore de la morphogenèse. Il voit les idées scientifiques comme des agents en interaction, desquels émergent de nouveaux phénotypes par des processus de différentiation, qui sont désignés comme la morphogenèse du champ.
}


\paragraph{A mathematical approach}{Une Approche Mathématique}


\bpar{
Ren{\'e} Thom, in \emph{Structural stability and Morphogenesis}~\cite{thom1974stabilite} has developed a theory of system dynamics, the ``catastrophe theory'', that studies in deep the impact of topological structure of phase space manifolds on a system dynamics. Let $M$ a differentiable manifold, in which system state $(m,\dot{m})$ is embedded. We assume the existence of a closed set $K$, called \emph{Catastrophe set}. The topological type of $K$ is indeed endogenously determined by system dynamics (in simple cases, it refers to the "classical" type of attractors/fixed points usually known: points, limit cycles). When $m$ encounters $K$, the system follows a \emph{qualitative} change in its form, what constitutes the basis of \emph{morphogenesis}. This abstract theory of morphogenesis is independent of the nature of the system studied, its main contribution being to classify local catastrophes that occur during morphogenesis. Differentiation and richness of patterns have thus a geometrical explanation through the topological types of catastrophes. Thom notes that at this time, the study of form has mainly be the focus of biology, but that many applications could be done in physics and geomorphology for example. He formulated the hypothesis that it is because it implies discontinuities and self-organisation, to which mathematicians were repulsive, that it was not applied easily to various fields. We can link this to the rise of complexity approaches, with complexity paradigms that slowly spreaded in various disciplines, and the study of morphogenesis seem to have followed.
}{
Ren{\'e} Thom a développé dans \emph{Stabilité Structurelle et Morphogenèse}~\cite{thom1974stabilite} une théorie de la dynamique des systèmes, la théorie des catastrophes, qui étudie en profondeur l'impact de la structure topologique des variétés de l'espace des phases sur les dynamiques du système. Soit $M$ une variété différentielle, dans laquelle l'état du système $(m,\dot{m})$ est embarqué. On suppose l'existence d'un ensemble fermé $K$ appelé \emph{Ensemble de Catastrophe}. Le type topologique de $K$ est en fait déterminé de manière endogène par la dynamique du système (dans les cas simples, il réfère au types ``classiques'' d'attracteurs/points fixes que l'on connait habituellement : points et cycles limites). Quand $m$ traverse $K$, le système encontre un changement \emph{qualitatif} dans sa forme, ce qui constitue la base de la \emph{morphogenèse}. Cette théorie abstraite de la morphogenèse est indépendante de la nature du système étudié, sa contribution principale étant de classifier les catastrophes locales qui surviennent lors de la morphogenèse. La différentiation et la richesse des motifs ont ainsi une explication géométrique à travers les types topologiques des catastrophes. \noun{Thom} note qu'à cette époque, l'étude de la forme a majoritairement été ciblée par la biologie, mais que de nombreuses applications pourraient être développées en physique et géomorphologie par exemple. Il formule l'hypothèse que parce que cela implique des discontinuités et de l'auto-organisation, à laquelle les mathématiciens étaient réticents, que cela n'a pas été appliqué facilement à divers champs. Nous pouvons lier cela à l'émergence des approches complexes, avec des paradigmes de la complexité qui se sont progressivement répandus dans diverses disciplines, et l'étude de la morphogenèse semble avoir suivi. 
}


%%%%%%%%%
% Quote from Villani on Morphogenesis (thanks Mario !)
% Morphogenesis as a discipline that is not yet very clearly identified, having still lots of mystery, it is in the intersection between mathematics, chemistry and biology. Typically it is in chemical reactions produced in the development of living beings, where some mathematical models play a role to make structures emerge, so it really makes these 3 intervene (math, chem, biol). And Turing identified that when we put together and at the same time a diffusion phenomenon -with a substance that spreads in an organism- and a chemical reaction, we can end up with instabilities. And for example, these instabilities make the leopard's skin, where you can see black spots, instead of black being evenly distributed. And this incredible article by Turing was a struck of genius, to think that by putting together two stable phenomena -because this chemical reaction is a stable phenomena, and the diffusion is a stable phenomena- one could have an instability. Completely counter-intuitive, but mathematics is precisely that which can permit us to go beyond intuition.


\bpar{
}{
Les mathématiques, peu mentionnées dans notre revue, sont toutefois concernées à la fois comme outil mais comme discipline à part entière, les constructions mathématiques obtenues à partir des questions liées à la morphogenèse sont des sujets de recherche à part entière. Comme l'a récemment rappelé \noun{Cedric Villani}~\cite{villani2017chauvesouris}, ``la morphogenèse est une discipline pas très bien identifiée ayant toujours un certain nombre de mystère, à l'intersection entre les mathématiques, la chimie et la biologie, (...) où des modèles mathématiques jouent un rôle pour faire émerger les structures''.
}






\paragraph{Autopoiesis and Morphogenesis}{Autopoièse et Morphogenèse} \comment[CC]{ca fait beaucoup d'allers-retours entre les 2, peut-etre dire ce que c'est des le debut et eviter des repetitions}


\bpar{
The notion of \emph{autopoiesis} expresses the ability for a system to reproduce itself. A basic characterization is a semi-permeable boundary produced within the system and the ability to reproduce its components. A more general definition is proposed by Bourgine and Stewart in~\cite{bourgine2004autopoiesis}: \textit{``An autopoietic system is a network of processes that produces the components that reproduce the network, and that also regulates the boundary conditions necessary for its ongoing existence as a network''}. The notion of dynamical processes is key, and could be linked to Thom's theory of morphogenesis. They furthermore introduce a definition of cognition (trigger actions as function of sensory inputs to ensure viability), and of living organism as autopoietic and cognitive, both notions being distinct~\cite{bitbol_autopoiesis_2004}. In that frame, for example, the arbotron~\cite{jun2005formation} is cognitive but not autopoietic. An example of link between autopoiesis and morphogenesis is shown in~\cite{niizato2010model}, where a type of Physarum organism has to play both on cell mobility and form evolution to be able to collect the food necessary for its survival. At this stage, we can postulate a strict inclusion from autopoietic systems, morphogenetic systems to self-organizing systems.
}{
La notion d'\emph{autopoièse}, déjà mentionnée ci-dessus, exprime la capacité d'un système à s'auto-reproduire. Une caractérisation rudimentaire est l'existence d'une frontière semi-perméable produite par le système et la capacité à reproduire ses composants. Une définition plus générale est proposée par~\cite{bourgine2004autopoiesis} : \textit{``un système autopoiétique est un réseau de processus qui produit les composants permettant de reproduire le réseau, et qui régule également les conditions au bord nécessaire pour son existence continue en tant que réseau''}. La notion de processus dynamique est une notion clé, et pourrait être liée à la théorie de la morphogenèse de \noun{Thom}. Ils introduisent de plus une définition de la cognition (déclenchement d'actions en fonction d'entrées sensorielles pour assurer la viabilité), et d'un organisme vivant comme autopoiétique et cognitif, les deux notions étant bien distinctes~\cite{bitbol_autopoiesis_2004}. Dans ce cadre par exemple, l'arbotron~\cite{jun2005formation} est cognitif mais pas autopoiétique. Un exemple de lien entre autopoièse et morphogenèse est montré dans~\cite{niizato2010model}, où un type d'organisme Physarum doit jouer à la fois sur la mobilité des cellules et sur l'évolution de la forme pour être capable de collecter la nourriture nécessaire à sa survie. A cette étape, nous pouvons déjà postuler une inclusion stricte des systèmes autopoiétiques, aux systèmes morphogénétiques, aux systèmes auto-organisés.
}




\paragraph{Co-evolution}{Co-évolution}

\bpar{
Since morphogenesis can be transposed to ecosystem or societies, and the components of the system are co-evolving in those cases, the existence of co-evolution may be linked with morphogenesis, as an other way of seeing the system. Symbiosis in biology can lead to very strong causalities in organism evolution (co-evolution) : this phenomenon has been designed as \emph{symbiogenesis}. The symbiosis induce an change in morphogenetic patterns of symbiotic organisms as exemplified for different species in~\cite{chapman1998morphogenesis}. Thus the strong link between morphogenesis and co-evolution (here morphogenesis designing more evolutionary paths of morphogenetic patterns, i.e. at a different time scale).
}{
La morphogenèse pouvant être transposée aux ecosystèmes ou aux sociétés, dont les composantes sont en co-évolution dans ce cas, la présence d'une co-évolution pourrait être liée à la morphogenèse, comme une autre façon de voir le système. La symbiose en biologie peut mener à des causalités très fortes dans l'évolution de l'organisme (co-évolution) : ce phénomène a été désigné comme \emph{symbiogenesis}. La symbiose induit un changement dans les motifs morphogénétiques des organismes symbiotiques comme montré pour différentes espèces par~\cite{chapman1998morphogenesis}. D'où un lien potentiellement fort entre morphogenèse et co-évolution : dans ce cas la morphogenèse est utilisée pour désigner plus des trajectoires évolutionaires de motifs morphogénétiques, i.e. sur une échelle de temps différente.
}



\paragraph{System definition and boundaries}{Definition et frontières du système}

\comment[JR]{en lien avec autopoiese et co-evol (signals and boundaries), et remarque de Bonin a TheoQuant : importance de systeme ouvert/ferme - def du systeme}





\subsection{Synthesis}{Synthèse}


\subsubsection{Key notions}{Notions clés}\comment[CC]{TB|}

\bpar{
We list here important concepts that come out from this review, and from which a synthetic vision should emerge. Each may be domain-dependent, and underlying conceptions may vary from one field to the other.
}{
Nous listons à présent les concepts importants découlant de cette revue, et dont une vision synthétique doit émerger. Chacun peut être dépendant du domaine, et les conceptions sous-jacentes peuvent varier d'un champ à l'autre.
}

\bpar{
\begin{itemize}
\item \textbf{Self-organisation} : Morphogenesis implies self-organisation but the contrary is not necessarily true, some aspects are specific of morphogenesis, such as the presence of functions resulting from the form.
\item \textbf{Patterns and shape} : The ``formation of shapes'' seems to be common to all approaches to morphogenesis.
\item \textbf{Embryogenesis / tissue modeling} In biology, typical processes of morphogenesis are generally observed at early stages of life, during empryogenesis, including the initial formation of tissues.
\item \textbf{Apoptosis} Morphogenesis is often related to life (see section on autopoiesis), but also to death : the programmed death of cells, apostosis, can in some cases be a part of morphogenetic processes.
\item \textbf{Qualitative vs Quantitative} Qualitative bifurcations are a fundamental concept in morphogenesis : e.g. differentiation of organs in biology ; emergence of differentiated urban functions
\item \textbf{Symmetry} Symmetry breaking occurs, mostly at early stages, but also at all stages of morphogenesis.
\item \textbf{Unit and Scale} Are systems top-down or bottom-up designed, self-organized or exhibiting architecture ? Both are not necessarily incompatible, fundamental units and scales playing a crucial role in defining morphogenesis. Fractal-like systems, such as corals (collaborating tissues) or cities, but also the self and the society, can be studied from the point of view of morphogenetic processes at different levels.
\item \textbf{Boundaries} Boundaries are a major aspect in Complex Adaptive systems (see e.g. Holland's approach as \emph{Signals and Boundaries}~\cite{holland2012signals}). Morphogenesis can imply clear boundaries (of an embryo e.g.) but not necessarily (social organisms, cities for which the definition of boundaries is still an open question~\cite{2015arXiv150707878C}).
\item \textbf{Relation between Form and Function} Causal relations between form and function are at the center of emerging architecture.
\end{itemize}
}{
\begin{itemize}
\item \textbf{Auto-organisation} : la morphogenèse implique auto-organisation mais le contraire n'est pas nécessairement vrai\comment[FL]{pourquoi penserait-on cela ?}, certains aspects sont spécifiques à la morphogenèse, comme la présence de fonctions résultant de la forme.
\item \textbf{Motifs et Forme} : ``l'émergence de formes'' semble être commun à toutes les approches de la morphogenèse.
\item \textbf{Embryogenèse / modélisation des tissus} en biologie, les processus typiques de la morphogenèse sont généralement observés au stades initiaux de la vie, durant l'embryogenèse, incluant la formation initiale des tissus.
\item \textbf{Apostosis} la morphogenèse est souvent liée à la vie (voir la section sur l'autopoièse), mais aussi à la mort : la mort programmée de cellules, l'apostosis, peut dans certains cas faire partie de processus morphogénétiques.
\item \textbf{Qualitatif vs Quantitatif} Les bifurcations qualitatives sont un concept fondamental pour la morphogenèse : e.g. la différentiation des organes en biologie ; l'émergence de fonctions urbaines différenciées.
\item \textbf{Symétrie} Des ruptures de symétrie se produisent, majoritairement dans les étapes initiales, mais aussi à tous les stades de la morphogenèse.
\item \textbf{Unité et Echelle} : les systèmes sont-ils conçus par le haut ou par le bas, auto-organisés, ou présentant une architecture ? Les deux ne sont pas nécessairement incompatibles, les unités fondamentales et les échelles jouant un rôle crucial dans la définition de la morphogenèse. Les systèmes semblables à des fractales, comme les coraux (tissus collaboratifs) ou les villes, mais aussi le sujet et la société peuvent être étudiés du point de vue des processus morphogénétiques à différents niveaux.
\item \textbf{Frontières} : les frontières sont un aspect crucial pour l'étude des Systèmes Complexes Adaptatifs (voir par exemple l'approche de \noun{Holland} par \emph{Signals and Boundaries}~\cite{holland2012signals}). La morphogenèse peut impliquer des frontières claires (d'un embryon e.g.) mais pas nécessairement (organismes sociaux, villes pour lesquelles la définition des frontières est toujours une question ouverte~\cite{2015arXiv150707878C}).
\item \textbf{Relation entre forme et fonction} : les relations causales entre forme et fonction sont au centre de l'architecture émergente.
\end{itemize}
}



%%%%%%%%%%%%%%%%%%%%%%%
\subsubsection{Common processes and differences}{Processus communs et divergences}\comment[CC]{TB}

%%%%%%%%%%%%%%%%%%%%%%%
\paragraph{From local interactions to global information flow}{Des interactions locales aux flux globaux d'information}


\bpar{
The interplay between agent-to-agent interactions, either through neighborhood effects such as mechanistic interactions and diffusion, or through network interactions such as signaling, and the feedback of a global information flow (i.e. a downward causation of the upper level) appears to be common to most use of morphogenesis. It highlights the fundamental multi-level nature of morphogenetic processes and the central role of emergence.
}{
Les imbrications des relations entre agents, soit par des effets de voisinage comme des interactions mécaniques et la diffusion, ou par des interactions de réseaux comme le signalement, et la retroaction d'un flux d'information global (i.e. une causation descendante du niveau supérieur) apparaît être commun à la majorité des utilisations de la morphogenèse. Cela souligne la nature fondamentalement multi-niveaux des processus morphogénétiques et le rôle central de l'émergence.
}


%%%%%%%%%%%%%%%%%%%%%%%
\paragraph{From self-organization to morphogenesis : the notion of architecture}{De l'auto-organisation à la morphogenèse : la notion d'architecture}

\bpar{
Most system studied seem to have the particularity to exhibit an architecture, what would make the distinction between self-organization and morphogenesis. This idea comes from the field of morphogenetic engineering (which can be seen as a subfield of artificial intelligence). This point may be a divergence point on some fields, as for example in physical science, where the ``morphogenesis'' of terrain patterns is a self-organization in our sense. The notion of architecture may be tricky to define. A way to do it is to consider the functions of macro-levels in the system : the emergence of function at an upper level implies an architecture, which is \emph{the link between the form and the function}. Here this last concept takes all its sense and importance in regard to morphogenesis.
}{
La plupart des systèmes étudiés semblent avoir la particularité de présenter une architecture, ce qui permettrait de faire la distinction entre auto-organisation et morphogenèse. Cette idée vient du champ du \emph{morphogenetic engineering}, qui peut être vu comme un sous-champ de l'intelligence artificielle. Ce point peut être une divergence pour certains champs, comme par exemple en géographie physique où la ``morphogenèse'' de motifs d'érosion est une auto-organisation en notre sens. La notion d'architecture peut être difficile à définir. Une façon d'y parvenir est de considérer les fonctions des niveaux macroscopiques du système : l'émergence d'une fonction à un niveau supérieur implique une architecture, qui est \emph{le lien entre la forme et la fonction}. Ici ce dernier concept prend tout son sens et son importance au regard de la morphogenèse.
}

%%%%%%%%%%%%%%%%%%%%%%%
\subsubsection{Proposition of a Meta-epistemological Framework}{Proposition d'un cadre meta-epistémologique}

\paragraph{Framework}{Cadre}


\bpar{
We propose a hierarchical organisation of concepts, that can be seen as a meta-epistemological framework, since definitions are built from synthesis of the many disciplines evoked here, and that their application in each particular discipline yields an epistemological frame. The concepts are organized the following way :

\begin{equation}
\textrm{Self-organization} \supsetneq \textrm{Morphogenesis} \supsetneq \textrm{Autopoiesis} \supsetneq \textrm{Life} 
\end{equation}

each having a generic definition, elaborated from the synthesis of disciplines.
}{
Nous proposons une imbrication hiérarchique des concepts, qui peut être vue comme un cadre meta-épistémologique, puisque les définitions sont construites de la synthèse des diverses disciplines évoquées ici, et que leur application dans chaque discipline particulière fournit un cadre épistémologique. Les concepts sont organisés de la façon suivante:

\begin{equation}
\textrm{Auto-organisation} \supsetneq \textrm{Morphogenèse} \supsetneq \textrm{Autopoïèse} \supsetneq \textrm{Vie}
\end{equation}

chacun ayant une définition générique, élaborée de la synthèse des disciplines. L'inclusion stricte signifie qu'un concept implique l'autre mais qu'ils sont différents. L'ensemble des concepts est nécessaire pour bien situer la morphogenèse.
}


\bpar{
\textbf{Definition : \textit{Self-organization}.} A system is self-organized if it exhibits weak emergence~\cite{bedau2002downward}.
}{
\textbf{Definition : \textit{Auto-organisation}.} Un système est dit auto-organisé s'il exhibe une émergence faible~\cite{bedau2002downward}.
}

\bpar{
\textbf{Definition : \textit{Morphogenesis}.} A self-organized system is the result of morphogenetic processes if it exhibits an emergent architecture, in the sens of causal relations between form and function at different levels.
}{
\textbf{Définition : \textit{Morphogenèse}.} Un système auto-organisé est le produit de processus morphogénétiques s'il présente une architecture émergente, au sens de relations causales circulaires entre forme et fonction à différents niveaux.

La \emph{forme} est comprise comme \emph{propriétés topologiques ou géométriques} d'un système ou de l'une de ses parties, tandis que la \emph{fonction} est son rôle au sein des chaînes de processus, dans une perspective \emph{téléonomique}\footnote{Au sens donné par \noun{Monod} dans~\cite{monod1970hasard}, c'est à dire participant à répondre à un projet, à un but donné. Les êtres vivants sont téléonomiques au sens que l'ensemble de leur fonctions visent à finalement reproduire leur ADN. Une vision non \emph{téléologique} de l'univers postule que celui-ci n'a pas de projet, et que la plupart des objets physiques ne rentrent pas dans cette catégorie. L'ensemble des autres cas d'étude que nous avons revu dans notre construction sont téléonomiques à différents niveaux : les systèmes territoriaux sont aménagés selon des logiques d'acteurs qui répondent à des projets; les systèmes de robots en \emph{morphogenetic engineering} répondent à un besoin; les idées ou pensées participent à l'écosystème de l'esprit. Nous postulons ainsi cette nécessité téléonomique de la fonction pour avoir morphogenèse, position qui peut être discutée, comme en géomorphologie le réseau de rivières sera supposé avoir la fonction de drainer l'eau de pluie. Dans tous les cas une dichotomie claire entre morphogenèse en notre sens et auto-organisation ne pourra être distinctement établie, et un continuum correspond plus sûrement à la réalité (de la même manière que \noun{Bedau} imagine un continuum entre émergence faible et émergence forte). En effet, dans une vision perspectiviste (voir~\ref{sec:epistemology}), l'observateur joue un rôle essentiel dans la définition d'une fonction : le Jeu de la Vie utilisé comme ordinateur (par ses propriétés de Turing-complétude) sera morphogénétique, tandis qu'il sera auto-organisé s'il est simulé sans raison, rejoignant l'absurdité de la définition d'un \emph{objet} sans \emph{sujet} soulevée par \noun{Morin} dans~\cite{morin1976methode}.}.
}

\bpar{
\textbf{Definition : \textit{Autopoiesis and Life}.} We take the definition of Bourgine \cite{bourgine2004autopoiesis} for autopoiesis, that extends Bitbol's~\cite{bitbol_autopoiesis_2004}, who also define life as autopoiesis with cognition.
}{
\textbf{Définition : \textit{Autopoièse et Vie}.} Nous prenons la définition de \noun{Bourgine} pour l'autopoièse~\cite{bourgine2004autopoiesis}, qui étend celle de \noun{Bitbol}~\cite{bitbol_autopoiesis_2004}, qui définit également la vie comme autopoièse avec cognition.
}

\bpar{
The boundary between self-organization and morphogenesis is the existence of causal links between form and function, which can be defined as \emph{architecture}~\cite{doursat2013review}, generally emergent from the bottom-up. We observe that the complexity of systems increase with notion depth, what can be loosely translated in the fact that :
\begin{itemize}
\item Emergence strength~\cite{bedau2002downward} diminishes with depth, in the sense that the number of autonomous scales increases.
\item Number of bifurcations increases~\cite{thom1974stabilite}, i.e. path-dependancy increases.
\end{itemize}
}{
La frontière entre auto-organisation et morphogenèse est l'existence de liens causaux entre forme et fonction, qui peut être définie comme une \emph{architecture}~\cite{doursat2013review}, généralement émergente de manière \emph{bottom-up}\comment[CC]{repetition}. Nous observons que la complexité du système augmente avec la profondeur de la notion, ce qui peut être traduit de façon simplifiée par :
\begin{itemize}
\item La force de l'émergence~\cite{bedau2002downward} diminue avec la profondeur, au sens que le nombre d'échelles autonomes, ainsi que le nombre de propriétés aux pouvoir causaux irréductibles, augmentent.
\item Le nombre de bifurcations augmente~\cite{thom1974stabilite}, i.e. la dépendance au chemin augmente.
\end{itemize}

Ce deux propriétés peuvent être interprétées comme \emph{l'une des} caractérisations de la complexité (voir~\ref{sec:epistemology}).
}



\paragraph{Application}{Application}

\bpar{
An ontological specification~\cite{livet2010ontology}, i.e. the definition of entities to which the notion apply, yields an application to a particular field, each one developing its own properties and level of inclusion between concepts. There is a priori no reason for a direct correspondence or equivalence of projected concepts, thus transfer of knowledge between fields may be subject to caution.
}{
Une spécification ontologique~\cite{livet2010ontology}, i.e. la définition des entités à laquelle la notion s'applique, fournit une application à un champ donné, chaque champ développant ses propres propriétés et niveaux d'inclusion entre les concepts. Il n'existe a priori pas de raison pour une correspondance directe ou une équivalence entre les concepts projetés, ainsi le transfert de connaissances entre les domaines doit rester sujet à caution.
}





\subsection{Discussion}{Discussion}



\paragraph{Towards a more systematic construction}{Vers une construction systématique}

\bpar{
Our work relies for now on a broad but not \emph{systematic} review, in the sense of the methodology used for example in therapeutic evaluation, and where they play a role as important as primary studies, new knowledge being created through systematic comparison of results and meta-analysis. It would imply in our case an iterative approach :
}{
Ce travail repose pour l'instant sur une revue large mais non \emph{systématique}, au sens de la méthodologie utilisée en évaluation thérapeutique par exemple, et où elle joue un rôle aussi important que les études primaires, une nouvelle connaissance étant créée par la comparaison systématique des résultats et la meta-analyse. Cela impliquerait dans notre cas une approche itérative, en utilisant de manière couplée les différents outils et méthodes développés en~\ref{sec:quantepistemo} :
}

\bpar{
\begin{itemize}
\item Blind systematic review, without any a priori on the fields concerned and on the way to express the notion.
\item Extraction of main fields ; extraction of synonyms and close notions (such as we did here with autopoiesis and self-assembly for example ; if needed iteration of the first general review.
\item Systematic reviews specific to each field, as each one has its own bibliographical databases, specific ways of communication, etc.
\item Confrontation of each notion from one field to other fields.
\end{itemize}
}{
\begin{itemize}
\item Une revue systématique aveugle, sans aucun a priori des champs concernés ou des moyens d'exprimer la notion.
\item Extraction des champs principaux ; extraction des synonymes et notions proches (comme il a été fait ici avec l'autopoièse et la \emph{self-assembly} par exemple) ; si besoin itération de la première revue générale.
\item Revue systématique spécifique à chaque champ, puisque chaque a ses propres bases bibliographiques, moyens spécifiques de communiquer, etc.
\item Confrontation de chaque notion depuis un champ vers les autres
\end{itemize}
}
 
\comment[CC]{peut-etre hors sujet dans le contexte de la these}

\paragraph{Quantitative Epistemology}{Epistemologie quantitative}

\bpar{
Our position may be also strengthen by quantitative approaches to literature analysis. With text-mining, keywords and concept extraction from abstracts (or even full texts) is possible, and would allow to confront our qualitative analysis to empirical data, by answering questions such as: is a concept indeed central, or what concept is used the same way in most disciplines. \cite{chavalarias2013phylomemetic} for example reconstructs scientific fields from the bottom-up through text-mining, and studies their lineage and dynamics in time. An other approach would be an iterative extraction of concept, by an algorithmic systematic review such as the one done in~\cite{raimbault2015models}.
}{
Notre position peut également être renforcée par des approches quantitatives à l'analyse de la littérature. Avec la fouille de texte, l'extraction de mots-clés et de concepts à partir des résumés (ou même des textes complets) est possible, et devrait permettre de confronter notre analyse qualitative à la réalité empirique, en répondant à des questions telles que : un concept est-il central, ou quel concept est utilisé de la même façon dans la plupart des disciplines. \cite{chavalarias2013phylomemetic} par exemple reconstruisent des champs scientifiques par le bas par une analyse textuelle, et étudie leur lignée et dynamique dans le temps. Une autre approche peut être la construction itérative des concepts, par une revue systématique algorithmique comme celle faite par~\cite{raimbault2015models}.
}


%%%%%%%%%%%%%%%%%%%%%%%
\subsubsection*{Potential Applications}{Application potentielles}


\paragraph{Transfer of Knowledge between fields}{Transfert de Connaissances}

\bpar{
Concrete applications of our framework include potential transfer of knowledge between fields. As biological systems inspire system architecture in morphogenetic engineering, or as the use of gravity models inspired from physics have flourishing applications in geography, we think that trying to decline the general framework in specific disciplines may bring analogies or new models that would have been difficult to formulate otherwise.
}{
Les applications concrètes de ce cadre incluent un transfert potentiel de connaissance entre champs. Comme les systèmes biologiques inspirant l'architecture en \emph{morphogenetic engineering}, ou comme l'usage des modèles gravitaires inspirés par la physique a eu des applications riches en géographie, nous postulons que les tentatives de déclinaison du cadre dans des disciplines spécifiques peuvent favoriser des analogies ou d'autre modèles qui auraient été difficiles à formuler autrement.
}




\stars





