


%----------------------------------------------------------------------------------------

\newpage


\section*{Chapter Conclusion}{Conclusion du Chapitre}


%Dans une logique de lecture linéaire, cette ouverture par l'introduction d'un cadre théorique, devrait avoir synthétisé et rassuré sur les questions ouvertes a priori réglées dans leur majorité - seul la conclusion pouvant encore apporter une chute dans la narration. Il s'agit d'un malentendu, et le lecteur qui voudrait être rassuré aurait du s'arrêter à la fin de la troisième partie, à la fin duquel nous avions fait un tour relativement conséquent des approches proposées.


\bpar{
This chapter thus allowed us to take a step back on our contributions and to put these into perspective. It indeed opens several doors, and recall the fact that the coverage of knowledge remain very low.
}{
Ce chapitre nous a permis ainsi de prendre du recul sur nos contributions et de les mettre en perspective. Il ouvre en fait de nombreuses portes, et fait prendre conscience que la portée des connaissances reste embryonnaire.
}

%Pour donner une allégorie, nous serions un peu dans la situation du périhélie de Mercure et du spectre de l'atome qui étaient des détails négligeables pour la physique classique à la fin du 19ème siècle, et ont mené aux gigantesques développements au cours du 20ème que sont la physique quantique et la relativité générale.

\bpar{
The questions raised by each of the levels are fundamental for the study of complex territorial systems but also of complex systems in general. The theory proposed in~\ref{sec:theory} again highlights the issue of spatio-temporal non-stationarity within a multi-scale context, that we postulate as crucial but under-explored in the case of territorial systems.We also distinguish the difficulty to integrate existing theories what implies an understanding of model coupling processes.
}{
Les questions soulevées par chacun des niveaux sont fondamentales pour l'étude des systèmes territoriaux complexes mais aussi des systèmes complexes en général. La théorie proposée en~\ref{sec:theory} pointe à nouveau la question de la non-stationnarité spatio-temporelle dans un contexte multi-échelle, que nous postulons cruciale mais peu explorée dans le cas des systèmes territoriaux. Nous distinguons également la difficulté d'intégration de théories existantes ce qui implique une compréhension des processus de couplage des modèles.
}


\bpar{
This issue is at the heart of the formal framework developed in the following~\ref{app:sec:csframework}, which also raises scale imbrication issues. The problem to obtain a consistent algebraic structure with a monoid action on data implies an integration of \noun{Krob}'s theory, what more generally questions the integration of system engineering approaches (``industrial'' complex systems) with the ones of natural complex systems.
}{
Ce problème est au coeur du cadre formel développé par la suite~\ref{app:sec:csframework}, qui soulève aussi des questions d'imbrication d'échelles. Le problème d'obtenir une structure algébrique cohérente avec une action de monoïde sur les données implique une intégration de la théorie de \noun{Krob}, ce qui questionne plus généralement l'intégration des approches d'ingénierie système (systèmes complexes ``industriels'') avec celles de systèmes complexes naturels.
}


\bpar{
The possibility of integrative theories is raised by the introduction of the knowledge framework~\ref{sec:knowledgeframework}, which also introduces more general questions of knowledge production and of the nature of complexity which was briefly evoked from an epistemological viewpoint in~\ref{sec:epistemology}.
}{
La possibilité de théories intégratives est soulevée par l'introduction du cadre de connaissance~\ref{sec:knowledgeframework}, qui pose également des problèmes plus généraux de production des connaissances et de nature de la complexité que nous avions brièvement abordé d'un point de vue épistémologique en~\ref{sec:epistemology}.
}


\bpar{
We propose to synthesize a part of these diverse open questions in a consistent research project on the long term, but which includes first immediate directions, and that we present in opening.
}{
Nous proposons de synthétiser une partie de ces diverses questions ouvertes dans un projet de recherche cohérent sur un long terme mais incluant des premières pistes concrètes immédiates, que nous présenterons en ouverture.
}



\stars
