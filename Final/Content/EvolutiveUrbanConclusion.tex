



%----------------------------------------------------------------------------------------

\newpage


\section*{Chapter Conclusion}{Conclusion du Chapitre}

\bpar{
The notion of co-evolution, which was until here in our work relatively conceptual, appears under multiple new complementary angles. This chapter allows to clarify its role within the evolutive urban theory. It will also be central for the theoretical construction that we will elaborate in~\ref{sec:theory}.
}{
La notion de co-évolution, qui était jusqu'ici dans notre travail relativement conceptuelle, apparaît sous de multiples angles nouveaux complémentaires. Ce chapitre permet d'éclairer son rôle au sein de la théorie évolutive des villes. Celle-ci sera également centrale pour la construction théorique que nous élaborerons en~\ref{sec:theory}.
}
 

 
\bpar{
Indeed, strong interdependencies can translate as variable local correlations, i.e. a spatial non-stationarity, induced on the one hand by the local patterns corresponding to a given interaction regime, of which we managed to capture the static manifestations in section~\ref{sec:staticcorrelations}, on the other hand by the multi-scalar nature of implied processes that we also showed, and thus by interaction at a small scale and long range between the different territorial entities, that we illustrated on a simple case with the interaction model studied in~\ref{sec:interactiongibrat}, which already allowed to indirectly reveal network effects in systems of cities.
}{
En effet, des interdépendances fortes peuvent se traduire par des corrélations locales variables, c'est-à-dire une non-stationnarité spatiale, induite d'une part par les motifs locaux correspondant à un régime d'interaction donné, dont nous avons pu capturer les manifestations statiques en section~\ref{sec:staticcorrelations}, d'autre part par le caractère multi-scalaire des processus impliqués que nous avons également montré, et donc par les interactions à petite échelle et longue portée entre les différentes entités territoriales, que nous avons illustré sur un cas simple par le modèle d'interaction étudié en~\ref{sec:interactiongibrat}, qui a déjà permis de révéler indirectement des effets de réseaux dans les systèmes de villes. 
}


\bpar{
We also shed light on a dynamical approach of co-evolution, by showing the potential complexity of the structure of causal relationships in the cas of a simple model of urban morphogenesis. The methodology developed was also shown efficient on real data for South Africa on long time, allowing to show an effect of segregation policies at the second order on the co-evolution itself. This method will be used as an empirical characterization of co-evolution in the following.
}{
On a également éclairé une approche dynamique de la co-évolution, en montrant la complexité potentielle de la structure des relations causales dans le cas d'un modèle de morphogenèse urbaine simple. La méthodologie développée s'est montrée également efficace sur les données réelles de l'Afrique du Sud sur le temps long, permettant de découvrir un effet des politiques de ségrégation au second ordre sur la co-évolution elle-même. Cette méthode nous servira de caractérisation empirique de la co-évolution par la suite.
}


%La question de la non-stationnarité dans les systèmes urbains est cruciale mais très peu comprise, et nous l'avons à peine effleurée. Dans notre cas, l'aspect le plus important de celle-ci pour la construction des modèles est son implication pour les échelles considérées, et les hypothèses d'équilibre ou de stochasticité correspondantes. On y reviendra par un point de vue différent en Chapitre~\ref{ch:morphogenesis}.
 % Nous proposons pour l'instant de renforcer l'épaisseur thématique des relations considérées : on a en effet pour l'instant seulement étudié des variables très simples (distribution de la population et propriétés du réseau) à certaines échelles seulement. On étudiera ainsi dans le prochain Chapitre~\ref{ch:micro} des ontologies et échelles sur des cas d'étude plus exotiques.



\stars

