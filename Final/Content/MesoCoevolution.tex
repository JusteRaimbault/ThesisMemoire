
% Chapter 

%\chapter{Co-evolution at the meso-scale}{Co-évolution à l'Echelle Mesoscopique}
\chapter{Co-évolution à l'Echelle Mesoscopique}


\label{ch:mesocoevolution} 

%----------------------------------------------------------------------------------------



Les processus sous-jacents à la co-evolution ne sont pas exactement similaires lorsqu'on passe de l'échelle macroscopique a l'échelle mesoscopique, comme le suggèrent nos différentes analyses empiriques : par exemple, les régimes de causalités obtenus à petite échelle pour l'Afrique du Sud en~\ref{sec:causalityregimes} sont plus clairs que ceux pour les transactions immobilières et le Grand Paris en~\ref{sec:casestudies}. À l'échelle métropolitaine, les processus de relocalisation sont essentiels pour expliquer l'évolution de la forme urbaine, et ceux-ci peuvent partiellement être attribués aux différentiels d'accessibilité, sachant que l'évolution des réseaux répond quant à elle à des logiques complexes conditionnées par les distributions territoriales. Centralité, densité, accessibilité, autant de propriétés potentiellement impliquées dans les processus co-évolutifs, et propres au concept de forme urbaine.


Nous faisons le choix d'appuyer le rôle de la forme urbaine à l'échelle mesoscopique, et utilisons la morphogenèse urbaine comme paradigme de modélisation de la co-évolution : le couplage fort de la forme urbaine avec le réseau par la co-évolution permet de considérer les fonctions urbaines plus explicitement. Ce chapitre fait suite au Chapitre~\ref{ch:morphogenesis}, et étend les modèles qui y ont été développés.

Différentes heuristiques de génération de réseau sont comparées dans une première section~\ref{sec:networkgrowth}, toujours dans un paradigme de couplage simple, afin d'établir les topologies produites par différentes règles.

Cette étape permet d'introduire un modèle de co-évolution par morphogenèse en~\ref{sec:mesocoevolmodel}, qui est calibré sur les objectifs couplé de morphologie urbaine et de topologie de réseau.

Enfin, nous décrivons en~\ref{sec:lutecia} un modèle permettant l'exploration de processus complexes pour la croissance du réseau, notamment des processus endogènes de gouvernance impliquant des agents décideurs à l'échelle métropolitaine.




\stars


\textit{Les résultats des deux premieres sections de ce chapitre ont été présentés à CCS 2017 comme~\cite{raimbault:halshs-01590624}, et paraitront prochainement de façon synthétique comme chapitre d'ouvrage~\cite{raimbault2018urban}; la structure du modèle et des résultats préliminaires pour la troisième section ont été présentés à ECTQG 2015 comme \cite{le2015modeling}.}


%----------------------------------------------------------------------------------------




