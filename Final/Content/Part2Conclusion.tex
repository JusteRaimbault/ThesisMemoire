



%\chapter*{Part II Conclusion : Co-evolution: a complex concept with multiple faces}{Conclusion de la Partie II}
\chapter*{Conclusion de la Partie II : co-évolution, un concept complexe aux visages multiples}


% to have header for non-numbered introduction
\markboth{Conclusion of Part II}{Conclusion de la partie II}


%\headercit{}{}{}




Cette partie nous a permis d'apporter divers premiers éléments de réponse à notre problématique de modélisation de la co-évolution, en construisant à la fois des outils et en ouvrant des perspectives particulières :
\begin{enumerate}
	\item Le premier chapitre, à la composition hétérogène, creuse des concepts fondamentaux issus de la théorie évolutive des villes, qui s'affirme ainsi comme partie intégrante de notre squelette conceptuel. L'étude des corrélations statiques confirme la non-stationnarité et suggère la multi-scalarité des interactions entre réseaux et territoires, et nous permet d'une part de confirmer la pertinence de notre approche à deux échelles distinctes, et d'autre part fournit une analyse empiriques construisant des données observées qui permettront de calibrer les modèles. Ensuite, la construction d'une caractérisation opérationnelle de la co-évolution, em termes de régimes de causalité, est essentielle à la fois du point de vue empirique et pour la caractérisation du comportement des modèles à venir. Enfin, nous explorons les potentialités des modèles d'interaction dans les systèmes de villes, ce qui nous permet de confirmer l'existence d'effet de réseau.
	\item Le second chapitre explore le concept de morphogenèse, en commençant par en construire une définition interdisciplinaire qui suggère les paradigmes de modélisation par la forme et la fonction et introduit un lien implicite avec la co-évolution. Nous développons alors un modèle simple se basant uniquement sur la forme, par des principes d'agrégation-diffusion, et montrons que celui-ci reproduit une large gamme de forme territoriales existantes en Europe. Nous posons alors la première brique d'un couplage avec un modèle de croissance de réseau et explorons l'espace des corrélations statiques potentielles.
\end{enumerate}


Faisons à ce stade un bilan conceptuel de notre construction progressive.


\subsection*{Conceptual definition}{Définition conceptuelle}


Rappelons la définition conceptuelle de la co-évolution construite en particulier par transfert multidisciplinaire en première partie : des systèmes territoriaux évolutifs pourront présenter des propriétés de co-évolution à trois niveaux distincts : (i) entités locales en interactions réciproques ; (ii) population régionale d'entités présentant des causalité circulaires d'un point de vue statistique ; (iii) interdépendances systémiques globales.


\subsection*{An operational approach}{Une caractérisation opérationnelle}


Cette partie aura également été cruciale puisqu'elle aura permis d'introduire une mesure opérationnelle de relations causales complexes, que nous proposons de considérer comme une méthode de caractérisation de la co-évolution, c'est-à-dire un proxy de celle-ci. Cette caractérisation, introduite et explorée en~\ref{sec:causalityregimes}, se base sur l'idée de \emph{régimes de causalité}, qui correspondent à des motifs de causalité au sens de Granger entre un ensemble de variable. Dans le cas de causalité réciproques entre deux populations d'entités, nous aurons bien \emph{co-évolution} au deuxième sens. Nous avons donc ainsi une caractérisation empirique et opérationnelle de la co-évolution.



\subsection*{Morphogenesis}{L'approche morphogénétique}


La morphogenèse appuie la question de l'autonomie et de l'interdépendance, des limites et de l'environnement, la question des échelles. Précisons dans quelle mesure celle-ci renforce la construction du concept de co-évolution. L'idée de sous-système indépendant rejoint celle de niche écologique équivalente à un système de frontières dans la théorie de \noun{Holland}~\cite{holland2012signals}. Or celui-ci suppose les entités d'une même niche en co-évolution : on voit ainsi en filigrane que ce concept nous permet d'une part une entrée pertinente pour des modèles à l'échelle macroscopique, mais qu'il tisse d'autre part indubitablement bien que subrepticement des liens profonds avec la sphère conceptuelle que nous mettons progressivement en place.




\subsection*{Towards a modeling approach}{Vers un approche de modélisation}


En se raccrochant au triptyque des domaines de connaissance concepts-empirique-modèles~\cite{livet2010}, nous pouvons considérer être armés pour la composante encore manquante et qui est notre objectif final : celle des modèles, puisque nous avons amplement traité la co-évolution du point de vue conceptuel et du point de vue empirique. 

% préciser direction précise / processus ? ou intro de III ?

L'enjeu de la partie suivante va donc être de produire une synthèse des briques que nous avons introduites, et construire progressivement des modèles de co-évolution aux deux échelles (macroscopique et mesoscopique), principalement en étendant les modèles déjà étudiés en profondeur.












