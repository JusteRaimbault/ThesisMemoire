



%\chapter*{Part II Conclusion}{Conclusion de la Partie II}
\chapter*{Conclusion de la Partie II}


% to have header for non-numbered introduction
\markboth{Conclusion}{Conclusion}


%\headercit{}{}{}




\section*{Co-evolution: a complex concept with multiple faces}{Co-évolution : un concept complexe aux visages multiples}



\subsection*{Conceptual definition}{Définition conceptuelle}


Rappelons la vision conceptuelle construite en particulier par transfert interdisciplinaire en première partie.



\subsection*{Morphogenesis}{L'approche morphogénétique}


La morphogenèse appuie la question de l'autonomie et de l'interdépendance, des limites et de l'environnement, la question des échelles. Précisons dans quelle mesure celle-ci renforce la construction du concept de co-évolution.

% evoquer les niches sans aller trop dans le détail (pour la conclusion)




\subsection*{An operational approach}{Une approche opérationnelle}


Cette partie aura également été cruciale puisqu'elle aura permis d'introduire une mesure opérationnelle de relations causales complexes, que nous proposons de considérer comme une méthode de caractérisation de la co-évolution, c'est-à-dire un proxy de celle-ci.





\subsection*{Towards a modeling approach}{Vers un approche de modélisation}


En se raccrochant au triptyque des domaines de connaissance concepts-empirique-modèles (voir~\ref{sec:knowledgeframework}), nous pouvons considérer être armés pour la composante encore manquante et qui est notre objectif final : celle des modèles.

% préciser direction précise / processus ? ou intro de III ?














