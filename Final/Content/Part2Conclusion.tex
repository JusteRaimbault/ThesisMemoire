



\bpar{
\chapter*{Conclusion of Part II: Co-evolution, a complex concept with multiple faces}
}{
\chapter*{Conclusion de la Partie II : co-évolution, un concept complexe aux visages multiples}
}


% to have header for non-numbered introduction
\bpar{
\markboth{Conclusion of Part II}{Conclusion of Part II}
}{
\markboth{Conclusion de la partie II}{Conclusion de la partie II}
}

%\headercit{}{}{}



\bpar{
This part allowed us to bring diverse first elements of answer to our problematic of modeling co-evolution, by both constructing tools and opening particular perspectives.
}{
Cette partie nous a permis d'apporter divers premiers éléments de réponse à notre problématique de modélisation de la co-évolution, en construisant à la fois des outils et en ouvrant des perspectives particulières. 
}


\bpar{
The first chapter, with an heterogenous composition, digs into fundamental concepts spanning from the evolutive urban theory, which is thus confirmed as a consequent part of our conceptual skeleton. The study of static correlations confirms the non-stationarity and suggests the multi-scalarity of interactions between networks and territories, and allows us on the one hand to confirm the relevance of our approach at two distinct scales, and on the other hand provides an empirical analysis constructing observed data which will allow to calibrate models. Then, the construction of an operational characterization of co-evolution, in terms of causality regimes, is essential both from the empirical viewpoint and for the characterization of the behavior of models which will be introduced in the following. Finally, we explore the potentialities of interaction models in systems of cities, what allows us to confirm the existence of network effects.
}{
Le premier chapitre, à la composition hétérogène, creuse des concepts fondamentaux issus de la théorie évolutive des villes, qui s'affirme ainsi comme partie intégrante de notre squelette conceptuel. L'étude des corrélations statiques confirme la non-stationnarité et suggère la multi-scalarité des interactions entre réseaux et territoires, et nous permet d'une part de confirmer la pertinence de notre approche à deux échelles distinctes, et d'autre part fournit une analyse empirique construisant des données observées qui permettront de calibrer les modèles. Ensuite, la construction d'une caractérisation opérationnelle de la co-évolution, en termes de régimes de causalité, est essentielle à la fois du point de vue empirique et pour la caractérisation du comportement des modèles à venir. Enfin, nous explorons les potentialités des modèles d'interaction dans les systèmes de villes, ce qui nous permet de confirmer l'existence d'effets de réseau.
}


\bpar{
The second chapter explores the concept of morphogenesis, starting by constructing for it an interdisciplinary definition which suggests the modeling paradigms through form and function and introduces an implicit link with co-evolution. We then develop a simple model based on form only, through aggregation-diffusion principles, and show that it reproduces a large spectrum of territorial forms existing in Europe. We finally construct the first building brick of a coupling with a network growth model and explore the space of potential static correlations.
}{
Le second chapitre explore le concept de morphogenèse, en commençant par en construire une définition interdisciplinaire qui suggère les paradigmes de modélisation par la forme et la fonction et introduit un lien implicite avec la co-évolution. Nous développons alors un modèle simple se basant uniquement sur la forme, par des principes d'agrégation-diffusion, et montrons que celui-ci reproduit une large gamme de formes territoriales existantes en Europe. Nous posons alors la première brique d'un couplage avec un modèle de croissance de réseau et explorons l'espace des corrélations statiques potentielles.
}


\bpar{
We can at this stage make a conceptual summary of our progressive construction.
}{
Nous pouvons faire à ce stade un bilan conceptuel de notre construction progressive.
}



\subsection*{Conceptual definition}{Définition conceptuelle}


\bpar{
We recall the conceptual definition of co-evolution constructed in particular through multi-disciplinary transfer in the first part: evolutive territorial systems can exhibit co-evolution properties at three distinct levels: (i) local entities in reciprocal interactions; (ii) regional population of entities exhibiting circular causalities from a statistical viewpoint; (iii) global systemic interdependencies.
}{
Rappelons la définition conceptuelle de la co-évolution construite en particulier par transfert multidisciplinaire en première partie : des systèmes territoriaux évolutifs pourront présenter des propriétés de co-évolution à trois niveaux distincts : (i) entités locales en interactions réciproques ; (ii) population régionale d'entités présentant des causalités circulaires d'un point de vue statistique ; (iii) interdépendances systémiques globales.
}


\subsection*{An operational characterization}{Une caractérisation opérationnelle}


\bpar{
This part will also have been crucial since it allowed us to introduce an operational measure of complex causal relationships, that we propose to consider as a method to characterize co-evolution, i.e. a proxy for it. This characterization, introduced and explored in~\ref{sec:causalityregimes}, is based on the idea of \emph{causality regimes}, which correspond to causality patterns in the Granger sense between an ensemble of variables. In the case of reciprocal causalities between two populations of entities, we will speak indeed of a \emph{co-evolution} in the second sense given above. We therefore have an empirical and operational characterization of co-evolution.
}{
Cette partie aura également été cruciale puisqu'elle aura permis d'introduire une mesure opérationnelle de relations causales complexes, que nous proposons de considérer comme une méthode de caractérisation de la co-évolution, c'est-à-dire un proxy de celle-ci. Cette caractérisation, introduite et explorée en~\ref{sec:causalityregimes}, se base sur l'idée de \emph{régimes de causalité}, qui correspondent à des motifs de causalité au sens de Granger entre un ensemble de variables. Dans le cas de causalités réciproques entre deux populations d'entités, nous aurons bien \emph{co-évolution} au deuxième sens donné ci-dessus. Nous avons donc ainsi une caractérisation empirique et opérationnelle de la co-évolution.
}




\subsection*{The morphogenetic approach}{L'approche morphogénétique}



\bpar{
Morphogenesis highlights the question of autonomy and interdependency, of boundaries and the environment, the question of scales. We can precise to what extent it reinforces the construction of the concept of co-evolution. The idea of independent subsystem rejoins the one of ecological niche which is equivalent to a system of boundaries in the theory of \noun{Holland}~\cite{holland2012signals}. This theory indeed consider the entities of a given niche as co-evolving: we can see implicitly that this concept allows on the one hand a relevant entry for models at the mesoscopic scale, but on the other hand that it creates inevitable yet unexpected deep links with the conceptual context that we are progressively building.
}{
La morphogenèse appuie la question de l'autonomie et de l'interdépendance, des limites et de l'environnement, la question des échelles. Précisons dans quelle mesure celle-ci renforce la construction du concept de co-évolution. L'idée de sous-système indépendant rejoint celle de niche écologique équivalente à un système de frontières dans la théorie de \noun{Holland}~\cite{holland2012signals}. Or celui-ci suppose les entités d'une même niche en co-évolution : on voit ainsi en filigrane que ce concept nous permet d'une part une entrée pertinente pour des modèles à l'échelle mesoscopique, mais qu'il tisse d'autre part indubitablement bien que subrepticement des liens profonds avec la sphère conceptuelle que nous mettons progressivement en place.
}




\subsection*{Towards a modeling approach of co-evolution}{Vers une approche de modélisation de la co-évolution}


\bpar{
By recalling the three knowledge domains conceptual-empirical-models~\cite{livet2010}, we can consider to be equiped for the still missing component and which is our final objective: the one of models, since we extensively developed co-evolution from a conceptual and empirical point of view.
}{
En se raccrochant au triptyque des domaines de connaissance concepts-empirique-modèles~\cite{livet2010}, nous pouvons considérer être armé pour la composante encore manquante et qui est notre objectif final : celle des modèles, puisque nous avons amplement traité la co-évolution du point de vue conceptuel et du point de vue empirique. 
}

% préciser direction précise / processus ? ou intro de III ?

\bpar{
The aim of the next part will thus be to produce a synthesis of the bricks we introduced, and progressively construct co-evolution models at the two scales (macroscopic and mesoscopic), mostly by extending the models already studied.
}{
L'enjeu de la partie suivante va donc être de produire une synthèse des briques que nous avons introduites, et construire progressivement des modèles de co-évolution aux deux échelles (macroscopique et mesoscopique), principalement en étendant les modèles déjà étudiés.
}












