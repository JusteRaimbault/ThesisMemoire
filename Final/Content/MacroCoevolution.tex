% Chapter 

%\chapter{Co-evolution at the macro-scale}{Co-évolution à l'Echelle Macroscopique}
\chapter{Co-évolution à l'Echelle Macroscopique}


\label{ch:macrocoevolution} 

%----------------------------------------------------------------------------------------

% chapter introduction


Les dynamiques des systèmes territoriaux, qui nous le rappelons impliquent dans notre cadre dynamiques couplées des territoires et des réseaux, à l'échelle macroscopique peuvent être partiellement appréhendées au moyen d'une approche par les interactions, comme montré au Chapitre~\ref{ch:evolutiveurban}. Pour rappeler les idées sous-jacentes de manière synthétique, en echo au point de vue par la morphogenèse développé en Chapitre~\ref{ch:morphogenesis} qui au contraire se concentre sur les règles autonomes au sein des sous-systèmes a une échelle intermédiaire\comment[FL]{cela sera pour le chap8 (wrapup)}, le principe dans cette ontologie est de raffiner\comment[FL]{sens} le rôle des interactions en capturant\comment[AB]{suppr. en particulier, nous avons montré qu'il est possible de capturer} les variations propres\comment[AB]{des interactions ? preciser} dans des processus abstraits endogènes simples. Le pouvoir explicatif est alors different de celui des modèles économiques classiques et concerne d'autre types de processus, basés sur les interactions à des échelles d'espace plus grandes et des échelles de temps plus longues. Le rôle des réseaux de transports dans ce cadre\comment[FL]{ton cadre conceptuel n'est pas tres coevolution : role transport $\rightarrow$ territoire} est crucial, comme suggéré par les résultats préliminaires obtenus précédemment. Dans quelle mesure la construction du lien ferroviaire par le tunnel sous la Manche a-t-elle pu conforter le pouvoir économique de Londres ou renforcer ses interactions avec ses proches voisins Européens\comment[FL]{quel rapport ? de plus echelle non evoquee jusqua present}, et dans quelle mesure les évènements politiques recents peuvent-ils conduire a une modification des trajectoires économiques puis par conséquent a une modification des motifs de transports par une rétroaction de la demande ? D'une façon similaire, les projets de lignes à grande vitesse sur la côte Est des Etats-Unis et dans le corridor Californien sont-ils une conséquence attendue des dynamiques régionales ou un choix de gouvernance plus compliqué à cerner, et s'ils sont effectivement réalisés %malgré le contexte politique récemment devenu plus hostile au rail\comment[FL]{suppr}
\comment[FL]{formulation maladroite, pas assez systemique} , dans quelle mesure influenceront-ils les trajectoires du système de ville ? Nous avons déjà étudié des questions analogues dans le cas de l'Afrique du Sud de manière empirique en~\ref{sec:causalityregimes}, et nous proposons dans ce chapitre d'éclairer celles-ci à un plus grand niveau de généralité, en introduisant les processus de co-evolution dans les modèles d'interactions déjà développés. Pour donner une idée de la nature des enseignements qu'il est possible de tirer d'une telle approche, nous commençons en~\ref{sec:macrocoevolexplo} par une exploration systématique du modèle SimpopNet, approche la plus avancée en termes de modélisation de la co-evolution au sein des systèmes de villes\comment[FL]{donc ce n'est plus le territoire ? harmoniser}, comme établi au Chapitre~\ref{ch:modelinginteractions}. Cela permet également d'introduire les indicateurs adaptés pour la compréhension des trajectoires des systèmes de villes en termes de dynamiques co-évolutives\comment[FL]{formulation a harmoniser}. Nous décrivons ensuite en~\ref{sec:macrocoevol} le modèle générique de co-évolution, qui est testé sur des données synthétiques à deux niveaux de détail pour la représentation du réseau, puis sur le système de villes français.
% de manière à pouvoir comparer avec les modèles statiques précédemment étudiés.
%Enfin, nous décrivons en~\ref{sec:simpopsino} le cas d'application potentiel au système de ville Chinois et ses enjeux particuliers.




\stars


\textit{Ce chapitre est inédit pour sa première section. La deuxième section reprend les résultats de~\cite{} % medium conference
pour les données synthétiques, et va paraître prochainement comme~\cite{}. % chapitre Rozenblat
%La dernière section est également inédite.
}


%----------------------------------------------------------------------------------------













