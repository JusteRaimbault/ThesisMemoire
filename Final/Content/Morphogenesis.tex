


% Chapter 

%\chapter{Urban Morphogenesis}{Morphogenèse Urbaine} % Chapter title
\chapter{Morphogenèse Urbaine}

\label{ch:morphogenesis} % For referencing the chapter elsewhere, use \autoref{ch:name} 

%----------------------------------------------------------------------------------------

%\headercit{}{}{}

%\bigskip


La géographie accorde une grande importance aux relations spatiales et à la mise en réseau, comme l'atteste par exemple la première loi de \noun{Tobler}. Nous l'avons mis en évidence pour les relations entre réseaux et territoires par exemple en section~\ref{sec:interactiongibrat}. Toutefois, nos résultats sur la non-stationnarité, ainsi que la mise en valeur d'échelles locales endogènes, suggèrent une certaine pertinence de l'idée de sous-système relativement indépendant. Il serait alors possible d'isoler certaines règles locales régissant un sous-système, un fois fixés certains paramètres exogènes capturant justement les relations avec d'autres sous-systèmes. Cette question porte à la fois sur l'échelle d'espace, de temps, mais aussi sur les éléments concernés.

Reprenons un exemple concret de terrain déjà évoqué en Chapitre~\ref{ch:thematic}: la laborieuse mise en place du tramway de Zhuhai. L'impact du retard de la mise en place et la remise en question de futures lignes (dus à un problème technique inattendu lié à une technologie de transfert de courant par troisième rail importée d'Europe qui n'avait jamais été testée dans les conditions climatiques locales assez exceptionnelles en termes d'humidité), aura une nature très différentes selon l'échelle et les acteurs urbains considérés. Le manque de coordination générale entre transports et urbanisme laisse supposer que les dynamiques urbaines en termes de populations et d'emplois y sont relativement insensibles dans l'immédiat. Le Bureau des Transports de la Municipalité de Zhuhai ainsi que le bureau technique Européen ayant conçu la technologie défectueuse ont pu subir des répercussions politiques et économiques bien plus conséquentes. D'autre part, que ce soit à Zhongshan, Macao ou Hong-Kong, nous pouvons supposer que le problème a une repercussion quasi-nulle, le projet ayant un rôle uniquement local.


Généralisant au système de transport local, celui-ci peut être relativement bien isolé des systèmes voisins, et donc ses relations avec la ville considérée dans un contexte local. On supposera à la fois une certaine forme de stationnarité locale (``régime urbain local'') mais aussi une certaine indépendance avec l'extérieur. Nous pouvons également noter que dans ce cadre, son auto-organisation locale impliquera nécessairement des relations fortes entre forme et fonction, de par la distribution spatiale des fonctions urbaines mais aussi car \emph{la forme fait la fonction} dans certains cas de figure, au sens des motifs d'utilisation entièrement conditionnés à cette forme. Le type de raisonnement que nous avons esquissé mobilise les éléments essentiels propres à l'idée de \emph{morphogenèse urbaine}.


Nous allons dans ce chapitre clarifier sa définition et montrer les potentialités qu'elle donne pour éclairer les relations entre réseaux et territoires. La morphogenèse, qui a été importée de la biologie vers de nombreux champs, a dans chaque cas ouvert des voies pour l'étude des systèmes complexes propres à ce champ selon un point particulier. Il est important de noter que le monument qu'est la Théorie des Catastrophes de \noun{René Thom} introduit une façon originale de comprendre la différentiation qualitative et donc la morphogenèse. Cette théorie a toujours un potentiel d'application considérable aux problèmes qui nous concernent, comme l'a suggéré \noun{Durand-Dastès}~\cite{durand2003geographes} en évoquant la systèmogenèse.

Dans un premier temps, un effort d'épistémologie par des points de vue complémentaires de plusieurs disciplines permet d'éclairer la nature de la morphogenèse dans la section~\ref{sec:interdiscmorphogenesis}. Cela permet de clarifier le concept en lui donnant une définition bien précise, distincte de celle de l'auto-organisation, qui appuie les relations causales circulaires entre forme et fonction. Nous explorons ensuite un modèle simple de morphogenèse urbaine, basé sur la densité de population seule, à l'échelle mesoscopique, dans la section~\ref{sec:densitygeneration}. La démonstration que les processus abstraits d'agrégation et de diffusion sont suffisants pour reproduire l'ensemble des formes d'établissements humains en Europe, en utilisant les résultats de~\ref{sec:staticcorrelations}, confirme la pertinence de l'idée de morphogenèse pour la modélisation à certaines échelles et pour les dimensions morphologiques. Ce modèle est ensuite couplé de manière séquentielle à un module de morphogenèse de réseau dans la section~\ref{sec:correlatedsyntheticdata}, afin d'établir un espace possible des correlations statiques entre indicateurs de forme urbaine et indicateurs de réseau, qui sont comme on l'a vu précédemment un témoin des relations locales entre réseaux et territoires.


Nous posons ainsi d'autres briques de modélisation de la co-évolution, à l'échelle mesoscopique par l'entrée de la morphogenèse urbaine.




\stars


\textit{Ce chapitre est composé de divers travaux. La première section est adaptée d'un travail en anglais en collaboration avec \noun{C. Antelope} (University of California), \noun{L. Hubatsch} (Francis Crick Institute) et \noun{J.M. Serna} (Université Paris VII) à la suite de l'école d'été 2016 du Santa Fe Institute~\cite{antelope2016interdisciplinary}; la deuxième section est traduite de~\cite{raimbault2017calibration}; et enfin la troisième section a été écrite pour les Actes des Journées de Rochebrune 2016~\cite{raimbault2016generation}.}



%\comment[CC]{Attention. Les parties traduites sont tres claires et avec un niveau de langage soigne, tandis que les introductions / transitions sont brouillonnes et touffues. Il faudrait harmoniser le tout, c'est a dire surtout fluidifier les transitions qui sont la partie en propre de ton manuscript de these, le reste etant deja publie ailleur}





%----------------------------------------------------------------------------------------









