

%\chapter*{Part I Introduction}{Introduction de la Partie I}
\chapter*{Introduction de la Partie I}


% to have header for non-numbered introduction
\markboth{Introduction de la Partie I}{Introduction de la Partie I}


%\headercit{}{}{}


%---------------------------------------------------------------------------



\bigskip

\textit{Un voyage, la découverte d'une ville, de nouvelles rencontres, un partage d'idées : autant de processus qui impliquent une générativité cognitive et une interaction complexe entre nos representations, nos actions et l'environnement. La construction d'une connaissance scientifique n'échappe pas à ces règles. On pourrait alors voir dans l'objet étudié lui-même, prenons la ville et ses agents, une allégorie du processus de production de connaissance sur l'objet. Comme Romain Duris qui débarque dans l'Auberge Espagnole, et découvre ces rues inconnues que plus tard on aura parcouru cent fois, où on aura vécu mille choses : on débarque dans un monde de concepts, d'approches, de point de vues complémentaires sur des choses qui ne sont pas la même chose. Cette discrépance ontologique est finalement tout aussi présente dans nos représentations de l'espace urbain : \emph{Oven Street} c'est un des centres de la connaissance pour le membre de Géocités; c'est le centre de Paris, donc de la France, donc du Monde pour le fier autochtone du 6ème; c'est le marché Saint-Germain et le shopping de luxe globalisé pour le touriste international; c'est un morceau d'histoire pour l'élève des Ponts pour qui cela évoque le temps des Saint-pères. Des objets, des concepts, compris et définis par de multiples disciplines et agents producteurs de connaissance : parle-t-on finalement vraiment de la même chose ? Comment tirer parti de cette richesse de points de vue, comment intégrer la complexité permise par cette diversité ? Apporter des éléments de réponse suppose une démarche constructive, générative et autant inclusive que possible. Les choix sont toujours plus éclairés si on a un aperçu d'un maximum d'alternatives. Le trader qui habite son loft en haut des \emph{mid-levels} et travaille dans son building à deux pas entre deux rails, connait bien Hong-Kong, mais un seul parmi ses multiples visages, et il lui sera difficilement concevable qu'existe une misère à Kwoloon, dont les habitants ne conçoivent pas le Hong-Kong éphémère mais parfois cyclique des travailleurs temporaires du mainland, qui eux ne conçoivent pas les difficultés administratives et financières de migrants de Thaïlande ou d'Inde, l'ensemble étant encore moins concevable pour un étudiant parisien égaré. Mais c'est justement l'égarement qui à dose appropriée sera source d'une connaissance plus large : les fourmis établissent leurs optimisations extrêmement précises à partir d'une marche qu'on peut considérer comme aléatoire. Les algorithmes génétiques, mais encore plus les processus d'évolution biologiques ancrés dans le physique, reposent sur un subtil compromis entre ordre et désordre, entre signal et bruit, entre stabilités et perturbations. Se perdre pour mieux se retrouver fait l'essence et le charme du voyage, qu'il soit physique, conceptuel, social. Finalement, pas de comparaison possible entre une orientation au Caylar ou sur la montagne de Bange à un ennui rectiligne en forêt d'Orléans.}



\bigskip

%\stars


Cet intermède littéraire soulève des problèmes fondamentaux induits par une exigence d'interdisciplinarité et la volonté de construction d'une connaissance complexe intégrative. Dans un premier temps, la réflexivité et la mise en relation d'une perspective prise avec un certain nombre d'autres perspectives existantes est nécessaire pour la pertinence de celle-ci. Il s'agit donc de construire solidement les concepts et spécifier les références empiriques, afin de préciser la problématique et ses objectifs \emph{de manière endogène}. D'autre part, le cadre épistémologique de la démarche se doit d'être précisé. Ci-dessus est finalement imagée une approche \emph{perspectiviste}, qui est une position épistémologique particulière que nous détaillerons ici. De plus, le statut des démonstrations est conditionné par la conception des méthodes et des outils, qui est particulière dans le cas des modèles de simulation.


Cette partie répond à ces contraintes, en posant les \emph{fondations} nécessaires à la suite de notre démarche. En terrain relativement mouvant, celles-ci devront dans certains cas être particulièrement profondes pour une stabilité de l'édifice global : ce sera par exemple le cas de l'état de l'art qui mobilisera des techniques d'épistémologie quantitative. Nous rappelons qu'elle s'organise de la manière suivante :
\begin{enumerate}
	\item Le premier chapitre construit les concepts et objets de manière théorique, et dégage un large éventail d'approches possibles aux interactions entre réseaux de transport et territoires.
	\item Le second chapitre développe les différentes approches de modélisation des interactions entre réseaux et territoires. Il établit un état de l'art, structuré par une typologie établie précédemment. Il dresse ensuite le paysage scientifique des disciplines concernées, et cherche les caractéristiques des modèles propres à chaque discipline ainsi que des possibles déterminants de celles-ci dans une modélographie.
	\item Le troisième chapitre est relativement indépendant et précise nos positions épistémologiques. Il permet notamment de situer la complexité dans laquelle nous cherchons à nous placer, de spécifier ce qui peut être attendu d'une démarche de modélisation, et de donner une définition plus large du concept de co-évolution.
\end{enumerate}




\stars











%---------------------------------------------------------------------------


%\chapter*{Définitions prélimimaires}

%Il est nécessaire de fixer pour commencer les définitions de notions qui joueront un rôle clé tout au long de notre raisonnement. Nous adoptons la stratégie suivante : les définitions données sont assez générales pour que les raffinements lorsqu'ils auront lieu précisent ces notions. Une fois qu'une notion aura été raffinée, son utilisation fera référence à l'ensemble de la profondeur (sauf utilisation particulière locale qui sera alors précisée explicitement). Cette stratégie permet d'une part d'alléger la lecture, et d'autre part favorise une lecture non-linéaire, vu que la profondeur complète ne sera pas nécessaire à toute étape pour une compréhension au premier ordre des connaissances construites. Lorsqu'une référence précise n'est pas donnée, les définitions sont inspirées de~\cite{hypergeo}. 


%\subsection*{System}{Système}

%Un \emph{Système} est composé ``\textit{d'un ensemble d'entités en interaction}''. Différentes formalisations équivalentes

% -> def en intro



%\subsection*{Models}{Modèles et Ontologies}

% -> def en intro


%\subsection*{Cities, System of Cities, Territories}{Villes, Systèmes de Villes, Territoires}

% -> def en intro


%\subsection*{Causality}{Causalité}

% -> def en intro ; dvlpmt CH4


%\subsection*{Model Coupling}{Couplage de Modèles, Modèles Intégratifs}

% -> def partielle en intro - besoin de mieux définir ?










