


% Chapter 

%\chapter{Theoretical Framework}{Cadre Théorique} % Chapter title
\chapter{Conclusion et Ouverture Théorique}


\label{ch:theory} % For referencing the chapter elsewhere, use \autoref{ch:name} 

%----------------------------------------------------------------------------------------


%\headercit{}{}{}


\bigskip


\bpar{
Theory is a key element of any scientific construction, especially in Human Sciences in which object definition and questioning are more open but also determining for research directions. We develop in this chapter a self-consistent theoretical background. It naturally emerges from thematic considerations of previous chapter, empirical explorations done in chapter~\ref{ch:empirical} and modeling experiments conducted in chapter~\ref{ch:modeling}, as a linear structure of knowledge is not appropriate to translate the type of scientific entreprise we are conducting, typically in the spirit of \noun{Sanders} in~\cite{livet2010} for which the simultaneous conjonction of empirical, conceptual and modeling domains is necessary for the emergence of knowledge. This theoretical construction is however presented to be understood independently, and is used as a structuring skeleton for the rest of the thesis.
}{
L'une des implications sous-jacentes du travail que nous avons mené jusqu'ici est l'introduction de pistes pour des \emph{théories intégrées}, c'est-à-dire s'appuyant sur une intégration horizontale et verticale au sens de la feuille de route des systèmes complexes~\cite{2009arXiv0907.2221B}, mais aussi permettant une intégration des domaines de connaissance et une réflexivité. Nous développons dans ce chapitre un ouverture théorique à plusieurs niveaux. Le cadre correspondant émerge de l'interaction des différentes composantes de la connaissance développées jusqu'ici.
}


Nous proposons dans un premier temps de mettre en perspective nos contributions sur le sujet de la co-évolution des réseaux de transport et des territoires, et d'ouvrir ainsi de manière thématique des développements potentiels.


\bpar{
We propose first to construct the \emph{geographical theory} that will pose the studied objects and their meaning in the real world (their ontology), with their interrelations. This yields precise assumptions that will be sought to be confirmed or proven false in the following.
}{
Nous élaborons ensuite dans une seconde section~\ref{sec:theory} une synthèse théorique des différentes approches prises jusqu'ici, permettant de faire le lien entre théorie évolutive des villes et morphogenèse, ce qui donne un point de vue synthétique sur la co-évolution.
}


\bpar{
Staying at a thematic level appears however to be not enough to obtain general guidelines on the type of methodologies and the approaches to use. More precisely, even if some theories imply a more natural use of some tools\footnote{to give a rough example, a theory emphasizing the complexity of relations between agents in a system will conduct generally to use agent-based modeling and simulation tools, whereas a theory based on macroscopic equilibrium will favorise the use of exact mathematical derivations.}, at the subtler level of contextualization in the sense of the approach taken to implement the theory (as models or empirical analysis), the freedom of choice may mislead into unappropriated techniques or questionings (see \cite{raimbault2016cautious} on the example of incautious use of big data and computation).
}{
Rester à un niveau thématique apparaît cependant ne pas être suffisant pour obtenir des lignes directrices générales sur le type de méthodologies et d'approches à utiliser. Plus précisément, même si certaines théories impliquent un usage plus naturel de certains outils\footnote{Pour donner un exemple basique, une théorie mettant l'emphase sur la complexité des relations entre agents dans un système conduira généralement à utiliser de la modélisation multi-agents et des outils de simulation, tandis qu'une théorie basée sur un équilibre macroscopique favorisera l'usage de dérivations mathématiques exactes.}, au niveau plus subtil de la mise en contexte au sens de l'approche prise pour implémenter la théorie (comme modèles ou analyses empiriques), la liberté de choix d'objets et d'approches en sciences sociales peut conduire à l'utilisation de techniques inappropriées ou des questionnements inadaptés (voir la section~\ref{sec:computation} pour l'exemple de l'usage inconsidéré des données massives et du calcul).
}



\bpar{}{
Ainsi, nous construisons dans une dernière section~\ref{sec:knowledgeframework} un cadre de connaissances appliqué visant à expliciter des processus de production de connaissance sur les systèmes complexes. Celui-ci est illustré par une analyse fine de la genèse de la Théorie Evolutive des Villes, puis est ensuite appliqué de manière réflexive à l'ensemble de notre travail. Une formalisation possible de ce cadre sous forme de structure algébrique est suggéré en~\ref{app:sec:csframework}.
}



\bpar{
}{
Ce chapitre doit être lu avec précaution car les constructions théoriques introduites sont à un niveau d'abstraction progressif : en quelque sorte, chaque niveau théorique est un cadre méta pour la précédente. On touche alors la question de la réflexivité, et dans quelle mesure les théories peuvent s'appliquer à elles-mêmes\footnote{En gardant à l'esprit que la séparation entre les niveaux n'est pas directement évidente : par exemple le cadre formel pour les systèmes socio-techniques de~\ref{app:sec:csframework} pourrait être appliqué comme une formalisation du cadre de connaissances.}. Ainsi, cette démarche nous permet de mener simultanément une synthèse et une ouverture.
}




\stars

\textit{Les deux premières sections de ce chapitre sont entièrement inédites. La troisième a été proposée par~\cite{raimbault:halshs-01505084} puis développée et appliquée dans~\cite{raimbault2017applied}, et son application réflexive a été présentée par~\cite{raimbault2017co}.}






