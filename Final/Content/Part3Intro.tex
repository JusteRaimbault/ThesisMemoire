



%\chapter*{Part III Introduction}{Introduction de la Partie III}
\chapter*{Introduction de la Partie III}


% to have header for non-numbered introduction
\markboth{Introduction}{Introduction}


%\headercit{}{}{}


\textit{Les contradictions ressenties au sein d'un contexte académique contraignant peuvent rapidement limiter les possibilités à la fois d'approfondissement mais aussi de synthèse. Le niveau de saturation est facilement atteint et la résignation à enterrer des illusions idéalistes passés rapidement de mise. Mais la réceptivité du public permet d'échapper invisiblement à ces contraintes, et certains media y jouent un rôle déterminant. La plus marquante aura été celle par modélisation expérimentale. Le modèle comme outil de communication. Le modèle comme un jeu pour l'enseignement. Le modèle comme prétexte au développement d'une réflexion personnelle. Le modèle comme approfondissement de notions effleurées. Le modèle comme croisement des points de vue et des sensibilités. Le modèle à la convergence des concepts compris. Le modèle comme synthèse complexe. Le pessimisme ne doit finalement pas être de mise, les moyens les plus originaux de s'évader seraient aussi les plus efficaces.}


% Les contradictions d'un système académique sclérosé et bureaucratique s'actualisent au détriment des acteurs qui essayent de s'en émanciper. Un enseignant qui n'en sait pas plus que ses élèves, s'étant retrouver à boucher un trou par nécessité budgétaire. Un cours qui ne change pas depuis une vingtaine d'année, s'enfermant dans une vision de plus en plus démodée et finissant par se crisper dogmatiquement sur des concepts insensés car déconnectés de leur base empirique et théorique. Des tentatives de changement du mode de validation des acquis, pour tendre vers une réflexion intégrée plutôt qu'une restitution à l'identique débilisante, tuées dans l'oeuf. Un cours de modélisation intégrée monté de toutes pièces mis au rebuts pour d'obscures raisons de pouvoir.


\bigskip


Au coeur de notre sujet, nous devons à la fois faire la synthèse des entrées conceptuelle et empirique sur la co-évolution, et l'approfondissement des entrées thématiques. Les modèles vont être à la fois produits et producteurs de cette synthèse et de cet approfondissement, et permettre de nous extraire du cadre disciplinaire restrictif mis en valeur précédemment, en explorant des frontières floues des domaines et de la connaissance, à l'image de l'expérience de modélisation collective en enseignement imagée ci-dessus\footnote{Qui a conduit à une réalisation concrète, voir \url{https://github.com/JusteRaimbault/ExperimentalModeling}.} dans laquelle le modèle a à la fois permis de s'extraire du cadre et d'opérer une synthèse et un approfondissement.


Cette partie vise ainsi à formuler et explorer des modèles de co-évolution, répondant à notre deuxième axe de la problématique, c'est à dire comment intégrer les processus de co-évolution dans des modèles. La question des échelles a été traitée de manière sous-jacente par les entrées thématiques complémentaires de la partie précédente : à une échelle mesoscopique, il sera plus pertinent de s'intéresser à la forme précise, tandis qu'à une échelle macroscopique les interactions entre agents sont fondamentales. Cette complémentarité des échelles fait par ailleurs écho à deux modèles séminaux de croissance urbaine, le modèle de Gibrat et le modèle de Simon. Nous démontrons en Annexe~\ref{app:sec:stochurbgrowth} que ceux-ci sont deux spécification d'un cadre plus global de modèles stochastique de croissance urbaine, ce qui suggère que nos deux approches sont non seulement complémentaires mais synthétisables.


Nous construisons ainsi les modèles dans deux chapitres, dont l'ordre a été fixé pour avoir un degré progressif de complexité des modèles. Le chapitre 6 développe les modèles à l'échelle macroscopique. Nous introduisons d'abord les indicateurs nécessaire pour qualifier le comportement de ce type de modèle, qui sont testés par application à un modèle de la littérature. Une extension directe du modèle d'interaction de~\ref{sec:interactiongibrat} est ensuite proposée comme modèle de co-évolution à l'échelle macroscopique.

Le chapitre 7 développe pour commencer un couplage fort du modèle de morphogenèse de~\ref{sec:densitygeneration} et de modèles de croissance de réseau, dans une démarche de multi-modélisation. Celui-ci est calibré sur données statiques calculées en~\ref{sec:staticcorrelations}. Nous introduisons ensuite un modèle à l'échelle métropolitaine prenant en compte des processus de gouvernance pour l'extension du réseau de transport.



\stars




%\paragraph{}{Echelles et processus}

%Partant des hypothèses tirées des enseignements empiriques et théoriques, on postulera \emph{a priori} que certaines échelles privilégient certains processus, par exemple que la forme urbaine aura une influence au niveaux micro et mesoscopiques, tandis que les motifs émergeant des flux agrégés entre villes au sein d'un système se manifesteront au niveau macroscopique. Toutefois la distinction entre échelles n'est pas toujours si claire et certains processus tels la centralité ou l'accessibilité sont de bons candidats pour jouer un rôle à plusieurs échelles\footnote{On entend ici par ``jouer un rôle'' avoir une autonomie propre à l'échelle correspondante, c'est à dire qu'ils émergent \emph{faiblement} des niveaux inférieurs.} : il s'agira par la modélisation d'également tester ce postulat, par comparaison des processus nécessaires et/ou suffisants dans les familles de modèles à différentes échelles que nous allons mettre en place, en gardant à l'esprit des possibles développements vers des modèles multi-scalaires dans lesquels ces processus intermédiaires joueraient alors un rôle crucial.



%They will be necessary a flexible family because of the variety of scales and concrete cases we can include and we already began to explore in preliminary studies. Processes already studied can serve either as a thematic bases for a reuse as building bricks in a multi-modeling context, or as methodological tools such as synthetic data generator for synthetic control. Finally, we mean by operational models hybrid models, in the sense of semi-parametrized or semi-calibrated on real datasets or on precise stylized facts extracted from these same datasets. This point is a requirement to obtain a thematic feedback on geographical processes and on theory.













