





%\chapter*{Part III Conclusion}{Conclusion de la Partie III}
\chapter*{Conclusion de la Partie III}


% to have header for non-numbered introduction
\markboth{Conclusion}{Conclusion}


%\headercit{}{}{}


Cette partie a ainsi donné des premiers éléments d'exploration de différentes entrées sur la modélisation de la co-évolution :
\begin{enumerate}
	\item 
\end{enumerate}




\subsection*{Processes in models}{Processus modélisés}


\begin{table}
\begin{tabular}[6pt]{m{4cm}|c|c|c|c}
Processus & Analyse Empirique & Echelles & Type & Modèle \\\hline
Attachement préférentiel & & Croissance Urbaine & & \\\hline
Diffusion/Etalement & & Forme Urbaine & &\\\hline
Accessibilité  & & Réseau / Ville & & \\\hline
Gouvernance des Transports & & & &\\\hline
Flux direct  & & & &\\\hline
Flux indirect/Effet tunnel & & & &\\\hline
Centralité de proximité (accessibilité : generalisation) & & & &\\\hline
Centralité de Chemin (correspond aux flux indirect : différents niveaux de généralité / sous-processus-sous-classif ?) & & & &\\\hline
Proximité au réseau & & & &\\\hline
Distance au centre (similar to agrégation ?) & & & RBD &\\\hline
\end{tabular}
\caption{}{Description des différents processus pris en compte dans les modèles de co-évolution}
\end{table}

% Type \textit{find a typology of processes}
%Ontologies : dans le macro, villes fixes, pas de nouvelles ville, mais nouveaux liens de réseau. Meso : tout évolue.


% Flux indirect/Effet tunnel : \textit{c'est le même processus, vu sous un angle différent : l'effet tunnel est l'absence de nw feedback}

%\comment{faire le même tableau pour les modèles existants : vue plus large de l'ensemble des processus. pour chacun de ces modèles et de nos modèles, lister tous les processus potentiels ; faire une typologie ensuite. Q : typologie différente d'une pure empirique ? a creuser, et peut être intéressant dans le cadre du knowledge framework, comme illustration coevol connaissances.}

%\comment{justifier ici poruquoi pas modèle très fins sur processus eco par exemple (//Levinson) : prix à payer pour être accross scales, disciplines et avoir vraiment de la coevol ? pour ces premières étapes oui. à justifier}



\subsection*{The trinity of co-evolution}{La trinité de la co-évolution}

% complementarite de la vision conceptuelle/empirique/modelisation : exemples de conclusions fondamentales / adequations / non-adequations pour chaque




\subsection*{A Roadmap for an Operational Family of Models of Coevolution}{Vers des Modèles Opérationnels de Coevolution}



\bpar{
Towards operational Models : what is possible ; what is desirable ; etc.
As previously stated, one of our principal aims is the validation of the network necessity assumption, that is the differentiating point with a classic evolutive urban theory. To do so, toy-model exploration and empirical analysis will not be enough as hybrid models are generally necessary to draw effective and well validated conclusions. We briefly give an overview of planned work in the following, that will be the conclusion of this Memoire.
}{

}









%----------------------------------------------------------------------------------------

%\subsubsection*{Case Studies}{Cas d'étude}

%$\rightarrow$ potential application cases ?

%Currently we expect to work on the following case studies to build these hybrid models :

%\begin{itemize}
%\item Dynamical data for Bassin Parisien should allow to parametrize and calibrate a model at this temporal and spatial scale.
%\item On larger scales, South African dataset of \noun{Baffi} will along empirical analysis also be used to parametrize hybrid co-evolution models.
%\item A possibility that is not currently set up (and that may however be difficult because of a disturbing closed-data policy among a frightening large number of scientists !) is the exploitation of French railway growth dataset (with population dataset) used in~\cite{bretagnolle:tel-00459720}, that would also provide an interesting case study on other regimes, scales and transportation mode.
%\end{itemize}






