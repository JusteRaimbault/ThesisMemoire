





%\chapter*{Part III Conclusion}{Conclusion de la Partie III}
\chapter*{Conclusion de la Partie III}


% to have header for non-numbered introduction
\markboth{Conclusion}{Conclusion}


%\headercit{}{}{}


Cette partie a ainsi donné des premiers éléments d'exploration de différentes entrées sur la modélisation de la co-évolution. Nous avons exploré dans le chapitre~\ref{ch:macrocoevolution} un modèle de co-évolution à l'échelle macroscopique, qui permet l'isolation de nombreux régimes de causalité, qu'on peut alors nommer régimes de co-évolution pour ceux présentant des causalités circulaires, et qui est calibré sur le système de villes français. Nous montrons ainsi que des mécanismes et une représentation simple permettent déjà de capturer synthétiquement et empiriquement la co-évolution à cette échelle.

Nous avons ensuite exploré des modèles à une échelle plus grande, impliquant une complexité croissante. Un modèle de co-évolution par morphogenèse permet de coupler la forme urbaine (distribution de la population et topologie du réseau) à une abstraction des fonctions urbaines (mesures de centralité et d'accessibilité dans le réseau). Les différentes heuristiques d'évolution du réseau qui ont été testées se révèlent complémentaires pour s'approcher de configurations réelles. Enfin, nous avons introduit des pistes pour la prise en compte des processus de gouvernance dans l'évolution des réseaux de transport.




\subsection*{Processes in models}{Processus modélisés}

%\comment{justifier ici poruquoi pas modèle très fins sur processus eco par exemple (//Levinson) : prix à payer pour être accross scales, disciplines et avoir vraiment de la coevol ? pour ces premières étapes oui. à justifier}

Les modèles que nous avons développé l'ont été faits dans une logique de parcimonie, tout en cherchant à effectivement capturer des processus de co-évolution à différentes échelles en en s'encrant dans différentes disciplines : ces contraintes se paient par un prix en raffinement des mécanismes intégrés\footnote{Et n'incluent pas des processus économiques élaborés comme par exemple le modèle de~\cite{levinson2007co}.}. Ils remplissent toutefois leurs objectifs et couvrent un spectre relativement large de processus. Ceux-ci sont synthétisés en Table~\label{tab:partIII:modeled}. 


%%%%%%%%%%%%%%%%
\begin{table}
\begin{tabular}[6pt]{|p{4cm}|c|p{4cm}|c|}
\hline
Processus & Échelles & Concept & Modèles \\\hline
Attachement préférentiel/Gibrat  & Meso/Macro & Croissance urbaine & Morphogenèse/Interactions \\\hline
Diffusion/Etalement & Meso & Forme Urbaine & Morphogenèse \\\hline
Centralité de proximité/Accessibilité & Meso/Macro & Accessibilité & Morphogenèse/Interactions \\\hline
Flux direct & Macro & Interactions & Interactions\\\hline
Flux indirect/Effet tunnel/Centralité de Chemin & Meso/Macro & Effet de réseau & Morphogenèse/Interactions \\\hline
Proximité au réseau & Meso & Accessibilité & Morphogenèse \\\hline
Relocalisations actifs/emplois & Meso & Mobilité résidentielle & Lutecia\\\hline
Gouvernance des Transports & Meso & Gouvernance & Lutecia\\\hline
\end{tabular}
\caption[Processus taken into account in our models][Processus pris en compte dans les modèles]{\textbf{Processus taken into account in our models.}\label{tab:partIII:modeled}}{\textbf{Liste des différents processus pris en compte dans les modèles de co-évolution.}\label{tab:partIII:modeled}}
\end{table}
%%%%%%%%%%%%%%%%



%\comment{faire le même tableau pour les modèles existants : vue plus large de l'ensemble des processus. pour chacun de ces modèles et de nos modèles, lister tous les processus potentiels ; faire une typologie ensuite. Q : typologie différente d'une pure empirique ? a creuser, et peut être intéressant dans le cadre du knowledge framework, comme illustration coevol connaissances.}



\subsection*{A full view on co-evolution}{Une vue complète de la co-évolution}

% complementarite de la vision conceptuelle/empirique/modelisation : exemples de conclusions fondamentales / adequations / non-adequations pour chaque

Nous avons à ce stade apporté des éléments de réponse aux deux axes de notre problématique générale (comment définir et caractériser la co-évolution, et comment la modéliser). Il est remarquable de noter que ceux-ci s'articulent dans les trois domaines de connaissance du conceptuel (définition), de l'empirique (caractérisation) et de la modélisation (modèles). Ces trois aspects s'auto-génèrent l'un l'autre, et notre point de vue forme une véritable trinité, c'est à dire un concept à la fois unique et triple, dans lequel aucune des approches ne peut être ignorée.

Ainsi, les modèles contiennent l'aspect microscopique (interactions réciproques entre entités), et dans certains cas l'aspect statistique au niveau d'une population ; cette conclusion étant rendue possible par l'outil de caractérisation opérationnelle, celui-ci permettant par ailleurs de renforcer la pertinence de la définition.



\subsection*{A Roadmap for an Operational Family of Models of Coevolution}{Vers des modèles opérationnels de co-évolution ?}



\bpar{
Towards operational Models : what is possible ; what is desirable ; etc.
As previously stated, one of our principal aims is the validation of the network necessity assumption, that is the differentiating point with a classic evolutive urban theory. To do so, toy-model exploration and empirical analysis will not be enough as hybrid models are generally necessary to draw effective and well validated conclusions. We briefly give an overview of planned work in the following, that will be the conclusion of this Memoire.
}{
Les défis futurs qui s'ouvrent à la suite de notre travail sont nombreux et seront développés par la suite en~\ref{sec:contributions}, mais au regard du point précédent nous pouvons déjà mentionner la perspective de modèles opérationnels. Est-il possible de construire des modèles de prospection ou de planification similaires à ceux que nous avons mis en place ?
}

\bpar{}{
Tout d'abord, nous subodorons que de tels modèles seraient effectivement efficaces dans des cadres particuliers, notamment au regard des échelles de temps concernées et de la flexibilité du système urbain : une application au cas de la Chine où les potentialités sont réalisées rapidement et relativement rationnellement peut être plus crédible qu'une application à la Métropole du Grand Paris qui comme nous l'avons vu présente une forte complexité dans les processus d'évolution des infrastructures.
}

\bpar{}{
Par ailleurs, il faut garder à l'esprit la difficulté d'une prospective sur le temps long : il est impossible à notre connaissance d'intégrer dans un modèle de manière endogène des changements structurels des systèmes territoriaux\footnote{Au sens d'une \emph{transition} des systèmes de peuplement comme élaboré par~\cite{tannier:hal-01666491}.}. Ainsi, la transition de la révolution industrielle est exogène dans le modèle Simpop2. Dans notre cas des interactions entre réseaux et territoires, il est possible que des pratiques de mobilité actuellement inconcevables (par exemple mobilité partagée dans des flottes de véhicules électriques autonomes sur demande) changent totalement la donne et bouleversent la donne actuelle d'accessibilité aux transports en commun.
}


\bpar{}{
Enfin, comme évoqué en Chapitre~\ref{ch:modelinginteractions}, il est possible que ce type de modèle soit en fait indésirable du point de vue des acteurs produisant ou gérant les territoires, puisque leur rôle de scénarisation et de prospection territoriale serait grandement réduit par des hypothétiques modèles opérationnels.
}


\subsection*{Perspectives}{Perspectives}

Notre point de vue sur la co-évolution a bien entendu été réducteur et limité, puisque l'état actuel de nos modes de production de connaissance est encore loin d'une intégration paradigmatique de la complexité~\cite{morin1991methode}, et que toute tentative d'appréhension d'un système complexe combine habilement analyse et synthèse, réductionnisme et holisme, modularité et interdépendance. Afin d'enrichir notre point de vue, nous proposons dans la partie suivante une ouverture empirique et théorique.




%----------------------------------------------------------------------------------------

%\subsubsection*{Case Studies}{Cas d'étude}

%$\rightarrow$ potential application cases ?

%Currently we expect to work on the following case studies to build these hybrid models :

%\begin{itemize}
%\item Dynamical data for Bassin Parisien should allow to parametrize and calibrate a model at this temporal and spatial scale.
%\item On larger scales, South African dataset of \noun{Baffi} will along empirical analysis also be used to parametrize hybrid co-evolution models.
%\item A possibility that is not currently set up (and that may however be difficult because of a disturbing closed-data policy among a frightening large number of scientists !) is the exploitation of French railway growth dataset (with population dataset) used in~\cite{bretagnolle:tel-00459720}, that would also provide an interesting case study on other regimes, scales and transportation mode.
%\end{itemize}






