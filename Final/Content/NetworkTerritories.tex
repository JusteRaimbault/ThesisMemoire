


\newpage


%-------------------------------


\section{Territories and Networks}{Territoires et Réseaux}

\label{sec:networkterritories}


%-------------------------------


Nous commençons par une construction plus précise des concepts mobilisés, qui permet de comprendre comment les concepts de territoire et de réseau sont rapidement en interdépendance forte, impliquant une importance ontologique des interactions entre les objets correspondants. Nous verrons que les territoires impliquent l'existence de réseaux, mais que réciproquement ceux-ci les influencent également. Un développement plus particulier sur les propriétés des réseaux de transport permet d'amener progressivement une vision précise de la \emph{co-évolution}, que nous prendrons jusque là dans son sens préliminaire donné précédemment, c'est à dire l'existence de relations causales circulaires entre réseaux de transports et territoires.


\subsection{Territories and Networks : There and Back Again}{Territoires et Réseaux : \emph{There and Back Again}}

\comment[AB]{titre : ?}[(AB) reference !]\comment[FL]{traduire}


\subsubsection{Territories}{Territoires}


\bpar{
The notion of territory can be taken as a basis to explore the scope of geographical objects we will study. In Ecology, a territory corresponds to a spatial extent occupied by a group of agents or more generally an ecosystem. \emph{Human Territories} are far more complex in the sense of semiotic representations of these that are a central part in the emergence of societies. For \noun{Raffestin} in~\cite{raffestin1988reperes}, the so-called \emph{Human Territoriality} is the ``conjonction of a territorial process with an informational process'', what means that physical occupation and exploitation of space by human societies is not dissociable from the representations (cognitive and material) of these territorial processes, driving in return its further evolutions. In other words, as soon as social constructions are assumed in the constitution of human settlements, concrete and abstract social structures will play a role in the evolution of the territorial system, through e.g. propagation of information and representations, political processes, conjonction or disjonction between lived and perceived territory.
}{
Le concept\footnote{Nous utiliserons le terme \emph{concept} pour des connaissances construites, plutôt que celui de \emph{notion}, qui suivant~\cite{raffestin1978construits} est plus proche d'une information empirique. Cette distinction peut être mise en perspective avec les domaines de connaissance théorique et empirique de~\cite{livet2010}, que nous approfondissons en~\ref{sec:knowledgeframework}.} de \emph{Territoire}, que nous avons introduit précédemment par ceux de Ville et de Système de Ville, sera central à nos raisonnements et nécessite d'être approfondi et enrichi. En Ecologie Spatiale, un groupe d'agents ou plus généralement un écosystème occupe une certaine étendue spatiale~\cite{tilman1997spatial}, qu'on peut identifier comme notion de territoire. Les \emph{Territoires Humains}\comment[AB]{en sciences sociales on peut aussi dire ``territoire'' tout seul} impliquent des dimensions supplémentaires, par exemple par l'importance de leur représentations sémiotiques\footnote{c'est à dire des signes marquants les territoires et leur sens, mais aussi leur représentations, cartographiques par exemple}. Celles-ci jouent un rôle significatif dans l'émergence des constructions sociétales, dont la genèse est profondément liée à celle des systèmes urbains. Selon~\cite{raffestin1988reperes}, la \emph{Territorialité Humaine} est ``la conjonction d'un processus territorial avec un processus informationnel'', ce qui implique que l'occupation physique et l'exploitation de l'espace par les sociétés humaines sont complémentaires des représentations (cognitives et matérielles) de ces processus territoriaux, qui influent en retour sur leur évolution.
}

\bpar{}{
En d'autres termes, à partir de l'instant où les constructions sociales déterminent la constitution des établissements humains, les structures sociales abstraites et concrètes joueront un rôle dans l'évolution des territoires, et ces deux objets seront intimement liés. Des exemples de tels liens se retrouvent à travers la propagation d'informations et de représentations, par des processus politiques, ou encore par la correspondance plus ou moins effective entre territoire vécu et territoire perçu. Une illustration concrète est donnée par une vision simplifiée de la construction de la Métropole du Grand Paris, qui témoigne des ajustements successifs des territoires administratifs (émergence d'un nouveau niveau de gouvernance), des territoires fonctionnels (partiellement par l'évolution des possibilités d'accessibilité), des territoires perçus (dépassement de l'opposition Paris-banlieue), des territoires vécus (nouvelles pratiques de mobilité ou de mobilité résidentielle potentiellement\comment[AB]{partiellement} induites\comment[FL]{pourquoi cette vision deterministe ?} par les dynamiques territoriales et du nouveau réseau de transport), l'ensemble de ces processus étant liés de manière complexe et ceux-ci étant loin d'être systématiques. Un territoire est ainsi compris comme une structure sociale organisée dans l'espace, qui comprend ses artefacts concrets et abstraits. Une étendue spatiale imaginaire avec des ressources potentielles qui n'aurait jamais connu de contact avec l'humain ne pourra pas être un territoire si elle n'est pas habitée, imaginée, vécue, exploitée, même si ces ressources pourraient être potentiellement exploitée le cas échéant. En effet, ce qui est considéré comme une ressource (naturelle ou artificielle) dépendra de la société (par exemple de ses pratiques et de ses capacités technologiques).\comment[FL]{je ne suis pas sur que tu doives garder cette digression}
}



% complementarité du point de vue de la theorie evolutive : boucler la boucle ?

\bpar{
Although this approach does not explicitly give the condition for the emergence of a seminal system of aggregated settlements (i.e. the emergence of cities), it insists on the role of these that become places of power and of creation of wealth through exchange. But the city has no existence without its hinterland and the territorial system can not be summarized by its cities as a system of cities. There is however compatibility on this subsystem between \noun{Raffestin} approach to territories and \noun{Pumain}'s evolutive theory of urban systems~\cite{pumain2010theorie}, in which cities are viewed as an auto-organized complex dynamical systems, and act as mediators of social changes : for example, cycles of innovation occur within cities and propagate between them. Cities are thus competitive agents that co-evolve (in the sense given before). The territorial system can be understood as a spatially organized social structure, including its concrete and abstract artifacts. A imaginary free-of-man spatial extent with potential ressources will not be a territory if not inhabited, imagined, lived, and exploited, even if the same ressources would be part of the corresponding habited territorial system. Indeed, what is considered as a ressource (natural or artificial) will depend on the corresponding society (e.g. of its practices and technological potentialities).
}{
Cette approche du territoire est compatible avec la définition préliminaire que nous en avions prise, qui vient alors la renforcer. L'approche raffestinienne\comment[FL]{de Raffestin} insiste sur le rôle des villes comme lieu de pouvoir (au sens d'un lieu rassemblant des processus décisionnel et de contrôle socio-économique) et de création de richesse au travers des échanges et interactions\footnote{une interaction sera comprise dans son sens le plus général, comme une action réciproque de plusieurs entités l'une sur l'autre. Celle-ci peut être physique, informationnelle, transformer les entités, etc. Voir~\cite{morin1976methode} pour une construction complète et complexe du concept, en lien intime avec celui d'organisation.} (sociaux, économiques). La ville n'a cependant pas d'existence sans son hinterland, ce qu'on pourrait appeler le \emph{territoire d'une ville}\comment[FL]{pas convaincu}\footnote{Une correspondance exacte entre territoires et villes n'est probablement qu'une simplification de la réalité, puisque les territoires peuvent s'entremêler à différentes échelles, selon différentes dimensions. Une lecture par lieux centraux de type \noun{Christaller}~\cite{banos2011christaller} permet de se faire une image conceptuelle de cette correspondance. Des définitions fonctionnelles comme celles des aires urbaines de l'Insee, qui définit l'aire autour d'un pôle dépassant une taille critique (10000 emplois) par les communes dont un seuil minimal d'actifs travaillent dans le pôle (40\%) - voir \url{https://www.insee.fr/fr/metadonnees/definition/c2070}, est une approche possible. La sensibilité des propriétés du système urbain à ces paramètres est testée par~\cite{2015arXiv150707878C}. La définition de la ville est alors intimement liée à celle de ses territoires, et celle du système urbain à l'ensemble des territoires. \comment[FL]{a eviter cela cree plus de confusion que de clarté}}. Cette correspondance permet de lire l'ensemble des territoires au prisme du système de villes, comme développé par la Théorie Evolutive des Villes~\cite{pumain2010theorie}. Celle-ci interprète les villes comme des systèmes complexes dynamiques\comment[FL]{terme non defini} auto-organisés, qui agissent comme des médiateurs du changement social : par exemple, les cycles d'innovation s'initialisent au sein des villes et se propagent entre elles (voir~\ref{app:sec:patentsmining} pour une entrée empirique sur la notion d'innovation) : cela permet de commencer à concevoir\comment[FL]{formulation maladroite} le territoire comme un espace des flux, ce qui permettra d'introduire la notion de réseau comme nous le verrons plus loin. Les villes sont par ailleurs vues comme des agents compétitifs qui co-évoluent, ce qui permet de préfigurer également l'importance de la co-évolution pour les dynamiques territoriales.\comment[FL]{bibilographie ?}
}


% approche historique et point de vue complementaires

\bpar{

}{
Ces visions complémentaires du territoire peuvent également être enrichies par une perspective historique. \cite{di1998espace} procède à une analyse historique des différentes conceptions de l'espace (qui aboutissent entre autres à l'espace vécu, l'espace social et l'espace classique de la géographie) et montre comment leur combinaison forme ce que \noun{Raffestin} décrit comme territoires. \cite{giraut2008conceptualiser} rappelle les différents usages récents qui ont été faits de la notion de territoire, de la géographie culturelle où il a plus été utilisé par effet de mode, à la géopolitique où c'est un terme bien spécifique lié aux structures de gouvernance, en passant par des utilisations où il sert plus de concept, et dégage l'avantage\comment[FL]{? pourquoi un avantage ?} d'un objet interdisciplinaire capturant une certaine complexité des systèmes étudiés, ce qui confirme la pertinence de la notion dans notre cas.\comment[FL]{le but ne doit pas etre de tout explorer $\rightarrow$ il faut hierarchiser le propos}
}




% transition

\bpar{
A crucial aspect of human settlements that were studied in geography for a long time, and that relate with the previous notion of territory, are \emph{networks}. Let see how we can switch from one to the other and how their definition may be indissociable.
}{
Un aspect central des établissements humains qui a une longue tradition d'étude en géographie, et qui est directement relié au concept de territoire, est celui des \emph{réseaux}. Nous allons préciser leur définition et voir comment le passage de l'un à l'autre est intrinsèque aux approches que nous en prenons.
}



\subsubsection{Networks}{Réseaux}


% definition des réseaux de manière generale


Un \emph{réseau} doit être compris au sens large de motifs de connectivité\comment[FL]{phrase floue. motif ? connectivité ?} entre entités d'un système, qui peuvent être vus comment relations, liens, interactions. \cite{haggett1970network} postule que l'existence d'un réseau est nécessairement liée à celle de flux, et rappelle la représentation topologique sous forme de graphe de tout système géographique dans lequel circulent des flux entre des entités ou des lieux qui sont abstraits sous la forme de noeuds, reliés par des liens. L'analyse topologique révèle déjà un certain nombre de propriétés du système, mais \cite{haggett1970network} précise l'importance de la spatialisation du réseau, incluse dans les propriétés de ses noeuds (localisation) et de ses liens (localisation, impédance), pour la compréhension des dynamiques dans le réseau (flux) ou du réseau lui-même (croissance du réseau). Cette spécificité a été rappelée par~\cite{barthelemy2011spatial} qui met en perspective les domaines empiriques concernés par les réseaux spatiaux, certains modèles de croissance de réseau, et certains modèles de processus dans les réseaux (par exemple de diffusion).\comment[FL]{ce n'est pas clair du tout. qu'a rellement appris le lecteur ?}

\comment[AB]{Bien :)}

% Les territoires impliquent des réseaux potentiels, selon Dupuy


\bpar{
We paraphrase \noun{Dupuy} in~\cite{dupuy1987vers} when he proposes elements for ``a territorial theory of networks'' based on the concrete case of Urban Transportation Networks. This theory sees \emph{real networks} (i.e. concrete networks, including transportation networks) as the materialization of \emph{virtual networks}. More precisely, a territory is characterized by strong spatio-temporal discontinuities induced by the non-uniform distribution of agents and ressources. These discontinuities naturally induce a network of ``transactional projects'' that can be understood as potential interactions between elements of the territorial system (agents and/or ressources). For example today, people need to access the ressource of employments, economic exchanges operate between specialized production territories.
}{
Pour approfondir le concept de réseau en appuyant sur sa forte interdépendance avec celui de Territoire, nous reprenons~\cite{dupuy1987vers} qui propose des éléments pour une ``théorie territoriale des réseaux'' s'inspirant du cas concret d'un réseau de transport urbain. Cette théorie distingue les \emph{réseaux réels} (auxquels appartiennent une catégorie qu'on peut désigner comme réseaux concrets, matériels ou physiques - nous utiliserons ces termes de manière interchangeable par la suite, à laquelle les réseaux de transport appartiennent; d'autres catégories comme les réseaux sociaux sont également des réseaux réels sur lesquels nous ne nous attarderons pas) et les \emph{réseaux virtuels}, eux-mêmes induits entre autre par la configuration territoriale. Les réseaux réels sont la matérialisation de réseaux virtuels. Plus précisément, un territoire est caractérisé par de fortes discontinuités spatio-temporelles induites par la distribution non-uniforme des agents et des ressources. Ces discontinuités induisent naturellement un réseau d'interactions potentielles entre les éléments du système territorial, notamment des agents et des ressources. \cite{dupuy1987vers} désigne ces interactions potentielles comme \emph{projets transactionnels}.\comment[FL]{la encore au bout de plusieurs lignes dont on ne comprend plus l'objet, il est difficile de suivre} Celles-ci induisent la notion de \emph{potentiel d'interaction}, c'est à dire une propriété de l'espace dont les interactions dérivent\footnote{Etant donné tout champ vectoriel de classe $\mathcal{C}^1$ sur $\mathbb{R}^3$, le théorème d'\noun{Helmoltz} fournit un potentiel vecteur et un potentiel scalaire dont ce champ dérive par rotationnel\comment[AB]{rotation ?}[non, bien $\vec{rot}\vec{A}$ - yes preciser que bien un cas bien particulier] et gradient. Cela justifie \comment[AB]{d'un point de vue formel} le passage d'un champ d'interaction entre agents à un champ de potentiel.}. Par exemple, de nos jours les actifs se doivent\comment[FL]{mal dit} d'accéder à la ressource qu'est l'emploi, et des échanges économiques s'effectuent entre les différents territoires spécialisés\comment[FL]{il ne le sont pas toujours (specialise)} dans les productions de différents types. Une distribution spatiale d'agents suffit à introduire des interactions potentielles et donc un champ de potentiel.\comment[FL]{phrase a supprimer}
}




\subsubsection{Real networks}{Réseaux réels}

% Les réseaux potentiels se transforment en réseaux réels sous certaines conditions.
%. -> effet des territoires sur les réseaux

\bpar{
The potential interaction network is concretized as offer adapts to demand, and results of the combination of economic and geographical constraints with demand patterns, in a non-linear way through agents designed as \emph{operators}. This process is not immediate, leading to strong non-stationarity and path-dependance effects : the extension of an existing network will depend on previous configuration, and depending on involved time scales, the logic and even the nature of operators may have evolved. \noun{Raffestin} points out in his preface of~\cite{offner1996reseaux} that a geographical theory articulating space, network and territories had never been consistently formulated. It appears to still be the case today, but the theory developed just before is a good candidate, even if it stays at a conceptual level. The presence of a human territory necessarily imply the presence of abstract interaction networks and concrete networks used for transportation of people and ressources (including communication networks as information is a crucial ressource). Depending on regime in which the considered system is, the respective role of different networks may be radically different. Following \noun{Duranton} in \cite{duranton1999distance}, pre-industrial cities were limited in growth because of limitations of transportation networks. Technological progresses have lead to the end of these limitations and the preponderance of land markets in shaping cities (and thus a role of transportation network as shaping prices through accessibility), and recently to the rising importance of telecommunication networks that induce a ``tyranny of proximity'' as physical presence is not replaceable by virtual communication.
}{
Il existe des cas où un réseau potentiel se matérialise en réseau réel.\comment[FL]{d'une certaine maniere OK, mais attention a la simplification} La question sous-jacente est alors de savoir si le champ de potentiel des territoires est en partie à l'origine de cette matérialisation, si celle-ci est totalement indépendante, ou si la dynamique des deux est fortement couplée, en d'autres termes en co-évolution. La matérialisation résultera généralement de la combinaison de contraintes économiques et géographiques avec des motifs de demande, de manière non-linéaire. Un tel processus est loin d'être immédiat, et conduit à de forts effets de non-stationnarité et de dépendance au chemin\footnote{La non-stationnarité spatiale consiste en la dépendance de la structure de covariance des processus à l'espace, tandis que la dépendance au chemin traduit le fait que les trajectoires prises par le passé influencent fortement les trajectoires actuelles du système.} : l'extension d'un réseau existant dépendra de la configuration précédente, et selon les échelles de temps impliquées, la logique et même la nature des opérateurs, c'est à dire des agents participant à sa production, peut avoir évolué. Les exemples de trajectoires concrètes peuvent être très variées : \comment[FL]{TB} \cite{kasraian2015development} montre par exemple dans le cas de la Randstad sur le temps long, une première période pendant laquelle le réseau ferré s'est développé pour suivre le développement urbain, tandis que des effets inverses ont été constaté plus récemment. A une échelle urbaine sur le temps long, la dépendance au chemin est montrée pour Boston par~\cite{block2012hysteresis} puisque l'environnement bâti et la distribution de la population sont montrés fortement dépendants des lignes de tramway passé\comment[FL]{orth?} même lorsqu'elles n'existent plus.\comment[FL]{c'est un point tres important, tu dois mieux permettre au lecteur de rentrer dans ce raisonnement} Ainsi, l'existence d'un territoire humain implique nécessairement\comment[FL]{s'appuie sur} la présence de réseaux d'interactions abstraites et de réseaux concrets utilisés pour transporter les individus et les ressources (incluant les réseaux de communication puisque l'information est une ressource essentielle\comment[FL]{affirmation gratuite}), mais les processus d'établissement de ceux-ci sont difficiles à identifier de manière générale. Nous insistons sur l'importance de notre choix ontologique, qui par positionnement dans la théorie de \noun{Dupuy}, induit dans la construction des objets même une imbrication complexe entre ceux-ci.\comment[FL]{tres flou. pas du niveau chap 1 a mon avis} La justification de  ce choix théorique sera progressive dans l'ensemble de nos développements par la suite.\comment[FL]{? pourquoi le dire ?}
}




% le contexte socio-eco/techno conditionne fortement la facon dont les réseaux agissent sur les territoires.

\bpar{
}{
Le statut du réseau par rapport au territoire est d'autre part fortement conditionné par le contexte socio-économique et technologique. Selon \noun{Duranton}~\cite{duranton1999distance}, un facteur influençant la forme des villes pré-industrielles était la performance des réseaux de transport. Les progrès technologiques, conduisant à une baisse des coûts de transport, ont induit un changement de régime, ce qui a mené à une prépondérance du marché foncier dans la formation des villes (et par conséquent un rôle des réseaux de transport qui déterminent les prix par l'accessibilité), et plus récemment à une importance croissante des réseaux de télécommunication ce qui a induit une ``tyrannie de la proximité'' puisque la présence physique n'est pas remplaçable par une communication virtuelle~\cite{duranton1999distance}. On s'attendra ainsi à l'existence de multiples processus d'interaction, potentiellement superposables de manière complexe.\comment[FL]{creux pour l'instant}
}



% Transition

\bpar{
This territorial approach to networks seems natural in geography, since networks are studied conjointly with geographical objects with an underlying theory, in opposition to network science that studies brutally spatial networks with few thematic background~\cite{ducruet2014spatial}.
}{
Cette approche territoriale des réseaux semble naturelle en géographie, puisque les réseaux sont étudiés conjointement avec des objets géographiques qu'ils connectent, en opposition aux travaux théoriques sur les réseaux complexes qui les étudient de manière relativement déconnectée de leur fond thématique~\cite{ducruet2014spatial}.
}





\subsubsection{Networks shaping territories ?}{Des réseaux qui façonnent les territoires ?}

% Effets des réseaux sur les territoires ? -> approfondir le debat des effets structurants

\bpar{
However networks are not only a material manifestation of territorial processes, but play their part in these processes as they evolution may shape territories in return. In the case of \emph{technical networks}, an other designation of real networks given in~\cite{offner1996reseaux}, many examples of such feedbacks can be found : the interconnectivity of transportation networks allows multi-scalar mobility patterns, thus shaping the lived territory. At a smaller scale, changes in accessibility may result in an adaptation of a functional urban space. Here emerges again an intrinsic difficulty : it is far from evident to attribute territorial mutations to some network evolutions and reciprocally materialization of a network to precise territorial dynamics. Coming back to Diderot should help, in the sense that one must not consider network nor territories as independent systems that would have causal relationships but as strongly coupled components of a larger system. These potential retroactions of networks on territories does not necessarily act on concrete components : \noun{Claval} shows in~\cite{claval1987reseaux} that transportation and communication networks contribute to the collective representation of territories by acting on territorial belonging feeling.
}{
Cependant les réseaux ne sont pas seulement une manifestation matérielle de processus territoriaux, mais jouent également leur rôle dans ces processus comme leur évolution peut influencer l'évolution des territoires en retour. Il emerge alors une difficulté intrinsèque : il n'est pas évident d'attribuer des mutations territoriales à une évolution du réseau et réciproquement la matérialisation d'un réseau à des dynamiques territoriales précises.\comment[FL]{dire aussi qu'il y a d'autre choses (prix energie, techno existante, etc)} Dans le cas des \emph{réseaux techniques}, une autre désignation des réseaux réels (au sens pris plus haut) donnée dans~\cite{offner1996reseaux}, de nombreux exemples de tels retroactions peuvent être mis en évidence : une accessibilité accrue peut être un facteur favorisant la croissance urbaine, ou bien l'interconnexion des réseaux de transport permet des motifs de mobilité multi-échelles\comment[FL]{sens ?} formant ainsi le territoire vécu.\comment[FL]{pas pertinent de parler de cela je pense} A une plus petite échelle, des changements de l'accessibilité peuvent induire l'adaptation d'un espace fonctionnel urbain.\comment[FL]{sens ?} Ces retroactions potentielles\comment[FL]{suppr} des réseaux sur les territoires n'agissent pas nécessairement sur des composantes concretes : \noun{Claval} montre dans~\cite{claval1987reseaux} que les réseaux de transport et de communication contribuent à la représentation collective d'un territoire en agissant sur un sentiment d'appartenance, qui peut alors jouer un rôle crucial dans l'émergence d'une dynamique régionale fortement cohérente. Développons d'abord plus en détail les possibles influences des réseaux sur les territoires.
}


\bpar{
The confusion on possible simple causal relationships has fed a scientific debate that is still active. Methodologies to identify so-called \emph{structural effects} of transportation networks were proposed by planners in the seventies~\cite{bonnafous1974detection,bonnafous1974methodologies}.
}{
La confusion autour de possibles relations causales simples a nourri un débat scientifique encore actif aujourd'hui. La question sous-jacente repose sur des attributions plus ou moins déterministes d'impact d'infrastructures ou d'un nouveau mode de transport sur des transformations territoriales. On peut trouver des précurseurs de ce raisonnement dès les années 1920 : \noun{McKenzie}, de l'école de Chicago, parle dans~\cite{burgess1925city} des `` modifications des formes du transport et de la communication comme facteurs déterminants des cycles de croissance et de déclin [des territoires]'' (p. 69). Des méthodologies pour identifier ce qui est alors nommé \emph{effets structurants} des réseaux de transport ont été développées pour la planification dans les années 1970 : \cite{bonnafous1974methodologies} situe le concept d'effet structurant dans le cadre d'une logique d'utilisation de l'offre de transport comme outil d'aménagement (les alternatives étant le développement d'une offre pour répondre à une congestion du réseau, et le développement simultané d'une offre et d'un aménagement associé). Ces auteurs identifient du point de vue empirique des effets directs d'une nouvelle offre sur le comportement des agents, sur les flux de transport et des possibles inflexions sur les trajectoires socio-économiques des territoires concernés. \cite{bonnafous1974detection} développe une méthode pour identifier de tels effets par modifications de la classe des communes dans une typologie établie a posteriori.
}


\bpar{
It took some time for a critical positioning on unreasoned and decontextualized use of these methods by planners and politics generally to technocratically justify transportation projects, that was first done by \noun{Offner} in~\cite{offner1993effets}. Recently the special issue~\cite{espacegeo2014effets} on that debate recalled that on the one hand misconceptions and misuses were still greatly present in operational and planning milieus as~\cite{crozet:halshs-01094554} confirmed, and on the other hand that a lot of scientific progresses still need to be made to understand relations between networks and territories as \noun{Pumain} highlights that recent works gave evidence of systematic effects on very long time scales (as e.g. the work of \noun{Bretagnolle} on railway evolution, that shows a kind of structural effect in the necessity of connectivity to the network for cities to ``stay in the game'', but that is not fully causal as not sufficient).
}{
Selon \cite{offner1993effets}\comment[FL]{j'ai peine a croire que ce soit le premier}, il s'est par la suite développé un usage non raisonné et hors contexte de ces méthodes par les planificateurs et les politiques qui les mobilisaient généralement pour justifier des projets de transports de manière technocratique : argumentant\comment[FL]{md} d'un effet direct d'une nouvelle infrastructure sur le développement local (par exemple économique), les élus sont en mesure de demander des financements et de justifier leur action auprès des contribuables. \cite{offner1993effets} est alors la première contribution à proposer un positionnement critique et à remettre en cause ces dérives\comment[FL]{?}. Une édition spéciale de l'Espace Géographique sur ce débat~\cite{espacegeo2014effets} a rappelé d'une part que l'instrumentalisation\comment[FL]{sens ?} était encore largement présents aujourd'hui dans les milieux opérationnels de la planification, ce qui peut s'expliquer par exemple par le besoin de justifier l'action publique, et d'autre part qu'une compréhension scientifique en profondeur\comment[FL]{sens ? peux tu definir ce qui est profond et ce qui ne l'est pas ?} des relations entre réseaux et territoires est encore à construire.
}


\bpar{}{
Une illustration concrète d'actualité permet de se faire une image de cette instrumentalisation : les débats en juillet 2017 relatifs à l'ouverture des LGV Bretagne et Sud-Ouest ont montré toute l'ambiguïté des positions, des conceptions, des imaginaires à la fois des politiques mais aussi du public : inquiétude quant à la spéculation dans les quartiers de gare, questionnements des pratiques de mobilité quotidienne mais aussi sociale\footnote{voir par exemple \url{http://www.liberation.fr/futurs/2017/07/02/immobilier-plus-de-parisiens-comment-les-bordelais-voient-l-arrivee-de-la-lgv_1580776}, ou \url{http://www.lemonde.fr/big-browser/article/2017/10/24/a-bordeaux-une-fronde-anti-parisiens-depuis-l-ouverture-de-la-ligne-a-grande-vitesse_5205282_4832693.html} pour une réaction ``à chaud'' de divers acteurs locaux, témoignant d'un impact au minimum sur les représentations.}\comment[FL]{tu ne dis rien ! alors autant ne pas en parler}. La complexité et la portée des sujets montrent bien la difficulté d'une compréhension systématique d'effets du transport sur les territoires.
}





\subsubsection{Territorial Systems}{Systèmes Territoriaux}


\bpar{
This detour from territories, to networks and back again, allows us to give a preliminary definition of a territorial system that will be the basis of our following theoretical considerations. As we emphasized the role of networks, the definition takes it into account.
\textbf{Preliminary Definition.} \textit{A territorial system is a human territory to which both interaction and real networks can be associated. Real 
 networks are a component of the system, involved in evolution processes, through multiples feedbacks with other components at various spatial and temporal scales.}
 This reading of territorial systems is conditional to the existence of networks and may discard some human territories, but it is a deliberate choice that we justify by previous considerations, and that drives our subject towards the study of interactions between networks and territories.
}{
Cet aperçu introductif, des territoires aux réseaux, nous permet ainsi de clarifier notre approche des systèmes territoriaux qui sera sous-jacente dans l'ensemble de la suite. Une prise en compte des diverses rétroactions potentielles des réseaux pour la compréhension des territoires est suggérée par un retour à la citation de Diderot ayant introduit le sujet devrait aider à ce point, au sens où il ne faut pas considérer le réseau ni les territoires comme des systèmes indépendants qui s'influenceraient soit l'une soit l'autre par des relations causales, mais comme des composantes fortement couplées d'un système plus large, et donc étant en relations causales circulaires.\comment[FL]{c'est au coeur de ta demarche. mieux expliquer (et nuancer)} Comme nous avons mis en exergue le rôle des réseaux dans de nombreux aspects des dynamiques territoriales, nous proposons une définition des systèmes territoriaux les incluant explicitement.
}

\bpar{}{
Nous considérons un \emph{Système Territorial} comme un \emph{territoire humain qui contient à la fois des réseaux d'interactions et des réseaux réels}. Les réseaux réels, et plus particulièrement les réseaux concrets\comment[FL]{$\neq$ entre les 2 ?}, sont une composante à part entière du système, jouant dans les processus d'évolution, au travers de multiples retroactions avec les autres composantes à plusieurs échelles spatiales et temporelles. Cette lecture des systèmes territoriaux est conditionnée à l'existence des réseaux\comment[FL]{je penche pour supprimer ce genre de phrase} et pourrait écarter certains territoires humains\footnote{Quoique nous doutions de cette affirmation et soyons convaincus \emph{qu'il n'existe de territoire humain sans réseau d'interaction}, il est évidemment impossible de prouver cette assertion.\comment[FL]{digression inutile, de plus il existe beaucoup de territoires humains sans reseaux : une chambre, un terrain de foot, un desert, etc.}}, mais il s'agit d'un choix délibéré justifié par les considérations précédentes, et qui confirme le positionnement de notre sujet vers l'étude des interactions entre réseaux et territoires.
}

\bpar{}{
Le réseau n'est pas nécessairement une composante en tant que telle du territoire, mais bien du \emph{Système Territorial} en notre sens\footnote{Ce choix ontologique n'est pas anodin et appuie la dialectique entre réseaux et territoires. Partant de l'époque lointaine où les réseaux physiques n'existaient pas, l'émergence d'un territoire humain, que nous supposons équivalent à un réseau d'interactions, induit la mise en place de la dialectique diachronique complexe entre réseaux physiques et territoires humains. On peut ainsi lire la genèse du système territorial comme une boucle morinienne~\cite{morin1976methode}, dans laquelle on entre par le territoire initial puis qui se boucle du réseau physique aux composantes territoriales pour former le système territorial (donc le territoire dans la majorité des cas) de la manière récursive suivante :\\Territoire initial $\rightarrow$ Territoire $=$ \tikzmark{Configuration} territoriale$\rightarrow$ Réseau \tikzmark{physique}\arrow{physique}{Configuration}\\}. Cette vision rejoint le positionnement de \comment{Dupuy systemes territoires reseaux}. Notons le raccourci sémantique pour désigner les composantes du système territorial qui ne sont pas les réseaux et qui interagissent avec celui-ci, par le terme de territoire. Celles-ci dépendent des ontologies et des échelles considérées, comme nous le verrons par la suite, et peuvent aller des agents microscopiques aux villes elle-mêmes. Comme nous le verrons aussi par la suite (voir~\ref{sec:modelingsa}), il existe des paradigmes où ce raccourci n'est pas fait, comme dans le cas particulier des interactions entre transport et usage du sol ou les entités sont spécifiques. Mais il est fait si on reste à un cadre plus général, comme en témoigne l'un des ouvrages de référence sur le sujet~\cite{offner1996reseaux}\footnote{Lorsque \comment[JR]{cit. moprhogen reseaux dans ouvrage Dupuy} propose un modèle conceptuel de morphogenèse des réseaux, il désigne les composantes territoriales par ``Le Monde'', ce qui n'apporte pas de solution au problème sémantique. Le parti pris de garder le territoire, au sein du territoire, suggère une récursivité, et donc une complexité dans la générativité du système~\cite{morin1976methode}. La mobilisation du concept de morphogenèse à partir du Chapitre~\ref{ch:morphogenesis} suggère que cette récursivité serait plus que fortuite, mais bien intrinsèque au problème.}. Nous assumerons également ce raccourci de langage, en désignant par \emph{interactions entre réseaux et territoires} ou \emph{co-évolution entre réseaux et territoires}, les interactions ou la co-évolution entre les réseaux physiques et les composantes qu'ils relient, au sein d'un système territorial et donc d'un territoire.
}




\subsection{Transportation Networks}{Réseaux de Transport}


Nous précisons à présent le cas particulier des réseaux de transport et développons des concepts spécifiques associés qui joueront un rôle prépondérant dans la précision de notre problématique.


\subsubsection{Specificity of transportation networks}{Spécificités des réseaux de transport}


\bpar{
}{
Centraux aux discussions déjà évoquées sur les effets structurants des réseaux, les réseaux de transports\comment[FL]{aux discussions ou aux processus ?} jouent un rôle significatif dans l'évolution des territoires, mais il n'est évidemment pas question de leur attribuer des effets causaux déterministes. On parlera de manière générale de réseau de transport pour désigner l'entité fonctionnelle permettant un déplacement des agents et des ressources au sein et entre les territoires\footnote{On désigne ainsi à la fois l'infrastructure, mais aussi ses conditions d'exploitation, le matériel roulant, les agents exploitants.}. Même si d'autres types de réseaux sont également fortement impliqués dans l'évolution des systèmes territoriaux (voir par exemple les débats sur l'impact des réseaux de communication sur la localisation des activités économiques), les réseaux de transport conditionnent d'autres types de réseaux (logistique, échanges commerciaux, interactions sociales concrètes pour donner quelques exemples) et sont une entrée privilégiée en rapport aux motifs d'évolution territoriale, en particulier dans nos sociétés contemporaines pour lesquelles les réseaux de transport jouent un rôle privilégié~\cite{bavoux2005geographie}. Nous nous concentrerons ainsi par la suite uniquement sur les réseaux de transport.
}


\bpar{
Already evoked in relation to the question of structural effects of networks, transportation networks play a determining role in the evolution of territories. Although other types of networks are also strongly involved in the evolution of territorial systems (see e.g. the discussions of impacts of communication networks on economic activities), transportation networks shape many other networks (logistics, commercial exchanges, social concrete interactions to give a few) and are prominent in territorial evolution patterns, especially in our recent societies that has become dependent of transportation networks~\cite{bavoux2005geographie}. The development of French High Speed Rail network is a good illustration of the impact of transportation networks on territorial development policies. Presented as a new era of railway transportation, a top-down planning of totally novel lines was introduced as central for developments~\cite{zembri1997fondements}. The lack of integration of these new networks with existing ones and with local territories is now observed as a structural weakness and negative impacts on some territories have been shown~\cite{zembri2008contribution}. A review done in~\cite{bazin2011grande} confirms that no general conclusions on local effects of High Speed lines connection can be drawn although it keeps a strong place in imaginaries. These are examples of how transportation networks have both direct and indirect impacts on territorial dynamics.
}{
Le développement du réseau français à grande vitesse est une illustration du rôle des réseaux de transport sur les politiques de développement territorial. Présenté comme une nouvelle ère de transport sur rail, une planification au niveau de l'Etat de lignes totalement nouvelles et relativement indépendantes\comment[FL]{non, justement c'est interpretable} de par leur vitesse deux fois plus élevée, a été défendu par les acteurs politiques entre autres comme central pour le développement~\cite{zembri1997fondements}. L'articulation faible\comment[FL]{discutable} de ces nouveaux réseaux avec le réseau classique et avec les territoires locaux est à présent observé comme une faiblesse structurelle (c'est à dire conséquence de la structure du réseau tel qu'il a été planifié dans le Schéma Directeur de 1990), et des impacts négatifs sur certains territoires, comme par la suppression de dessertes intermédiaires sur les lignes classiques empruntées par le TGV, qui contribue à un accroissement de l'effet tunnel\footnote{L'effet tunnel désigne le processus de télescopage du territoire traversé par une infrastructure, celle-ci n'étant utilisable à partir de celui-ci.} ont été montrés~\cite{zembri2008contribution}. Une revue faite dans~\cite{bazin2011grande} confirme qu'aucune conclusion générale sur des effets locaux d'une connection à une ligne à grande vitesse ne peut être tirée, bien que ce sésame garde une place conséquente dans les imaginaires des élus.\comment[FL]{mais pe des conclusions particulieres ? Lille Europe incr. ; Dijon decr.} Le développement des différentes Lignes à Grande Vitesse s'inscrit dans des contextes territoriaux très différents, et il est dans tous les cas délicat d'interpréter des processus en les sortant de leur contexte\comment[FL]{la encore tu vas trop vite : tout le monde ne connait pas par coeur tous ces projets} : par exemple, les lignes LGV Nord et LGV Est s'inscrivent dans des échelles européennes plus vastes que la LGV Bretagne ouverte en juillet 2017. Les effets de l'ouverture d'une ligne peuvent s'étendre au delà des seuls territoires directement concernés : \cite{l2014contribution} montre par l'utilisation d'indicateurs issus de la \emph{Time Geography}\footnote{La \emph{Time Geography}, introduite par le géographe suédois \noun{T. Hägerstrand}, s'intéresse majoritairement aux trajectoires des individus dans le temps et l'espace, et de leurs implications dans les interactions avec l'environnement~\cite{chardonnel2007time}.} (mesurant une quantité de temps de travail disponible dans le cadre d'un aller-retour journalier) que la ligne Tours-Bordeaux a des répercussions potentielles dans le Nord et l'Est de la France. Ces exemples illustrent la manière dont les réseaux de transport peuvent avoir des effets à la fois directs et indirects, positifs ou négatifs, et à différentes échelles, ou bien aucun effet sur les dynamiques territoriales.
}





\subsubsection{The question of scales}{La question des échelles}

La question des échelles temporelles et spatiale concernées a été jusqu'ici abordée de manière auxiliaire aux concepts introduits. Nous proposons à présent de les intégrer de manière structurelle à notre raisonnement, c'est à dire guidant le développements de nouveaux concepts. Ainsi, les concepts de \emph{Mobilité}, d'\emph{Accessibilité}\comment[FL]{accessibilite me semble multi-scalaire}, puis de \emph{Dynamique structurelle sur le temps long}, correspondent chacun à des échelles de temps et d'espace décroissantes : intra-urbain et journalier, métropolitain et décennal, régional (au sens large et flexible de la portée d'un système de villes) et centennal. La correspondance que nous postulons ici entre échelles de temps et échelles d'espace, loin d'être évidente, sera montrée lors du développement de chacun de ces concepts. Par contre, la prise en compte d'échelles multiples est importante, comme le montre \cite{RIETVELD1994329} par une revue des approches économiques des interactions, qui appuie la différence entre l'intra-urbain et l'intra-régional.\comment[FL]{a developper}



%On sait que sur des échelles de temps relativement courtes allant de l'année à la dizaine d'année, les effets observés sur les mobilités quotidiennes et mobilités résidentielles peuvent être significatifs. 

% Nous proposons à présent de détailler ces concepts, toujours dans une logique de raffinement progressif de notre cadre ontologique.


\subsubsection{Transportation and Mobility}{Transports et Mobilité}


\bpar{
}{
La notion de mobilité et l'ensemble des approches associées, capturent en partie nos questionnements à grande échelle. Nous définirons la mobilité de manière générale comme un déplacement d'agents\comment[FL]{dans l'espace ?} territoriaux. Elle relève des motifs d'utilisation des réseaux de transport. \cite{hall2005reconsidering} introduit un cadre théorique permettant une typologie des pratiques de mobilité. En particulier, il montre une décroissance rapide de la fréquence des déplacements avec la portée spatiale et la durée, et donc que les motifs ``micro-micro''\comment[FL]{pourquoi appeler ca micro-micro ?} (journalier intra-urbain), qu'on désigne par \emph{mobilité quotidienne}, sont majoritaires. Cela ne signifie pas pour autant une absence de lien avec d'autres échelles : d'une part les motifs de mobilité sont très fortement conditionnés par la distribution des activités comme l'illustre~\cite{lee2015relating}, mais également corrélés à la structure sociale~\cite{camarero2008exploring}, qui évoluent tous deux à des échelles de temps d'un ordre différent (supérieur à la dizaine d'année, donc au moins un ordre de grandeur de différence). Ainsi, infrastructure et superstructure déterminent pratiques de mobilité, donnant un rôle important aux réseaux de transports dans celle-ci.\comment[FL]{est il necessaire d'evoquer les mobilites ici ?}
}


\bpar{}{
Réciproquement, les motifs d'utilisation des réseaux de transport sont le produit des dynamiques de mobilité quotidiennes, et ceux-ci s'y adaptent, tout en induisant des relocalisations des actifs et emplois : il existe une co-évolution entre transports et composantes territoriales aux échelles microscopiques et mesoscopiques, qui sont un objet d'étude à part entière. Par exemple, \cite{fusco2004mobilite} révèle une relation causale\comment[FL]{pas convaincu par la causalite dans son papier} de la mobilité sur la structure urbaine, l'offre d'infrastructure et ses propriétés ayant cependant des effets simultanément sur la mobilité et sur la structure urbaine. Dans le cas des réseaux autoroutiers, \cite{faivr2003} rappelle la nécessité de construire un cadre d'analyse dépassant la logique des effets structurants sur le temps long, et montre également des interactions à petite échelle propres à la mobilité sur lesquelles des conclusions plus systématiques peuvent être établies, comme une évolution des pratiques de mobilité impliquant une utilisation différente du réseau de transport. Nous avons donc à grande échelle une première interdépendance forte entre réseaux de transports et territoires, une première échelle de co-évolution.\comment[FL]{schemas/exemples aideraient ici}
%\comment[FL]{ce sont des points importants : ne pas etre aussi elliptique}
}


\bpar{
The mystification of the notion of \emph{mobility} was shown by \noun{Commenges} in~\cite{commenges:tel-00923682}, which proved than most of debates on modeling mobility and corresponding notions were mostly made-of by transportation administrators of \emph{Corps des Ponts} who roughly imported ideas from the United States without adaptation and reflexion fit to the totally different French context.
}{
Enfin, il est important de garder à l'esprit la forte contingence des concepts mobilisés ici. La co-construction du concept de mobilité et des solutions techniques modélisant celle-ci dans un but opérationnel, a été montrée\comment[FL]{simplificateur} par~\cite{commenges:tel-00923682} pour le contexte français, qui révèle entre autre une application peu adaptée au contexte français de cadres et méthodes importés des Etats-Unis. Cette contingence signifie que le choix des concepts même est emprise de sens ontologique dépassant largement celui qu'on peut leur attribuer directement\comment[FL]{un peu flou...}, et une inscription systémique implicite globale dans le \emph{Système Territorial}.
}

% Mobility as a service https://maas-alliance.eu/homepage/what-is-maas/



\subsubsection{Transportation and Accessibility}{Transports et Accessibilité}


% lien entre transports et accessibilite ; potentielle implication de l'accessibilite dans les transformations territoriales.

\bpar{
Reformulate positioning. \cite{miller1999measuring} on three different way to approach accessibility : time-geography and constraints, user utility based measures, and transportation time. It derives measures for each in perspective of \noun{Weibull}'s axiomatic frameworks and reconcile the three in a way.
The notion of accessibility comes rapidly when considering transportation networks. Based on the possibility to access a place through a transportation network (including transportation speed, difficulty of travel), it is generally described as a potential of spatial interaction\footnote{and often generalized as \emph{functional accessibility}, for example employments accessible for actives at a location. Spatial interaction potentials ruling gravity law can also been understood this way.}~\cite{bavoux2005geographie}. This object is often used as a planning tool or as an explicative variable of agents localisation for example. One has to be however careful on its unconditional use. More precisely, it may be a construction that misses a consistent part of territorial dynamics. Accessibility may be such a social construct and have no theoretical root since it is mostly a modeling and planning tool. Recent debates on the planification of \emph{Grand Paris Express}~\cite{confMangin}, a totally novel metropolitan transportation infrastructure planned to be built in the next twenty years, have revealed the opposition between a vision of accessibility as a right for disadvantaged territories against accessibility as a driver of economic development for already dynamic areas, both being difficultly compatible since corresponding to very different transportation corridors. Such operational issues confirm the complexity of the role of transportation networks in the dynamics of territorial systems, and we shall give in our work elements of response to a definition of accessibility that would integrate intrinsic territorial dynamics.
}{
Le concept d'\emph{Accessibilité} est fondamental pour notre question, puisqu'il se positionne à la croisée même des réseaux et des territoires. Basée sur la possibilité d'accéder un lieu par un réseau de transport (pouvant prendre en compte la vitesse, la difficulté de se déplacer), elle est généralement définie comme un potentiel d'interaction spatiale\footnote{et souvent généralisée comme une \emph{accessibilité fonctionnelle}, par exemple les emplois accessibles aux actifs d'un lieu. Les potentiels d'interaction spatiale s'exprimant dans les lois gravitaires peuvent aussi être compris de cette façon.}~\cite{bavoux2005geographie}. Elle a été introduite sous cette forme initialement par~\cite{hansen1959accessibility}, dans un but d'application à la planification. Diverses formulations et formalisations d'indicateurs correspondants ont été proposées, tout en pouvant être liées implicitement\comment[AB]{reformuler} par un même cadre théorique \comment[FL]{source Hansen 1959 ? Autre ?} : \cite{miller1999measuring} développe une approche axiomatique, c'est à dire proposant de la caractériser à partir d'un nombre minimal d'hypothèses fondamentales (les axiomes), pour unifier trois façons\comment[FL]{pourquoi voudrait-on unifier ? ta demarche intellectielle n'est aps forcement comprehensible de ton lecteur : il faut expliciter} de comprendre l'accessibilité. Celles-ci sont respectivement celle basée sur la \emph{Time Geography} et les contraintes, celle sur les mesures d'utilité pour l'utilisateur, et celle sur un temps de trajet moyen. Les mesures correspondantes sont dérivées dans un cadre mathématique unifié, ce qui permet un lien à la fois théorique et opérationnel entre des approches du concept a priori différentes.
}


\bpar{}
{
On peut voir dans un premier temps dans quelle mesure des motifs d'accessibilité induisent une évolution du réseau. Ce concept est souvent utilisé comme un outil de planification ou comme une variable explicative de localisation des agents par exemple, puisqu'il s'agit par exemple d'un bon indicateur pour la quantité de personnes affectées par un projet de transport.}

\bpar{}{
 Les débats récents sur la planification du \emph{Grand Paris Express}~\cite{mangin2013paris}, cette nouvelle infrastructure de transport métropolitaine planifiée pour les vingts prochaines années, a révélé l'opposition entre une vision de l'accessibilité comme nécessaire pour désenclaver des territoires désavantagés, et une vision de l'accessibilité comme moteur du développement économique pour des zones déjà dynamiques, les deux n'étant pas forcément compatibles car correspondent à des corridors de transport différents. L'un était initialement porté par l'Etat dans la perspective des pôles de compétitivité, l'autre par la région dans une perspective d'équité territoriale.\comment[FL]{un peu schematique car tous les miveaux ont ces $\neq$ objectifs de competitivite/d'equite} Nous reviendrons sur cet exemple précis du Grand Paris en détails par la suite.
 }

\bpar{}{
Cet exemple permet de suggérer un effet des motifs de potentiels sur l'évolution du réseau : même si celui-ci passe par des structures sociales complexes (nous y reviendrons aussi en détail plus loin), il existe de nombreuses situations où une croissance du réseau de transport (qui peut se manifester par une évolution topologique, c'est à dire l'ajout d'un lien, mais aussi une évolution des capacités des liens) est directement ou indirectement induite par une distribution d'accessibilité~\cite{zhang2007economics}. Ce phénomène peut concerner des modifications fondamentales du réseau comme des modifications mineures : \cite{rouleau1985villages} étudie l'évolution sur le temps long (de 1800 à 1980) des villages satellites à Paris qui ont été progressivement intégrés à son tissu urbain et montre à la fois une persistance de la trame viaire et parcellaire, mais aussi ces ruptures locales dont certaines témoignent d'une satisfaction d'une différence de potentiel\comment[FL]{tu es deja dans les modeles $\rightarrow$ garde cet exemple pour plus tard}, sans être partie essentielle de l'évolution globale (comme dans le cas d'Haussmann). Nous désignerons ce processus abstrait par \emph{rupture de potentiel}\footnote{En analogie avec le phénomène de \emph{dieletric breakdown}, ou décharge partielle, qui correspond au passage du courant dans un isolant quand la différence de potentiel électrique est trop grande.}.
}

\bpar{}{
Un autre processus intéressant est l'impact d'une évolution de l'accessibilité par relocalisations sur les motifs d'utilisation du réseau, et particulièrement la congestion, induisant une modification de la capacité\comment[FL]{sens ?} : ce phénomène est montré dans le cas de Beijing\comment[FL]{Pékin en Francais} par~\cite{yang2006transportation}, qui révèle des modifications d'impédance (vitesse effective dans le réseau routier) allant jusqu'à 30\%. Il peut être mis en correspondance avec les processus liés à la mobilité, même si on se situe ici plutôt dans des échelles meso-meso\comment[FL]{c'est a dire} : évolution du réseau et relocalisations sur des temps de l'ordre de la dizaine d'année (le réseau étant plus lent, de l'ordre de la vingtaine d'années), sur des échelles spatiales métropolitaines\footnote{qui correspondent à des étendues spatiales de 100 à 200km, mais à diverse réalités urbaines. Une métropole sera une ville d'importance dans un système de villes à grande échelle, et sera vue avec son territoire fonctionnel (par exemple Paris et une grande partie de l'Ile-de-France). L'émergence de nouvelles formes métropolitaines, comme les \emph{Mega-city-regions} qui sont composés de métropoles de taille comparable, sur une faible étendue spatiale, et en très forte interaction, complique cette question de l'échelle. Nous reviendrons sur ces objets en~\ref{sec:casestudies}.}.

}


\bpar{
}{
Réciproquement, une évolution du réseau implique une reconfiguration immédiate de la distribution spatiale des accessibilités (au sens de l'ensemble des approches existantes, puisque toutes mobilisent le réseau), et aussi potentiellement des transformations territoriales sur une plus longue durée : on rejoint finalement le débat des effets structurants que nous avons déjà commenté. On a déjà vu que l'accessibilité co-évolue\comment[FL]{un peu rapide \ldots est-ce le meme mot pour toutes echelles s/t ?} avec les pratiques de mobilité, ce qui suppose un effet à cette échelle. Concernant les relocalisations et la distribution des populations, il existe des cas où il est en effet possible d'attribuer à la croissance du réseau des dynamiques des territoires. \comment[AB]{exemples ? references ? : suite}\comment[FL]{! il faut absolument un tableau avec les echelles t/s que tu appelles micro/meso/macro et des references}
}

\bpar{}{
\cite{duranton2012urban} montrent ainsi à une échelle de temps moyenne de 20 ans pour les Etats-unis, par l'utilisation de variables instrumentales\footnote{La méthode des variables instrumentales permet de dégager des relations causales entre une variable explicative et une variable expliquée. Le choix d'une troisième variable, appelée variable instrumentale, soit être fait tel que celle-ci n'influence que la variable explicative mais pas la variable expliquée, en quelque sorte un choc exogène.}, que la croissance de l'accessibilité dans une ville cause une croissance de l'emploi. Sur une échelle temporelle similaire, mais à l'échelle spatiale du pays pour la Suède, \cite{johansson1993infrastructure} montre que l'accessibilité locale (``intra-régionale'') et globale (``inter-régionale'') explique la croissance de la production et la productivité des entreprises. \cite{doi:10.1080/01441647.2016.1168887} procède à une revue systématique des études empiriques des impacts à moyen terme des infrastructures de transport, et montre qu'une densification urbaine à proximité des nouvelles infrastructures est très probable, celle-ci étant résidentielle dans le cas d'une infrastructure ferroviaire et pour les emplois et l'activité industrielle et commerciale dans le cas d'une infrastructure routière.\comment[FL]{est ce que son approche fait consensus ? est elle applicable a tous les contextes ?} De même, on peut montrer des effets forts de la présence d'infrastructures pour des types particuliers d'usage du sol : \cite{nilsson2016measuring} l'illustre par exemple pour les fast food dans deux villes aux Etats-Unis, en montrant statistiquement que l'accès à une infrastructure importante induit une agrégation spatiale des commerces.
}

\bpar{}{
Ces derniers exemples suggèrent l'existence potentielle d'effets de l'accessibilité, et donc du réseau, sur les dynamiques territoriales. Dans certain cas, les effets structurants sont ainsi présents. Mais ceux-ci sont toujours liés au contexte précis ainsi qu'aux échelles. Cela nous permet de faire la transition vers les concepts liés aux dynamiques des systèmes urbains sur le temps long.
}




\subsubsection{Transportation and Urban Systems}{Transports et Systèmes Urbains}

\bpar{
}{
La troisième entrée conceptuelle sur les interactions entre réseaux et territoires, et qui sera particulièrement liée à l'idée de co-évolution, est celle par les systèmes urbains, à petite échelle spatiale et sur le temps long. Celle-ci est organiquement\comment[FL]{?} dépendante à la Théorie Evolutive des Villes\comment[FL]{pas besoin d'en reparler a ce stade}, dont nous avons esquissé une description en introduction et lors de la construction du concept de territoire, dans la façon dont nous la présenterons, mais ne lui est bien sûr pas subordonnée dans les études existantes. Nous désignerons le concept par \emph{Dynamique structurelle du système urbain}.
}

\bpar{}{
La Théorie Evolutive des Villes considère les systèmes de villes comme des systèmes de systèmes à de multiples échelles, du niveau microscopique intra-urbain, au niveau macroscopique du système entier, par le niveau mesoscopique de la ville~\cite{pumain2008socio}. Ces systèmes sont complexes, dynamiques hors-équilibre\comment[AB]{?}, et adaptatifs : leur composants \emph{co-évoluent} et le système répond à des perturbations intérieures ou extérieures par des modifications de sa structure et de sa dynamique. Nous développerons longuement les multiples implications de cette approche tout au long de notre travail, et retenons ici les processus d'interactions entre villes. Celles-ci\comment[AB]{l'importance des interactions ?} consistent en des échanges informationnels ou matériels, et la diffusion de l'innovation en est une composante cruciale~\cite{pumain2010theorie}. Ces interactions sont nécessairement portées par les réseaux physiques, et plus particulièrement les réseaux de transport. On s'attend ainsi du point de vue théorique à une interdépendance forte entre villes et réseaux de transport à ces échelles, c'est à dire à une co-évolution.\comment[FL]{$\rightarrow$ chapitre TEV}
}


\bpar{
\cite{bretagnolle:tel-00459720} highlighted an increasing correlation in time between urban hierarchy and network hierarchy for French railway network, marker of positive feedbacks between urban rank and network centralities. Different regimes in space and times were identified: for French railway network evolution e.g., a first phase of adaptation of the network to the existing urban configuration was followed by a phase of co-evolution i.e. in the sense that causal relations became difficult to identify. The impact of space-time contraction by the network on patterns of growth potential had already been shown for Europe with an exploratory analysis in~\cite{bretagnolle1998space}. Railway evolution in the United States followed a different pattern, without hierarchical diffusion, shaping locally urban growth.
}{
Du point de vue empirique, celle-ci a déjà été mise en valeur : \cite{bretagnolle:tel-00459720} souligne une corrélation croissante dans le temps entre la hiérarchie urbaine et la hiérarchie de l'accessibilité temporelle pour le réseau ferroviaire français (a priori plus claire pour cette mesure que pour les mesures intégrées d'accessibilité soumises à l'auto-corrélation comme nous le verrons en~\ref{sec:causalityregimes}). Celle-ci est un marqueur de rétroactions positives entre le rang urbain et la centralité de réseau. Différents régimes dans le temps et l'espace ont été identifiés : pour l'évolution du réseau ferroviaire français, une première phase d'adaptation du réseau à la configuration urbaine existante a été suivie par une phase de co-évolution, au sens où les relations causales sont devenues difficiles à identifier. L'impact de la contraction de l'espace-temps par les réseaux sur le potentiel de croissance des villes avait déjà été montré pour l'Europe par des analyses exploratoires dans~\cite{bretagnolle1998space}. Les résultats de modélisation par~\cite{bretagnolle2010comparer}, et plus particulièrement les paramétrisations différentes du modèle Simpop2\comment[FL]{que tu dois decrire a minima}, montrent que l'evolution du réseau ferroviaire aux Etats-unis a suivi une dynamique bien différente, sans diffusion hiérarchique, donnant forme localement à la croissance urbaine dans certains cas. Ce contexte particulier de conquête d'un espace vierge d'infrastructures implique un régime particulier au système territorial. D'autres contextes révèlent des impacts différents du réseau à court en long terme : \cite{berger2017locomotives} étudient l'impact de l'établissement du réseau ferroviaire suédois sur la croissance des populations urbaines, de 1800 à 2010, et trouvent un effet causal immédiat de la croissance de l'accessibilité sur la croissance de la population, suivi sur le temps long d'une forte inertie de la hiérarchie des populations, témoignant d'une dépendance au chemin du système dans son ensemble.\comment[FL]{tu y reviens $\rightarrow$ c'est trop dispersé} Dans chaque cas, on a bien existence de \emph{dynamiques structurelles} sur le temps long, qui correspondent aux dynamiques lentes de la structure du système urbain, et témoignent en ce sens d'\emph{effets structurants sur le temps long} comme le souligne~\cite{pumain2014effets}.
}





\bpar{
It emphasizes the presence of path-dependance for trajectories of urban systems: the presence in France of a previous city system and network (postal roads) strongly shaped railway development.
}{
Il ne s'agit bien de différencier \comment[AB]{?} ces derniers de ceux sujets des débats mentionnés précédemment. Souvent, les effets attendus par les planificateurs ou politiques relèvent du moyen terme\comment[FL]{affirmation gratuite (ou semblant l'etre)}. Au niveau du système urbain, on\comment[FL]{qui est on ?} regarde globalement la réalisation de trajectoires qui étaient possibles, et localement l'effet a nécessairement un aspect probabiliste. D'autre part, il faut mettre l'accent sur le rôle de la dépendance au chemin pour les trajectoires des systèmes urbains : par exemple la présence en France d'un système préalable de villes et de réseau (routes postales) a fortement influencé le développement du réseau ferré, ou comme \cite{berger2017locomotives} l'a montré pour la Suède. De même, \cite{doi:10.1068/b39089} souligne l'importance des évènements historiques dans les dynamiques couplées du réseau routier et des territoires, choc historiques pouvant être vus comme exogènes et induisant des bifurcations du système qui accentuent l'effet de la dépendance au chemin. Ainsi, pour ces dynamiques de structure sur le temps long, des prévisions ne sont guère envisageables.
}

\bpar{}{
Cette troisième approche nous a permis de dégager un point de vue complémentaire de la co-évolution, à une autre échelle.
}


%%%%%
% digression with no future, or eventually in a DynSys-ABM opening ?
%\bpar{
%(different \emph{regimes} in the sense of settlement systems transitions introduced in the current ANR Research project TransMonDyn, as a transition can be understood as a change of stationarity for meta-parameters of a general dynamic). In terms of dynamical systems formulation, it is equivalent to ask if dynamics of attractors (long time scale components) obey similar equations as the position and nature of attractors for a stochastic dynamical system that give its current regime, in particular if it is in a divergent state (positive local Liapounov exponent) or is converging towards stable mechanisms~\cite{sanders1992systeme}.
%}{
%, c'est à des \emph{régimes} différents au sens des transitions des systèmes de peuplement\comment[FL]{concept non connu}, puisqu'une transition entre deux régimes peut être comprise comme un changement de stationnarité des méta-paramètres d'une dynamique plus générale. En termes de systèmes dynamiques, cela revient à se demander si les dynamiques des ensembles de catastrophes (composantes à plus grandes échelles temporelles) obéissent à des équations similaires à la position et nature des attracteurs pour un système dynamique stochastique qui donne son régime courant, en particulier si le système est dans un état local divergent (exposant de Liapounov local positif\comment[FL]{ce n'est pas comprehensible}) ou en train de converger vers des mécanismes stables\comment[FL]{c'est flou}~\cite{sanders1992systeme}.
%}



\subsubsection{Scaling laws}{Lois d'échelles}

% avant la transition : rappeler que le schema par échelles est simplifié, mais grille de lecture permettant de construire.


Notre grille de lecture par échelles progressives, qui permet de dégager une assez bonne correspondance entre échelle spatiale et temporelle, ainsi que d'y associer les concepts adaptés, ne capture bien sûr pas l'ensemble des processus possibles : ceux qui seraient fondamentalement multi-échelles\comment[FL]{pourquoi as tu besoin de discuter cela ?}, ou qui impliqueraient l'émergence de leur propre niveau intermédiaire, ne sont pas évoqués. Nous y reviendrons ci-dessous. Dans un premier temps, nous proposons d'effectuer un lien conceptuel entre les échelles par l'intermédiaire des \emph{lois d'échelles} (que nous comprenons au sens général donné en introduction).


\bpar{
An incontournable aspect of transportation networks that we will need to take into account in further developments is hierarchy. Transportation networks are by essence hierarchical, depending on scales they are embedded in. \cite{10.1371/journal.pone.0102007} showed empirical scaling properties for public transportation networks for a consequent number of metropolitan areas across the world, and scaling laws reveal the presence of hierarchy within a system, as for size hierarchy for system of cities expressed by Zipf's law~\cite{nitsch2005zipf} or other urban scaling laws~\cite{2013arXiv1301.1674A,2015arXiv151000902B}. Transportation network topology has been shown to exhibit such scaling also for the distribution of its local measures such as centrality~\cite{samaniego2008cities}.
}{
Les réseaux de transport sont par essence hiérarchiques, cette propriété dépendant des échelles dans lesquelles ils sont intégrés, et se manifestant par l'émergence de lois d'échelles pour leurs propriétés. Par exemple, \cite{10.1371/journal.pone.0102007} montrent empiriquement des propriétés de loi d'échelle pour un nombre conséquent d'aires métropolitaines à travers la planète. Or les lois d'échelle révèlent la présence de hiérarchies dans un système, comme pour la hiérarchie de tailles dans les systèmes de villes exprimée par la loi de Zipf~\cite{nitsch2005zipf} ou d'autres lois d'échelles urbaines~\cite{2013arXiv1301.1674A,2015arXiv151000902B}\comment[FL]{et alors ?}. La topologie du réseau de transport suit de telles lois pour la distribution de ses mesures locales comme la centralité~\cite{samaniego2008cities}, celles-ci étant directement liées au motifs d'accessibilité à différentes échelles. De plus, la topologie du réseau fait partie des facteurs induisant la hiérarchie d'usage, se retrouvant dans les externalités négatives de congestion, en relation avec la distribution spatiale de l'usage du sol~\cite{Tsekeris20131}. Ainsi, la considération des lois d'échelles pour les réseaux de transport, et plus généralement pour les systèmes territoriaux, permet un lien implicite entre les échelles.\comment[FL]{cela me semble une conclusion un peu deceptive}
}




%%%%%%
% transition : synthese breve et preliminaire des processus mis en evidence

\subsubsection{Processus}{Processus}

A ce stade, nous pouvons d'ores et déjà synthétiser des processus d'interaction que nous avons introduit. Des composantes territoriales peuvent agir sur les réseaux de transport par :

\begin{itemize}
	\item Impact des motifs de mobilité sur les impédances
	\item Rupture de potentiel
	\item Sélection hiérarchique de l'accessibilité ; effets systémiques structurels et bifurcations
\end{itemize}

\comment[FL]{TB $\rightarrow$ a developper je pense, tu n'es pas exhaustif a ce stade}

Réciproquement, des processus où les propriétés des réseaux agissent sur les territoires incluent :
\begin{itemize}
	\item Relocalisations induite par des contraintes de mobilité
	\item Changement d'usage du sol du à une infrastructure de transport
    \item Motifs d'accessibilité induits par les réseaux, pouvant induire des relocalisations
	\item Interactions entre territoires portées par les réseaux, incluant l'effet tunnel lorsque celles-ci sont télescopées
	
	
\end{itemize}

Ces différents processus n'ont pas tous le même statut d'abstraction ni les mêmes échelles. Nous avons de plus volontairement occulté des processus déjà évoqués, au sein desquels le couplage est plus fort et pour lesquels la circularité est déjà présente dans l'ontologie, comme les processus liés à la planification. Nous allons détailler à présent ceux-ci, ce qui nous permettra par la suite de raffiner la liste ci-dessus et de la présenter sous forme de typologie après l'avoir enrichie par des études empiriques.




%%%%%%%%%%%%%%%%%%%
\subsection{From interactions to co-evolution}{Des interactions à la co-évolution}


\bpar{
At this state of progress, we have naturally identified a research subject that seems to take a significant place in the complexity of territorial systems, that is the study of interactions between transportation networks and territories. In the frame of our preliminary definition of a territorial system, this question can be reformulated as the study of networked territorial systems with an emphasis on the role of transportation networks in system evolution processes.
}{
A ce stade, nous avons identifié des processus d'interaction entre réseaux de transport et territoires jouant un rôle significatif dans la complexité des systèmes territoriaux. Dans le cadre de l'approche d'un système territorial par la définition donnée lors de la construction première des concepts, cette question peut être reformulée comme l'étude de systèmes territoriaux réticulaires, avec une emphase sur le rôle des systèmes de transports. On a vu que l'étendue des échelles spatiales et temporelles va de celle de la mobilité quotidienne (micro-micro) à des processus sur le temps long dans les systèmes de villes (macro-macro), avec la possibilité de combinaisons intermédiaires. La précision des échelles particulièrement pertinentes fera l'objet de la majorité des préliminaires (Partie 1) et des fondations (Partie 2), jusqu'au Chapitre~\ref{ch:morphogenesis} qui conclura les fondations. Etendons à présent cette liste et donnons des exemples concrets précisant la complexité des interactions et la nécessité de considérer une co-évolution.\comment[FL]{NB : tu as deja parle plusieurs fois de coevolution}
}



\subsubsection{Importance of the geographical context}{Importance du contexte géographique}


\bpar{}{
La mise en contexte de notre question dans un cadre bien particulier révèle l'importance de la prise en compte du contexte géographique. L'exemple des territoires de montagne, où les contraintes de ressources et de déplacement sont fortes, montre la richesse des situations possibles lorsqu'un schéma générique est appliqué à un système mis sous contrainte.\comment[FL]{?}
}

\bpar{
For example, on comparable French mountain territories, \cite{berne2008ouverture} shows that reactions to a same context of evolution of the transportation network can lead to very different reactions of territories, some finding a huge benefit in the new connectivity, whereas others become more closed.
}{
Par exemple, sur des territoires de montagne français comparables, \cite{berne2008ouverture} montre que les réactions à un même contexte d'évolution du réseau de transport peuvent mener à des dynamiques territoriales très diverses, certains trouvant de forts bénéfices de l'accessibilité accrue, d'autres au contraire devenant plus fermés. Dans le même cadre, ces potentiels processus antagonistes sont examinés plus en détail par~\cite{bernier2007dynamiques}, pour lesquels il propose un typologie basée sur le potentiel d'ouverture à la fois des dynamiques territoriales et des dynamiques des réseaux : par exemple, un territoire peut présenter de riches opportunités d'attractivité, par exemple des opportunités touristiques, tout en gardant une faible accessibilité. Réciproquement, il donne l'illustration des contraintes douanières pouvant freiner le potentiel d'ouverture d'une infrastructure performante.
}


\bpar{}{
En écho aux approches par systèmes de villes, \cite{torricelli2002traversees} montre comment dans ce contexte il est possible de faire un lien entre nature des flux de transport et développement local du système urbain : les villes de montagne ont d'abord émergé comme point de passage sur les chemins de col, puis ont perdu de leur importance avec l'avènement des routes. L'arrivée du chemin de fer a pu les re-dynamiser, par le tourisme et l'industrie, et enfin l'autoroute a encore plus récemment induit une déstructuration par des effets de périurbanisation par exemple. Ainsi, les dynamiques structurelles sur le temps long sont particulières, en conséquence du contexte géographique.\comment[FL]{B}
}



\subsubsection{Planification}{Planification}

\bpar{}{
Comme nous l'avons déjà suggéré, les potentiels impacts des dynamiques territoriales sur les réseaux impliquent des processus à plusieurs niveaux. Ainsi, les projets d'infrastructure sont généralement planifiés\footnote{Nous parlerons de \emph{planification} en général, urbaine, territoriale, d'un projet d'infrastructure, pour désigner la conception volontaire d'un projet et d'un plan par un acteur d'aménagement, dans le but de transformer l'espace selon certaines motivations propres à l'acteur et à ses interactions avec les autres acteurs.}, afin de répondre à certains objectifs fixés par des acteurs souvent institutionnels. Ces objets nous amènent progressivement vers le concept de gouvernance, mais prenons d'abord un instant pour illustrer des projets planifiés.
}

\bpar{
}{
L'exemple de l'échec de planification de l'aéroport de Ciudad Real en Espagne montre que la réponse d'une infrastructure planifiée n'est pas systématique\comment[FL]{oui}. Les explications à celui-ci découlent très probablement d'une combinaison complexe de multiples facteurs, difficiles à séparer. \cite{otamendi2008selection} prédisait avant l'ouverture de l'aéroport une gestion complexe due à la dimension des flux attendus et propose un modèle approprié, or les ordres de grandeurs de flux effectifs étaient plus proches des milliers que des millions planifiés et l'aéroport a rapidement fermé. Il est compliqué de savoir la raison de l'échec, s'il s'agit de l'optimisme quand au polycentrisme régional (l'aéroport est à mi-chemin de Madrid et Séville), la non-réalisation de la gare sur la ligne à grande vitesse, ou des facteurs purement économiques.
}


\bpar{}{
\cite{heddebaut:hal-01355621}\footnote{Le possible jeu de mot par le titre ambigu sur l'existence du ``Tunnel effect'' rappelle l'effet tunnel, qui réside en la non-interaction d'une infrastructure sur un territoire le traversant sans s'y arrêter.} montrent pour l'impact des infrastructures sur le long terme, dans le cas du tunnel sous la Manche\footnote{Mis en service en 1994 entre Calais en France et Folkestone au Royaume-uni, ce tunnel ferroviaire sous-marin de 50km permet une liaison physique entre l'Europe continentale et le Royaume-uni.}, par une analyse des investissements et des politiques dans le temps, que les effets effectivement constatés pour la région Nord-Pas-de-Calais comme un gain de centralité et de visibilité au niveau Européen, sont en fort décalage avec les discours justifiant le projet, et que le renouvellement des acteurs implique un non-accompagnement du projet sur le long terme, rendant son impact plus hasardeux : on rejoint l'idée défendue par \noun{Bretagnolle} dans \cite{espacegeo2014effets} selon laquelle des ``effets de structure'' effectivement existent mais que ceux-ci se manifestent sur le temps long en termes de dynamiques systémiques pour lesquelles une vision locale courte n'a que peu de sens. A l'échelle intra-urbaine, \cite{fritsch2007infrastructures} prend l'exemple du Tramway de Nantes pour montrer, par une étude localisée des transformations urbaines à proximité d'une nouvelle ligne, que les dynamiques de densification urbaine sont en décalage avec ce qu'en attendaient les élus et planificateurs, c'est à dire une association forte entre proximité à la ligne et densification.
}

\bpar{}{
Ces exemples confirment que la compréhension des effets des territoires sur les infrastructures impliquent la prise en compte de la notion de \emph{gouvernance}.
}




\subsubsection{Governance}{Gouvernance}


\bpar{}{
Le développement d'un réseau de transport nécessite des acteurs disposant à la fois des moyens concrets et économiques de mener à bien la construction, et d'autre part ayant la légitimité de mener ce développement. Il s'agit donc nécessairement d'acteurs de la superstructure sociale, pouvant être différents niveaux de pouvoirs publics, parfois associés à des acteurs privés. Le concept de \emph{gouvernance}, que nous comprenons comme la gestion d'une organisation disposant de ressources communes dans des buts liés à l'intérêt de la communauté concernée (pouvant être définis de différentes façons, par exemple de manière \emph{top-down} par les acteurs de gouvernance ou de manière \emph{bottom-up} par consultation des agents concernés par la décision), est alors essentiel pour comprendre l'évolution des projets de transports et donc des réseaux de transport. Nous parlerons de \emph{gouvernance territoriale} lorsque les décisions concernent directement ou indirectement des composantes de systèmes territoriaux. 
}


\bpar{}{
Certains aspects de la gouvernance territoriale peuvent avoir un impact déterminant sur le développement des infrastructures de transport. \cite{deng2007potential} montre dans le cas des villes Chinoises que les nouvelles directives en terme de logement peuvent fortement détériorer la performance des infrastructures, et que des dispositions spécifiques en termes de \emph{Transit Oriented Development} (TOD)\comment[FL]{tu n'en as pas parle avant} doivent être prises pour anticiper ces externalités négatives. Le TOD est une approche particulière de l'aménagement urbain visant à articuler développement de l'offre de transport en commun et développement urbain. Il s'agit en quelque sorte d'une co-évolution volontaire de la part des développeurs (autorités administratives et/ou de planification), dans laquelle l'articulation est pensée et planifiée. Nous reviendrons sur le TOD lors d'études empiriques par la suite.\comment[FL]{tu vas un peu vite en amenant ensemble gouvernance (un gros morceau) et modeles urbains (un gros morceau), mais tu as raison d'aller vers cela !}
}


\bpar{}{
Ces concepts ne sont pas nouveaux, puisqu'ils étaient implicites par exemple dans l'aménagement des villes nouvelles en Ile-de-France, sous une forme différente puisque celles-ci étaient également fortement zonées (c'est à dire planifiées en zones relativement cloisonnées et monofonctionnelles) et dépendantes de l'automobile pour certains quartiers~\cite{es119}. \cite{l2012ville} est un exemple de projet européen ayant exploré des mises en pratiques de paradigmes du TOD : des détails d'aménagement comme un réseau de qualité pour les modes actifs à courte portée sont cruciaux pour une concrétisation des principes. Par exemple, \cite{lhostis:hal-01179934} utilise une analyse multi-critères\footnote{Dans le cadre de l'aide à la décision pour la planification des infrastructures de transport, l'analyse multi-critère est une alternative aux analyses coût-bénéfices (qui comparent des projets en agrégeant un coût généralisé) qui permet de prendre en compte de multiple dimensions, souvent contradictoires (par exemple coût de construction et robustesse pour un réseau), et obtenir des solutions optimales au sens de Pareto} pour comprendre les facteurs déterminants dans la sélection des stations de la ville planifiée, incluant densité urbaine et temps d'accès aux stations. \cite{LIU2014120} montre que si certaines politiques de planification, en particulier en France, ne se réclament pas directement de cette approche, leurs caractéristiques sont très similaires comme le révèle le cas de Lille.
}


\bpar{}{
L'articulation entre transport et aménagement doit souvent être opérée de façon fortement couplée pour parvenir à ses\comment[AB]{?} objectifs, d'autant plus que le projet est spécialisé : \cite{larroque2002paris} rappelle l'anecdote du metro SK de Noisy-le-Grand montre un cas de dépendance complète de la fonctionnalité du transport à l'aménagement local. Pour desservir un projet de complexe de bureau, une ligne spécifique avec une matériel roulant léger est construite pour faire le lien avec la gare RER de Mont-d'Est. Le projet immobilier avortera alors que la ligne est inaugurée en 1993, celle-ci sera d'abord entretenue régulièrement puis laissée a l'abandon sans jamais avoir été ouverte au public.\comment[FL]{B mais la formulation est-elle de toi ? car le style est different du reste je trouve}
}


\bpar{}{
Ainsi, les processus de gouvernance, qui peuvent se décliner de plusieurs manières, comme ceux de planification, ou plus spécifiques de TOD, jouent un rôle important dans les interactions entre réseaux de transports et territoires. Ceux-ci s'ajoutent à notre panorama, étant d'un type particulier car impliquant leur propre niveau d'émergence et une forte autonomie.
}


\comment{\cite{offner2000territorial} deregulation of network governance}







\subsubsection{Co-evolution of networks and territories}{Co-évolution des réseaux et des territoires}


\bpar{}{
Cette construction progressive nous a permis de souligner la complexité des interactions entre réseaux et territoires, ce qui confirme \comment[AB]{?} la nécessité de se placer dans l'ontologie particulière de la \emph{co-évolution} comme nous l'avons définie en introduction.\comment[FL]{la encore c'est dommage tu fermes le jeu. TB de proposer de travailler sur la coevolution. cela ne signifie pas que toutes les autres pistes sont ste} \cite{levinson2011coevolution} souligne la difficulté de la compréhension de la co-évolution entre transport et usage du sol en termes de causalités circulaires, en partie à cause des différentes échelles de temps impliquées, mais aussi par l'hétérogénéité des composantes. \cite{offner1993effets} parle de congruence, qu'on peut comprendre comme une dynamique systémique impliquant des corrélations fortuites ou non, à lier avec la vision systémique de l'époque\comment[AB]{?}, ce qui serait une vision préliminaire de la co-évolution.
}

\bpar{}{
La nécessité de dépasser les approches réductrices des effets structurants, tout en capturant la complexité des interactions entre réseaux et territoires par leur co-évolution, est confirmée par le cas des effets économiques des trains à grande vitesse : \cite{Blanquart2017} procède à une revue à la fois théorique et empirique, incluant la littérature grise, des études de ce cas spécifiques, et conclut, au delà des retombées directes liées à la construction sur lesquelles il y a consensus, à des effets en apparence aléatoires si les sujets sont considérés hors contexte\comment[FL]{? c'est enigmatique}, témoignant de situation locales bien plus complexes, un grand nombre d'aspect conjoncturels entrant en jeu dans la production d'effets, qu'on ne peut alors pas attribuer seulement au transport : il y a bien co-évolution entre les différentes composantes du système\comment[FL]{donc tu as une definition operationelle de la coevolution}. Cette revue confirme le décalage entre les discours politiques et techniques prévalant aux projets de transports et les analyses effectives a posteriori révélée par~\cite{bazin:hal-00615196}. \cite{bazin2007evolution} procèdent d'autre part à une étude ciblée du marché immobilier à Reims en anticipation de l'arrivée du TGV Est. En procédant à une analyse diachronique pour chaque année entre 1999 et 2005, par quartier, des prix immobiliers et de la provenance des acheteurs (Franciliens ou locaux), ils concluent que seul des opérations très localisées pouvaient être directement reliées au TGV, l'ensemble du marché répondant à une dynamique globale indépendante.
}


\bpar{}{
Ainsi, notre aperçu constructif, large et voulu circulaire, des interactions entre réseaux de transports et territoires, confirme la pertinence de cette notion de \emph{co-évolution} d'une part, mais suggère un approfondissement et une clarification de celle-ci. Nous nous appliquerons dans la section suivante à approfondir de manière empirique différents aspects abordés ici.
}




\stars




