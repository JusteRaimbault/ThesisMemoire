


% Chapter 

%\section[Open tools and workflows][Outils et pratiques pour l'ouverture]{Tools and Workflow for an open Reproducible Research}{Outils et pratiques pour une recherche ouverte et reproductible} % Chapter title
\section{Tools and Workflow for an open Reproducible Research}{Outils et pratiques pour une recherche ouverte et reproductible}

\label{app:workflow} % For referencing the chapter elsewhere, use \autoref{ch:name} 

%----------------------------------------------------------------------------------------


%\headercit{Open for Discovery}{PLoS}{}


%% 
%   Technical elements / workflow to be included :
%
%   - NLaggregator : useful ?
%   - NLDocumentation : yes
%   - git usage
%   - towards a git-compatible metafig ? / metadata-handler 
%   !! importance of metadata for transparency/reproducibility
%
%


\bpar{
We briefly evoke here tools or workflows currently under development or testing, aimed at easing an open reproducible research and making it more transparent.
}{
Nous décrivons ici brièvement des outils, des pratiques et des pistes de développement pour une recherche plus transparente, libre, ouverte et fluide.
}


\subsection{NetLogo documentation generator}{Générateur de Documentation Netlogo}


\bpar{
Documentation generation is central for reproducibility as it can automatize implementation description. NetLogo does not provide a documentation generator and we experimented a \texttt{Doxygen} wrapper for NetLogo code, that basically consists in transforming NetLogo code into Java code and parsing documentation comment blocks. An experimental version is available at \url{https://github.com/JusteRaimbault/CityNetwork/tree/master/Models/Doc}.
}{
La génération de documentation est centrale pour la reproductibilité, permettant d'automatiser la description de l'implémentation d'un modèle. NetLogo ne fournit pas de générateur de documentation. Nous avons implémenté un wrapper du logiciel \texttt{Doxygen} (génération de documentation pour divers langages dont Java) pour son application au langage NetLogo. Il repose sur le principe basique de génération d'un code Java intermédiaire, miroir du code NetLogo dans ses structures objet et reprenant les blocs de commentaires dans le code NetLogo. Une version expérimentale est disponible à \url{https://github.com/JusteRaimbault/CityNetwork/tree/master/Models/Doc}.
}


\subsection{git as a reproducibility tool}{git comme outil de reproductibilité}


\bpar{
The use if \texttt{git} as a reproducibility and transparency tool was emphasized in~\cite{ram2013git}, for various reasons such as exact history tracing, easy cloning, past commit branching.
}{
L'utilisation de \texttt{git} comme outil de reproductibilité et de transparence a été mis en valeur par~\cite{ram2013git}, qui soulignent de nombreux avantages tels le suivi exact de l'historique du processus de production de connaissance, un clonage immédiat (en combinaison avec des dépôts publics, pour lesquels existent des sites collaboratifs comme github ou gitlab), une possibilité de branchage à partir de commits passés.
}

\bpar{It furthermore can help individual workflow for advantages such as automatic backup, organisation, experiments tracking.
%We use it actively and develop extensions for it.
}{
Cet outil permet d'autre part de faciliter le flux de travail individuel, permettant par exemple le backup automatique, l'organisation, le suivi des expériences.
}



\subsection{Open Review}{Revue Ouverte}

Le processus de revue de ce manuscrit a expérimentalement testé la revue ouverte, par l'utilisation du dépôt git et de commandes \LaTeX spécifiques. La commande basique \texttt{{\textbackslash}comment} permet aux relecteurs d'insérer leur commentaires à l'endroit approprié (et se place alors en annotation de marge dans le manuscrit) et permet une discussion jusqu'à 5 réponses successives par des arguments optionnels. Une \emph{pull request} depuis la branche du relecteur permet d'intégrer les retours. D'autres commandes permettent par exemple de marquer les changements ou d'insérer des listes de tâches.

L'un des intérêts de cette démarche est qu'il est possible a posteriori de reconstruire le processus de revue, et que celui-ci est entièrement ouvert (pour une éventuelle revue de la revue). L'automatisation par parcours du réseau de l'historique du dépôt git est même facilement envisageable.



% git data totally obsolete with git-lfs ? not exactly the same ?

%\subsection{git-data}{git-data}

%\texttt{git-data} is a shell based (experimental) git extension, available at \url{https://github.com/JusteRaimbault/gitdata}, that allows automatized backup of large file within a git repository, their transparent integration in ignored files and the creation of symbolic links for a transparent local use.





%%%%%%%%%%%%%%%%%

\subsection{Towards a git-compatible figures metadata handler}{Vers un gestionnaire de métadonnées compatible avec git}

\bpar{
The issue of meta-data for figures is a crucial issue, as it is often difficult to keep a trace of all parameter values that have generated it, along with the corresponding code. Tricks may furthermore happen in script environments such as R or python when variables are accidentally modified without code modification.
}{
La question de la conservation des métadonnées pour les figures est cruciale pour la reproductibilité, puisqu'il est souvent difficile de garder une trace de l'ensemble de la configuration ayant généré une figure, ainsi que le code correspondant, celui-ci pouvant être modifié par des versions antérieures. L'utilisation d'environnements de scripts comme R ou python peuvent également être piégeurs puisque les variables peuvent être modifiées sans modification du code, et il faut garder alors l'ensemble de l'historique des commandes exécutées.
}


\bpar{
Keeping an exhaustive trace of the exact dataset, code and history that has generated a precise figure is a necessary condition for exact reproducibility. We are elaborating a git-compatible tool that would automatically handle these metadata, for example by branching and associating the unique commit hash to the figure. %To become not an organizational burden nor a repository perturbation, we must still make some experiments.
The final idea would be to have under each figure a unique identifier linking to the associated reproducting environment.
}{
Le stockage exhaustif des données, de l'environnement, du code et de l'historique qui a conduit à la génération d'une figure précise sont une condition nécessaire pour une reproductibilité exacte. Une piste pour répondre à ce problème est l'élaboration d'un outil compatible avec git qui générerait automatiquement ces métadonnées, par exemple en créant une branche propre et en conservant le hash du commit associé à la figure. L'idée finale serait d'avoir pour chaque figure un identifiant unique la reliant à l'environnement exact l'ayant produite, impliquant également une automatisation du système d'indexation au sein des documents les utilisant.
}


%%%%%%
% eventuellement getUncited ?





\stars










