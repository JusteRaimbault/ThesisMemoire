%\chapter{Datasets}{Données} % Chapter title
\chapter{Données}

\markboth{\thechapter\space Données}{\thechapter\space Données}


\label{app:data} % For referencing the chapter elsewhere, use \autoref{ch:name} 

%----------------------------------------------------------------------------------------


%\headercit{}{}{}


\bpar{
This appendix lists and describes the different open datasets created and used in the thesis.
}{
Cette annexe liste et décrit les différents jeux de données ouvertes que nous avons été amenés à créer et à utiliser dans la thèse. Les données sont en effet bien un domaine de connaissance propre, et les opérations de collecte et de consolidation sont une étape scientifique à part entière.
}


when possible, specify data citation (ex. traffic data : TransportationEquilibrium paper) ; try to put all on dataverse ; laius sur dataverse, partage des données etc.




%%%%%%%%%%%%%%
\section{Grand Paris Traffic Data}{Données de Traffic du Grand Paris}

% données syntadin : checker la licence




%%%%%%%%%%%%%%
\section{US Gaz Prices}{Prix de l'Essence aux Etats-Unis}






%%%%%%%%%%%%%%
\section{Topological Road Network}{Graphes topologiques des Réseaux Routiers}

La simplification des réseaux routiers, opérée à grande échelle pour l'Europe et la Chine sur les données d'OpenStreetMap, produit les graphes topologiques correspondants. 





%%%%%%%%%%%%%%
%% ON HOLD

%\section{French Freeway Dynamical Network}{Réseau Dynamique des Autoroutes Françaises}

%\comment{Merger avec la base bassin parisien de Florent, faire un data paper.}
%  -> dans une autre vie !





%%%%%%%%%%%%%%
\section{Interviews}{Interviews}

\label{app:sec:interviews}

% laius sur pourquoi données "quali" devraient pas être plus dispo (quand accord intervié) ; outils idem ex. git, dissocié quanti : cf exemple galère excel notes ridicule, refus systématique et catégorique d'une alternative stable et fiable...


\subsection{}{Entretien avec Denise Pumain, 2017/03/31}

Genèse de la Théorie Evolutive de Ville

\subsection{}{Entretien avec Romain Reuillon, 2017/04/11}

Genèse d'OpenMole

\subsection{}{Entretien avec Clémentine Cottineau, 2017/05/05}

Géographe à l'interface interdisciplinaire















