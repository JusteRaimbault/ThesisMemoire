

%\section[Description of implementations][Description des implémentations]{Architecture and Sources for Algorithms and Models of Simulation}{Description des implémentations des algorithmes et des modèles de simulation} % Chapter title
\section{Architecture and Sources for Algorithms and Models of Simulation}{Description des implémentations des algorithmes et des modèles de simulation}

\label{app:code} % For referencing the chapter elsewhere, use \autoref{ch:name} 

%----------------------------------------------------------------------------------------


% do not list all codes, but roughly gives architectures overview
%   and links to git repo

% : script that generates this directly from metadata files ? INCLUDING temporal statistics from git

% idem for work stats ! from git history

% Q : current state of programs ? -> frozen state on specific branch for each model -> could use metafig that way also ?

%\headercit{You must not be afraid of putting code in your thesis, code is not dirty}{Alexis Drogoul}{PhD defense of \cite{rey2015plateforme}}


\bpar{
And yet it is. It makes no sense to put code listings in the core of the text if there is no particular algorithmic detail that requires attention. As soon as implementation biases are avoided, architecture and source for a computational model should be independent from its formal description (but provided along model description with source code as already mentioned before).
}{
Il n'est à notre sens pas particulièrement pertinent d'alourdir un corps de texte avec du listing de code s'il n'y a pas de détails algorithmique nécessitant une attention particulière. Tant que les biais d'implémentation sont évités, l'architecture et le code source de l'implémentation d'un modèle de simulation devraient être indépendants de sa description formelle (mais naturellement fournis avec celle-ci, comme nous l'avons développé en~\ref{sec:reproducibility}).
}

\bpar{
We give in this appendix architectural details on main models of simulation or algorithms we used. Langage and size (in code lines) are provided, along with architectural remarkable features. See \url{https://github.com/JusteRaimbault/CityNetwork/tree/master/Models} for all models, empirical analysis and small experiments.
}{
Nous donnons ainsi dans cette section la liste et une description minimale des implémentations des modèles de simulation et des algorithmes que nous avons utilisé. Le langage et la taille (en termes de nombre de lignes de code) sont fournis, ainsi que les détails à noter le cas échéant. L'ensemble des modèles et des analyses sont regroupés à \url{https://github.com/JusteRaimbault/CityNetwork/tree/master/Models}.
}

%----------------------------------------------------------------------------------------

%\newpage

\subsection{Algorithmic Systematic Review}{Revue Systématique Algorithmique}

\paragraph{Objective}{Objectifs}

\bpar{
Implement systematic literature review algorithm.
}{
Implémentation de l'algorithme de revue systématique.
}

\paragraph{Location}{Localisation}

\url{https://github.com/JusteRaimbault/CityNetwork/tree/master/Models/QuantEpistemo/AlgoSR/AlgoSRJavaApp}

\paragraph{Characteristics}{Caractéristiques}

\begin{itemize}
\item Langage : \texttt{Java}
\item Taille : 7116
\end{itemize}

\paragraph{Particularities}{Particularités}

\bpar{
\begin{itemize}
\item HashConsing used for unique bibliography object, specific hashcode switching if id available or only titles (proceed to lexical distance comparison in that latest case).
\item API to cortext currently being replaced by Python scripts.
\end{itemize}
}{
La technique de HashConsing est utilisée pour conserver des objets bibliographiques uniques.
}

\paragraph{Architecture}{Architecture}

\bpar{
Classical object oriented, see code.
}{
Voir le diagramme en~\ref{fig:quantepistemo:algo}.
}

\paragraph{Additional scripts}{Scripts additionnels}

\bpar{
\texttt{R} for result exploration and visualization.
}{
Exploration des résultats (R).
}


%----------------------------------------------------------------------------------------


\subsection{Indirect Bibliometrics}{Bibliométrie Indirecte}

\paragraph{Objective}{Objectifs}

\bpar{
Multi-layer network analysis of scientific corpuses : cybergeo journal, corpus in~\ref{sec:quantepistemo}
}{
Analyse par réseau de citation et réseau sémantique de corpus scientifiques : corpus de~\ref{sec:quantepistemo}, journal Cybergeo (\ref{app:sec:cybergeo}) ; modélographie (section~\ref{sec:modelography}).
}

\paragraph{Location}{Localisation}

%\url{https://github.com/Geographie-cites/cybergeo20/tree/master/HyperNetwork}
%\url{https://github.com/JusteRaimbault/CityNetwork/tree/master/Models/Biblio/AlgoSR/AlgoSRJavaApp} for common Java part.

\url{https://github.com/JusteRaimbault/CityNetwork/tree/master/Models/QuantEpistemo/HyperNetwork}


\paragraph{Characteristics}{Caractéristiques}

\begin{itemize}
\item Langage : \texttt{Python}, \texttt{R} and \texttt{Java}.
\item Taille : 2210
\end{itemize}


\paragraph{Particularities}{Particularités}

%Polyglot 
Utilise des bases de données sqlite, sql ou Mongodb selon les opérations.

\paragraph{Architecture}{Architecture}

\bpar{
See schema chapter 3.
}{
Voir Fig.~\ref{fig:cybergeo:fig1} en Annexe~\ref{app:sec:cybergeo}.
}
%\paragraph{Additional scripts}{Scripts additionnels}



%----------------------------------------------------------------------------------------


%\subsection{Network simplification}{Simplification du réseau}

% --> cf packages


%\paragraph{Objective}{Objectif}

%\bpar{
%Simplification of european road network, Package \texttt{LargeNetwoRk}
%}{

%}

%\paragraph{Location}{Localisation}

%\url{https://github.com/JusteRaimbault/CityNetwork/tree/master/Models/StaticCorrelations}

%\paragraph{Characteristics}{Caractéristiques}

%\begin{itemize}
%\item Language : \texttt{R}, \texttt{Shell}, \texttt{PostgreSQL}
%\item Size : 919
%\end{itemize}


%\paragraph{Particularities}{Particularités}

%\bpar{
%Handling of large size databases imposes sequential processing ; use of external program \texttt{osmosis} for conversion from \texttt{osm} data to pgsql.
%}{
%L'utilisation sur données massives requière un traitement en parallèle ; le programme externe \texttt{osmosis} est utilisé pour la conversion des données OpenStreetMap (\texttt{osm pbf}) et leur import dans postgresql.
%}

%\paragraph{Architecture}{Architecture}

%Shell script lead maneuvers.

%\paragraph{Additional scripts}{Scripts additionnels}


%----------------------------------------------------------------------------------------


\subsection{Static correlations}{Corrélations statiques}

\paragraph{Objective}{Objectif}

\bpar{
}{
Calcul des indicateurs morphologiques, indicateurs de réseau et de leur corrélations.
}

\paragraph{Location}{Localisation}

\url{https://github.com/JusteRaimbault/CityNetwork/tree/master/Models/StaticCorrelations}

\paragraph{Characteristics}{Caractéristiques}

\begin{itemize}
\item Langage : \texttt{R}
\item Taille : 1862
\end{itemize}


%\paragraph{Particularities}{Particularités}

%\bpar{
%Handling of large size databases imposes sequential processing ; use of external program \texttt{osmosis} for conversion from \texttt{osm} data to pgsql.
%}{
%L'utilisation sur données massives requière un traitement en parallèle ; le programme externe \texttt{osmosis} est utilisé pour la conversion des données OpenStreetMap (\texttt{osm pbf}) et leur import dans postgresql.
%}

%\paragraph{Architecture}{Architecture}

%Shell script lead maneuvers.

%\paragraph{Additional scripts}{Scripts additionnels}



%----------------------------------------------------------------------------------------


\subsection{Spatio-temporal causalities}{Causalités spatio-temporelles}

\paragraph{Objective}{Objectif}

\bpar{
}{
Régimes de causalité, données synthétiques (arma et modèle rbd) et analyses empiriques (Grand Paris, Afrique du Sud, France).
}

\paragraph{Location}{Localisation}

\url{https://github.com/JusteRaimbault/CityNetwork/tree/master/Models/SpatioTempCausalities}

\paragraph{Characteristics}{Caractéristiques}

\begin{itemize}
\item Langage : \texttt{R}
\item Taille : 8627
\end{itemize}


%\paragraph{Particularities}{Particularités}

%\bpar{
%Handling of large size databases imposes sequential processing ; use of external program \texttt{osmosis} for conversion from \texttt{osm} data to pgsql.
%}{
%L'utilisation sur données massives requière un traitement en parallèle ; le programme externe \texttt{osmosis} est utilisé pour la conversion des données OpenStreetMap (\texttt{osm pbf}) et leur import dans postgresql.
%}

%\paragraph{Architecture}{Architecture}

%Shell script lead maneuvers.

%\paragraph{Additional scripts}{Scripts additionnels}



%----------------------------------------------------------------------------------------


\subsection{Macroscopic model of interactions}{Modèle d'interaction macroscopique}

\paragraph{Objective}{Objectif}

\bpar{
Interaction macroscopic model
}{
Modèle d'interaction macroscopique, section~\ref{sec:interactiongibrat}
}

\paragraph{Location}{Localisation}

\url{https://github.com/JusteRaimbault/CityNetwork/tree/master/Models/InteractionGibrat}

\paragraph{Characteristics}{Caractéristiques}

\begin{itemize}
\item Langage : \texttt{NetLogo}, \texttt{scala}, \texttt{R}
\item Taille : 5918
\end{itemize}


\paragraph{Particularities}{Particularités}

\bpar{
Morphological indicators in scala implemented with Fast Fourier transform ; with R communication in NetLogo.
}{
Le modèle est implémenté dans différents langages pour des raisons complémentaires : \texttt{NetLogo} pour l'exploration interactive, \texttt{R} pour intégration avec les tests statistiques, \texttt{scala} pour la calibration par OpenMole.
}

%\paragraph{Architecture}{Architecture}
%Nothing particular.

%\paragraph{Additional scripts}{Scripts additionnels}
%\bpar{
%\texttt{R} for result exploration and morphological analysis.

%\texttt{oms} for model exploration.
%}{
%\texttt{R} pour l'exploration des résultats et l'analyse morphologique
%}





%----------------------------------------------------------------------------------------


\subsection{Density Urban Growth}{Morphogenèse de la densité}

\paragraph{Objective}{Objectif}

\bpar{
Density-based urban morphogenesis model
}{
Modèle de morphogenèse pour la densité (section~\ref{sec:densitygeneration}).
}

\paragraph{Location}{Localisation}

\url{https://github.com/JusteRaimbault/CityNetwork/tree/master/Models/Synthetic/Density}

\paragraph{Characteristics}{Caractéristiques}

\begin{itemize}
\item Langage : \texttt{NetLogo}, \texttt{scala}, \texttt{R}
\item Taille : 5065
\end{itemize}



%----------------------------------------------------------------------------------------


\subsection{Correlated data generation}{Génération des Données Synthétiques Corrélées}

\paragraph{Objective}{Objectifs}

\bpar{
Weak coupling of density generation and network generation.
}{
Couplage faible de la génération de densité et de la génération de réseau.
}


\paragraph{Location}{Localisation}

\url{https://github.com/JusteRaimbault/CityNetwork/tree/master/Models/Synthetic/Network}

\paragraph{Characteristics}{Caractéristiques}

\begin{itemize}
\item Langage : \texttt{NetLogo} (réseau) and \texttt{scala} (densité)
\item Taille : 3188
\end{itemize}


\paragraph{Particularities}{Particularités}

\bpar{
Network heuristic easier to implement and explore in netlogo
}{
Les heuristiques de réseau sont plus naturelles à implémenter et explorer en NetLogo.
}


\paragraph{Architecture}{Architecture}

\bpar{
OpenMole allows coupling between modules through exploration script.
}{
Le couplage faible entre les modules est réalisé par l'intermédiaire d'un script OpenMole.
}

%\paragraph{Additional scripts}{Scripts additionnels}

%\texttt{R} for result exploration.
%\texttt{oms} for model exploration.





%----------------------------------------------------------------------------------------


\subsection{Macroscopic co-evolution}{Co-évolution à l'échelle macroscopique}

\paragraph{Objective}{Objectif}

\bpar{
Implementation of macro-coevolution model
}{
Implémentation du modèle de co-évolution à l'échelle macroscopique (section~\ref{sec:macrocoevol}).
}


\paragraph{Location}{Localisation}

\url{https://github.com/JusteRaimbault/CityNetwork/tree/master/Models/MacroCoevol}

\paragraph{Characteristics}{Caractéristiques}

\begin{itemize}
\item Langage : \texttt{NetLogo}
\item Taille : 4950
\end{itemize}


\paragraph{Particularities}{Particularités}

\bpar{}{
Représentation duale liens/matrice de distance du réseau.
}

%\paragraph{Architecture}{Architecture}


\paragraph{Data used}{Données utilisées}

Population des aires urbaines Françaises 1830-1999

\paragraph{Additional scripts}{Scripts additionnels}

Exploration et calibration (oms), exploration des résultats (R)






%----------------------------------------------------------------------------------------


\subsection{Morphogenesis co-evolution}{Co-évolution par morphogenèse}

\paragraph{Objective}{Objectif}

\bpar{
Implementation of meso-coevolution model
}{
Implémentation du modèle de co-évolution à l'échelle mesoscopique (sections~\ref{sec:networkgrowth} et~\ref{sec:mesocoevolmodel}).
}


\paragraph{Location}{Localisation}

\url{https://github.com/JusteRaimbault/CityNetwork/tree/master/Models/MesoCoevol}

\paragraph{Characteristics}{Caractéristiques}

\begin{itemize}
\item Langage : \texttt{NetLogo}
\item Taille : 5386
\end{itemize}


%\paragraph{Particularities}{Particularités}


%\paragraph{Architecture}{Architecture}


\paragraph{Additional scripts}{Scripts additionnels}

Exploration et calibration (oms), exploration des résultats (R)





%----------------------------------------------------------------------------------------


\subsection{Lutecia Model}{Modèle Lutecia}

\paragraph{Objective}{Objectif}

\bpar{
Implementation of Lutecia model, chapter~\ref{sec:lutecia}.
}{
Implémentation du modèle Lutecia (section~\ref{sec:lutecia}).
}

\paragraph{Location}{Localisation}

\url{https://github.com/JusteRaimbault/CityNetwork/tree/master/Models/Governance/Lutecia/Lutecia}

\paragraph{Characteristics}{Caractéristiques}

\begin{itemize}
\item Langage : \texttt{NetLogo}
\item Taille : 8866
\end{itemize}


\paragraph{Particularities}{Particularités}

\bpar{
Shortest path dynamical programming using matrices.
}{
La matrice des distances effectives est mise à jour par programmation dynamique.
}


%\paragraph{Architecture}{Architecture}
%Pseudo object architecture in agent environment.

\paragraph{Additional scripts}{Scripts additionnels}

\bpar{
\texttt{oms} for model exploration, \texttt{R} for result exploration
}{
Exploration/calibration du modèle (oms), exploration des résultats (\texttt{R}).
}




%----------------------------------------------------------------------------------------


\subsection{Static User Equilibrium}{Equilibre Utilisateur Statique}

\paragraph{Objective}{Objectif}

\bpar{
}{
Collecte et analyse des données de traffic pour la métropole du Grand Paris (section~\ref{sec:reproducibility}).
}

\paragraph{Location}{Localisation}

\url{https://github.com/JusteRaimbault/TransportationEquilibrium/tree/master/Models}

\paragraph{Characteristics}{Caractéristiques}

\begin{itemize}
\item Langage : \texttt{python}, \texttt{R}
\item Taille : $\simeq$ 300
\end{itemize}


%\paragraph{Particularities}{Particularités}

%\bpar{
%}{
%}

%\paragraph{Architecture}{Architecture}

%\paragraph{Additional scripts}{Scripts additionnels}




%----------------------------------------------------------------------------------------


\subsection{Geography of fuel prices}{Géographie des prix du carburant}

\paragraph{Objective}{Objectif}

\bpar{
}{
Collecte et analyse des données des prix du carburant aux Etats-unis (section~\ref{sec:energyprice}).
}

\paragraph{Location}{Localisation}

\url{https://github.com/JusteRaimbault/EnergyPrice/tree/master/Models}

\paragraph{Characteristics}{Caractéristiques}

\begin{itemize}
\item Langage : \texttt{python}, \texttt{R}
\item Taille : 1469
\end{itemize}


\paragraph{Particularities}{Particularités}

\bpar{
}{
Utilisation du logiciel TorPool (voir~\ref{app:sec:packages}) pour la collecte des données.
}

%\paragraph{Architecture}{Architecture}

%\paragraph{Additional scripts}{Scripts additionnels}












\stars





