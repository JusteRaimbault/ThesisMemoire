


% Chapter 

%\chapter{Methodological Developments}{Développements Méthodologiques} % Chapter title
\chapter{Développements Méthodologiques}


\markboth{\thechapter\space Développements Méthodologiques}{\thechapter\space Développements Méthodologiques}


\label{app:methodology} % For referencing the chapter elsewhere, use \autoref{ch:name} 

%----------------------------------------------------------------------------------------

\headercit{We are now building a rigorous Science of Cities, contrarily to what was done before.}{Marc Barth{\'e}l{\'e}my}{EMCSSS Fall 2014, Network Course Introduction}


%\headercit{C'est hardcore tes calculs.}{Anonyme}{}

This chapter gathers various methodological and technical developments, that have the common points to be not essential to the core of the thesis and difficult to digest.


\bigskip

Such a shocking phrase \comment{(Florent) je crois que si tu t'appuies explicitement sur la mise en exergue alors ce n'est plus une mise en exergue}
was pronounced during the introduction of a \emph{Network} course for students of Complex System Science. Besides the fact that the spirit of CSS \comment{(Florent) pas mettre trop d'acronymes que tu ne réutiliseras pas}
 is precisely the opposite, {\ie} the construction of integrative disciplines (vertical integration that is necessarily founded on the existing body of knowledge of concerned fields) that answer transversal questions (horizontal integration that imply interdisciplinarity) - see {\eg} the roadmap for CS~\cite{2009arXiv0907.2221B}, it reveals how methodological considerations shape the perceptions of disciplines. From a background in Physics, \comment{(Florent) soit on connaitre ton background ?}
 ``rigorous'' implies the use of tools and methods judged more rigorous (analytical derivations, large datasets statistics, etc.).\comment{(Florent) je ne suis pas sur que cela soit ca la rigueur physicienne. ce serait plutôt un raisonnement sans trou du début à la fin sur des objets clairement définis ; en sciences sociales il y a fréquemment des trous}
  But what is rigorous for someone will not be for an other discipline\footnote{a funny but sad anecdote told by a friend comes to mind : defending his PhD in statistics, he was told at the end by economists how they were impressed by the mathematical rigor of his work, whereas a mathematician judged that ``he could have done everything on the back of an enveloppe''.\comment{(Florent) ce n'est pas lié à la rigueur}}, depending on the purpose of each piece of research (perspectivism~\cite{giere2010scientific} poses the \emph{model}, that includes methods, as the articulating core of research entreprises). Thus the full role of methodology aside and not beside theory and experiments. We go in this chapter into various methodological developments which may be precisely used later or contribute to the global background.


\bpar{
We then derive technical results on models of urban growth and on the sensitivity of scaling laws, that are both recurrent themes in the modeling of complex urban systems. We then introduce a method in the context of systematic model exploration and model behavior. We finally work on a link between static and dynamic correlations in a geographical system. This chapter is rather heteroclite as sections may correspond to a particular technical need at a point in the thesis, to global methodological directions, or global research directions.
}{

}




%----------------------------------------------------------------------------------------


\section{An unified framework for stochastic models of urban growth}{Modèles stochastiques de croissance urbaine}

\bpar{
Urban growth modeling fall in the case of tentatives to find self-consistent rules reproducing dynamics of an urban system, and thus in our logic of system morphogenesis. \comment{(Florent) est ce que faire de la morphogenese est le but ou le moyen ? ce n'est pas clair en lisant}[(JR)les deux, cf chapitre morphogenese, c'est une façon de comprendre les systèmes complexes] We examine here methodological issues linked to different frameworks of urban growth.
}{

}


%%%%%%%%%%%%%%%%%%%%
\subsection{Introduction}{Introduction}
%%%%%%%%%%%%%%%%%%%%



Various stochastic models aiming to reproduce population patterns on large temporal and spatial scales (city systems) have been discussed across various fields of the literature, from economics to geography, including models proposed by physicists. We propose here a general framework that allows to include different famous models (in particular Gibrat, Simon and Preferential Attachment model) within an unified vision. It brings first an insight into epistemological debates on the relevance of models. Furthermore, bridges between models lead to the possible transfer of analytical results to some models that are not directly tractable.


Seminal models of urban growth are Simon~\cite{simon1955class} (later generalized as e.g. \cite{haran1973modified}) and Gibrat models.
 Many examples of variants and extensions can be given across disciplines. \cite{benguigui2007dynamic} give an equation-based dynamical model, whereas \cite{gabaix1999zipf} solves a stationary model. \cite{Gabaix20042341} reviews urban growth approaches in economics. A model adapted from evolutive urban theory is described in~\cite{favaro2011gibrat} and extends the Gibrat model by adding propagation of innovation between cities. The question of empirical scales at which it is consistent to study urban growth was also tackled in the particular case of France~\cite{bretagnolle2002time}.
%We stay to a certain level of tractability to include models as essence of our approach is links between models but do not make ontologic assumptions \comment[JR]{sens ?}.




%%%%%%%%%%%%%%%%%%%%
\subsection{Framework}{Cadre de Travail}
%%%%%%%%%%%%%%%%%%%%


\paragraph{Presentation}
What we propose as a framework can be understood as a meta-model in the sense of~\cite{cottineau2015incremental}, i.e. an modular general modeling process within each model can be understood as a limit case or as a specific case of another model. More simply it should be a diagram of formal relations between models.\comment{(Florent) à ce stade on ne sait pas si tu vas faire 1 ou N modèles, c'est un choix qu'il te faut défendre avant d'en arriver là}
 The ontological aspect is also tackled by embedding the diagram into an ontological state space (which discretization corresponds to the ``bricks'' of the incremental construction of~\cite{cottineau2015incremental}). It constructs a sort of model classification or modelography. \comment{(Florent) PAS UTILE ICI JE PENSE}

We are still at the stage of different derivations of links between models that are presented hereafter.

%\subsubsection{Models Included}

%The following models are included in our framework. The list is arbitrary but aims to offer a broad view of disciplines concerned

%\subsubsection{Thematic Classification}


%\subsubsection{Framework Formulation}
%Diagram linking various models ; first embedded into time/population plane, cases Discrete/Continous. Other aspects more sparse (ex. spatialization) ; how represent it ?

%%%%%%%%%%%%%%%%%%%%
%\subsection{Models formulation}



%%%%%%%%%%%%%%%%%%%%
\subsection{Derivations}{Dérivations}

\subsubsection{Generalization of Preferential Attachment}{Généralisation de l'Attachement Préférentiel}

\cite{yamasaki2006preferential} give a generalization of the classical Preferential Attachment Network Growth model, as a birth and death model with evolving entities. More precisely, network units gain and lose population (equivalent to links connexions) at fixed probabilities, and new unit can be created at a fixed rate.

\subsubsection{Link between Gibrat and Preferential Attachment Models}{Lien entre Gibrat et Attachement Préférentiel}


\bpar{
Let consider a strictly positive growth Gibrat model given by $P_i(t)=R_i(t)\cdot P_{i}(t-1)$ with $R_i(t)>1$, $\mu_i(t)=\Eb{R_i(t)}$ and $\sigma_i(t)=\Eb{R_i(t)^2}$. On the other hand, we take a simple preferential attachment, with fixed attachment probability $\lambda \in [0,1]$ and new arrivants number $m>0$. We derive that Gibrat model can be statistically equivalent to a limit of the preferential attachment model, assuming that the moment-generating function of $R_i(t)$ exists. Classical distributions that could be used in that case, e.g. log-normal distribution, are entirely defined by two first moments, making this assumption reasonable.
}{
\comment{(Florent) est-ce standard d'introduire de la stochasticité dans Gibrat : Pt+1=RPt}[c'est la formulation standard a priori]
Considérons un modèle de croissance strictement positive de Gibrat donnée par $P_i(t)=R_i(t)\cdot P_{i}(t-1)$ avec $R_i(t)>1$, $\mu_i(t)=\Eb{R_i(t)}$ et $\sigma_i(t)=\Eb{R_i(t)^2}$.\comment{(Florent) expliquer le sens desP,R etc.}
 D'autre part, soit un modèle simple d'attachement préférentiel, avec une probabilité d'attachement $\lambda \in [0,1]$ et un nombre de nouveau arrivants $m>0$. \comment{(Florent) quelle est l'équation $P_{t+1}=P_t \cdot m \cdot lambda$}
  Il est possible de dériver que le Gibrat est statistiquement équivalent à une limite de l'attachement préférentiel, sous l'hypothèse que toutes les fonctions génératrices des moments de $R_i(t)$ existent. Les distributions classiques qui peuvent être utilisées dans ce cas, e.g. une distribution normale ou log-normale, sont entièrement déterminées par leur deux premiers moments, ce qui rend cette hypothèse raisonnable.
  \comment{(Florent) on a déjà discuté de cette eq Gibrat/att pref mais tu ne peux pas faire l'économie d'expliquer pourquoi tu t'es posé la question, i.e. à quoi cela va te servir ensuite}
}



\begin{lemma}
The limit of a Preferential Attachment model when $\lambda \ll 1$ is a linear-growth Gibrat model, with limit parameters $\mu_i(t)=1+\frac{\lambda}{m\cdot (t-1)}$.
\end{lemma}


\bpar{
The proof is given in Appendix~\ref{app:technical}.
}{
La preuve est donnée en Annexe~\ref{app:technical}.
}

\comment{(Florent) certain limit : à qualifier plus précisément}

\comment{(Florent) je n'arrive pas à te suivre : si tu as besoin d'être relu sur ces développements, il faut convenir d'un rendez-vous pour que tu m'expliques le cheminement}

\subsubsection{Link between Simon and Preferential Attachment}{Lien entre Simon et Attachement Préférentiel}
%\label{subsubsec:gibrat-simon}

A rewriting of Simon model yields a particular case of the generalized preferential attachment, in particular by vanishing death probability.

\subsubsection{Link between Favaro-Pumain and Gibrat}{Lien entre Favaro-Pumain et Gibrat}

\cite{favaro2011gibrat} generalizes Gibrat models with innovation propagation dynamics, being therefore a generalization of that model. Theoretically, a process-based model equivalent to the Favaro-Pumain should then fill the missing case in model classification at the corresponding discretization. Simpop models do not fill that case as they stay at the scale of city systems, as for Marius models~\cite{cottineau2014evolution}. These must also have their counterparts in discrete microscopic formulation.

\comment{(Florent) la encore tu parles de modèles que tu ne décris pas par ailleurs ; or Personnes connaissant FavaraoPumain $\cap$ Personnes connaissant Gibrat $\cap$ \ldots = quelques personnes sur terre! }

\subsubsection{Link between Bettencourt-West and Pumain}{Lien entre Bettencourt-West et Pumain}

We are considering to study Bettencourt-West model for urban scaling laws \cite{bettencourt2008large} as entering the stochastic urban growth framework as stationary component of a random growth model, but investigation are still ongoing.

%\comment{(Florent) on ne sait toujours pas dans quelle perspective tu fais cela}


\subsubsection{Other Models}{Autres modèles}

\cite{gabaix1999zipf} develops an economic model giving a Simon equivalent formulation. They in particular find out that in upper tail, proportional growth process occurs. We find the same result as a consequence of the derivation of the link between Gibrat and Preferential attachment models.



\comment{(Florent) je pense que tu as intérêt soit à présenter moins de modèles, mais plus en détails, soit à partir d'angles d'attaque précis et faire des typologies de modèles}





\subsection{Derivations for Urban Growth Models}{Dérivations pour les modèles de croissance urbaine}





\begin{lemma}
The limit of a Preferential Attachment model when $\lambda \ll 1$ is a linear-growth Gibrat model, with limit parameters $\mu_i(t)=1+\frac{\lambda}{m\cdot (t-1)}$.
\end{lemma}

\begin{proof}

Starting with first moment, we denote $\bar{P}_i(t)=\Eb{P_i(t)}$. Independence of Gibrat growth rate yields directly $\bar{P}_i(t)=\Eb{R_i(t)}\cdot \bar{P}_i(t-1)$. Starting for the preferential attachment model, we have $\bar{P}_i(t) = \Eb{P_i(t)} = \sum_{k=0}^{+\infty}{k\Pb{P_i(t)=k}}$. But
\[
\{P_i(t)=k\}=\bigcup_{\delta=0}^{\infty}{\left(\{P_i(t-1)=k-\delta\}\cap \{P_i\leftarrow P_i + 1\}^{\delta}\right)}
\]

where the second event corresponds to city $i$ being increased $\delta$ times between $t-1$ and $t$ (note that events are empty for $\delta \geq k$). Thus, being careful on the conditional nature of preferential attachment formulation, stating that $\Pb{\{P_i\leftarrow P_i + 1\} | P_i(t-1)=p} = \lambda\cdot\frac{p}{P(t-1)}$ (total population $P(t)$ assumed deterministic), we obtain

\begin{equation*}
\begin{split}
\Pb{\{P_i\leftarrow P_i + 1\}} & = \sum_{p}{\Pb{\{P_i\leftarrow P_i + 1\} | P_i(t-1)=p}\cdot \Pb{P_i(t-1)=p}}\\
&=\sum_{p}{\lambda\cdot\frac{p}{P(t-1)}\Pb{P_i(t-1)=p}}=\lambda\cdot\frac{\bar{P}_i(t-1)}{P(t-1)}\\
\end{split}
\end{equation*}

It gives therefore, knowing that $P(t-1)=P_0 + m\cdot (t-1)$ and denoting $q=\lambda\cdot\frac{\bar{P}_i(t-1)}{P_0 + m\cdot (t-1)}$

\[
\begin{split}
\bar{P}_i(t) & =\sum_{k=0}^{\infty}{\sum_{\delta=0}^{\infty}{k\cdot \left(\lambda\cdot\frac{\bar{P}_i(t-1)}{P_0 + m\cdot (t-1)}\right)^{\delta}\cdot \Pb{P_i(t-1)=k-\delta}}}\\
& = \sum_{\delta^{\prime}=0}^{\infty}{\sum_{k^{\prime}=0}^{\infty}{\left(k^\prime + \delta^{\prime}\right)\cdot q^{\delta^{\prime}} \cdot \Pb{P_i(t-1)=k^\prime}}}\\
& = \sum_{\delta^{\prime}=0}^{\infty}{q^{\delta^{\prime}}\cdot \left(\delta^{\prime} + \bar{P}_i(t-1)\right)} = \frac{q}{(1-q)^2} + \frac{\bar{P}_i(t-1)}{(1-q)}\\
& = \frac{\bar{P}_i(t-1)}{1-q}\left[1 + \frac{1}{\bar{P}_i(t-1)}\frac{q}{(1-q)}\right]
\end{split}
\]

%& = \bar{P}_i(t-1)\cdot \frac{1}{1-\lambda\cdot\frac{\bar{P}_i(t-1)}{P_0 + m\cdot (t-1)}} \left[1 + \frac{\lambda}{P_0 + m\cdot (t-1)}\cdot \frac{1}{1-\lambda\cdot\frac{\bar{P}_i(t-1)}{P_0 + m\cdot (t-1)}} \right]


As it is not expected to have $\bar{P}_i(t)\ll P(t)$ (fat tail distributions), a limit can be taken only through $\lambda$. Taking $\lambda \ll 1$ yields, as $0 < \bar{P}_i(t)/P(t) < 1$, that $q=\lambda\cdot\frac{\bar{P}_i(t-1)}{P_0 + m\cdot (t-1)} \ll 1$ and thus we can expand in first order of $q$, what gives $\bar{P}_i(t)=\bar{P}_i(t-1)\cdot \left[1 + \left(1+\frac{1}{\bar{P}_i(t-1)}\right)q + o(q))\right]$

\[
\bar{P}_i(t) \simeq \left[1 + \frac{\lambda}{P_0 + m\cdot (t-1)}\right]\cdot \bar{P}_i(t-1)
\]

It means that this limit is equivalent in expectancy to a Gibrat model with $\mu_i(t) = \mu(t)=1 + \frac{\lambda}{P_0 + m\cdot (t-1)}$.

For the second moment, we can do an analog computation. We have still \[\Eb{P_i(t)^2} = \Eb{R_i(t)^2}\cdot \Eb{P_i(t-1)^2}\]
and
\[\Eb{P_i(t)^2}=\sum_{k=0}^{+\infty}{k^2 \Pb{P_i(t)=k}}\] 

We obtain the same way 

\[
\begin{split}
\Eb{P_i(t)^2} & = \sum_{\delta^{\prime}=0}^{\infty}{\sum_{k^{\prime}=0}^{\infty}{\left(k^\prime + \delta^{\prime}\right)^2\cdot q^{\delta^{\prime}} \cdot \Pb{P_i(t-1)=k^\prime}}}\\ 
& = \sum_{\delta^{\prime}=0}^{\infty}{q^{\delta^{\prime}}\cdot \left(\Eb{P_i(t-1)^2}+2\delta^{\prime}\bar{P}_i(t-1) + {\delta^{\prime}}^2\right)}\\
& = \frac{\Eb{P_i(t-1)^2}}{1-q} + \frac{2 q \bar{P}_i(t-1)}{(1-q)^2} + \frac{q(q+1)}{(1-q)^3}\\
& = \frac{\Eb{P_i(t-1)^2}}{1-q}\left[1 + \frac{q}{\Eb{P_i(t-1)^2}}\left(\frac{2\bar{P}_i(t-1)}{1-q} + \frac{(1+q)}{(1-q)^2}\right)\right]
\end{split}
\]



We have therefore an equivalence between the Gibrat model as a continuous formulation of a Preferential Attachment (or Simon model) in the limit given before. \qed

\end{proof}







%----------------------------------------------------------------------------------------

\newpage


%  section : scaling sensitivity : useful ?


\section{Sensitivity of Urban Scaling Laws to Spatial Extent}{Sensibilité des Lois d'Echelle Urbaines}


\bpar{
At the center of evolutive urban theory are hierarchy and associated scaling laws. We develop here a brief methodological investigation on the sensitivity of scaling laws to city definition.
}{
Au centre de la théorie évolutive des villes se trouvent la hiérarchie et les lois d'échelle associées. Nous proposons ici un bref développement méthodologique sur la sensibilité des lois d'échelle à la définition de la ville. \comment{(Florent) présenté comme cela ce n'est pas évident de comprendre le rapport avec ta thèse}
}


%%%%%%%%%%%%%%%%%%%%
%\subsection{Introduction}



\bpar{
Scaling laws have been shown to be universal of urban systems at many scales and for many indicators. Recent studies question however the consistence of scaling exponents determination, as their value can vary significantly depending on thresholds used to define urban entities on which quantities are integrated, even crossing the qualitative border of linear scaling, from infra-linear to supra-linear scaling. We use a simple theoretical model of spatial distribution of densities and urban functions to show analytically that such behavior can be derived as a consequence of the type of spatial distribution and the method used. Numerical simulation confirm the theoretical results and reveals that results are reasonably independent of spatial kernel used to distribute density.
}{
Les lois d'échelle ont été montrées universelles des systèmes urbains à de nombreuses échelles et pour différents indicateurs. \comment{(Florent) pas très précis}
Des études récentes questionnent toutefois la cohérence de la détermination des exposants d'échelle, puisque leur valeur peut varier significativement selon les seuils utilisés pour définir les entités urbaines sur lesquelles les quantités urbaines sont intégrées, franchissant même dans certains cas la barrière qualitative de l'échelle linéaire, d'une loi infra-linéaire à une loi super-linéaire. Nous utilisons un modèle théorique simple de distribution spatiale des densités et des fonctions urbaines pour montrer analytiquement qu'un tel comportement peut être dérivé comme conséquence du type de distribution spatiale et de la méthode utilisée. Les simulations numériques confirment les résultats théoriques et révèle que les résultats sont raisonnablement indépendants du noyau spatial utilisé pour distribuer la densité.
}


\bpar{
Scaling laws for urban systems, starting from the well-known rank-size Zipf's law for city size distribution~\cite{gabaix1999zipf}, have been shown to be a recurrent feature of urban systems, at many scales and for many types of indicators. They reside in the empirical constatation that indicators computed on elements of an urban system, that can be cities for system of cities, but also smaller entities at a smaller scale, do fit relatively well a power-law distribution as a function of entity size, i.e. that for entity $i$ with population $P_i$, we have for an integrated quantity $A_i$, the relation $A_i \simeq A_0\cdot \left(\frac{P_i}{P_0}\right)^{\alpha}$. Scaling exponent $\alpha$ can be smaller or greater than 1, leading to infra or supralinear effects. Various thematic interpretation of this phenomena have been proposed, typically under the form of processes analysis. The economic literature has produced abundant work on the subject (see~\cite{Gabaix20042341} for a review), but that are generally weakly spatial, thus of poor interest to our approach that deals precisely with spatial organization. Simple economic rules such as energetic equilibria can lead to simple power-laws~\cite{bettencourt2008large} but are difficult to fit empirically. A interesting proposition by \noun{Pumain} is that they are intrinsically due to the evolutionary character of city systems, where complex emergent interaction between cities generate such global distributions~\cite{pumain2006evolutionary}. Although a tempting parallel can be done with self-organizing biological systems, \noun{Pumain} insists on the fact that the ergodicity assumption for such systems is not reasonable in the case of geographical systems and that the analogy can difficultly be exploited~\cite{pumain2012urban}. Other explanations have been proposed at other scales, such as the urban growth model at the mesoscopic scale (city scale) given in~\cite{2014arXiv1401.8200L} \comment{is this quote also relevant here ?}
 that shows that the congestion within transportation networks may be one reason for city shapes and corresponding scaling laws. Note that ``classic'' urban growth models such as Gibrat's model do provide first order approximation of scaling systems, but that interactions between agents have to be incorporated into the model to obtain better fit on real data, such as the Favaro-Pumain model for innovation cycles propagation proposed in~\cite{favaro2011gibrat}, that generalize a Gibrat model for French cities with an ontology similar to Simpop models.
}{
Les lois d'échelle pour les systèmes urbains, en commençant par la bien connue loi rang-taille de Zipf pour la distribution des tailles des villes~\cite{gabaix1999zipf},\comment{(Florent) déjà dit}
 ont été montrées être une caractéristique récurrente des systèmes urbains, à différentes échelles et pour différents types d'indicateurs. Elles reposent sur la constatation empirique que des indicateurs calculés sur des éléments du système urbain, qui peuvent être les villes dans le cas d'un système de villes, mais aussi des entités plus petites à une plus petite échelle, suivent relativement bien une distribution en loi de puissance en fonction de la taille de l'entité, i.e. pour l'entité $i$ avec population $P_i$, on a pour une quantité intégrée $A_i$, la relation $A_i \simeq A_0\cdot \left(\frac{P_i}{P_0}\right)^{\alpha}$. Les exposants d'échelle $\alpha$ peuvent être plus petits ou plus grands que 1, menant à des effets infra ou supra-linéaires. Diverses interprétations thématiques de ce phénomène ont été proposées, typiquement sous la forme d'analyse des processus. La littérature économique contient une production abondante sur le sujet (voir~\cite{Gabaix20042341} pour une revue), mais est généralement faiblement spatiale, donc de faible intérêt pour notre approche qui s'intéresse particulièrement à l'organisation spatiale. Des règles économiques simples comme un équilibre énergétique peut conduire à de simples lois d'échelles~\cite{bettencourt2008large} mais sont difficiles à ajuster empiriquement. Une proposition intéressante par \noun{Pumain} est qu'elles sont intrinsèquement dues au caractère évolutionnaire des systèmes de villes, où l'émergence complexe par les interactions entre villes génère de telles distributions globales~\cite{pumain2006evolutionary}. Même si un parallèle tentant peut être fait avec les système biologiques auto-organisés \comment{possibly here make a link with morphogenesis - depending if introduced before or not}
 , \noun{Pumain} insiste sur le fait que l'hypothèse d'ergodicité \comment{(Florent) préciser ce que cela signifie}
  pour de tels systèmes n'est pas raisonnable dans le cas de système géographiques et que l'analogie est difficilement exploitable~\cite{pumain2012urban}. D'autres explications ont été proposées à d'autres échelles, comme le modèle de croissance urbaine à échelle mesoscopique (échelle de la ville) donné dans~\cite{2014arXiv1401.8200L} qui montre que la congestion dans les réseaux de transport pourrait être une raison de la forme des villes et des lois d'échelle correspondantes. On peut noter que les modèles ``classiques'' de croissance urbaine comme le modèle de Gibrat~\cite{favaro2011gibrat} fournissent une approximation au premier ordre des systèmes % scaling systems = systèmes scalant ? ~ systèmes exhibant des lois d'échelle <- too long..
  exhibant des lois d'échelles, mais que les interactions entre agents doivent être incorporées dans le modèle pour obtenir un résultat plus fidèle aux données réelles, comme le modèle de Favaro-Pumain pour la propagation des cycles d'innovation proposé dans~\cite{favaro2011gibrat}, qui généralise un modèle de Gibrat pour la croissance des villes françaises avec une ontologie similaire à celle des modèles Simpop.
  \comment{(Florent) ok : modèles qui reproduisent scaling, est-ce un des critères de validation des modèles que tu vas développer ?}
}


\comment{IDEE - take the FavaroPumain again, try to fit/compare with the IntGib-network model ? -- sort of benchmark, should be easy to implement}


\bpar{
However, the blind application of scaling exponents computations was recently pointed as misleading in most cases~\cite{louf2014scaling} \comment{should not cite this ``paper''} 
, confirmed by empirical works such as~\cite{2013arXiv1301.1674A} that showed the variability of computed exponents to the parameters defining urban areas, such as density thresholds. An ongoing work by Cottineau \& \textit{al.} presented at~\cite{cottineau2015scaling}, studies empirically for French Cities the influence of 3 parameters playing a role in city definition, that are a density threshold $\theta$ to delimitate boundaries of an urban area, a number of commuters threshold $\theta_c$ that is the proportion of commuters going to core area over which the unity is considered belonging to the area, and a cut-off parameter $P_c$ under which entities are not taken into account for the linear regression providing the scaling exponent. Remarquable results are that exponents can significantly vary and move from infra-linear to supra-linear when threshold varies. A systematic exploration of parameter space produces phase diagrams of exponents for various quantities. One question raising immediately is how these variation can be explained by the features of spatial distribution of variables. Do they result from intrinsic mechanisms present in the system or can they be explained more simply by the fact that the system is particularly spatialized ? We prove on a toy analytical model that even simple distributions can lead to such significant variations in the exponents, along one dimension of parameters (density threshold), directing the response towards the second explanation.
}{
\comment{(Florent) qu'est ce que ça veut dire, blind application of models ?}

}





%The rest of the section is organized as follows : we formalize the simple framework used and derive an analytical relation between estimated exponent and density threshold parameter. We then present a numerical implementation of the model that confirms numerically theoretical results, explore other form of kernels that would be less tractable, and study the sensitivity along two parameters. We finally discuss the implications of our results and further work needed.

The derivations in the simple case of exponential mixture density, are done in Appendix~\ref{app:technical}.

\comment{mention way of fitting ; golden standard to fit power laws ? check thèse d'Olivier pour voir si le cutoff est appliqué ?}






We formalize the simple theoretical context in which we will derive the sensitivity of scaling to city definition. Let consider a polycentric city system, which spatial density distributions can be reasonably constructed as the superposition of monocentric fast-decreasing spatial kernels, such as an exponential mixture model~\cite{anas1998urban}. Taking a geographical space as $\mathbb{R}^2$, we take for any $\vec{x}\in\mathbb{R}^2$ \comment{(Florent) attention à la sensibilité de certains géographes}
the density of population as
\begin{equation}
d(\vec{x}) = \sum_{i=1}^{N}{d_i(\vec{x})} = \sum_{i=1}^{N}{d_i^0\cdot \exp{\left(\frac{-\norm{\vec{x}-\vec{x}_i}}{r_i}\right)}}
\end{equation}

where $r_i$ are spread parameters of kernels, $d_i^0$ densities at origins, $\vec{x}_i$ positions of centers. We furthermore assume the following constraints :

\begin{enumerate}
\item To simplify, cities are monocentric, in the sense that for all $i\neq j$, we have $\norm{\vec{x}_i - \vec{x}_j}\gg r_i$.
\item It allows to impose structural scaling in the urban system by the simple constraint on city populations $P_i$. One can compute by integration that $P_i=2\pi d_i^0 r_i^2$, what gives by injection into the scaling hypothesis $\ln{P_i}=\ln{P_{max}}-\alpha \ln{i}$, the following relation between parameters : $\ln{\left[d_i^0 r_i^2\right]}=K' - \alpha \ln{i}$.
\end{enumerate}

To study scaling relations, we consider a random scalar spatial variable $a(\vec{x})$ representing one aspect of the city, that can be everything but has the dimension of a spatial density, such that the indicator $A(D)=\Eb{\iint_D{a(\vec{x})d\vec{x}}}$ represents the expected quantity of $a$ in area $D$. We make the assumption that $a\in \{0;1\}$ (``counting'' indicator) and that its law is given by $\Pb{a(\vec{x})=1}=f(d(\vec{x}))$. Following the empirical work done in~\cite{cottineau2015scaling}, the integrated indicator on city $i$ as a function of $\theta$ is given by
\[
A_i(\theta) = A(D(\vec{x}_i, \theta))
\]

where $D(\vec{x}_i, \theta)$ is the area centered in $\vec{x}_i$ where $d(\vec{x})>\theta$. Assumption 1 ensures that the area are roughly disjoint circles. We take furthermore a simple amenity such that it follows a local scaling law in the sense that $f(d)=\lambda\cdot d^\beta$. It seems a reasonable assumption since it was shown that many urban variable follow a fractal behavior at the intra-urban scale~\cite{keersmaecker2003using} and that it implies necessarily a power-law distribution~\cite{chen2010characterizing}. We make the additional assumption that $r_i=r_0$ does not depend on $i$, what is reasonable if the urban system is considered from a large scale. This assumption should be relaxed in numerical simulations. The estimated scaling exponent $\alpha(\theta)$ is then the result of the log-regression of $(A_i(\theta))_i$ against $(P_i(\theta))_i$ where $P_i(\theta)=\iint_{D(\vec{x}_i,\theta)}{d}$.


%%%%%%%%%%%%%%%%%%%%
\subsection{Analytical Derivation of Sensitivity}{Dérivation Analytique de la Sensibilité}

With above notations, let derive the expression of estimated exponent for quantity $a$ as a function of density threshold parameter $\theta$. The quantity computed for a given city $i$ is, thanks to the monocentric assumption and in a spatial range and a range for $\theta$ such that $\theta \gg \sum_{j\neq i}{d_j(\vec{x})}$, allowing to approximate $d(\vec{x})\simeq d_i(\vec{x})$ on $D(\vec{x}_i,\theta)$, is computed by
\[
\begin{split}
A_i(\theta) & = \lambda\cdot \iint_{D(\vec{x}_i,\theta)}{d^\beta} = 2\pi\lambda {d_i^0}^{\beta} \int_{r=0}^{r_0 \ln{\frac{d_i^0}{\theta}}}{r\exp{\left(-\frac{r\beta}{r_0}\right)}dr}\\
& = \frac{2\pi {d_i^0}^\beta r_0^2}{\beta^2} \left[1 + \beta \ln{\frac{\theta}{d_i^0}\left(\frac{\theta}{d_i^0}\right)^\beta} - \left(\frac{\theta}{d_i^0}\right)^\beta\right]
\end{split}
\]

We obtain in a similar way the expression of $P_i(\theta)$
\[
P_i(\theta) = 2\pi d_i^0 r_0^2 \left[1 + \ln{\left[\frac{\theta}{d_i^0}\right]}\frac{\theta}{d_i^0} - \frac{\theta}{d_i^0}\right]
\]

The Ordinary-Least-Square estimation, solving the problem $\inf_{\alpha,C}\norm{(\ln{A_i(\theta)} - C - \alpha \ln{P_i(\theta)})_i}^2$, gives the value $\alpha(\theta) = \frac{\Covb{(\ln{A_i(\theta)})_i}{(\ln{P_i(\theta)})_i}}{\Varb{(\ln{P_i(\theta)})_i}}$. As we work on city boundaries, threshold is expected to be significantly smaller than center density, i.e. $\theta / d_i^0 \ll 1$. We can develop the expression in the first order of $\theta / d_i^0$ and use the global scaling law for city sizes, what gives $\ln{A_i(\theta)} \simeq K_A - \alpha \ln{i} + (\beta - 1)\ln{d_i^0} + \beta \ln{\frac{\theta}{d_i^0}\left(\frac{\theta}{d_i^0}\right)^\beta} $ and $\ln{P_i(\theta)} = K_P - \alpha \ln{i} + \ln{\left[\frac{\theta}{d_i^0}\right]}\frac{\theta}{d_i^0}$. Developing the covariance and variance gives finally an expression of the scaling exponent as a function of $\theta$, where $k_j,{k_j}'$ are constants obtained in the development :

\begin{equation}
\label{eq:th}
\alpha(\theta) = \frac{k_0 + k_1 \theta + k_2 \theta^\beta + k_3 \theta^{\beta + 1} +  k_4 \theta \ln{\theta} + k_5 \theta^\beta \ln{\theta} + k_6 \theta^\beta (\ln{\theta})^2 + k_7 \theta^{\beta + 1}(\ln{\theta})^2 + k_8 \theta^{\beta + 1}\ln{\theta}}{k_0'+k_1' \ln{\theta} + k_2' \theta \ln{\theta} + k_3' \theta^2 + k_4' \theta^2\ln{\theta} + k_5' \theta^2 (\ln{\theta})^2}
\end{equation}

This rational fraction predicts the evolution of the scaling exponent when the threshold varies. We study numerically its behavior in the next section, among other numerical experiments.


%%%%%%%%%%%%%%%%%%%%
%\subsection{Numerical Simulations}{Simulations Numériques}

%\paragraph{Implementation}{Implémentation}


%\comment{(Florent) définir ton champ d'investigation (des grilles carrées de taille prédéfinies, ce n'est pas du tout standard)}

%We implement empirically the density model given in section~\ref{sec:formalization}. Centers are successively chosen such that in a given region of space only one kernel dominates in the sense that the sum of other contributions are above a given threshold $\theta_e$.\comment{(Florent) est-ce toujours possible, y'a t-il unicité du centre ? Par quelle méthode précise détermine tu le centre ?}
%In practice, adapting $N$ to world size allows to respect the monocentric condition. Population are distributed in order to follow the scaling law with fixed $\alpha$ and $r_i$ (arbitrary choice) by computing corresponding $d_i^0$. Technical details of the implementation done in R~\cite{R-Core-Team:2015fk} and using the package \texttt{kernlab} for efficient kernel mixture methods~\cite{Karatzoglou:2004uq} are given as comments in source code\footnote{available at \texttt{https://github.com/JusteRaimbault/CityNetwork/tree/master/Models/Scaling}}. \comment{(Florent) cela ne suffit pas, il faut en dire plus sur la méthode}[sure, surtout qu'on formule cette requete dans la partie méthodologique précédente, tout cela est un peu contradictoire..]
%We show in figure~\ref{fig:ex-distrib} example of synthetic density distributions on which the numerical study is conducted. The validation of theoretical results on these experimental mixtures must still be conducted, along with sensitivity tests to random perturbations, influence of kernel type, and two-parameters phase diagram when adding in the computational model functional density distribution and associated cut-off threshold.
%Theoretical result obtained in Eq.~\ref{eq:th} are studied and confronted to emprically computed values for various parameter as shown in Fig.~\ref{fig:th_results}.

%\comment{(Florent) TB mais la encore, on ne sait pas précisément pourquoi tu te lances là dedans}


%%%%%%%%%%%%%%%%%%
%\begin{figure}
%\centering
%\includegraphics[width=0.4\textwidth]{Figures/Scaling/example_exp_mixture}
%\appcaption{Example of a synthetic density distribution obtained with the exponential mixture, with a grid of size $400\times 400$ and parameters $N=20$, $r_0=10$, $P_{max}=200$, $\alpha=0.5$, $\theta_C = 0.01$.}{Exemple de distribution synthétique}
%\label{fig:ex-distrib}
%\end{figure}
%%%%%%%%%%%%%%%%%%



%\begin{figure}
%\centering
%
%\caption{Validation of theoretical result through numerical simulation.}
%\label{fig:th_results}
%\end{figure}



%\paragraph{Random Perturbations}{Perturbations aléatoires}

%The simple model used is quite reducing for maximal densities and radius distribution. We aim to proceed to an empirical study of the influence of noise in the system by fixing $d_i^0$ and $r_i$ the following way :
%\begin{itemize}
%\item $d_i^0$ follows a reversed log-normal distribution with maximal value being a realistic maximal density
%\item Radiuses are computed to respect rank-size law and then perturbed by a white noise. \comment{(Florent)  pourquoi ?}
%\end{itemize}

%Results shown in Fig.~\ref{fig:random-density} are quantitatively different from previous one, as expected, but the same qualitative behavior is reproduced.


%\begin{figure}
%\centering

%\caption{Variation of exponents with variable origin density and radius.}
%\label{fig:random-density}
%\end{figure}



%\paragraph{Kernel Type}{Type de Noyau}

%We shall test the influence of the type of spatial kernel used on results. We can test gaussian kernels and quadratic kernels with parameters within reasonable ranges analog to the exponential kernel. %As shown in Fig.~\ref{fig:other-kernels}, we obtain the same qualitative results that is the significant variation of $\alpha(\theta)$ as a function of $\theta$.


%
%\begin{figure}
%\centering

%\caption{Scaling exponents for other kernels.}
%\label{fig:other-kernels}
%\end{figure}

%\paragraph{Two-parameters phase diagram}

%We introduce now a second spatial variable that has also an influence on the definition of urban entities, that is the proportion of actives working in city center, as done on empirical data in~\cite{cottineau2015scaling}. To simplify, it is used only to define urban parameter but assumed as having no influence on the local probability distribution of the amenity which stays the same function of the density. We write 

%\begin{figure}
%\centering

%\caption{Two parameters phase diagram.}
%\label{fig:two-params}
%\end{figure}

%%%%%%%%%%%%%%%%%%%%
%\subsection{Discussion}

%%%%%%%%%%%%%%%%%%%%
%\subsection{Conclusion}














%----------------------------------------------------------------------------------------

\newpage




%\section[Spatio-temporal Correlations][Correlations spatio-temporelles]{Linking dynamic and static spatio-temporal correlations under simplified assumptions}{Lien entre correlation spatio-temporelles statiques et dynamiques sous hypothèses simplifiées}
\section{Spatio-temporal Correlations}{Correlations spatio-temporelles}



\label{sec:app:spatiotempcorrs}


\bpar{
Space and Time are both crucial for the study of geographical systems when aiming to understand \emph{processes} (by definition dynamical~\cite{hypergeo}) 
 evolving in a \emph{spatial structure} in the sense of~\cite{dollfus1975some}. Space is more than coordinates for elements of the system, but a dimension in itself that drives interactions and thus system properties. Reading geographical systems from the point of view of \emph{spatio-temporal processes} emphasizes the fact that \emph{space actually matters}. Space and time are closely linked in such processes, and depending on the underlying mechanisms, one can expect to extract useful information from one on the other : in certain cases that we will investigate in this part, it is for example possible to learn about dynamics from static information. Spatio-temporal correlations approaches, linked to spatio-temporal dynamics, are present in very broad fields such as dynamical image processing (including video compression)~\cite{chalidabhongse1997fast,hansen2004accelerated,ke2007spatio}, target tracking~\cite{belouchrani1997direction,vuran2004spatio}, climate science~\cite{cressie1999classes}, Earth sciences~\cite{ma2002spatio}, city systems dynamics~\cite{hernando2015memory,pigozzi1980interurban}, among others.
}{
L'espace et le temps sont cruciaux pour l'étude des systèmes géographiques quand on cherche à comprendre les \emph{processus} (par définition dynamiques~\cite{hypergeo}) 
\comment{(Florent)  c'est déjà une lecture, certes processus renvoie à une évolution,mais les échelles de temps du modèle/processus ne sont pas nécessairement les mêmes} 
qui évoluent dans une structure spatiale au sens de~\cite{dollfus1975some}.
\comment{(Florent) citer Cottineau ?}
}

\cite{cross1994spatiotemporal} : spatio-temporal chaos


The capture of neighborhood effects in statistical models is a wisely used practice in spatial statistics, as the technique of Geographically Weighted Regression illustrates~\cite{brunsdon1998geographically}. A possible interpretation among many definitions of spatial autocorrelation~\cite{griffith1992spatial} yields that by estimating a plausible characteristic distance for spatial correlations or auto-correlations, one can isolate independent effects between variables from effects due to neighborhood interactions\footnote{note that the formal link between models of spatial autocorrelation (see e.g. \cite{griffith2012advanced}) is not clear and should be further investigated}. The study of the spatial covariance structure is a cornerstone of advanced spatial statistics that was early formulated~\cite{griffith1980towards}. \comment{(Florent) cela semble tout de même loin du sujet ou alors il faut que tu expliques clairement}
 We propose now to study possible links between spatial and temporal correlations, using spatio-temporal covariance structure to infer information on dynamical processes.


\subsection{Notations}{Notations}

We consider a multivariate spatio-temporal stochastic process denoted by $\vec{Y}\left[\vec{x},t\right]$. At a given point $\vec{x}_0$ in space, we can define temporal covariance structure by
\[
\mathbf{C}_t (\vec{x}_0) = \Varb{\vec{Y}\left[\vec{x}_0, \cdot\right]}
\]

and spatial covariance structure at fixed time by
\[
\mathbf{C}_x (t) = \Varb{\vec{Y}\left[\cdot, t\right]}
\]

It is clear that these quantities will be in practice first ill-defined because of the difficulty in interpreting such a process by a spatio-temporal random variable, secondly highly non-stationary in space and time. We stay however at a theoretical level to gain structural knowledge,\comment{(Florent) sens ?}
 reviewing simple cases in which a formal link can be established.


\subsection{Wave Equation}{Equation des Ondes}

\comment{(Florent) pourquoi aborder cela ?}[cas idéal des STARMA, ondes d'innovation etc. : approche fondamentalement liée à l'analyse spatiale, mais bien plus complexe qu'une simple équation. justifie cette approche de lien spatio-temporel ?]

In the case of propagating waves, there is an immediate link. Let assume that a wave equation if verified by ``deterministic'' parts of components

\begin{equation}
c^2 \cdot \partial^2_{t} \bar{Y}_i = \Delta \bar{Y}_i
\end{equation}

with $Y_i = \bar{Y}_i + \varepsilon_i$. If errors are uncorrelated and processes are stationary, we have then directly

\begin{equation}
\mathbf{C}_t \left[ \partial^2_t Y_i , \partial^2_t Y_j \right] = \frac{1}{c^2} \cdot\mathbf{C}_x \left[ \Delta Y_i , \Delta Y_j \right]
\end{equation}

This gives us however few insight on real systems as local diffusion, stationary assumptions and uncorrelated noises are far from being verified in empirical situations.

\subsection{Fokker-Planck Equation}{Equation de Fokker-Planck}

An other interesting approach may when the process verifies a Fokker-Planck equation on probabilities of the state of the system when it is given by its position (diffusion of particles in that case)

\begin{equation}
\partial_t P(x_i,t) = - d \cdot \partial_x P(x_i,t) + \frac{\sigma^2}{2} \partial^2_x P(x_i,t)
\end{equation}

With no cross-correlation terms in the Fokker-Planck equation, covariance between processes vanish. We have finally in that case only a relation between averaged spatial and temporal variances that brings no information to our question.

\subsection{Master Equation}{Equation Maitresse}

In the case of a master equation on probabilities of discrete states of the system

\begin{equation}
\partial_t \vec{P} = \mathbb{W} \vec{P}
\end{equation}

we have then for state $i$, $\partial_t P_i = \sum_j W_{ij}P_j$. As this relation is at a fixed time we can average in time to obtain an equation on temporal covariance. It is not clear how to make the link with spatial covariance as these will depend on spatial specification of discrete states. This question is still under investigation.


\subsection{Consistent spatio-temporal sampling}{Echantillonnage spatial cohérent}

In a more empirical way, we propose to not assume any contraint of process dynamics but to however investigate how the computation of spatial correlations can inform on temporal correlations. We try to formulate easily verifiable assumptions under which this is possible.

We make the following assumptions on the spatio-temporal stochastic processes $Y_i\left[\vec{x},t\right]$ :
\begin{enumerate}
\item Local spatial autocorrelation is present and bounded by $l_{\rho}$ (in other words the processes are continuous in space) : at any $\vec{x}$ and $t$, $\left|\rho_{\norm{\Delta \vec{x}} < l_{\rho}}\left[Y_i (\vec{x}+\Delta \vec{x},t), Y_i (\vec{x},t) \right]\right| > 0$. \comment{(Florent) je ne comprends pas ce qui est écrit, qu'une abs soit > 0 ok, donc c'est autre mais quoi ?}[c'est le strict > 0 qui compte, c'est une façon de postuler que les processus sont continus à une certaine échelle fine]
\item Processes are locally parametrized : $Y_i = Y_i\left[\alpha_i\right]$, where $\alpha_i (\vec{x})$ varies with $l_{\alpha}$, with $l_{\alpha} \gg l_{\rho}$.
\item Spatial correlations between processes have a sense at an intermediate scale $l$ such that $l_{\alpha}\gg l \gg l{\rho}$.
\item Processes covariance stationarity times scale as $\sqrt{l}$.
\item Local ergodicity is present at scale $l$ and dynamics are locally chaotic.
\end{enumerate}


Assumptions one to three can be tested empirically and allow to compare spatial correlation estimated on spatial samplings at scale $l$. Assumption four is more delicate as we are precisely constructing this methodology because we have no temporal information on processes. It is however typical of spatial diffusion processes, and population or innovation diffusion should verify this assumption. \comment{(Florent)  cela devrait être un point de départ (expliquerait pourquoi ces modèles ; te ferait peut être en considérer d'autres}
 The last assumption can be tested if feasible space is known, by checking cribbing on image space on the spatial sample. Under these conditions, local spatial sampling is equivalent to temporal sampling and spatial correlation estimators provide estimator of temporal correlations.









%----------------------------------------------------------------------------------------

\newpage



\section{Generation of Correlated Synthetic Data}{Génération de Données Synthétiques Corrélées}


\label{app:sec:syntheticdata}



\bpar{
Generation of hybrid synthetic data resembling real data to some criteria is an important methodological and thematic issue in most disciplines which study complex systems. Interdependencies between constituting elements, materialized within respective relations, lead to the emergence of macroscopic patterns. Being able to control the dependance structure and level within a synthetic dataset is thus a source of knowledge on system mechanisms. We propose a methodology consisting in the generation of synthetic datasets on which correlation structure is controlled. The method is applied in a first example on financial time-series and allows to understand the role of interferences between components at different scales on performances of a predictive model. A second application on a geographical system is then proposed, in which the weak coupling between a population density model and a network morphogenesis model allows to simulate territorial configurations. The calibration on morphological objective on european data and intensive model exploration unveils a large spectrum of feasible correlations between morphological and network measures. We demonstrate therein the flexibility of our method and the variety of possible applications.
}{
La génération de données synthétiques hybrides similaires à des données réelles présente des enjeux méthodologiques et thématiques pour la plupart des disciplines dont l'objet est l'étude de systèmes complexes. Comme l'interdépendance entre les éléments constitutifs d'un système, matérialisée par leur relations, conduit à l'émergence de ses propriétés macroscopiques, une possibilité de contrôle de l'intensité des dépendances dans un jeu de données synthétiques est un instrument de connaissance du comportement du système. Nous proposons une méthodologie de génération de données synthétiques hybrides sur lequel la structure de correlation est contrôlée. La méthode est illustrée sur des séries temporelles financières et permet l'étude de l'interférence entre composantes à différentes fréquences sur la performance d'un modèle prédictif, en fonction des correlations entre composantes à différentes échelles. On présente ensuite une application à un système géographique, dans laquelle le couplage faible d'un modèle de distribution de densité de population avec un modèle de génération de réseau permet la simulation de configurations territoriales, qui sont calibrées selon des objectifs morphologiques sur l'ensemble de l'Europe. L'exploration intensive du modèle permet l'obtention d'un large spectre de valeurs pour la matrice de correlation entre mesures morphologiques et mesures du réseau. On démontre ainsi les possibilités d'applications variées et les potentialités de la méthode.
}




\subsection{Context}{Contexte}

\bpar{

The use of synthetic data, in the sense of statistical populations generated randomly under constraints of patterns proximity to the studied system, is a widely used methodology, and more particularly in disciplines related to complex systems such as therapeutic evaluation~\cite{abadie2010synthetic}, territorial science~\cite{moeckel2003creating,pritchard2009advances}, machine learning~\cite{bolon2013review} or bio-informatics~\cite{van2006syntren}. It can consist in data desegregation by creation of a microscopic population with fixed macroscopic properties, or in the creation of new populations at the same scale than a given sample, with criteria of proximity to the real sample. These criteria will depend on expected applications and can for example vary from a restrictive statistical fit on given indicators, to weaker assumptions of similarity in aggregated patterns. In the case of chaotic systems, or systems where emergence plays a strong role, a microscopic property does not directly imply given macroscopic patterns, which reproduction is indeed one aim of modeling and simulation practices in complexity science. With the rise of new computational paradigms~\cite{arthur2015complexity}, data (simulated, measured or hybrid) shape our understanding of complex systems. Methodological tools for data-mining and modeling and simulation (including the generation of synthetic data) are therefore crucial to be developed.
}{
L'utilisation de données synthétiques, au sens de populations statistiques d'individus générées aléatoirement sous la contrainte de reproduire certaines caractéristiques du système étudié, est une pratique méthodologique largement répandue dans de nombreuses disciplines, et particulièrement pour des problématiques liées aux systèmes complexes, telles que par exemple l'évaluation thérapeutique~\cite{abadie2010synthetic}, l'étude des systèmes territoriaux~\cite{moeckel2003creating,pritchard2009advances}, l'apprentissage statistique~\cite{bolon2013review} ou la bio-informatique~\cite{van2006syntren}. Il peut s'agir d'une désagrégation par création d'une population au niveau microscopique présentant des caractéristiques macroscopiques données, ou bien de la création de nouvelles populations au même niveau d'agrégation qu'un échantillon donné avec un critère de ressemblance aux données réelles.  Le niveau de ce critère peut dépendra des applications attendues et peut par exemple aller de la fidélité des distributions statistiques pour un certain nombre d'indicateurs à des contraintes plus faibles de valeurs pour des indicateurs agrégés, c'est à dire l'existence de motifs macroscopiques similaires. Dans le cas de systèmes chaotiques ou présentant de fortes caractéristiques d'émergence, une contrainte microscopique n'implique pas nécessairement le respect des motifs macroscopiques, et arriver à les reproduire est justement un des enjeux des pratiques de modélisation et simulation en sciences de la complexité. La donnée, qu'elle soit simulée, mesurée ou hybride est au coeur de l'étude des systèmes complexes de par la maturation de nouvelles approches computationelles~\cite{arthur2015complexity}, il est donc essentiel d'étudier des procédures d'extraction d'information des données (fouille de données) et de simulation d'une information similaire (génération de données synthétiques).
}




\bpar{
Whereas first order (in the sense of distribution moments) is generally well used, it is not systematic nor simple to control generated data structure at second order, i.e. covariance structure between generated variables. Some specific examples can be found, such as in~\cite{ye2011investigation} where the sensitivity of discrete choices models to the distributions of inputs and to their dependance structure is examined. It is also possible to interpret complex networks generative models~\cite{newman2003structure} as the production of an interdependence structure for a system, contained within link topology. We introduce here a generic method taking into account dependance structure for the generation of synthetic datasets, more precisely with the mean of controlled correlation matrices.
}{
Si le premier ordre est de manière générale bien maîtrisé, il n'est pas systématique ni aisé de contrôler le second ordre, c'est à dire les structures de covariance entre les variables générées, même si des exemples spécifiques existent, comme dans~\cite{ye2011investigation} où la sensibilité des sorties de modèles de choix discrets à la forme des distributions des variables aléatoires ainsi qu'à leur structures de dépendance. Il est également possible d'interpréter les modèles de génération de réseaux complexes~\cite{newman2003structure} comme la création d'une structure d'interdépendance au sein d'un système, représentée par la topologie des liens. Nous proposons ici une méthode générique prenant en compte l'interdépendance lors de la génération de données synthétiques, sous la forme de correlations.
}



\bpar{
Domain-specific methods aforementioned are too broad to be summarized into a same formalism. We propose a framework as generic as possible, centered on the control of correlations structure in synthetic data.
}{
L'ensemble des méthodologies mentionnées en introduction sont trop variées pour être résumées par un même formalisme. Nous proposons ici une formulation générique ne dépendant pas du domaine d'application, ciblée sur le contrôle de la structure de correlation des données synthétiques.
}


%%%%%%%%%%%%%
\subsection{Formalization}{Formalisation}


\bpar{
Let $\vec{X}_I$ a multidimensional stochastic process (that can be indexed e.g. with time in the case of time-series, but also space, or discrete set abstract indexation). We assume given a real dataset $\mathbf{X}=(X_{i,j})$, interpreted as a set of realizations of the stochastic process. We propose to generate a statistical population $\mathbf{\tilde{X}}=\tilde{X}_{i,j}$ such that
\begin{enumerate}
\item a given criteria of proximity to data is verified, i.e. given a precision $\varepsilon$ and an indicator $f$, we have $\norm{f(\mathbf{X})-f(\mathbf{\tilde{X}})} < \varepsilon$
\item level of correlation is controlled, i.e. given a matrix $R$ fixing correlation structure (symmetric matrix with coefficients in $[-1,1]$ and unity diagonal), we have $\hat{\Var{}}\left[(\tilde{X}_i)\right] = \Sigma R \Sigma$, where the standard deviation diagonal matrix $\Sigma$ is estimated on the synthetic population.
\end{enumerate}
}{
Soit un processus stochastique multidimensionnel $\vec{X}_I$ (l'ensemble d'indexation pouvant être par exemple le temps dans le cas de séries temporelles, l'espace, ou une indexation quelconque). On se propose, à partir d'un jeu de réalisations $\mathbf{X}=(X_{i,j})$, de générer une population statistique $\mathbf{\tilde{X}}=\tilde{X}_{i,j}$ telle que
\begin{enumerate}
\item d'une part un certain critère de proximité aux données est vérifié, i.e. étant donné une précision $\varepsilon$ et un indicateur $f$, $\norm{f(\mathbf{X})-f(\mathbf{\tilde{X}})} < \varepsilon$
\item d'autre part le niveau de correlation est controlé, i.e. étant donné une matrice fixant une structure de covariance $R$, $\Varb{(\tilde{X}_i)} = R$, où la matrice de variance/covariance est estimée sur la population synthétique.
\end{enumerate}
}


\bpar{
The second requirement will generally be conditional to parameter values determining generation procedure, either generation models being simple or complex ($R$ itself is a parameter). Formally, synthetic processes are parametric families $\tilde{X}_i[\vec{\alpha}]$. \comment{explicit the fact that real data may come out of different parameter values ?}
We propose to apply the methodology on very different examples, both typical of complex systems : financial high-frequency time-series and territorial systems. We illustrate the flexibility of the method, and claim to help building interdisciplinary bridges by methodology transposition and reasoning analogy. In the first case, proximity to data is the equality of signals at a fundamental frequency, to which higher frequency synthetic components with controlled correlations are superposed. It follows a logic of hybrid data for which hypothesis or model testing is done on a more realistic context than on purely synthetic data. This example that has no thematic link with the thesis, is presented in Appendix~\ref{app:syntheticdata}. In the second case, morphological calibration of a population density distribution model allows to respect real data proximity. Correlations of urban form with transportation network measures are empirically obtained by exploration of coupling with a network morphogenesis model. The control is in this case indirect as feasible space is empirically determined.
}{
La satisfaction du deuxième point sera généralement conditionnée par la valeur de paramètres, dont dépendra la procédure de génération, qu'il s'agisse de modèles simples ou complexes. Formellement, les processus synthétiques sont des familles paramétriques $\tilde{X}_i[\vec{\alpha}]$. Nous proposons de décliner cette méthode sur deux exemples très différents mais tous deux typiques des systèmes complexes : des séries temporelles financières à haute fréquence, et les systèmes territoriaux. On illustre ainsi la flexibilité de la logique, ouvrant des portes interdisciplinaires par l'exportation de méthodes ou raisonnements par exemple. Dans le premier cas, la proximité aux données est l'égalité des signaux à une fréquence fondamentale, auxquels on superpose des composantes synthétiques dont il est facile de contrôler le niveau de correlation. On se place dans une logique de données hybrides, pour tester des hypothèses ou modèles dans un contexte plus proche de la réalité que sur des données purement synthétiques. Cet exemple, sans rapport thématique avec la thèse, est présenté en Appendice~\ref{app:syntheticdata}. Dans le deuxième cas, la calibration morphologique d'un modèle de distribution de densité de peuplement permet de respecter le critère de proximité aux données. Les correlations de la forme urbaine avec celle d'un réseau de transport sont ensuite obtenues empiriquement par exploration du couplage avec un modèle de génération de réseau. Leur contrôle est dans ce cas indirect puisque constaté empiriquement.
}




%%%%%
% from space matters

\paragraph{An alternative view : synthetic data}{Une vue alternative: données synthétiques}

Let $M_{m}$ a stochastic model of simulation, which inputs are to simplify initial conditions $D_0$ and parameters $\vec{\alpha}$, and output $M_{m}\left[\vec{\alpha},D_0\right](t)$ at a given time $t$. We assume that it is partially data-driven in the sense that $D_0$ is supposed to represent a real situation at a given time, and model performance is measured by the distance of its output at final time to the real situation at the corresponding time, i.e. error function is of the form $\norm{\Eb{\vec{g}(M_{m}\left[\vec{\alpha},D_0\right](t_f))}-\vec{g}(D_f)}$ where $\vec{g}$ is a deterministic field corresponding to given indicators.


Evaluating the model on real data is rapidly limited in control possibilities, being restricted to the search of datasets allowing natural control groups. Furthermore, statistical behaviors are generally poorly characterized because of the small number of realizations. Working with synthetic data first allows to solve this issue of robustness of statistics, and then gives possibilities of control on some ``meta-parameters'' in the sense described before.

%Some remarks can be made on the approach: We can ask what knowledge are brought by adding the upstream model, rather than exploring a large set of initial geometries ? To obtain a sufficiently large set of initial configuration, one quickly needs a model to generate them ; in that case a quasi-random generation followed by a filtering on morphological constraint will be a morphogenesis model, which parameters are the ones of the generation and the filtering methods. Furthermore, as detailed further, the determination of the derivative of the downstream model is made possible by the coupling and knowledge of the upstream model.
%\comment{(Florent) est-ce important pour la thèse ?}
%If statistical noise is added by coupling models, indeed repetitions needed for convergence are indeed larger as the final expectance has to be determined by repeating on the first times the second model ; but it is exactly the same as exploring directly many configuration, to obtain statistical robustness in that case one must repeat on similar configurations.
%Finally, if complexity is added by coupling models, coupling is simple and no complexity is thus added.

\comment{link between synthetic data and model coupling ?}

% necessity of separation of scales and ontologies for space matters to work ? not necessarily (deleted paragraph). think to that.















