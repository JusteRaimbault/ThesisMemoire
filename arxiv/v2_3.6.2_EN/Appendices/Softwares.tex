%----------------------------------------------------------------------------------------

%\newpage

\section{Softwares and packages}{Packages et logiciels} % Chapter title

\label{app:sec:packages} % For referencing the chapter elsewhere, use \autoref{ch:name} 

%----------------------------------------------------------------------------------------


%\headercit{}{}{}

% PUBLIER ces logiciels / packages sur des repos après cleaning !
% Q : also density generator ; space matters : generic workflows / grids ?



\bpar{
This section describes the significant software contributions, which were the object of a packaging in the spirit of an open science. It is difficult to decide at which time an implementation and possibly a library developed in a particular context can be made generic and distributed in an autonomous way. We made the choice (i) of relatively general functions; (ii) of a strong potential impact; and (iii) of a certain level of maturity in the packaging.
}{
Cette section recense les contributions logicielles significatives, qui ont fait l'objet d'un \emph{packaging} dans l'esprit d'une science ouverte. Il est difficile de décider à quel moment une implémentation et éventuellement une bibliothèque développées dans un cadre particulier peuvent être rendus génériques et distribués de manière autonome. Nous avons fait le choix (i) de fonctions relativement générales ; (ii) d'un fort impact potentiel ; et (iii) d'un certain niveau de maturité au niveau du packaging.
}





%%%%%%%%%%%%%%%%%%
\subsection{largeNetwoRk: network import and simplification for R}{largeNetwoRk : import de réseau et simplification pour R}


\paragraph{Description}{Description}

\bpar{
The \texttt{largeNetwoRk} package for the \texttt{R} language is aimed at the import and the simplification of massive transportation networks. It is constructed in particular for the import of OpenStreetMap data, but can tackle other formats such as shp. The objective is to allow analyses of networks on large surfaces while having access to modest computational capabilities, and to make transparent the import of spatial data into a topological graph.
}{
Le package \texttt{largeNetwoRk} pour le langage \texttt{R} est destiné à l'import et la simplification des réseaux de transport massifs. Il est particulièrement construit pour l'import de données OpenStreetMap, mais peut traiter d'autres formats comme shp. L'objectif est de permettre des analyses des réseaux sur de grandes surfaces tout en disposant de capacités de calcul modestes, et de rendre transparent l'import des données spatiales en un graphe topologique.
}


\paragraph{Characteristics}{Caractéristiques}

%\begin{itemize}
%\item Language : \texttt{R}, \texttt{Shell}, \texttt{PostgreSQL}
%\item Size : 919
%\end{itemize}


\bpar{
The package is fully written in \texttt{R}, and requires a connection with a \texttt{PostgreSQL} database (the the PostGis extension installed). Source code is available at \url{https://github.com/JusteRaimbault/CityNetwork/tree/master/Models/TransportationNetwork/NetworkSimplification} with the documentation.
}{
Le package est intégralement écrit en \texttt{R}, et requiert une connexion avec une base \texttt{PostgreSQL} (avec extension PostGis installée). Le code source est disponible à \url{https://github.com/JusteRaimbault/CityNetwork/tree/master/Models/TransportationNetwork/NetworkSimplification} avec la documentation.
}


\paragraph{Functions}{Fonctions}

\bpar{
The main functions implemented are the following:
\begin{itemize}
	\item \texttt{constructLocalGraph}: construct a topological graph from spatial lines queried from the postgis database (in a given spatial extent)
	\item \texttt{graphFromSpdf}: constructs a topological graph from a spatial data structure (allows for example to import from a \texttt{shp} file)
	\item \texttt{mergeGraphs}: merge two graphs neighbors in space
	\item \texttt{simplifyGraph}: simplification of a graph (see algorithm in~\ref{app:sec:staticcorrelations})
	\item \texttt{connexify}: gives a connected graph from an arbitrary graph, through the addition of connectors
	\item \texttt{exportGraph}: exports a topological graph in the database
\end{itemize}
}{
Les principales fonctions suivantes sont implémentées :
\begin{itemize}
	\item \texttt{constructLocalGraph} : construit un graphe topologique à partir des lignes spatiales issues de la base postgis (dans une étendue spatiale précisée)
	\item \texttt{graphFromSpdf} : construit un graphe topologique à partir d'une structure de données spatiale (permet d'importer depuis un fichier \texttt{shp} par exemple)
	\item \texttt{mergeGraphs} : fusion de deux graphes voisins dans l'espace
	\item \texttt{simplifyGraph} : simplification d'un graphe (voir algorithme en~\ref{app:sec:staticcorrelations})
	\item \texttt{connexify} : donne un graphe connecté à partir d'un graphe quelconque, par l'ajout de connecteurs
	\item \texttt{exportGraph} : exporte un graphe topologique dans la base de données
\end{itemize}
}

\bpar{
A complete script allows moreover to execute the \emph{split and merge} algorithm described in~\ref{app:sec:staticcorrelations} for the simplification of large spatial extents.
}{
Un script complet permet par ailleurs l'exécution de l'algorithme \emph{split and merge} décrit en~\ref{app:sec:staticcorrelations} pour la simplification de grandes étendues spatiales.
}


\paragraph{Particularities}{Particularités}

\bpar{
The use on massive data requires a parallel processing. Furthermore, the external program \texttt{osmosis} is used for the initial conversion of OpenStreetMap data (\texttt{osm pbf} for example) and their import into the postgis database.
}{
L'utilisation sur données massives requière un traitement en parallèle. De plus, le programme externe \texttt{osmosis} est utilisé pour la conversion initiale des données OpenStreetMap (\texttt{osm pbf} par exemple) et leur import dans la base postgis.
}






%%%%%%%%%%%%%%%%%%
\subsection{Transportation networks and accessibility in R}{Réseaux de transports et accessibilité en R}


\paragraph{Description}{Description}

\bpar{
The package \texttt{tRansport} for the \texttt{R} language provides transparent primitives for computing indicators for public transportation networks and the associated accessibility computations. Starting from datasets including lines and stations for different transportation modes, it allows constructing a multimodal topological network and to compute different measures given geographical variables.
}{
Le package \texttt{tRansport} pour le langage \texttt{R} rend transparent les calculs d'indicateurs pour les réseaux de transport en commun et les calculs d'accessibilité associés. A partir de jeux de données comprenant lignes et stations pour différents modes de transports en commun, il permet de construire un réseau topologique multimodal et de calculer différentes mesures étant donné des variables géographiques.
}


\paragraph{Characteristics}{Caractéristiques}

\bpar{
The package is written in language \texttt{R} and produces graphs following the \texttt{igraph} package structure. Source code and documentation are available at \url{https://github.com/JusteRaimbault/CityNetwork/tree/master/Models/TransportationNetwork/NetworkAnalysis}.
}{
Le package est écrit en langage \texttt{R} et produit des graphes selon la structure du package \texttt{igraph}. Le code source et la documentation sont disponibles à \url{https://github.com/JusteRaimbault/CityNetwork/tree/master/Models/TransportationNetwork/NetworkAnalysis}.
}


\paragraph{Functions}{Fonctions}

\bpar{
The main following functions are available:
\begin{itemize}
	\item \texttt{addTransportationLayer}: constructs a graph from a layer of the network, or adds a layer to an existing network, from a shapefile description of links and nodes (stations) of the network.
	\item \texttt{addPointsLayer}: adds a layer of points, which can then be origin or destination of itineraries. They are linked to the closest station by connector links which speed is specified.
	\item \texttt{addAdministrativeLayer}: similar function, which connects the centroids of a polygon layer typically representing administrative areas, keeping their attributes as node attributes.
	\item \texttt{computeAccess}: computes the accessibility between points of the transportation network, following different specifications: travel time, weighting at the origin and/or destination by specified data.
\end{itemize}
}{
Les principales fonctions suivantes sont disponibles :
\begin{itemize}
	\item \texttt{addTransportationLayer} : construit un graphe à partir d'une couche du réseau, ou ajoute une couche à un réseau existant, à partir d'une description shapefile des liens et des noeuds (gares) du réseau.
	\item \texttt{addPointsLayer} : ajoute une couche de points, qui peuvent être alors origine ou destination d'itinéraires. Ils sont reliés à la gare la plus proche par des connecteurs dont la vitesse est spécifiée.
	\item \texttt{addAdministrativeLayer} : fonction similaire, qui connecte les centroïdes d'une couche de polygones représentant typiquement des zones administratives, conservant leur attributs comme attributs de noeuds.
	\item \texttt{computeAccess} : calcule l'accessibilité entre des point du réseau de transport, selon plusieurs spécifications : temps de trajet, pondération à l'origine et/ou à la destination par des données spécifiées.
\end{itemize}
}


%%%%%%%%%%%%%%%%%%
\subsection{morphology: a NetLogo extension to measure urban form}{morphology : extension NetLogo pour mesurer la forme urbaine}

\label{app:subsec:morphologyextension}

\paragraph{Description}{Description}

\bpar{
The \texttt{morphology} extension for NetLogo5 allows computing in an efficient and transprent way the morphological indicators introduced in~\ref{sec:staticcorrelations} (Moran index, entropy, average distance, hierarchy), for the spatial distribution of an arbitrary patch variable.
}{
L'extension \texttt{morphology} pour NetLogo5 permet de calculer de manière efficiente et transparente les indicateurs morphologiques introduits en~\ref{sec:staticcorrelations} (indice de Moran, entropie, distance moyenne, hiérarchie), pour la distribution spatiale d'une variable de patch quelconque.
}


\paragraph{Characteristics}{Caractéristiques}

\bpar{
The extension is written in \texttt{scala} and is compatible the version 5 of NetLogo. It is available at \url{https://github.com/JusteRaimbault/nl-spatialmorphology}.
}{
L'extension est écrite en \texttt{scala} et est compatible avec les versions 5 de NetLogo. Elle est disponible à \url{https://github.com/JusteRaimbault/nl-spatialmorphology}.
}


\paragraph{Particularities}{Particularités}

\bpar{
The indicators implying a convolution (Moran index, average distance) are implemented with a fast Fourier transform, allowing decreasing the complexity from a $O(N^4)$ to a $O(N^2\cdot \log^2 N)$ if $N$ is the width of the grid.
}{
Les indicateurs impliquant une convolution (indice de Moran, distance moyenne) sont implémentés par transformée de Fourier rapide, permettant de faire passer la complexité d'un $O(N^4)$ à un $O(N^2\cdot \log^2 N)$ si $N$ est la taille d'un côté de la grille. 
}


%%%%%%%%%%%%%%%%%%%%%%%%%%%
\subsection{TorPool}{TorPool}


\paragraph{Description}{Description}

\bpar{
\texttt{TorPool} is a java wrapper for the tor software, which allows maintaining a pool of instance in parallel, and to renew these instances on demand. An interface with TorPool is available with java with a dedicated library. This tool allows in particular facilitating the automatic collection of data. It is available as source and executable at \url{https://github.com/JusteRaimbault/TorPool}.
}{
\texttt{TorPool} est un wrapper java du logiciel tor, qui permet de maintenir une équipe d'instances en parallèle, et de renouveler ces instances sur demande. Une interface avec TorPool est disponible avec java par une bibliothèque dédiée. Cet utilitaire permet entre autres de faciliter la collection automatique de données. Il est disponible comme source et sous forme executable à \url{https://github.com/JusteRaimbault/TorPool}.
}

\paragraph{Functions}{Fonctions}

\bpar{
The software is launched as a \texttt{jar} executable, and opens a specified range of ports as local \texttt{socks5} proxies to the Tor network.
}{
Le logiciel se lance sous forme d'executable \texttt{jar}, et ouvre une plage spécifiée de ports en proxy \texttt{socks5} local vers le réseau Tor.
}


\bpar{
The associated java library allows to (i) establish a connexion with the proxies, (ii) to ask for a renewal of instances, allowing a change of circuit in the network.
}{
La bibliothèque java associée permet (i) d'établir une connexion avec les proxys (ii) de demander un renouvellement des instances, permettant un changement de circuit dans le réseau.
}





%%%%%%%%%%%%%%%%%%
\subsection{Scientific corpus mining}{Fouille de corpus scientifique}

\paragraph{Description}{Description}

\bpar{
The tools developed in the context of Chapter~\ref{ch:modelinginteractions}, and of Appendices~\ref{app:sec:cybergeo} and~\ref{app:sec:patentsmining} allow in a general way the mining of scientific corpuses, from the viewpoint of the citation network and the semantic network.
}{
Les outils développés dans le cadre du Chapitre~\ref{ch:modelinginteractions}, et des Annexes~\ref{app:sec:cybergeo} et~\ref{app:sec:patentsmining} permettent de manière générale la fouille de corpus scientifiques, du point de vue du réseau de citation et du réseau sémantique.
}

\paragraph{Characteristics}{Caractéristiques}

\bpar{
As recalled in~\ref{app:sec:cybergeo}, the tasks required are relatively heterogenous, and different languages are therefore used: Java for data collection, python for textual analysis, R for network analysis. The version of the different scripts used for Chapter~\ref{ch:modelinginteractions} is available at \url{https://github.com/JusteRaimbault/CityNetwork/tree/master/Models/QuantEpistemo/HyperNetwork}.
}{
Comme rappelé en~\ref{app:sec:cybergeo}, les tâches requises sont assez hétérogènes, et différents langages sont alors mobilisés : Java pour la collection des données, python pour l'analyse textuelle, R pour les analyses de réseau. La version des différents scripts utilisés pour le Chapitre~\ref{ch:modelinginteractions} est disponible à \url{https://github.com/JusteRaimbault/CityNetwork/tree/master/Models/QuantEpistemo/HyperNetwork}.
}


\paragraph{Functions}{Fonctions}

\bpar{
The following functions are ensured: (i) collection of the citation network from an initial corpus, collection of abstracts for a corpus; (ii) extraction of keywords as n-grams, estimation of the relevance of keywords; (iii) construction of semantic and citation networks.
}{
Les fonctions suivantes sont assurées : (i) collecte du réseau de citation à partir d'un corpus initial, collecte des résumés d'un corpus ; (ii) extraction des mots-clés sous forme de n-grams, estimation de la pertinence des mots clés ; (iii) construction des réseaux sémantiques et de citation.
}








