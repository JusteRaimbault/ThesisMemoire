

%\chapter{Modeling Interactions between Networks and Territories}{Modéliser les Interactions entre Réseaux et Territoires} % Chapter title
\bpar{
\chapter{Modeling interactions between networks and territories}
}{
\chapter{Modéliser les interactions entre réseaux et territoires}
}

\label{ch:modelinginteractions}

%----------------------------------------------------------------------------------------



\bpar{
The empirical and thematic literature, together with the case studies previously developed, seem to converge towards a consensus on the complexity of relations between transportation networks and territories. In some configurations and at some scales, it is possible to exhibit circular causal relationships between territorial dynamics and transportation networks dynamics. We designate their existence through the concept of \emph{co-evolution}. It seems to be difficult to introduce simple or systematic explanations for these dynamics, as recall for example the debates around structuring effects of infrastructures~\cite{offner1993effets}.
}{
La littérature empirique et thématique, ainsi que les cas d'études développés précédemment, semblent converger vers un consensus sur la complexité des relations entre réseaux de transport et territoires. Dans certaines configurations et à certaines échelles, il est possible de mettre en valeur des relations circulaires causales entre dynamiques territoriales et dynamiques des réseaux de transports. Nous désignons leur existence par le concept de \emph{co-évolution}. Il semble difficile d'introduire des explications simples ou systématiques de ces dynamiques, comme le rappelle par exemple les débats autour des effets structurants des infrastructures~\cite{offner1993effets}.
}

\bpar{
Furthermore, the multiple geographical situations suggest a strong dependency to the context, giving a relevance to fieldwork and to targeted studies. But geographical explanation and the understanding of processes remains quickly limited in this approach, and intervenes a need for a certain level of generality. Its on such a point that the evolutive urban theory is focused in particular, since it allows to combine schemes and general models to the geographical particularities. On the contrary, some theories coming from physics applying to the study of urban systems~\cite{west2017scale} can be more difficult to accept for geographers because of their universality positioning which is on the opposite of their ordinary epistemologies.
}{
Par ailleurs, les multiples situations géographiques suggèrent une forte dépendance au contexte, donnant une pertinence au travail de terrain et aux études ciblées. Or l'explication géographique et la compréhension des processus est très vite limitée dans cette approche, et intervient un besoin d'un certain niveau de généralisation. C'est sur un tel point que la théorie évolutive des villes se concentre en particulier, puisqu'elle permet de combiner des schémas et modèles généraux aux particularités géographiques. Au contraire, certaines théories issues de la physique s'appliquant à l'étude des systèmes urbains~\cite{west2017scale} peuvent être plus difficiles à accepter pour les géographes de par leur positionnement d'universalité qui est à l'opposé de leurs épistémologies habituelles.
}


\bpar{
In any case, the \emph{medium} which allows to gain in generality on processes and structures of systems is always the model. As \noun{J.P. Marchand} puts it\footnote{Personal communication, May 2017.}, ``\textit{our generation has understood that there was a co-evolution, yours aims at understanding it}'', what insists on the power of understanding brought by modeling and simulation that we judge to be today still with a very high potential for development.
}{
Dans tous les cas, le \emph{medium} qui permet de gagner en généralité sur les processus et structures des systèmes est toujours le modèle. Comme le rappelle \noun{J.P. Marchand}\footnote{Communication personnelle, Mai 2017.}, ``\textit{notre génération a compris qu'il y avait une co-évolution, la votre cherche à la comprendre}'', ce qui appuie le pouvoir de compréhension apporté par la modélisation et la simulation que nous jugeons être encore aujourd'hui à très fort potentiel de développement.
}


\bpar{
Without developing for now the numerous functions that a model can have, we will rely on the positioning of \noun{Banos} which states that ``modeling is learning'', and following our positioning within a complex systems science suggested in introduction, we will thus make \emph{modeling interactions between networks and territories} our principal subject of study, tool, object\footnote{Even if after a rereading of this positioning at the light of~\ref{sec:knowledgeframework}, it has no meaning since our appproach already contained models as soon as it was scientific.}. This chapter must be taken as a ``state-of-the-art'' of approaches modeling interactions between networks and territories. It aims in particular at capturing different dimensions of knowledge: therefore, we will uses quantitative epistemology analyses.
}{
Sans développer pour le moment les nombreuses fonctions que peut avoir un modèle, nous nous baserons sur la position de \noun{Banos} qui soutient que ``modéliser c'est apprendre'', et suivant notre positionnement dans une science des systèmes complexes suggéré en introduction, nous ferons ainsi de la \emph{modélisation des interactions entre réseaux et territoires} notre principal sujet d'étude, outil, objet\footnote{Même si dans une relecture à la lumière de~\ref{sec:knowledgeframework} de ce positionnement n'a pas de sens puisque notre démarche contenait déjà des modèles à partir du moment où elle était scientifique.}. Ce chapitre doit être pris comme un ``état de l'art'' des démarches de modélisation des interactions entre réseaux et territoires. Il vise en particulier à capturer différentes dimensions des connaissances : pour cela, nous mobiliserons des analyses en épistémologie quantitative.
}


\bpar{
In a first section~\ref{sec:modelingsa}, we review in an interdispclinary perspective the models that can be concerned, even remotely, without a priori of temporal or spatial scale, of ontologies, of structure, or of application context. This overview is possible thanks to the diverse disciplinary entries revealed in the previous chapter: for example geography, transportation geography, planning. This overview suggests relatively independent knowledge structures and disciplines that rarely communicate.
}{
Dans une première section~\ref{sec:modelingsa}, nous passons en revue de manière interdisciplinaire les modèles pouvant être concernés, même de loin, sans a priori d'échelle temporelle ou spatiale, d'ontologies, de structure, ou de contexte d'application. Cet aperçu est possible par les entrées disciplinaires diverses révélées au chapitre précédent : par exemple géographie, géographie des transports, planification. Cet aperçu suggère des structures de connaissances assez indépendantes et des disciplines ne communiquant que rarement.
}


\bpar{
We proceed in~\ref{sec:quantepistemo} to an algorithmic systematic review, which corresponds to a reconstruction by iterative exploration of a scientific landscape. Its results tend to confirm this compartmentalization. The study is completed by a multilayer network analysis, combining citation network and semantic network obtained through text-mining, which allows to better grasp the relations between disciplines, their lexical field and their interdisciplinarity patterns.
}{
Nous procédons dans~\ref{sec:quantepistemo} à une revue systématique algorithmique, qui correspond à une reconstruction par exploration itérative d'un paysage scientifique. Ses résultats tendent à confirmer ce cloisonnement. L'étude est complétée par une analyse de réseau multi-couches, combinant réseau de citations et réseau sémantique issu d'analyse textuelle, qui permet de mieux cerner les relations entre disciplines, leur champs lexicaux et leur motifs d'interdisciplinarité.
}


\bpar{
This study allows the construction of a corpus used for the modelography (typology of models) and the meta-analysis (characterization of this typology) done in the last section~\ref{sec:modelography}. It dissects the nature of several models and link it to the disciplinary context, what sets up the foundations and the precise frame of the modeling efforts that will be developed in the following.
}{
Cette étude permet la constitution d'un corpus utilisé pour la modélographie (typologie de modèles) et la méta-analyse (caractérisation de cette typologie) effectuée en dernière section~\ref{sec:modelography}. Celle-ci dissèque la nature d'un certain nombre de modèles et la relie au contexte disciplinaire, ce qui pose les bases et le cadre précis des efforts de modélisation qui seront développés par la suite.
}



%Les modèles de changement d'usage du sol très appliqués en planification sont tout autant concernés que des modèles totalement abstraits issus de la biologie ou de la physique, que des approches intégrées en géographie ou spécifiques en économie\comment[FL]{pas utile ici tu detailleras en 2.1 : ici affirmer les objectifs (quelle information en retirer) et les moyens (par ez meme automatique tu as bien un (des) points d'entrees $\rightarrow$ lesquels et pourquoi?}.



\stars


\bpar{
\textit{This chapter is unpublished for its first section; uses in its second section the text of~\cite{raimbault2015models}, and then in its second subsection the methodology introduced by \cite{raimbault2016indirect} and developed in~\cite{raimbault2017exploration} and also the tools of cite{bergeaud2017classifying}; it is finally unpublished for its last part.}
}{
\textit{Ce chapitre est inédit pour sa première section ; reprend dans sa deuxième section le texte traduit de~\cite{raimbault2015models}, puis pour sa deuxième partie la méthodologie introduite par \cite{raimbault2016indirect} et développée dans~\cite{raimbault2017exploration} ainsi que les outils de \cite{bergeaud2017classifying} ; et enfin est inédit pour sa dernière partie.}
}








