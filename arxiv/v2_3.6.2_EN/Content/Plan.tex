


%%%%%%%%%%%%%%%%%%%%
%% Organisation and general points



%\comment{(Florent) cf receuil articles du Monde sur Grd Paris (numériser)}[est ce aue je te les ai donnes, sinon redemande moi.]


%\comment{(Florent) trop peu ancré concrètement dans le champ des interactions transport/ville - enchainement idée ok mais revoir granularité info. Catalogue de situations complexes d'interactions forme urbaine/transport à reproduire.}[Personnes interessantes avec qui discuter de ces sujets au lvmt (liste non exhaustive) : Jean Laterasse, Caroline Gallez.]


%\comment[FL]{Lu Aout 17 : chap 4-2,4-3 : mal dit, on n'est pas vraiment dans la coevolution ; Annexes : a vue de nez il y en a trop ce nest pas evident de dire a priori ce qui est en trop mais des annexes = 50\% de la these, cela fairt bizarre ; Titre des figures : bien sur a ajuster au fil de la lecturem mais ces titres semblent non explicites, il faut pouvoir les comprendre hors du contexte, a minima.}


%\comment[FL]{(05/01/2018) manque d'objets (titres, figures,\ldots )qui ``structurent'' le document $\rightarrow$ ici notés des etonnements d'une lecture a vol d'oiseau}




%%%%%%%%%%%%%%%%%%%%
% TODO - todo
%
% -- TODO --


% - checker redondances bibliographie, et exactitude des records, publications arxiv, etc. !! biblio des appendices dans la générale ? a voir.
% - inclure éthique de la connaissance de Monod.
% - why not include artwork as appendix. science<->art (// poésies ?)
% - appendix : ecogeo - include results and discussions ; idem conf Medium ?
% - Morphogenesis : * reclarify link between dynamics and form in Thom's theory ; * find time to do quant epistemo ; * on the definition of the system and boundaries : missing part ?
% - somewhere clarification and discussion on definitions of emergence, Bedau etc. ? done on several points.
% - sur multi-scale modeling : who does some and how much ? à ajouter quelque part
% - pour la PDE : voir Villani, mentionner ?
% - effective dimension of urban system : sens ? (comment density) : question assez profonde à méditer - lié info sémantique portugali ? lié représentation minimale ?
% - on density multi-scale dvlpmt : pas clair ce que tu as en tête ici : je ne sais pas si tu auras le temps de creuser cela, mais pour du multi-scalaire, les schémas sont très aidant car c'est vite difficile à visualiser
% - morphogénétiques : est-ce que le mot existe, orthographe pas logique !
% - correlated synthetic data : appendix : inclure regressions
% - revoir discussion correlated synthetic data
% - aller interviewer JP Marchand (Theo Quant) (citation comprendre la co-évolution)
% - make tables uniform and clean
% - recompute indicators with capacity and/or hierarchy when possible with speed limits, to check how they change, and also correlations ? -> speeds pas assez fiables a priori ; fait toujours avec dans tous les cas
% - \cite{2015arXiv150600348A} : Enhanced Gravity Model, how to take better into account heterogeneous network topology, using entropy maximisation combined to gravity model. : a caser quelque part, ressemble sur le coupling, et gravité (sugg. intgib au début)

% - Knowledge framework : 
%    * portugali IRSSN dans la partie theory - conceptualisation.
%    * introduce necessary domains : at least of necessary domains, that we postulate these ones ; there may be an infinity  ?
%    * ajouter des éléments interview Clem.

% - Modelography : 
%    * tenter une classif endogène des modèles : selon les caractéristiques récupérées ? pas forcément pertinent vu taille des données
%    * processus extraction de manière ``synthétique'' : implique des catégories, des choix subjectif. ; idem région géographique : suppose une typologie de classement, niveaux etc. Both out of purpose for now.
%    * idem  pour modèles statistiques : conclusion sens causalité/signif. Travail plus poussé en lui-même. A Faire plus tard.
%

% - interviews : voir où inclure, selon quantité et contenu.

% - partie supplémentaire dans Modélographie : dissection précise des vrais modèles de coévolution ? (pas tant que ça).

% - StaticCorrelations : 
%    * finir-choisir maps pour la Chine en annexe ; complémentaires celles Europe ?
%
% - CausalityRegimes : \cite{kasraian2010interaction} à caser dans les études stat ?
%
% - observation flottante : La disposition d'esprit peut être rapprochée de la philosophie -> zen ? pshychédélique ici ?

% - Macro exploration : explorer le Portugali ?

% - Macro application FR : base autoroutes ("data paper" en annexe ? -> penser à faire un vrai data paper avec Florent.)

% - Gaelle Lesteven Metro toulouse - thèse : http://confins.revues.org/7653 ; exemple postdoc ? pas publié a priori.



%% -- DONE --
%
%  X - attention aspect uniformisateur/limite totalitariste vision sci ? -> ok clarifier avec anarchisme et Feyerabend.
%  X - trouver un moyen d'élargir systématiquement toutes les figures ?
%  X - (Arnaud) (sur le plan) : dans méthodo : ajouter Modèle agents - hors équilibre -> small part in review. -- OK partie sur why modeling, modélisation générative, simulation.
%  X - StaticCorrelations : * GWR pour échelle optimale.
%. X - Intro Generale : Florent : donner ici exemples dans champ transports/urba ; plus de détails sur les disciplines CS ?  --> deux pas nécessaire dans la logique de l'intro generale qui positionne CS
%  X - (thematic 1.1.1) \comment{(Florent)Une strategie à adopter serait d'abord de decrire de facon basique, avec exemples concrets, la complexité des interactoins réseau/espace/settelements, puis de rappeler CS et proprietes, puis de decrire lesquelles de ces propriétés presentes dans ces interactions, lequelles modèles vont essayer de reproduire et pquoi.} -> le problème, comme le révèle la modelography et la biblio, puis les expériences de modélisation, trop vaste, notion coevol mal définie, peut d'info systématiques. la partie fondations fait partiellement ce travail, seulement après celle-ci on peut avoir une vision un peu plus claire. --  \comment{(Florent) TB mais en parler avant, c'est cela le coeur} (difficulté intrinsèque, réseau qui façonnent territoire) -- OK finalement proche stratégie finale
%  X - (thematic 1.1.1) \comment{(Florent)pas besoin ni interet de se positionner sur emergence des societes} : si car lié à évolution culturelle, les villes comme étape de celle-ci : dans une perspective plus large (cf projet postdoc), fondamental.


%%%%%%%%%%%%%%%%%%%%
% TODO - include notes

% -> notes.md




%%%%%%%%%%%%%%%%%%%%
% TODO - Reading Records
%
%% -- A LIRE -- 

% - lire Morin sur la pensée complexe
% - Moore Nature of Computation
% - (Arnaud) A LIRE : R Brunet, Discontinuités en géographie ; Pierre Dumolard (Espace Différencié) ; Guy DiMeo (L'homme, la société, l'espace)


%% -- A INCLURE --
%



%%%%%%%%%%%%%%%%%%%%
% TODO - Ideas

% - cours : la ville est la combinaison de forces opposées, souvent contradictoire. (cours analyse spatiale). far-from-equilibrium of course.

%  -- URGENT -- (when writing comments on network growth models) : do a sort of list of processes, implied objects, etc. / or a tab , from modeling, theories and CASES STUDIES --


% -  companion site à la Seb ?

% -  a note on open review via git ?



%%%%%%%%%%%%%%%%%%%%
% TODO - Citations
%
% -> potential citations


%  - Nous vivons avec quelques arpents du passé, les gais mensonges du présent et la cascade furieuse de l'avenir. Autant continuer à sauter à la corde, l'enfant chimère à notre cote. - Rene Char (Fenetre dormantes et porte sur le toit)

%  - Le sérieux n'est que la crasse accumulée dans les têtes vides - Roland Topor

%  - Science is an essentially anarchic enterprise - Paul Feyerabend, Against Method

%  - Victor Hugo : le fond c'est la forme qui remonte à la surface [SOURCE ???]

%\headercit{Mais ce n'est pas une question d'{\^a}ge, de chiffres et de stats\\ Moi je te parle surtout de rage, de kif et d'espoir}{Youssoupha}{\textit{, Esperance de Vie}}

% %\headercit{Do or do not. There is no try.}{Yoda}{}

% %\headercit{We need to find Banos' tenth modeling law}{Ren{\'e} Doursat}{}

%%%%
% voie de garage

%  - % The Social Construction of What ?}{Ian Hacking}{\cite{hacking1999social} -> pas vraiment une citation, et pas adapté à grand chose..
 
%\headercit{Your theory is crazy, but not enough to be true.\comment{(Florent) rigolo mais le rapport avec le sujet est discutable}}{Niels Bohr}{}





%%%%%%%%%%%%%%%%%%%%
% TODO - Uncited
%
% -> uncited refs, why.
%
%2017arXiv170108673P -> number of states of HMM : ?
%levy1993t -> Levy theory on territory : need a deeper read and connections.
%pumain2017geography
%batty2017cities
%2017arXiv170107861D
%nicosia2009extending
%10.1371/journal.pone.0170830
%varma2017hpc
%Munafo:2017uq
%railsback2017
%2017arXiv170102973L
%2017arXiv170102383G
%shashok2017can
%miandoabchi2013multi
%farahani2013review
%fujita1982multiple
%bitbol2004autopoiesis
%dollens2014alan
%Chavalarias2016
%2016arXiv161208111S
%2016arXiv161208338T
%friesz1985transportation
%sui2004tobler
%miller2004tobler
%tobler2004first : Tobler giving precisions on the first law of geography
%2016arXiv161205463G
%raimbault2016discrepancy
%raimbault2016investigating
%2016arXiv161102269V
%2015arXiv150402550T
%batty2005agents
%xue2006spatial
%shen2002urban
%xu2005city
%10.1371/journal.pone.0166011
%10.1371/journal.pone.0166004
%2016arXiv161103232L
%hou2011transport
%cao2012accessibility
%huang2016association
%10.1371/journal.pone.0164553
%chan2005location
%tardy2004role
%perret2015roads
%makse1995modelling
%2016arXiv160904636V
%clauset2004finding
%2016arXiv160902000G
%2016arXiv160808839C
%Downey30082016
%osmosis
%xie2009topological
%rozenfeld2008laws
%2016arXiv160805770C
%2016arXiv160806897H
%2015arXiv151207603T
%duranton2007urban
%dimeo2016geographie
%2016arXiv160608103M
%raimbault2016generation
%raimbault2015hybrid
%raimbault2015user
%raimbault2016system
%swerts2013systemes
%swerts2015megacities
%florida2008rise
%hall1982great
%2016arXiv160805266R
%aveline2016medium
%lenechet:halshs-01272236
%2016arXiv160804472J
%ioannidis2005most
%2016arXiv160803608M
%batty2007creative
%chen2012wider
%hall1997modelling
%knowles2016sir
%reid2016decision
%carreira2000mode
%yu2012solving
%wang2015resilience
%masucci2014exploring
%fessel2012physarum
%gaughan2016spatiotemporal
%raimbault2016simpopsan
%10.1371/journal.pone.0160471
%10.1371/journal.pone.0159496
%pan2003croissance
%2016arXiv160708472M
%lyu2016developing
%emangard2009transports
%e18060197
%ishiguro1997bootstrapping
%Woodhouse19072016
%10.1371/journal.pone.0158826
%10.1371/journal.pcbi.1004947
%2016arXiv160703186A
%10.1371/journal.pone.0157261
%saichev2009theory
%karrer2011stochastic
%gastner2006shape
%10.1371/journal.pone.0157728
%2015arXiv150607608T
%2016arXiv160601959F
%sornette1997convergent
%solomon1996spontaneous
%gabaix2003theory
%newling1966urban
%bretagnolle2000long
%pumain2006villes
%okabe1987theoretical
%dellaposta2016endogenous
%Squartini:2013fk
%Takeuchi20153109
%2016arXiv160501949B
%klimek2012empirical
%10.1371/journal.pone.0154839
%2016arXiv160408816Z
%2016arXiv160407876G
%jiang2007topological
%akaike1998information
%heiss2008likelihood
%2005physics..12106P
%hall1990methodology
%burnham2004multimodel
%manning2014stanford
%thom1974stabilite
%benguigui1991suburban
%durand1990notion
%batty1991generating
%dauphine1995chaos
%dupuy:halshs-00438867
%damm1980response
%coffman1998railroad
%knight1977evidence
%goldberg1972evaluation
%alcaly1976transportation
%aveline2003ville
%2016arXiv160402872L
%2016arXiv160400758D
%2016arXiv160404155A
%2016arXiv160403904B
%echenique2012growing
%Brunton12042016
%banos2011christaller
%10.1371/journal.pone.0152686
%dupuy1993geographie
%newman1996land
%kenworthy2002transport
%tirnakli2015standard
%benettin1980lyapunov
%hijmans2015geographic
%10.1371/journal.pone.0151676
%offner2000territorial
%bourgine2010morphogenesis
%Fujita199693
%lenormand2015comparing
%soler2014calculating
%gallotti2015transportation
%krugman1998space
%lenormand2012generating
%decraene2013emergence
%varenne2008epistemologie
%derrible2010complexity
%10.1371/journal.pone.0150932
%10.1371/journal.pone.0148660
%le2015modeling
%mehaffy2007notes
%cuyala2013diffusion
%ciotti2015homophily
%el2006access
%2015arXiv151003797G
%2016arXiv160208451P
%choi2014patent
%shibata2008detecting
%zembri2010new
%2015arXiv150901940M
%schmid1994probabilistic
%noruzi2005google
%bohannon2014scientific
%2015arXiv150601280B
%2016arXiv160106075O
%10.1371/journal.pone.0147913
%barthelemy2015time
%raffestin1982remarques
%Gao:2016ty
%fujita1996economics
%2016arXiv160203774H
%blondel2008fast
%min2011real
%schultz2014random
%jedwab2013transportation
%offner2002x
%bitner2009complex
%raimbault2015models
%2015arXiv151205659M
%achlioptas2009explosive
%10.1371/journal.pone.0146491
%2015arXiv151207715C
%2015arXiv151205259R
%2015arXiv151200946S
%2015arXiv151201423W
%di1998espace
%2015arXiv151105468E
%2014arXiv1403.3005S
%Brummitt20150712
%beckman1996creating
%simini2012universal
%masucci2013gravity
%witten1981diffusion
%kuhnert2006scaling
%aghion2002schumpeterian
%andersson2002urban
%frankhauser1998fractal
%2015arXiv151006326H
%Sinatra:2015yu
%su2008effect
%fraedrich1986estimating
%chen2009spatial
%2015arXiv150909055P
%Perret:2015fk
%2015arXiv150907599C
%2015arXiv150803542B
%pumain2006hierarchy
%vattay2015quantum
%2014arXiv1403.7686B
%2015arXiv150904486M
%2015arXiv150903678R
%2015arXiv150905183R
%2015arXiv150905590H
%2015arXiv150904558J
%Pohlert:2015fk
%cheng2004notes
%cristelli2012there
%fox2005revisiting
%newman2005power
%clauset2009power
%kyriakidou2011applying
%bretagnolle:halshs-00159894
%sanders2006artificial
%courbaud2015applying
%2015arXiv150707878C
%10.1371/journal.pone.0133780
%gilli2005bassin
%servais2004polycentrisme
%leveque:halshs-00280396
%urbanek2011emdist
%wikle1999dimension
%galka2004solution
%vitanov2015test
%hens2015extreme
%2015arXiv150507372L
%pan2013urban
%tretyakov2011fast
%pumain2004urban
%Capozza1990187
%thevenin2013mapping
%schwartz2011spatial
%roth2012long
%10.1371/journal.pone.0029721
%Ye2014200
%donaldson2010railroads
%lucas1998mechanics
%moretti2004human
%banos2012network
%gauvin2009phase
%chen2010characterizing
%kang2012intra
%Karatzoglou:2004uq
%R-Core-Team:2015fk
%anderson1991turning
%underhill1990soviet
%baklanov2015projects
%perret2010multi
%crucitti2006centrality
%frankhauser2008fractal
%pumain1998urban
%Schwarz201029
%hebert2011structural
%batty2006hierarchy
%lugovoy2007analysis
%dorogovtsev2000structure
%pumain2002role : epistemology on TQG
%Mandelbrot1961198
%Mandelbrot195990
%bailey1999funk
%roopkumar2006generalized
%pumain2015multilevel
%reuillon2015
%10.1371/journal.pcbi.1004101
%anas1998urban
%ribeiro2010game
%Roumboutsos2008209
%le2010approche
%ordeshook1986game
%lenechet:halshs-00674059
%lenechet2012
%keitt2011rgdal
%baddeley2004spatstat
%gallego2010population
%hirtzel:tel-01121665
%martin2004generating
%abadie2011synth
%coulombel:tel-00601262
%strano2012elementary
%networkQgis
%mills2000thematic
%fujita2004new
%cookbookForR
%leurent2012disaggregate
%fields1999city
%dori2002object
%zeigler1989devs
%louail:tel-00584495
%goldspink2000modelling
%Liu201326
%lechner2006procedural
%parish2001procedural
%beauguitte2014r
%andersson2003urban
%brown2005spatial
%Magliocca2011183
%Ettema20111
%phan2010agent
%clarke1998loose
%He2006323
%kocabas2006coupling
%Kunz2000597
%zanette1997role
%clarke2007decade
%achibetmorphogenese
%rui2013urban
%lodin2011road
%porta2006network
%andersson2006complex
%eboli2012exploring
%duranton2012urban
%horner2012analyzing
%2013PhRvL.111s8702L
%antoni:hal-00914269
%berroir2005contribution
%guerois2002commune
%hall2006polycentric
%offner:halshs-00438903
%portugali2012complexity
%
%

% X biernacki2000assessing -> empirical likelihood (paper interactionGibrat)
% X bon2017novel -> SJS
% X bettencourt2007growth : Bettencourt scaling theory of cities
% X bourgine2010morphogenesis : in morphogenesis
% X loo1999development : planning transportation in DPR
% X favaro2007croissance : thèse JM Favaro
% X paulus2004coevolution : specialisation of urban areas






