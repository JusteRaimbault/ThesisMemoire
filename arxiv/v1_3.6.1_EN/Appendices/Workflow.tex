
\section{Tools and workflow for an open reproducible research}{Outils et pratiques pour une recherche ouverte et reproductible}

\label{app:workflow} % For referencing the chapter elsewhere, use \autoref{ch:name} 

%----------------------------------------------------------------------------------------


%\headercit{Open for Discovery}{PLoS}{}


%% 
%   Technical elements / workflow to be included :
%
%   - NLaggregator : useful ?
%   - NLDocumentation : yes
%   - git usage
%   - towards a git-compatible metafig ? / metadata-handler 
%   !! importance of metadata for transparency/reproducibility
%
%


\bpar{
We briefly evoke here tools, practices, and development directions for a more transparent, free, open and fluid research.
}{
Nous décrivons ici brièvement des outils, des pratiques et des pistes de développement pour une recherche plus transparente, libre, ouverte et fluide.
}


\subsection{NetLogo documentation generator}{Générateur de Documentation Netlogo}


\bpar{
Documentation generation is central for reproducibility, as it can automatize the description of a model implementation. NetLogo does not provide a documentation generator. We implemented a \texttt{Doxygen} software (generation of documentation for different languages including Java) for its application to the NetLogo language. It basically consists in generating intermediate Java code, mirror of the NetLogo code in its object structures and containing the comment blocks of the NetLogo code. An experimental version is available at \url{https://github.com/JusteRaimbault/CityNetwork/tree/master/Models/Doc}.
}{
La génération de documentation est centrale pour la reproductibilité, permettant d'automatiser la description de l'implémentation d'un modèle. NetLogo ne fournit pas de générateur de documentation. Nous avons implémenté un wrapper du logiciel \texttt{Doxygen} (génération de documentation pour divers langages dont Java) pour son application au langage NetLogo. Il repose sur le principe basique de génération d'un code Java intermédiaire, miroir du code NetLogo dans ses structures objet et reprenant les blocs de commentaires dans le code NetLogo. Une version expérimentale est disponible à \url{https://github.com/JusteRaimbault/CityNetwork/tree/master/Models/Doc}.
}
% TODO may be more relevant to produce a template or a fork for a robust doc generation engine ?


\subsection{git as a reproducibility tool}{git comme outil de reproductibilité}


\bpar{
The use if \texttt{git} as a reproducibility and transparency tool has been emphasized by~\cite{ram2013git}, which list numerous advantages such as the exact tracking of the history of the knowledge production process, an immediate cloning (in combination with public repositories, for which collaborative sites exist such as github or gitlab), a possibility to branch from past commits.
}{
L'utilisation de \texttt{git} comme outil de reproductibilité et de transparence a été mis en valeur par~\cite{ram2013git}, qui soulignent de nombreux avantages tels le suivi exact de l'historique du processus de production de connaissance, un clonage immédiat (en combinaison avec des dépôts publics, pour lesquels existent des sites collaboratifs comme github ou gitlab), une possibilité de branchage à partir de commits passés.
}

\bpar{
This tool furthermore allows facilitating the individual workflow, providing for example an automatic backup, an organisational support, the following of experiments.
}{
Cet outil permet d'autre part de faciliter le flux de travail individuel, permettant par exemple le backup automatique, l'organisation, le suivi des expériences.
}



\subsection{Open review}{Revue ouverte}

\bpar{
The review process of this manuscript has experimentally tested an open review, through the use of the git repository and specific \LaTeX commands. The basic \texttt{{\textbackslash}comment} command allows the reviewers to insert their comments in the appropriate place (and is then placed as a margin annotation of the manuscript) and allows a discussion up to 5 consecutive answers through optional arguments. A \emph{pull request} from the reviewer branch allows integrating the feedbacks. Other commands for example allow marking changes or inserting lists of tasks.
}{
Le processus de revue de ce manuscrit a expérimentalement testé la revue ouverte, par l'utilisation du dépôt git et de commandes \LaTeX spécifiques. La commande basique \texttt{{\textbackslash}comment} permet aux relecteurs d'insérer leur commentaires à l'endroit approprié (et se place alors en annotation de marge dans le manuscrit) et permet une discussion jusqu'à 5 réponses successives par des arguments optionnels. Une \emph{pull request} depuis la branche du relecteur permet d'intégrer les retours. D'autres commandes permettent par exemple de marquer les changements ou d'insérer des listes de tâches.
}


\bpar{
One of the advantages of this approach is that it is a posteriori possible to reconstruct the review process, and that it is totally open (for a potential review of the review). The automation by traversing the network of the git repository history is even easily considerable. 
}{
L'un des intérêts de cette démarche est qu'il est possible a posteriori de reconstruire le processus de revue, et que celui-ci est entièrement ouvert (pour une éventuelle revue de la revue). L'automatisation par parcours du réseau de l'historique du dépôt git est même facilement envisageable.
}



% git data totally obsolete with git-lfs ? not exactly the same ?

%\subsection{git-data}{git-data}

%\texttt{git-data} is a shell based (experimental) git extension, available at \url{https://github.com/JusteRaimbault/gitdata}, that allows automatized backup of large file within a git repository, their transparent integration in ignored files and the creation of symbolic links for a transparent local use.





%%%%%%%%%%%%%%%%%

\subsection{Towards a git-compatible metadata handler}{Vers un gestionnaire de métadonnées compatible avec git}

\bpar{
The issue of conserving metadata for figures is crucial for reproducibility, since it is often difficult to keep trace of the full configuration having generated a figure, and also of the corresponding code, since it can be modified by older versions. The use of script environments such as R or python can also build some traps since variables can be modified without modifying the code, and the full history of executed commands must then be kept.
}{
La question de la conservation des métadonnées pour les figures est cruciale pour la reproductibilité, puisqu'il est souvent difficile de garder une trace de l'ensemble de la configuration ayant généré une figure, ainsi que le code correspondant, celui-ci pouvant être modifié par des versions antérieures. L'utilisation d'environnements de scripts comme R ou python peuvent également être piégeurs puisque les variables peuvent être modifiées sans modification du code, et il faut garder alors l'ensemble de l'historique des commandes exécutées.
}


\bpar{
The exhaustive storage of data, the environment, code and the history which led to the generation of a precise figure are a necessary condition for an exact reproducibility. A direction to answer this issue is the construction of a tool compatible with git which would automatically generate these metadata, for example by creating a proper branch and conserving the commit hash associated to the figure. The final idea would be to have for each figure a unique identifier linking it to the exact environment having produced it, also implying an automation of the index system within the documents using them.
}{
Le stockage exhaustif des données, de l'environnement, du code et de l'historique qui a conduit à la génération d'une figure précise sont une condition nécessaire pour une reproductibilité exacte. Une piste pour répondre à ce problème est l'élaboration d'un outil compatible avec git qui générerait automatiquement ces métadonnées, par exemple en créant une branche propre et en conservant le hash du commit associé à la figure. L'idée finale serait d'avoir pour chaque figure un identifiant unique la reliant à l'environnement exact l'ayant produite, impliquant également une automatisation du système d'indexation au sein des documents les utilisant.
}
% TODO rq: containers are a brutal but sure solution (could be containers by project with history in it?) - link with reprod. lit


%%%%%%
% eventuellement getUncited ?





\stars










