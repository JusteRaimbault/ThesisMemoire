




%----------------------------------------------------------------------------------------


%%%%%%%%%%%%%%%%%%%%%%%
\section{Quantitative Epistemology}{Épistémologie quantitative}

\label{app:sec:quantepistemo}


%----------------------------------------------------------------------------------------

\subsection{Algorithmic systematic review}{Revue systématique algorithmique}



\paragraph{Algorithm description}{Description de l'algorithme}



\bpar{
Let $A$ be an alphabet (an arbitrary set of symbols), $A^{\ast}$ corresponding words (strings of finite length on it). Texts of finite length on it are then $T = \cup_{k\in \mathbb{N}} {A^{\ast}}^k$. What we call a reference is for the algorithm a record with text fields representing title, abstract and keywords. Set of references at iteration $n$ will be denoted $\mathcal{C}_n \subset T^3$: it is a subset of text triplets. We assume the existence of a set of keywords $\mathcal{K}_n$, initial keywords being $\mathcal{K}_0$, specified by the user\footnote{We could also start from a corpus $\mathcal{C}_0$, but it is more the spirit of the methodology presented in the next sub-section. We remain here for this preliminary exploration by assuming the necessarily arbitrary biased aspect of this specification. The choice of the initial corpus must thus be done with a good knowledge of existing domains, and necessarily done after the literature review of~\ref{sec:modelingsa}.}. An iteration proceeds the following way:
}{
Soit $A$ un alphabet (un ensemble arbitraire de symboles), $A^{\ast}$ les mots correspondants (chaînes de longueur finie sur l'alphabet). Les textes de longueur finie sur celui-ci sont donc $T = \cup_{k\in \mathbb{N}} {A^{\ast}}^k$. Ce qu'on nomme une référence est pour l'algorithme un enregistrement avec des champs textuels représentant le titre, le résumé et les mots-clés. L'ensemble de références à l'itération $n$ est ainsi noté $\mathcal{C}_n \subset T^3$ : il s'agit d'un sous-ensemble de triplets de textes. Nous supposons l'existence d'un ensemble de mots-clés $\mathcal{K}_n$, les mots-clés initiaux étant $\mathcal{K}_0$, spécifiés par l'utilisateur\footnote{On pourrait également partir d'un corpus $\mathcal{C}_0$, mais il s'agit plutôt de l'esprit de la méthodologie présentée dans la sous-section suivante. Nous nous en tiendrons ici pour cette exploration préliminaire en assumant le caractère arbitraire forcément biaisé de cette spécification. Le choix du corpus initial doit donc être fait en bonne connaissance des domaines existants, et fait nécessairement suite à la revue de littérature de~\ref{sec:modelingsa}.}. Une itération procède de la manière suivante :
}


\bpar{
\begin{enumerate}
\item A raw intermediate corpus $\mathcal{R}_n$ is obtained through a catalog request\footnote{The catalog is a function providing references as an answer to a request composed by regular expressions of keywords. In practice, we use the online bibliographic catalog Mendeley. Catalog dependency should surely introduce a bias which can not be controlled, since a sensitivity analysis or a cross-search through diverse catalogs being out of the scope of this exploratory analysis.} to which we provide the previous keywords $\mathcal{K}_{n-1}$.
\item Overall corpus is actualized by $\mathcal{C}_n = \mathcal{C}_{n-1} \cup \mathcal{R}_n$.
\item The new keywords $\mathcal{K}_n$ are extracted from corpus through Natural Language Processing (NLP) treatment, given a parameter $N_k$ fixing the number of keywords extracted at this stage.
\end{enumerate}
}{
\begin{enumerate}
\item Un corpus intermédiaire brut $\mathcal{R}_n$ est obtenu par une requête à un catalogue\footnote{Le catalogue est une fonction fournissant des références en réponse à une requête composée d'expression régulières de mots-clés. En pratique, nous utilisons le catalogue bibliographique en ligne Mendeley. La dépendance au catalogue devant sûrement introduire un biais que nous ne pouvons contrôler, une analyse de sensibilité ou le croisement de divers catalogues étant hors de propos pour cette analyse exploratoire.}
 auquel on fourni les mots-clés précédents $\mathcal{K}_{n-1}$.
\item Le corpus total est actualisé par $\mathcal{C}_n = \mathcal{C}_{n-1} \cup \mathcal{R}_n$.
\item Les nouveaux mot-clés $\mathcal{K}_n$ sont extraits du corpus par Traitement du Language Naturel (NLP), étant donné un paramètre fixé $N_k$ donnant le nombre de mot-clés extraits à cette étape.
\end{enumerate}
}


\bpar{
The algorithm stops when corpus size becomes stable (experiments on tested requests shows that for these, the corpus does not contain new references after a certain number of iterations) or when a maximal number of iterations defined by the user is reached. Fig.~\ref{fig:quantepistemo:algo} synthesizes the global workflow.
}{
L'algorithme s'arrête quand la taille du corpus ne varie plus (l'expérience sur les requêtes testées montre pour celles-ci que le corpus ne contient plus de nouvelles références après un certain nombre d'itérations) ou quand un nombre maximal d'itérations défini par l'utilisateur est atteint. La figure~\ref{fig:quantepistemo:algo} synthétise le processus général.
}




\paragraph{Implementation}{Implémentation}


\bpar{
Because of the heterogeneity of operations required by the algorithm (references organisation, catalog requests, text processing), it was found a reasonable choice to implement it in Java. Source code is available on the open repository of the project\footnote{at the adress \texttt{https://github.com/JusteRaimbault/CityNetwork/tree/master/Models/QuantEpsitemo/AlgoSR}}. Catalog requests, consisting in retrieving a set of references from a set of keywords, are done using the Mendeley software API \cite{mendeley} as it allows an open access to a large database. Keyword extraction is done by Natural Language Processing (NLP) techniques, following the workflow given in~\cite{chavalarias2013phylomemetic}, through a Python script that uses \cite{bird2006nltk}.
}{
De par l'hétérogénéité des opérations requises par l'algorithme (organisation des références, requêtes au catalogue, analyse textuelle), le language Java s'est présenté comme une alternative raisonnable. Le code source est disponible sur le dépôt ouvert du projet\footnote{à l'adresse \url{https://github.com/JusteRaimbault/CityNetwork/tree/master/Models/QuantEpistemo/AlgoSR}}. Les requêtes au catalogue, qui consistent à récupérer un ensemble de références à partir d'un ensemble de mots-clés, sont faites via l'API du logiciel Mendeley~\cite{mendeley} qui permet un accès ouvert à une base de données conséquente. L'extraction des mots-clés est effectuée par techniques d'Analyse Textuelle (NLP) selon le processus donné dans~\cite{chavalarias2013phylomemetic}, via un script Python qui utilise~\cite{bird2006nltk}.
}


\paragraph{Convergence and sensitivity analysis}{Convergence et analyse de sensibilité}



\bpar{
A formal proof of algorithm convergence is not possible as it will depend on the empirical unknown structure of request results and keywords extraction. We need thus to study empirically its behavior. Good convergence properties but various sensitivities to $N_k$ were found as presented in Fig.~\ref{fig:app:quantepistemo:sensitivity-algosr}. We also study the internal lexical consistence of final corpuses as a function of keywords number. As expected, small number yields more consistent corpuses, but the variability when increasing stays reasonable.
}{
Une preuve formelle de convergence de l'algorithme n'est guère envisageable puisque qu'elle dépendra de la structure empirique inconnue des résultats de requête et d'extraction de mots-clés. Il est donc nécessaire d'étudier le comportement de l'algorithme de manière empirique. Comme présenté en Fig.~\ref{fig:app:quantepistemo:sensitivity-algosr}, l'algorithme a de bonnes propriétés de convergence mais diverses sensibilités à $N_k$. Nous étudions également la cohérence lexicale interne des corpus finaux et fonction du nombre de mots-clés. Comme attendu, des valeurs faibles produisent des corpus plus cohérents, mais la variabilité lorsque qu'elles augmentent reste raisonnable.
}

\bpar{
We take the weakest assumption for the parameter $N_k=100$. Indeed, the larger $N_k$ is, the less constrained the explored domain will be, what increases the chances of overlap between two corpuses originating from different initial requests. In this case, a small final distance between corpuses will be more significant for larger $N_k$ values.
}{
Nous prenons l'hypothèse la plus faible pour le paramètre $N_k=100$. En effet, plus $N_k$ est grand, moins le domaine exploré sera restreint, ce qui augmente les chances de recouvrement de deux corpus provenant de requêtes initiales différentes. Dans ce cas, une faible distance finale entre corpus sera plus significative pour des valeurs de $N_k$ grandes.
}


%%%%%%%%%%%%%%%%%%%%%%%%%%%%
\begin{figure}
%\includegraphics[width=\linewidth]{Figures/QuantEpistemo/explo}
%\includegraphics[width=\linewidth]{Figures/QuantEpistemo/lexicalConsistence_MeanSd}
\includegraphics[width=\linewidth,height=0.85\textheight]{Figures/Final/A-quantepistemo-sensitivity-algosr.jpg}
\appcaption{\textbf{Convergence and sensitivity analysis of the algorithmic systematic review.} (\textit{Top}) Plots of number of references as a function of iteration, for various queries linked to our theme (see further), for various values of $N_k$ (from 2 to 30). We obtain a rapid convergence for most cases, around 10 iterations needed. Final number of references appears to be very sensitive to keyword number depending on queries, what confirms a strong variability of the encountered landscape depending on terms. (\textit{Bottom}) Mean lexical consistence and standard error bars for various queries, as a function of $n_k$. Lexical consistence is defined though co-occurrences of keywords by, with $N$ final number of keywords, $f$ final step, and $c(i)$ co-occurrences in references, $k = \frac{2}{N(N-1)}\cdot \sum_{i,j \in \mathcal{K}_f}{\left| c(i) - c(j) \right|}$. The stability confirms the consistence of final corpuses.\label{fig:app:quantepistemo:sensitivity-algosr}}{\textbf{Convergence et analyse de sensibilité de l'algorithme de revue systématique.} (\textit{Haut}) Graphes des nombres de références en fonction de l'itération, pour différentes requêtes liées à notre thème, et pour différentes valeurs de $N_k$ (de 2 à 30, couleur). On obtient une convergence rapide dans la majorité des cas, autour de 10 itérations étant nécessaires. Le nombre final de références semble très sensible au nombre de mots-clés selon les requêtes, ce qui confirme une forte variabilité du paysage rencontré selon les termes. (\textit{Bas}) Consistence lexicale moyenne et déviation standard sur différentes requêtes, en fonction de $N_k$. La consistence lexicale est définie par les co-occurrences des mots-clés, comme $k = \frac{2}{N_k(N_k-1)}\cdot \sum_{i,j \in \mathcal{K}_f}{\left| c(i) - c(j) \right|}$, avec $f$ temps final, $c(i)$ co-occurrence des mots dans les références. La stabilisation confirme la consistence des corpus finaux.\label{fig:app:quantepistemo:sensitivity-algosr}}
\end{figure}
%%%%%%%%%%%%%%%%%%%%%%%%%%%%








%----------------------------------------------------------------------------------------

\subsection{Indirect bibliometrics}{Bibliométrie indirecte}


%%

\paragraph{Initial corpus}{Corpus initial}

\bpar{
The Table~\ref{tab:app:quantepistemo:corpus} gives the composition of the initial corpus for the construction of the citation network.
}{
Le tableau~\ref{tab:app:quantepistemo:corpus} donne la composition du corpus initial pour la construction du réseau de citation.
}


%%%%%%%%%%%%%
\begin{table}
\apptabcaption{\textbf{Composition of the initial corpus for the construction of the citation network.}\label{tab:app:quantepistemo:corpus}}{\textbf{Composition du corpus initial pour la construction du réseau de citation.}\label{tab:app:quantepistemo:corpus}}
\bpar{
\begin{tabular}{|l|p{6cm}|l|}
	\hline
	Discipline & Title & Reference \\\hline
	Political science & \textit{Les effets structurants du transport: mythe politique, mystification scientifique} & \cite{offner1993effets} \\\hline 
	Interdisciplinary & \textit{Réseaux et territoires-significations croisées} & \cite{offner1996reseaux} \\\hline
	Geography & \textit{Villes et réseaux de transport: des interactions dans la longue durée (France, Europe, Etats-Unis)} & \cite{bretagnolle:tel-00459720} \\\hline
	Transportation & Land-use transport interaction: state of the art & \cite{wegener2004land} \\\hline
	Economics & The co-evolution of land use and road networks & \cite{levinson2007co} \\\hline
	Economics & Modeling the growth of transportation networks: a comprehensive review & \cite{xie2009modeling} \\\hline
	Physics & Co-evolution of density and topology in a simple model of city formation & \cite{barthelemy2009co} \\\hline
	\end{tabular}
}{
	\begin{tabular}{|l|p{6cm}|l|}
	\hline
	Domaine & Titre & Référence \\\hline
	Sciences politiques & Les effets structurants du transport: mythe politique, mystification scientifique & \cite{offner1993effets} \\\hline 
	Interdisciplinaire & Réseaux et territoires-significations croisées & \cite{offner1996reseaux} \\\hline
	Géographie & Villes et réseaux de transport: des interactions dans la longue durée (France, Europe, Etats-Unis) & \cite{bretagnolle:tel-00459720} \\\hline
	Transports & Land-use transport interaction: state of the art & \cite{wegener2004land} \\\hline
	Économie & The co-evolution of land use and road networks & \cite{levinson2007co} \\\hline
	Économie & Modeling the growth of transportation networks: a comprehensive review & \cite{xie2009modeling} \\\hline
	Physique & Co-evolution of density and topology in a simple model of city formation & \cite{barthelemy2009co} \\\hline
	\end{tabular}
}
\end{table}
%%%%%%%%%%%%%




\paragraph{Sensitivity analysis}{Analyse de sensibilité}

\bpar{
The sensitivity analysis allowing to fix the optimal parameters for the semantic network is shown in Fig.~\ref{fig:app:quantepistemo:sensitivity}.
}{
L'analyse de sensibilité permettant de fixer les paramètres optimaux pour le réseau sémantique est montrée en Fig.~\ref{fig:app:quantepistemo:sensitivity}.
}

% degré non pondéré ? -> justifié dans le Cybergeo paper


%%%%%%%%%%%%%%%%%%
\begin{figure}
%\includegraphics[width=0.49\linewidth]{Figures/Quantepistemo/pareto-com-vertices}
%\includegraphics[width=0.49\linewidth]{Figures/Quantepistemo/pareto-modularity-vertices}\\
%\includegraphics[width=\linewidth]{Figures/Quantepistemo/sensitivity_freqmin0_normalized}
\includegraphics[width=\linewidth]{Figures/Final/A-quantepistemo-sensitivity.jpg}
\appcaption{\textbf{Sensitivity analysis of modular properties of the semantic network as a function of filtering parameters.} \textit{(Top Left)} Pareto front of the number of communities and the number of vertices (two objectives to be maximized), the color giving the value of $\theta_w$; \textit{(Top Right)} Pareto front of the modularity as a function of number of vertices, for varying $\theta_w$; (\textit{Bottom}) Values of possible objectives (modularity, number of communities, number of connnected components, number of vertices, density, size balance between communities), each objective being normalized in $\left[0;1\right]$, as a function of parameters $\theta_w$ and $k_{max}$.\label{fig:app:quantepistemo:sensitivity}}{\textbf{Analyse de sensibilité des propriétés modulaires du réseau sémantique en fonction des paramètres de filtrage.} \textit{(Haut Gauche)} Front de Pareto du nombre de communauté et du nombre de sommets (deux objectifs à maximiser), la couleur donnant la valeur de $\theta_w$ ; \textit{(Haut Droite)} Front de Pareto de la modularité en fonction du nombre de sommets, pour $\theta_w$ variant ; \textit{(Bas)} Valeurs des objectifs possibles (modularité, nombre de communautés, nombre de composantes connexes, nombre de sommets, densité, équilibre de taille entre communautés), chaque objectif étant normalisé dans $\left[0;1\right]$, en fonction des paramètres $\theta_w$ et $k_{max}$.\label{fig:app:quantepistemo:sensitivity}}
\end{figure}
%%%%%%%%%%%%%%%%%%



\paragraph{Semantic network}{Réseau sémantique}

% Visualisation du réseau sémantique.

\bpar{
A visualization of the semantic network is given in Fig~\ref{fig:app:quantepistemo:semanticnw}.
}{
Une visualisation du réseau sémantique est donnée en Fig.~\ref{fig:app:quantepistemo:semanticnw}.
}


%%%%%%%%%%%%%%%%%%
\begin{figure}
%\includegraphics[width=\linewidth]{Figures/Quantepistemo/semantic}
\includegraphics[width=\linewidth]{Figures/Final/A-quantepistemo-semanticnw.jpg}
\appcaption{\textbf{Semantic network of domains.} The color of links gives the community and the size of keywords is fixed by their degree.\label{fig:app:quantepistemo:semanticnw}}{\textbf{Réseau sémantique des domaines.} La couleur des liens donne la communauté et la taille des mots-clés est fixée par leur degré.\label{fig:app:quantepistemo:semanticnw}}
\end{figure}
%%%%%%%%%%%%%%%%%%





\stars





%----------------------------------------------------------------------------------------

\newpage

%%%%%%%%%%%%%%%%%%%%%%%
\section{Modelography}{Modélographie}

\label{app:sec:modelography}


%----------------------------------------------------------------------------------------



\subsection{Systematic review methodology}{Méthodologie de la revue systématique}



\bpar{
For the choice of initial keywords for the indirect construction (through semantic request), a possible alternative is to extract the relevant keywords for each sub-community of the citation network, and then select the most relevant for each domain. We make the choice of extracting them on the complete corpus, and then to collect them by sub-community thereafter. For a small corpus, the second option is more suitable, since the notion of relevance is less importance than for very large corpuses, in which some relevant words may be drowned and less relevant to emerge in a spurious way. In other words, the keyword selection method appears to be more robust on smaller corpuses, as suggest the comparison of this application with the one done on the Cybergeo journal and the one done on the patent corpus (see~\ref{app:sec:patentsmining}).
}{
Pour le choix des mots-clés initiaux pour la construction indirecte (via requête sémantique), une alternative possible est d'extraire les mots-clés pertinents par sous-communautés du réseau de citations, puis sélectionner les plus pertinents ensuite pour chaque domaine. Nous faisons le choix de les extraire sur le corpus complet, puis de les récupérer par sous-communautés ensuite. Pour un petit corpus, la deuxième option est plus souhaitable, puisque la notion de pertinence moins importante que pour des très grands corpus, ou certains mots pertinents pourront être noyés et des moins pertinents ressortir de manière fortuite. En d'autre termes, la méthode de selection des mots-clés parait plus robuste sur des petits corpus, comme le suggère la comparaison de cette application avec celle faite sur le journal Cybergeo et celle faite sur le corpus de brevets (voir~\ref{app:sec:patentsmining}).
}

%(mentionner Patents et Cybergeo, ressemblances et différences).




% random stuff removed manual screening 
%computer sci, neurosci, geology, chem/bio terrorism WTF ? , theoretical physics, eco des orgs, socio (aids ethiopia ?), target tracking, rice storage culinary qual, bulimic, smart cities, bullshit territorialité Die, chinese migrants !, chinese tourist HongKong, papier CN desakota (cité migrdyn), health, hydrology, paysage hautes corbières, musical implication network, finance capitalism, produit territoire pomme des alpes, mobilité sociale, vibration pont hsr, ontological knowledge industri., signal processing, anger/agression, flocks/schools, friendship and mobility, consumer brand,  aghion ? , colorectal cancer texas, lidar urban, mediteranean cities, sun and sand tourism, gender car use, gold rush movies, urban informatics, "best outside US", badger, robots beacon, milieux innovateurs, politique univ villes la rochelle etc, cosmology, economie de la qualité,  handbook adult resilience, stock price, archeo chefferie, speculative urb, Mendoza arg, subway platfrom ventilation, magaliths, crowding HK light rail !, terrritoire familiaux naples classes sup, performing arts series, sncf tunnel rig, china snow disaster; fabriq rue paris 19e, geopol inuit, espace public beyrouth, greenway., desire named streetcar, jeddah, french competitino rail myth, Equip the warrior instead of manning the equipment, aparheid namibia, baselIII finance, tokyo, espace transfrtontalier, transport proteins !, hedonic rail road noise, croissance syst urbain loi metropol, maritime nw indonesia, eco sociale au quebec, first world urban activism, hybrid.. , gentrif nvel urba, [before] Estimation of local spatial scale : otpicians !, stations de montange, street gang sptail pattern LA, SFI adress !, bogota BRT, transport noise, brussel urban geology, drosohiplia embryo


\subsubsection{First corpus review}{Première revue du corpus}

\bpar{
The methods used do not allow to be cleared from a ``noise'', i.e. of articles that are not relevant to the subject, even with a very low tolerance threshold. We obtained for example articles as absurd as gender and car use, colorectal cancer in Texas, vibration mechanics at the passage of a high speed train, protein transport within the cell, public space in Beyrouth, spatial patterns of \emph{street gangs} in Los Angeles, urban geology in Brussels. This confirms that the manual filtering stage is essential.
}{
Les méthodes utilisées ne permettent pas de s'affranchir d'un ``bruit'', c'est-à-dire d'article ne relevant a priori pas même de loin à la thématique. Nous avons obtenu par exemple des articles aussi divers qu'incongrus sur le genre et l'usage de la voiture, le cancer colorectal au Texas, la mécanique des vibrations au passage d'un train à grande vitesse, le transport des protéines dans la cellule, l'espace public à Beyrouth, les motifs spatiaux des \emph{street gangs} à Los Angeles, la géologie urbaine à Bruxelles. Cela confirme que l'étape de filtrage manuel est essentielle.
}


\bpar{
This noise can for example be due to:
\begin{itemize}
	\item Effective citations for diverse reasons, but having only a low relevance in the citing article.
	\item Noise intrinsic to the keyword search.
	\item Catalog classification errors.
\end{itemize}
}{
Ce bruit peut être du par exemple à :
\begin{itemize}
	\item Des citations effectives pour diverses raisons, mais n'ayant que peu de pertinence dans l'article citant.
	\item Du bruit intrinsèque à la recherche par mots-clés.
	\item Des erreurs de classification du catalogue.
\end{itemize}
}



\subsubsection{Remarks on manual screening}{Remarques sur la classification manuelle}

\bpar{
During the manual classification achieved when screening abstracts, the following points appear: 
}{
Lors de la classification manuelle opérée lors de l'inspection des résumés, les points suivants ressortent :
}



\bpar{
\begin{itemize}
	\item The ``a priori'' disciplines are judged based on the journal in which the article was published. In particular, we operate the following particular choices (for other journals such as physics journals there is no ambiguity): Journal of Transport Geography, Environment and Planning B: Geography; Journal of Transport and Land-use, Transportation Research: Transportation.
	\item Geography in our sense includes urbanism and urban studies is these are not too close from planning (urban sustainability for example).
\end{itemize}
}{
\begin{itemize}
	\item Les disciplines ``a priori'' sont jugées par le journal dans lequel l'article a été publié. En l'occurence, nous opérons les choix particuliers suivants (pour d'autres journaux comme des journaux de physique il n'y a pas d'ambiguïté) : Journal of Transport Geography, Environment and Planning B : geography ; Journal of Transport and Land-Use, Transportation Research : Transportation.
	\item La géographie en notre sens inclut l'urbanisme et les études urbaines si celles-ci ne sont pas trop proches de la planification (urbain durable par exemple).
\end{itemize}
}





\subsection{Meta-analysis}{Meta-analyse}

\bpar{
We give here the full numerical results of statistical analysis linking model characteristics and explicative variables.
}{
Nous donnons ici les résultats numériques complets des analyses statistiques reliant caractéristiques de modèles et variables explicatives.
}


\subsubsection{Modalities of variables}{Modalités des variables}

\bpar{
We recall here the variables used in the meta-analysis and their modalities. These are:
}{
Rappelons ici les variables utilisées dans la méta-analyse et leur modalités. Celles-ci sont :
}


\bpar{
\begin{itemize}
	\item Type of model (\texttt{TYPE}): strong, territory, network.
	\item Publication year (\texttt{YEAR}), integer number.
	\item Citation community (\texttt{CITCOM}), defined within the citation network: Accessibility, Geography, Infra Planning, LUTI, Networks, TOD.
	\item A priori discipline (\texttt{DISCIPLINE}): biology, computer science, economics, engineering, environment, geography, physics, planning, transportation.
	\item Semantic community (\texttt{SEMCOM}): brt, complex networks, hedonic, hsr, infra planning, networks, tod.
	\item Methodology used: ca (Cellular Automaton), eq (analytical equations), map (cartography), mas (Multi-agent simulation), ro (operations research), sem (Structural Equation Modeling), sim (simulation), stat (statistics).
	\item Interdisciplinarity index (\texttt{INTERDISC}): real number in $[0,1]$.
	\item Temporal scale (\texttt{TEMPSCALE}): given in years, is set to 0 for static analyses.
	\item Spatial scale (\texttt{SPATSCALE}): continent (10000), country (1000), region (100), metro (10). These modalities are numerically transformed in km by the values given in parenthesis (stylized scales).
\end{itemize}
}{
\begin{itemize}
	\item Type de modèle (\texttt{TYPE}) : strong, territory, network.
	\item Année de publication (\texttt{YEAR}), nombre entier.
	\item Communauté de citation (\texttt{CITCOM}), définies par le réseau de citations : Accessibility, Geography, Infra Planning, LUTI, Networks, TOD.
	\item Discipline a priori (\texttt{DISCIPLINE}) : biology, computer science, economics, engineering, environment, geography, physics, planning, transportation.
	\item Communauté sémantique (\texttt{SEMCOM}) : brt, complex networks, hedonic, hsr, infra planning, networks, tod.
	\item Méthodologie utilisée : ca (\emph{Cellular Automaton}), eq (équations analytiques), map (cartographie), mas (\emph{Multi-agent simulation}), ro (recherche opérationnelle), sem (\emph{Structural Equation Modeling}), sim (simulation), stat (statistiques).
	\item Indice d'interdisciplinarité (\texttt{INTERDISC}) : réel dans $[0,1]$.
	\item Echelle temporelle (\texttt{TEMPSCALE}) : donnée en année, vaut 0 pour les analyses statiques.
	\item Echelle spatiale (\texttt{SPATSCALE}) : continent (10000), country (1000), region (100), metro (10). Ces modalités sont transformées numériquement en km par les valeurs données entre parenthèses (échelles stylisées).
\end{itemize}
}

%La variable d'équilibre (modèles supposant un équilibre ou non) n'a pas pu être construite.



\subsubsection{Model selection}{Sélection des modèles}


\bpar{
Regarding model selection, it is not achieved following a unique criteria, because of the low number of observations for some models, but by the optimization in the Pareto sense of contradictory objectives of adjustment (adjusted $R^2$, to be maximized) and of the overfitting (corrected Akaike criterion AICc, to be minimized), while controlling the number of observation points. The Fig.~\ref{fig:app:quantepistemo:regressions} gives for each variable to be explained the localization of the set of potential models within the objective space, and also the corresponding number of observations. For interdisciplinarity, two point clouds correspond to different compromises, and we select the two optimal models (one for each cloud). For the spatial scale, we postulate a positive $R^2$, and a single optimal model then emerges. For the temporal scale, we have as for interdisciplinarity two compromise models. Finally for the year, the AICc gain between the two potential optima is negligible in comparison to the $R^2$ loss, and we thus select the optimal model such that $R^2>0.25$ and AICc$<600$. The results of models are given in the following.
}{
Concernant la sélection des modèles, celle-ci n'est pas opérée en critère unique, de par le faible nombre d'observations pour certains modèles, mais par l'optimisation au sens de Pareto des objectifs contradictoires de l'ajustement ($R^2$ ajusté, à maximiser) et du sur-ajustement (critère d'Akaike corrigé AICc, à minimiser), tout en contrôlant le nombre de points d'observation. La Fig.~\ref{fig:app:quantepistemo:regressions} donne pour chaque variable à expliquer la localisation de l'ensemble des modèles potentiels dans l'espace des objectifs, ainsi que le nombre d'observations correspondantes. Pour l'interdisciplinarité, deux nuages de points correspondent à des compromis différents, et nous sélectionnons les deux modèles optimaux (un pour chaque nuage). Pour l'échelle d'espace, nous postulons un $R^2$ positif, et un seul modèle optimal émerge alors. Pour l'échelle de temps, on a comme pour l'interdisciplinarité deux modèles compromis. Enfin, pour l'année, le gain en AICc entre les deux optimaux potentiels est négligeable en comparaison à la perte en $R^2$, et nous sélectionnons donc le modèle optimal tel que $R^2>0.25$ et AICc$<600$. Les résultats des modèles sont donnés par la suite.
}


%%%%%%%%%%%%%%
\begin{figure}
%\includegraphics[width=0.48\linewidth]{Figures/QuantEpistemo/lm_adjr2-aicc_INTERDISC.pdf}
%\includegraphics[width=0.48\linewidth]{Figures/QuantEpistemo/lm_adjr2-aicc_SPATSCALE.pdf}
%\includegraphics[width=0.48\linewidth]{Figures/QuantEpistemo/lm_adjr2-aicc_TEMPSCALE.pdf}
%\includegraphics[width=0.48\linewidth]{Figures/QuantEpistemo/lm_adjr2-aicc_YEAR.pdf}
\includegraphics[width=\linewidth]{Figures/Final/A-quantepistemo-regressions.jpg}
\appcaption{\textbf{Multi-objective selection of linear models.} For each variable to be explained, we represent the position of all linear models in the objective space (corrected Akaike criterion AICc and adjusted $R^2$). The color of points gives the number of observations.\label{fig:app:quantepistemo:regressions}}{\textbf{Sélection multi-objectif des modèles linéaires.} Pour chaque variable à expliquer, nous représentons la position de l'ensemble des modèles linéaires dans l'espace des objectifs (critère d'Akaike corrigé AICc et $R^2$ ajusté). La couleur des points donne le nombre d'observations.\label{fig:app:quantepistemo:regressions}}
\end{figure}
%%%%%%%%%%%%%%



\subsubsection{Model fitting}{Ajustement des modèles}


\paragraph{Interdisciplinarity}{Interdisciplinarité}


\bpar{
Interdisciplinarity is adjusted according to the linear models presented in Table~\ref{tab:app:modelography:interdisc}.
}{
L'interdisciplinarité est ajustée selon les modèles linéaires présentés en Table~\ref{tab:app:modelography:interdisc}.
}



%%%%%%%%%%%%%%
\begin{table}%[!htbp] \centering 
  \apptabcaption{\textbf{Linear models for interdisciplinarity}\label{tab:app:modelography:interdisc}}{\textbf{Modèles linéaires pour l'interdisciplinarité.}\label{tab:app:modelography:interdisc}}
\bpar{
\begin{tabular}{@{\extracolsep{5pt}}lcc} 
\footnotesize
\\[-1.8ex]\hline 
\hline \\[-1.8ex] 
 %& \multicolumn{2}{c}{\textit{Dependent variable:}} \\ 
%\cline{2-3} 
\\[-1.8ex] & \multicolumn{2}{c}{INTERDISC} \\ 
\\[-1.8ex] & (1) & (2)\\ 
\hline \\[-1.8ex] 
 YEAR & $-$0.004 ($-$0.008, $-$0.00002), p = 0.055$^{*}$ & $-$0.002 ($-$0.005, 0.0001), p = 0.061$^{*}$ \\ 
  TEMPSCALE & $-$0.0003 ($-$0.001, 0.001), p = 0.615 &  \\ 
  DISCIPLINEengineering & 0.144 ($-$0.082, 0.371), p = 0.218 &  \\ 
  DISCIPLINEenvironment & 0.092 ($-$0.132, 0.316), p = 0.425 &  \\ 
  DISCIPLINEgeography & 0.036 ($-$0.043, 0.114), p = 0.378 &  \\ 
  DISCIPLINEphysics & $-$0.103 ($-$0.287, 0.080), p = 0.275 &  \\ 
  DISCIPLINEplanning & $-$0.047 ($-$0.135, 0.041), p = 0.30 &  \\ 
  DISCIPLINEtransportation & 0.062 ($-$0.025, 0.149), p = 0.169 &  \\ 
  TYPEstrong &  & $-$0.026 ($-$0.134, 0.081), p = 0.633 \\ 
  TYPEterritory &  & 0.044 ($-$0.026, 0.114), p = 0.222 \\ 
  SEMCOMcomplex networks &  & $-$0.217 ($-$0.522, 0.087), p = 0.166 \\ 
  SEMCOMhedonic & $-$0.179 ($-$0.407, 0.049), p = 0.130 & $-$0.184 ($-$0.400, 0.032), p = 0.100$^{*}$ \\ 
  SEMCOMhsr & $-$0.100 ($-$0.361, 0.162), p = 0.459 & $-$0.122 ($-$0.357, 0.112), p = 0.309 \\
  SEMCOMinfra planning & $-$0.032 ($-$0.273, 0.209), p = 0.797 & $-$0.096 ($-$0.321, 0.128), p = 0.404  \\ 
  SEMCOMnetworks & $-$0.038 ($-$0.272, 0.195), p = 0.750 & $-$0.107 ($-$0.324, 0.109), p = 0.335 \\ 
  SEMCOMtod & $-$0.105 ($-$0.332, 0.121), p = 0.366 & $-$0.152 ($-$0.364, 0.060), p = 0.165 \\ 
  Constant & 8.962 (0.776, 17.147), p = 0.037$^{**}$ & 5.531 (0.575, 10.487), p = 0.032$^{**}$ \\
 \hline \\[-1.8ex] 
Observations & 64 & 98 \\ 
R$^{2}$ & 0.314 & 0.155 \\ 
Adjusted R$^{2}$ & 0.136 & 0.068 \\ 
Residual Std. Error & 0.109 (df = 50) & 0.107 (df = 88) \\ 
F Statistic & 1.761$^{*}$ (df = 13; 50) & 1.789$^{*}$ (df = 9; 88) \\ 
\hline 
\hline \\[-1.8ex] 
\textit{Note:}  & \multicolumn{2}{r}{$^{*}$p$<$0.1; $^{**}$p$<$0.05; $^{***}$p$<$0.01} \\ 
\end{tabular}
}{
\begin{tabular}{@{\extracolsep{5pt}}lcc} 
\footnotesize
\\[-1.8ex]\hline 
\hline \\[-1.8ex] 
 %& \multicolumn{2}{c}{\textit{Dependent variable:}} \\ 
%\cline{2-3} 
\\[-1.8ex] & \multicolumn{2}{c}{INTERDISC} \\ 
\\[-1.8ex] & (1) & (2)\\ 
\hline \\[-1.8ex] 
 YEAR & $-$0.004 ($-$0.008, $-$0.00002), p = 0.055$^{*}$ & $-$0.002 ($-$0.005, 0.0001), p = 0.061$^{*}$ \\ 
  TEMPSCALE & $-$0.0003 ($-$0.001, 0.001), p = 0.615 &  \\ 
  DISCIPLINEengineering & 0.144 ($-$0.082, 0.371), p = 0.218 &  \\ 
  DISCIPLINEenvironment & 0.092 ($-$0.132, 0.316), p = 0.425 &  \\ 
  DISCIPLINEgeography & 0.036 ($-$0.043, 0.114), p = 0.378 &  \\ 
  DISCIPLINEphysics & $-$0.103 ($-$0.287, 0.080), p = 0.275 &  \\ 
  DISCIPLINEplanning & $-$0.047 ($-$0.135, 0.041), p = 0.30 &  \\ 
  DISCIPLINEtransportation & 0.062 ($-$0.025, 0.149), p = 0.169 &  \\ 
  TYPEstrong &  & $-$0.026 ($-$0.134, 0.081), p = 0.633 \\ 
  TYPEterritory &  & 0.044 ($-$0.026, 0.114), p = 0.222 \\ 
  SEMCOMcomplex networks &  & $-$0.217 ($-$0.522, 0.087), p = 0.166 \\ 
  SEMCOMhedonic & $-$0.179 ($-$0.407, 0.049), p = 0.130 & $-$0.184 ($-$0.400, 0.032), p = 0.100$^{*}$ \\ 
  SEMCOMhsr & $-$0.100 ($-$0.361, 0.162), p = 0.459 & $-$0.122 ($-$0.357, 0.112), p = 0.309 \\
  SEMCOMinfra planning & $-$0.032 ($-$0.273, 0.209), p = 0.797 & $-$0.096 ($-$0.321, 0.128), p = 0.404  \\ 
  SEMCOMnetworks & $-$0.038 ($-$0.272, 0.195), p = 0.750 & $-$0.107 ($-$0.324, 0.109), p = 0.335 \\ 
  SEMCOMtod & $-$0.105 ($-$0.332, 0.121), p = 0.366 & $-$0.152 ($-$0.364, 0.060), p = 0.165 \\ 
  Constant & 8.962 (0.776, 17.147), p = 0.037$^{**}$ & 5.531 (0.575, 10.487), p = 0.032$^{**}$ \\
 \hline \\[-1.8ex] 
Observations & 64 & 98 \\ 
R$^{2}$ & 0.314 & 0.155 \\ 
R$^{2}$ ajusté & 0.136 & 0.068 \\ 
Erreur Std. Résiduelle & 0.109 (df = 50) & 0.107 (df = 88) \\ 
Statistique F & 1.761$^{*}$ (df = 13; 50) & 1.789$^{*}$ (df = 9; 88) \\ 
\hline 
\hline \\[-1.8ex] 
\textit{Note:}  & \multicolumn{2}{r}{$^{*}$p$<$0.1; $^{**}$p$<$0.05; $^{***}$p$<$0.01} \\ 
\end{tabular}
}
\end{table} 
%%%%%%%%%%%%%%



\paragraph{Spatial scale}{Echelle d'espace}


\bpar{
The spatial scale is adjusted following the linear model which adjustment is given in Table~\ref{tab:app:modelography:spatscale}.
}{
L'échelle spatiale est ajustée selon le modèle linéaire dont l'ajustement est donné en Table~\ref{tab:app:modelography:spatscale}.
}

%\[
%\begin{split}
%	SPATSCALE \sim & YEAR+CITCOM+TYPE+TEMPSCALE+ \\
%	& FMETHOD+DISCIPLINE+INTERDISC+SEMCOM
%\end{split}
%\]


%%%%%%%%%%%%%
\begin{table}%[!htbp] \centering 
  \apptabcaption{\textbf{Linear model for the spatial scale.}\label{tab:app:modelography:spatscale}}{\textbf{Modèle linéaire pour l'échelle spatiale.}\label{tab:app:modelography:spatscale}}
\bpar{
\begin{tabular}{@{\extracolsep{5pt}}lc} 
\\[-1.8ex]\hline 
\hline \\[-1.8ex] 
 %& \multicolumn{1}{c}{\textit{Dependent variable:}} \\ 
%\cline{2-2} 
\\[-1.8ex] & SPATSCALE \\ 
\hline \\[-1.8ex] 
 TEMPSCALE & $-$5.179 ($-$16.259, 5.901) \\ 
  & p = 0.363 \\ 
  DISCIPLINEengineering & $-$154.461 ($-$3,003.326, 2,694.405) \\ 
  & p = 0.916 \\ 
  DISCIPLINEenvironment & $-$5.878 ($-$3,977.974, 3,966.219) \\ 
  & p = 0.998 \\ 
  DISCIPLINEgeography & 1,445.457 (389.349, 2,501.565) \\ 
  & p = 0.009$^{***}$ \\ 
  DISCIPLINEphysics & 292.559 ($-$2,717.659, 3,302.777) \\ 
  & p = 0.850 \\ 
  DISCIPLINEplanning & $-$143.554 ($-$1,361.357, 1,074.249) \\ 
  & p = 0.818 \\ 
  DISCIPLINEtransportation & 568.329 ($-$606.167, 1,742.826) \\ 
  & p = 0.346 \\ 
  Constant & 235.357 ($-$458.201, 928.914) \\ 
  & p = 0.508 \\ 
 \hline \\[-1.8ex]
Observations & 94 \\ 
R$^{2}$ & 0.100 \\ 
Adjusted R$^{2}$ & 0.027 \\ 
Residual Std. Error & 1,995.272 (df = 86) \\ 
F Statistic &  1.369 (df = 7; 86) \\ 
\hline 
\hline \\[-1.8ex] 
\textit{Note:}  & \multicolumn{1}{r}{$^{*}$p$<$0.1; $^{**}$p$<$0.05; $^{***}$p$<$0.01} \\ 
\end{tabular}
}{
\begin{tabular}{@{\extracolsep{5pt}}lc} 
\\[-1.8ex]\hline 
\hline \\[-1.8ex] 
 %& \multicolumn{1}{c}{\textit{Dependent variable:}} \\ 
%\cline{2-2} 
\\[-1.8ex] & SPATSCALE \\ 
\hline \\[-1.8ex] 
 TEMPSCALE & $-$5.179 ($-$16.259, 5.901) \\ 
  & p = 0.363 \\ 
  DISCIPLINEengineering & $-$154.461 ($-$3,003.326, 2,694.405) \\ 
  & p = 0.916 \\ 
  DISCIPLINEenvironment & $-$5.878 ($-$3,977.974, 3,966.219) \\ 
  & p = 0.998 \\ 
  DISCIPLINEgeography & 1,445.457 (389.349, 2,501.565) \\ 
  & p = 0.009$^{***}$ \\ 
  DISCIPLINEphysics & 292.559 ($-$2,717.659, 3,302.777) \\ 
  & p = 0.850 \\ 
  DISCIPLINEplanning & $-$143.554 ($-$1,361.357, 1,074.249) \\ 
  & p = 0.818 \\ 
  DISCIPLINEtransportation & 568.329 ($-$606.167, 1,742.826) \\ 
  & p = 0.346 \\ 
  Constant & 235.357 ($-$458.201, 928.914) \\ 
  & p = 0.508 \\ 
 \hline \\[-1.8ex]
Observations & 94 \\ 
R$^{2}$ & 0.100 \\ 
R$^{2}$ ajusté & 0.027 \\ 
Erreur Std. Résiduelle & 1,995.272 (df = 86) \\ 
Statistique F & 1.369 (df = 7; 86) \\ 
\hline 
\hline \\[-1.8ex]
\textit{Note:}  & \multicolumn{1}{r}{$^{*}$p$<$0.1; $^{**}$p$<$0.05; $^{***}$p$<$0.01} \\ 
\end{tabular}
}
\end{table} 
%%%%%%%%%%%%%



\paragraph{Time scale}{Echelle de temps}


\bpar{
The temporal scale is adjusted according to the linear models presented in Table~\ref{tab:app:modelography:tempscale}.
}{
L'échelle de temps est ajustée selon les modèles linéaires présentés en Table~\ref{tab:app:modelography:tempscale}.
}

%\[
%\begin{split}
%	TEMPSCALE \sim & YEAR+CITCOM+TYPE+SPATSCALE+FMETHOD\\
%	& +DISCIPLINE+INTERDISC+SEMCOM
%\end{split}
%\]

%%%%%%%%%
\begin{table}%[!htbp] \centering 
    \apptabcaption{\textbf{Linear models for the temporal scale.}\label{tab:app:modelography:tempscale}}{\textbf{Modèles linéaires pour l'échelle temporelle.}\label{tab:app:modelography:tempscale}}
\bpar{
\begin{tabular}{@{\extracolsep{5pt}}lcc} 
\\[-1.8ex]\hline 
\hline \\[-1.8ex] 
% & \multicolumn{2}{c}{\textit{Dependent variable:}} \\ 
%\cline{2-3} 
\\[-1.8ex] & \multicolumn{2}{c}{TEMPSCALE} \\ 
\\[-1.8ex] & (1) & (2)\\ 
\hline \\[-1.8ex] 
 YEAR & 0.674 ($-$0.294, 1.643) &  \\ 
  & p = 0.179 &  \\ 
  TYPEstrong &  & 100.271 (58.312, 142.230) \\ 
  &  & p = 0.00002$^{***}$ \\ 
  TYPEterritory & $-$38.933 ($-$64.249, $-$13.617) & $-$14.988 ($-$37.411, 7.435) \\ 
  & p = 0.004$^{***}$ & p = 0.194 \\ 
  DISCIPLINEengineering & $-$52.107 ($-$110.950, 6.735) & $-$9.609 ($-$55.841, 36.624) \\ 
  & p = 0.089$^{*}$ & p = 0.685 \\ 
  DISCIPLINEenvironment & 17.110 ($-$37.350, 71.569) & 17.886 ($-$45.319, 81.090) \\ 
  & p = 0.541 & p = 0.581 \\ 
  DISCIPLINEgeography & 3.640 ($-$15.364, 22.644) & 9.126 ($-$7.590, 25.843) \\ 
  & p = 0.709 & p = 0.288 \\ 
  DISCIPLINEphysics & 46.879 (0.638, 93.120) & 77.897 (28.225, 127.570) \\ 
  & p = 0.053$^{*}$ & p = 0.003$^{***}$ \\ 
  DISCIPLINEplanning & 1.304 ($-$19.336, 21.945) & 4.553 ($-$14.865, 23.971) \\ 
  & p = 0.902 & p = 0.648 \\ 
  DISCIPLINEtransportation & $-$14.718 ($-$34.978, 5.543) & 8.753 ($-$9.864, 27.371) \\ 
  & p = 0.161 & p = 0.360 \\ 
  INTERDISC & 2.357 ($-$59.200, 63.915) &  \\ 
  & p = 0.941 &  \\ 
  Constant & $-$1,305.126 ($-$3,252.499, 642.247) & 22.103 ($-$0.951, 45.156) \\ 
  & p = 0.195 & p = 0.064$^{*}$ \\ 
 \hline \\[-1.8ex] 
Observations & 64 & 94 \\ 
R$^{2}$ & 0.385 & 0.393 \\ 
Adjusted R$^{2}$ & 0.282 & 0.336 \\ 
Residual Std. Error & 26.984 (df = 54) & 31.747 (df = 85) \\ 
F Statistic & 3.755$^{***}$ (df = 9; 54) & 6.871$^{***}$ (df = 8; 85) \\ 
\hline 
\hline \\[-1.8ex] 
\textit{Note:}  & \multicolumn{2}{r}{$^{*}$p$<$0.1; $^{**}$p$<$0.05; $^{***}$p$<$0.01} \\ 
\end{tabular} 
}{
\begin{tabular}{@{\extracolsep{5pt}}lcc} 
\\[-1.8ex]\hline 
\hline \\[-1.8ex] 
% & \multicolumn{2}{c}{\textit{Dependent variable:}} \\ 
%\cline{2-3} 
\\[-1.8ex] & \multicolumn{2}{c}{TEMPSCALE} \\ 
\\[-1.8ex] & (1) & (2)\\ 
\hline \\[-1.8ex] 
 YEAR & 0.674 ($-$0.294, 1.643) &  \\ 
  & p = 0.179 &  \\ 
  TYPEstrong &  & 100.271 (58.312, 142.230) \\ 
  &  & p = 0.00002$^{***}$ \\ 
  TYPEterritory & $-$38.933 ($-$64.249, $-$13.617) & $-$14.988 ($-$37.411, 7.435) \\ 
  & p = 0.004$^{***}$ & p = 0.194 \\ 
  DISCIPLINEengineering & $-$52.107 ($-$110.950, 6.735) & $-$9.609 ($-$55.841, 36.624) \\ 
  & p = 0.089$^{*}$ & p = 0.685 \\ 
  DISCIPLINEenvironment & 17.110 ($-$37.350, 71.569) & 17.886 ($-$45.319, 81.090) \\ 
  & p = 0.541 & p = 0.581 \\ 
  DISCIPLINEgeography & 3.640 ($-$15.364, 22.644) & 9.126 ($-$7.590, 25.843) \\ 
  & p = 0.709 & p = 0.288 \\ 
  DISCIPLINEphysics & 46.879 (0.638, 93.120) & 77.897 (28.225, 127.570) \\ 
  & p = 0.053$^{*}$ & p = 0.003$^{***}$ \\ 
  DISCIPLINEplanning & 1.304 ($-$19.336, 21.945) & 4.553 ($-$14.865, 23.971) \\ 
  & p = 0.902 & p = 0.648 \\ 
  DISCIPLINEtransportation & $-$14.718 ($-$34.978, 5.543) & 8.753 ($-$9.864, 27.371) \\ 
  & p = 0.161 & p = 0.360 \\ 
  INTERDISC & 2.357 ($-$59.200, 63.915) &  \\ 
  & p = 0.941 &  \\ 
  Constant & $-$1,305.126 ($-$3,252.499, 642.247) & 22.103 ($-$0.951, 45.156) \\ 
  & p = 0.195 & p = 0.064$^{*}$ \\ 
 \hline \\[-1.8ex] 
Observations & 64 & 94 \\ 
R$^{2}$ & 0.385 & 0.393 \\ 
R$^{2}$ ajusté & 0.282 & 0.336 \\ 
Erreur Std. Résiduelle & 26.984 (df = 54) & 31.747 (df = 85) \\ 
Statistique F & 3.755$^{***}$ (df = 9; 54) & 6.871$^{***}$ (df = 8; 85) \\
\hline 
\hline \\[-1.8ex] 
\textit{Note:}  & \multicolumn{2}{r}{$^{*}$p$<$0.1; $^{**}$p$<$0.05; $^{***}$p$<$0.01} \\ 
\end{tabular} 
}
\end{table} 
%%%%%%%%%





\paragraph{Year}{Année}


\bpar{
The publications year is adjusted following the linear model which adjustement is given in Table~\ref{tab:app:modelography:year}.
}{
L'année de publication est ajustée selon le modèle linéaire dont l'ajustement est donné en Table~\ref{tab:app:modelography:year}.
}


%%%%%%%%%%%%%%
\begin{table}%[!htbp]
  \apptabcaption{\textbf{Linear model for the publication year.}\label{tab:app:modelography:year}}{\textbf{Modèle linéaire pour l'année de publication.}\label{tab:app:modelography:year}}
\bpar{
\begin{tabular}{@{\extracolsep{5pt}}lc} 
\footnotesize
\\[-1.8ex]\hline 
\hline \\[-1.8ex] 
% & \multicolumn{1}{c}{\textit{Dependent variable:}} \\ 
%\cline{2-2} 
\\[-1.8ex] & YEAR \\ 
\hline \\[-1.8ex] 
 TYPEterritory & 10.898 (3.045, 18.750), p = 0.010$^{***}$ \\ 
  TEMPSCALE & 0.035 ($-$0.033, 0.103), p = 0.320 \\ 
  FMETHODeq & $-$6.224 ($-$20.162, 7.714), p = 0.387 \\ 
  FMETHODmap & 4.747 ($-$7.595, 17.089), p = 0.456 \\ 
  FMETHODro & 6.128 ($-$11.694, 23.950), p = 0.504 \\ 
  FMETHODsem & 1.009 ($-$16.659, 18.676), p = 0.912 \\ 
  FMETHODsim & 5.153 ($-$6.809, 17.114), p = 0.404 \\ 
  FMETHODstat & $-$0.357 ($-$10.925, 10.211), p = 0.948 \\ 
  DISCIPLINEengineering & 13.486 ($-$7.238, 34.210), p = 0.210 \\ 
  DISCIPLINEenvironment & $-$3.668 ($-$21.605, 14.269), p = 0.691 \\ 
  DISCIPLINEgeography & 1.121 ($-$4.528, 6.769), p = 0.700 \\ 
  DISCIPLINEphysics & 3.392 ($-$8.461, 15.245), p = 0.578 \\ 
  DISCIPLINEplanning & $-$2.850 ($-$8.873, 3.173), p = 0.359 \\ 
  DISCIPLINEtransportation & 5.503 (0.006, 11.000), p = 0.057$^{*}$ \\ 
  INTERDISC & $-$12.876 ($-$29.567, 3.815), p = 0.138 \\ 
  SEMCOMhedonic & $-$5.769 ($-$19.931, 8.393), p = 0.430 \\ 
  SEMCOMhsr & 6.135 ($-$9.889, 22.159), p = 0.458 \\ 
  SEMCOMinfra planning & $-$4.123 ($-$18.910, 10.663), p = 0.588 \\ 
  SEMCOMnetworks & 4.711 ($-$9.736, 19.158), p = 0.527 \\ 
  SEMCOMtod & $-$1.653 ($-$15.837, 12.532), p = 0.821 \\ 
  Constant & 2,004.945 (1,981.531, 2,028.359), p = 0.000$^{***}$ \\ 
 \hline \\[-1.8ex]
Observations & 64 \\ 
R$^{2}$ & 0.510 \\ 
Adjusted R$^{2}$ & 0.281 \\ 
Residual Std. Error & 6.617 (df = 43) \\ 
F Statistic & 2.234$^{**}$ (df = 20; 43) \\ 
\hline 
\hline \\[-1.8ex] 
\textit{Note:}  & \multicolumn{1}{r}{$^{*}$p$<$0.1; $^{**}$p$<$0.05; $^{***}$p$<$0.01} \\ 
\end{tabular}
}{
\begin{tabular}{@{\extracolsep{5pt}}lc} 
\footnotesize
\\[-1.8ex]\hline 
\hline \\[-1.8ex] 
% & \multicolumn{1}{c}{\textit{Dependent variable:}} \\ 
%\cline{2-2} 
\\[-1.8ex] & YEAR \\ 
\hline \\[-1.8ex] 
 TYPEterritory & 10.898 (3.045, 18.750), p = 0.010$^{***}$ \\ 
  TEMPSCALE & 0.035 ($-$0.033, 0.103), p = 0.320 \\ 
  FMETHODeq & $-$6.224 ($-$20.162, 7.714), p = 0.387 \\ 
  FMETHODmap & 4.747 ($-$7.595, 17.089), p = 0.456 \\ 
  FMETHODro & 6.128 ($-$11.694, 23.950), p = 0.504 \\ 
  FMETHODsem & 1.009 ($-$16.659, 18.676), p = 0.912 \\ 
  FMETHODsim & 5.153 ($-$6.809, 17.114), p = 0.404 \\ 
  FMETHODstat & $-$0.357 ($-$10.925, 10.211), p = 0.948 \\ 
  DISCIPLINEengineering & 13.486 ($-$7.238, 34.210), p = 0.210 \\ 
  DISCIPLINEenvironment & $-$3.668 ($-$21.605, 14.269), p = 0.691 \\ 
  DISCIPLINEgeography & 1.121 ($-$4.528, 6.769), p = 0.700 \\ 
  DISCIPLINEphysics & 3.392 ($-$8.461, 15.245), p = 0.578 \\ 
  DISCIPLINEplanning & $-$2.850 ($-$8.873, 3.173), p = 0.359 \\ 
  DISCIPLINEtransportation & 5.503 (0.006, 11.000), p = 0.057$^{*}$ \\ 
  INTERDISC & $-$12.876 ($-$29.567, 3.815), p = 0.138 \\ 
  SEMCOMhedonic & $-$5.769 ($-$19.931, 8.393), p = 0.430 \\ 
  SEMCOMhsr & 6.135 ($-$9.889, 22.159), p = 0.458 \\ 
  SEMCOMinfra planning & $-$4.123 ($-$18.910, 10.663), p = 0.588 \\ 
  SEMCOMnetworks & 4.711 ($-$9.736, 19.158), p = 0.527 \\ 
  SEMCOMtod & $-$1.653 ($-$15.837, 12.532), p = 0.821 \\ 
  Constant & 2,004.945 (1,981.531, 2,028.359), p = 0.000$^{***}$ \\ 
 \hline \\[-1.8ex]
Observations & 64 \\ 
R$^{2}$ & 0.510 \\ 
R$^{2}$ ajusté & 0.281 \\ 
Erreur Std. Résiduelle & 6.617 (df = 43) \\ 
Statistique F & 2.234$^{**}$ (df = 20; 43) \\ 
\hline 
\hline \\[-1.8ex] 
\textit{Note:}  & \multicolumn{1}{r}{$^{*}$p$<$0.1; $^{**}$p$<$0.05; $^{***}$p$<$0.01} \\ 
\end{tabular}
}
\end{table} 
%%%%%%%%%%%%%%



\stars





