


% Chapter 

%\chapter{Urban Morphogenesis}{Morphogenèse Urbaine} % Chapter title

\bpar{
\chapter{Urban Morphogenesis}
}{
\chapter{Morphogenèse urbaine}
}

\label{ch:morphogenesis} % For referencing the chapter elsewhere, use \autoref{ch:name} 

%----------------------------------------------------------------------------------------

%\headercit{}{}{}

%\bigskip



\bpar{
Geography gives a great importance to spatial relations and to the establishment of networks, as shows for example the first law of \noun{Tobler} combined to the fact that networks are the support of interactions. We unveiled it for the relations between networks and territories for example in section~\ref{sec:interactiongibrat}. However, our results on non-stationarity, together with the evidence of endogenous spatial scales, suggest a certain relevance of the idea of relatively independent sub-systems. It would be then possible to isolate some local rules ruling a sub-system, after some exogenous parameters have been fixed which in particular capture the relations with other sub-systems. This question is related simultaneously to the spatial scale, the temporal scale, but also the elements concerned.
}{
La géographie accorde une grande importance aux relations spatiales et à la mise en réseau, comme l'atteste par exemple la première loi de \noun{Tobler} combinée au fait que les réseaux sont vecteurs des interactions. Nous l'avons mis en évidence pour les relations entre réseaux et territoires par exemple en section~\ref{sec:interactiongibrat}. Toutefois, nos résultats sur la non-stationnarité, ainsi que la mise en valeur d'échelles locales endogènes, suggèrent une certaine pertinence de l'idée de sous-systèmes relativement indépendants. Il serait alors possible d'isoler certaines règles locales régissant un sous-système, un fois fixés certains paramètres exogènes capturant justement les relations avec d'autres sous-systèmes. Cette question porte à la fois sur l'échelle d'espace, de temps, mais aussi sur les éléments concernés.
}


\bpar{
We can go back to a concrete fieldwork example already evoked in chapter~\ref{ch:thematic}: the difficult launch of Zhuhai tramway. The impact of the delay in the operationalization and the questioning of future lines (due to an unexpected technical problem linked to an electrical current transfer technology with a third rail, imported from Europe but which had never been tested in the local climatic conditions which are rather exceptional in terms of humidity), will have a very different nature depending on the scale and the urban actors considered. The lack of general coordination between transportation and urbanism leads to assume that urban dynamics in terms of population and employment are relatively insensitive to it for the moment. The Transportation Bureau of Zhuhai Municipality and also the European technical company having conceived the failing technology may have been the subject of much more significant political and economical impacts, whereas otherwise, let it be in Zhongshan, Macao or Hong-Kong, we can assume that the issue has a nearly inexistant consequence, since the project has a fully local role. There therefore exist some complex interplays of relative independencies and interdependencies in territorial systems.
}{
Reprenons un exemple concret de terrain déjà évoqué au chapitre~\ref{ch:thematic} : la laborieuse mise en place du tramway de Zhuhai. L'impact du retard de la mise en place et la remise en question de futures lignes (dus à un problème technique inattendu lié à une technologie de transfert de courant par troisième rail importée d'Europe qui n'avait jamais été testée dans les conditions climatiques locales assez exceptionnelles en termes d'humidité), aura une nature très différente selon l'échelle et les acteurs urbains considérés. Le manque de coordination générale entre transports et urbanisme laisse supposer que les dynamiques urbaines en termes de populations et d'emplois y sont relativement insensibles dans l'immédiat. Le Bureau des Transports de la Municipalité de Zhuhai ainsi que le bureau technique européen ayant conçu la technologie défectueuse ont pu subir des répercussions politiques et économiques bien plus conséquentes, tandis que par ailleurs, que ce soit à Zhongshan, Macao ou Hong-Kong, nous pouvons supposer que le problème a une repercussion quasi-nulle, le projet ayant un rôle uniquement local. Il existe ainsi des jeux complexes d'indépendances et d'interdépendances relatives dans les systèmes territoriaux.
}

% TODO rq : vraiment pas impact in Macao ? pas si sur en fait, a raffiner comme reflexion.



\bpar{
Generalizing to the local transport system, it can be relatively well isolated from neighbor systems, and thus its relations with the city considered in a local context. It is possible to assume both a certain form of local stationarity but also a certain independency with the exterior. The type of reasoning we just sketched implies the crucial elements which are proper to the idea of \emph{urban morphogenesis}.
}{
Généralisant au système de transport local, celui-ci peut être relativement bien isolé des systèmes voisins, et donc ses relations avec la ville considérée dans un contexte local. Il est possible de supposer à la fois une certaine forme de stationnarité locale mais aussi une certaine indépendance avec l'extérieur. Le type de raisonnement que nous avons esquissé mobilise les éléments essentiels propres à l'idée de \emph{morphogenèse urbaine}.
}


% Nous pouvons également noter que dans ce cadre, son auto-organisation locale impliquera nécessairement des relations fortes entre forme et fonction, de par la distribution spatiale des fonctions urbaines mais aussi car \emph{la forme fait la fonction} dans certains cas de figure, au sens des motifs d'utilisation entièrement conditionnés à cette forme.

%La morphogenèse, qui a été importée de la biologie vers de nombreux champs, a dans chaque cas ouvert des voies pour l'étude des systèmes complexes propres à ce champ selon un point particulier. Il est important de noter que le monument qu'est la Théorie des Catastrophes de \noun{René Thom} introduit une façon originale de comprendre la différentiation qualitative et donc la morphogenèse. Cette théorie a toujours un potentiel d'application considérable aux problèmes qui nous concernent, comme l'a suggéré \noun{Durand-Dastès}~\cite{durand2003geographes} en évoquant la systèmogenèse.


\bpar{
We will in this chapter clarify its definition and show the potentialities it gives to shed light on the relations between networks and territories. First of all, an epistemological effort through complementary viewpoints from different disciplines allow to shed light on the nature of morphogenesis in section~\ref{sec:interdiscmorphogenesis}. This allows to clarify the concept by giving it a very precise definition, distinct from self-organisation, which insists on the causal circular relations between form and function.
}{
Nous allons dans ce chapitre clarifier sa définition et montrer les potentialités qu'elle donne pour éclairer les relations entre réseaux et territoires. Dans un premier temps, un effort d'épistémologie par des points de vue complémentaires de plusieurs disciplines permet d'éclairer la nature de la morphogenèse dans la section~\ref{sec:interdiscmorphogenesis}. Cela permet de clarifier le concept en lui donnant une définition bien précise, distincte de celle de l'auto-organisation, qui appuie les relations causales circulaires entre forme et fonction.
}


\bpar{
We then explore a simple model of urban morphogenesis, based on population density only, at the mesoscopic scale, in section~\ref{sec:densitygeneration}. The demonstration that abstract processes of aggregation and diffusion are sufficient to reproduce a large diversity of forms of human settlements in Europe, by using the results of section~\ref{sec:staticcorrelations}, confirms the relevance of the idea of morphogenesis for modeling at certain scales and for the morphological dimensions.
}{
Nous explorons ensuite un modèle simple de morphogenèse urbaine, basé sur la densité de population seule, à l'échelle mesoscopique, dans la section~\ref{sec:densitygeneration}. La démonstration que les processus abstraits d'agrégation et de diffusion sont suffisants pour reproduire une grande diversité de formes d'établissements humains en Europe, en utilisant les résultats de la section~\ref{sec:staticcorrelations}, confirme la pertinence de l'idée de morphogenèse pour la modélisation à certaines échelles et pour les dimensions morphologiques.
}



\bpar{
This model is then coupled in a sequential way to a network morphogenesis model in section~\ref{sec:correlatedsyntheticdata}, in order to establish a possible space of static correlations between indicators of urban form and network indicators, which are as we previously saw a witness of local relations between networks and territories.
}{
Ce modèle est ensuite couplé de manière séquentielle à un modèle de morphogenèse de réseau dans la section~\ref{sec:correlatedsyntheticdata}, afin d'établir un espace possible des correlations statiques entre indicateurs de forme urbaine et indicateurs de réseau, qui sont comme on l'a vu précédemment un témoin des relations locales entre réseaux et territoires.
}


\bpar{
We thus introduce other building bricks for modeling co-evolution, at the mesoscopic scale through the entry of urban morphogenesis.
}{
Nous posons ainsi d'autres briques de modélisation de la co-évolution, à l'échelle mesoscopique par l'entrée de la morphogenèse urbaine.
}




\stars


\bpar{
\textit{This chapter is composed by various works. The first section is adapted from a work in collaboration with \noun{C. Antelope} (University of California), \noun{L. Hubatsch} (Francis Crick Institute) and \noun{J.M. Serna} (Université Paris VII) following the Santa Fe Institute 2016 summer school~\cite{antelope2016interdisciplinary}; the second section corresponds to~\cite{raimbault2017calibration}; and finally the third section has been written for \emph{Actes des Journées de Rochebrune 2016}~\cite{raimbault2016generation}.}
}{
\textit{Ce chapitre est composé de divers travaux. La première section est adaptée d'un travail en anglais en collaboration avec \noun{C. Antelope} (University of California), \noun{L. Hubatsch} (Francis Crick Institute) et \noun{J.M. Serna} (Université Paris VII) à la suite de l'école d'été 2016 du Santa Fe Institute~\cite{antelope2016interdisciplinary} ; la deuxième section est traduite de~\cite{raimbault2017calibration} ; et enfin la troisième section a été écrite pour les Actes des Journées de Rochebrune 2016~\cite{raimbault2016generation}.}
}

% TODO : rq Rochebrune : written in French, should mention ? if submitted one day, in English ?





%----------------------------------------------------------------------------------------









