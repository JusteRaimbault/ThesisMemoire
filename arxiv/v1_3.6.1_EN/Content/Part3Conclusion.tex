





%\chapter*{Part III Conclusion}{Conclusion de la Partie III}

\bpar{
\chapter*{Conclusion of Part III: a complete view of co-evolution}
}{
\chapter*{Conclusion de la Partie III : une vue complète de la co-évolution}
}


% to have header for non-numbered introduction
\markboth{Conclusion}{Conclusion}


%\headercit{}{}{}


\bpar{
This part thus gave the first elements of the exploration of different entries on the modeling of co-evolution. We explored in chapter~\ref{ch:macrocoevolution} a co-evolution model at the macroscopic scale, which allows to isolate numerous causality regimes, which we can thus designate as co-evolution regimes for the ones exhibiting circular causalities, and which is calibrated on the French system of cities. We therein show that a simple representation and mechanisms already allow to synthetically and empirically capture co-evolution at this scale.
}{
Cette partie a ainsi donné des premiers éléments d'exploration de différentes entrées sur la modélisation de la co-évolution. Nous avons exploré dans le chapitre~\ref{ch:macrocoevolution} un modèle de co-évolution à l'échelle macroscopique, qui permet l'isolation de nombreux régimes de causalité, qu'on peut alors nommer régimes de co-évolution pour ceux présentant des causalités circulaires, et qui est calibré sur le système de villes français. Nous montrons ainsi que des mécanismes et une représentation simple permettent déjà de capturer synthétiquement et empiriquement la co-évolution à cette échelle.
}


\bpar{
We then explored models at a larger scale, implying an increasing complexity. A morphogenesis co-evolution model allows to couple urban form (distribution of population and network topology) with an abstraction of urban functions (centrality and accessibility measures within the network). The different heuristics for network evolution which have been tested appear to be complementary to approach real configurations. Finally, we introduced elements to take into account governance processes in the evolution of transportation networks.
}{
Nous avons ensuite exploré des modèles à une échelle plus grande, impliquant une complexité croissante. Un modèle de co-évolution par morphogenèse permet de coupler la forme urbaine (distribution de la population et topologie du réseau) à une abstraction des fonctions urbaines (mesures de centralité et d'accessibilité dans le réseau). Les différentes heuristiques d'évolution du réseau qui ont été testées se révèlent complémentaires pour s'approcher de configurations réelles. Enfin, nous avons introduit des pistes pour la prise en compte des processus de gouvernance dans l'évolution des réseaux de transport.
}




\subsection*{Processes in models}{Processus modélisés}

%\comment{justifier ici poruquoi pas modèle très fins sur processus eco par exemple (//Levinson) : prix à payer pour être accross scales, disciplines et avoir vraiment de la coevol ? pour ces premières étapes oui. à justifier}


\bpar{
The models we developed have been so in a logic of parsimony, while seeking to effectively capture co-evolution processes at different scales and by being anchored into various disciplines: these constraints are payed by a price on the refinement of integrated mechanisms. We will come back on this compromise in~\ref{sec:contributions}.
}{
Les modèles que nous avons développés l'ont été dans une logique de parcimonie, tout en cherchant à effectivement capturer des processus de co-évolution à différentes échelles et en s'ancrant dans des disciplines variées : ces contraintes se paient par un prix en raffinement des mécanismes intégrés. Nous reviendrons sur ce compromis en~\ref{sec:contributions}.
}


\subsection*{A full view of co-evolution}{Une vue complète de la co-évolution}

% complementarite de la vision conceptuelle/empirique/modelisation : exemples de conclusions fondamentales / adequations / non-adequations pour chaque


\bpar{
At this stage we brought elements of answer to the two axis of our general problematic (how to define and characterize co-evolution, and how to model it). It is remarkable to note that these are articulated within the three knowledge domains of the conceptual (definition), of the empirical (characterization) and of modeling (models). These three aspect reciprocally auto-generate the others, and our viewpoint consists in a true trinity, i.e. a concept which is together unique and triple, in which none of the approaches can be ignored (the same way that \cite{morin2001methode} does for complex anthropology).
}{
Nous avons à ce stade apporté des éléments de réponse aux deux axes de notre problématique générale (comment définir et caractériser la co-évolution, et comment la modéliser). Il est remarquable de noter que ceux-ci s'articulent dans les trois domaines de connaissance du conceptuel (définition), de l'empirique (caractérisation) et de la modélisation (modèles). Ces trois aspects s'auto-génèrent l'un l'autre, et notre point de vue forme une véritable trinité, c'est-à-dire un concept à la fois unique et triple, dans lequel aucune des approches ne peut être ignorée (de la manière dont le fait \cite{morin2001methode} pour l'anthropologie complexe).
}


\bpar{
Thus, models contain the individual aspect of co-evolution (reciprocal interactions between entities), and in some cases the statistical aspect at the scale of a population. This conclusion is made possible through the operational characterization tool, which in its turn allows to reinforce the relevance of the definition.
}{
Ainsi, les modèles contiennent l'aspect individuel de la co-évolution (interactions réciproques entre entités), et dans certains cas l'aspect statistique au niveau d'une population. Cette conclusion est rendue possible par l'outil de caractérisation opérationnelle, celui-ci permettant par ailleurs de renforcer la pertinence de la définition.
}




\subsection*{Perspectives}{Perspectives}

\bpar{
Our point of view on co-evolution has naturally been reducing and limited, since the current state of our modes of knowledge production is still far from a paradigmatic integration of complexity~\cite{morin1991methode}, and that any tentative to apprehend a complex system combines with elegance analysis and synthesis, reductionism and holism, modularity and interdependency. In order to enrich our viewpoint, we finally propose an opening chapter.
}{
Notre point de vue sur la co-évolution a bien entendu été réducteur et limité, puisque l'état actuel de nos modes de production de connaissance est encore loin d'une intégration paradigmatique de la complexité~\cite{morin1991methode}, et que toute tentative d'appréhension d'un système complexe combine habilement analyse et synthèse, réductionnisme et holisme, modularité et interdépendance. Afin d'enrichir notre point de vue, nous proposons finalement un chapitre d'ouverture.
}








